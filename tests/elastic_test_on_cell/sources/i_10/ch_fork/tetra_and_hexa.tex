\documentclass[a4paper,12pt]{article}
\usepackage{graphicx}
\usepackage[utf8]{inputenc}
\usepackage[russian]{babel}
\usepackage{amsmath,amssymb}
\usepackage{tikz}
\usepackage{pgfplots}
\pgfplotsset{compat=1.7}
\begin{document}

\begin{figure}
\centering
\begin{tikzpicture}
\begin{axis}[
%tiny,
enlargelimits,
axis equal,
xmin=-3,xmax=3,
ymin=-3,ymax=3,
xtick=\empty,
ytick=\empty,
]
\fill[gray] (axis cs:-3,-3) circle[radius=1]; \fill[gray] (axis cs:-3,0) circle[radius=1];
\fill[gray] (axis cs:-3,+3) circle[radius=1]; \fill[gray] (axis cs:0,-3) circle[radius=1];
\fill[gray] (axis cs:0,0) circle[radius=1]; \fill[gray] (axis cs:0,+3) circle[radius=1];
\fill[gray] (axis cs:+3,-3) circle[radius=1]; \fill[gray] (axis cs:+3,0) circle[radius=1];
\fill[gray] (axis cs:+3,+3) circle[radius=1];
\end{axis}
\end{tikzpicture}
\caption{Тетрагональная симметрия} \label{fig1}
\end{figure}



\begin{figure}
\centering
\begin{tikzpicture}
\begin{axis}[
%tiny,
enlargelimits,
axis equal,
xmin=-3,xmax=3,
ymin=-3,ymax=3,
xtick=\empty,
ytick=\empty,
]
\fill[gray] (axis cs:-1.5,+2.5980762) circle[radius=1]; \fill[gray] (axis cs:+1.5,+2.5980762) circle[radius=1];
 \fill[gray] (axis cs:-3,0) circle[radius=1]; \fill[gray] (axis cs:0,0) circle[radius=1]; \fill[gray] (axis cs:+3,0) circle[radius=1];
\fill[gray] (axis cs:-1.5,-2.5980762) circle[radius=1]; \fill[gray] (axis cs:+1.5,-2.5980762) circle[radius=1];
\end{axis}
\end{tikzpicture}
\caption{Гексагональная симметрия} \label{fig2}
\end{figure}

\begin{equation} \label{e1}
\begin{pmatrix} 
C_{xxxx} & C_{xxyy} & C_{xxzz} & 0 & 0 & 0\\
C_{yyxx} & C_{yyyy} & C_{yyzz} & 0 & 0 & 0\\
C_{zzxx} & C_{zzyy} & C_{zzzz} & 0 & 0 & 0\\
0 & 0 & 0 & C_{xzxz} & 0 & 0 \\
0 & 0 & 0 & 0 & C_{yzyz} & 0 \\
0 & 0 & 0 & 0 & 0 & C_{xyxy} \\
\end{pmatrix} 
\end{equation} 
\\
\begin{equation} \label{e2}
\begin{array}{c}
C_{xxxx} == C_{yyyy}, C_{xxyy} == C_{yyxx}, C_{xxzz} == C_{yyzz}, C_{xzxz} == C_{yzyz}, \\ \\
C_{xyxy} = \frac{C_{xxxx} - C_{xxyy}}{2}, \\ \\
C_{xyxy} = \frac{C_{yyyy} - C_{yyxx}}{2}, \\ \\
\end{array} 
\end{equation} 

$G_{xy} = C{xyxy}$ - модуль поперечного сдвига.\\

Материал имеет трансверсальную изотропию упругих свойств в случае его гексагональной симметрии (Рис. \ref{fig1}). 
Тогда его матрица упругих своств принимает вид (\ref{e1}), для которой выполняются соотношения (\ref{e2}). 
Если же материал обладает тетрагональной симметрией (Рис. \ref{fig2}), то он не является трансверсально изотропным 
и соотношения (\ref{e2}) не выполняются.

При повороте системы координат вокруг оси симметрии четвертого порядка (здесь ось Oz), 
компоненты тензора меняются, тензор принимает вид: 

\begin{equation} \label{e3}
\begin{pmatrix} 
C_{xxxx} & C_{xxyy} & C_{xxzz} & 0 & 0 & C_{xxxy}\\
C_{yyxx} & C_{yyyy} & C_{yyzz} & 0 & 0 & C_{yyxy}\\
C_{zzxx} & C_{zzyy} & C_{zzzz} & 0 & 0 & 0\\
0 & 0 & 0 & C_{xzxz} & 0 & 0 \\
0 & 0 & 0 & 0 & C_{yzyz} & 0 \\
C_{xyxx} & C_{xyyy} & 0 & 0 & 0 & C_{xyxy} \\
\end{pmatrix} 
\end{equation} 

В нем $C_{xxxy} = -C_{yyxy}$ и $C_{xyxx} = -C_{xyyy}$.

Величина $C_{x'x'x'y'}$ в новой системе координат $Ox'y'z$, при повороте на угл $\alpha$, 
для тетрагональной симметрии получается таким образом:

\begin{equation} \label{e4}
\begin{gathered}
C_{x'x'x'y'} = \\
(C_{xxxx} a^3 + C_{xyxy}  a  b^2 + C_{xyxy}  a  b^2 + C_{yyxx}  a  b^2)  b - \\
(C_{xxyy}  a^2  b + C_{xyxy}  a^2  b + C_{xyxy}  a^2 b + C_{yyyy}  b^3)  a = \\
C_{xxxx} a^3 b + 2 C_{xyxy}  a  b^3 + C_{yyxx}  a  b^3 - C_{xxyy}  a^3  b - 
2 C_{xyxy}  a^3  b - C_{yyyy}  b^3 a = \\
(C_{xxxx} - C_{xxyy} - 2 C_{xyxy}) a^3 b + (C_{yyyy} - C_{yyxx} - 2 C_{xyxy}) a b^3 = \\
(C_{xxxx} - C_{xxyy} - 2 C_{xyxy}) (a^3 b - a b^3)
\end{gathered}
\end{equation} 

где $a = cos(\alpha)$, $b = sin(\alpha)$.

Матрица поворота вокруг оси Oz:
\begin{equation} \label{e5}
\begin{pmatrix} 
cos(\alpha) & -sin(\alpha) & 0 \\
sin(\alpha) & cos(\alpha) & 0 \\
0 & 0 & 1 \\
\end{pmatrix} 
\end{equation} 

\begin{equation} \label{e6}
a^3 b - a b^3 = cos(\alpha)^3sin(\alpha) - cos(\alpha)sin(\alpha)^3 = \frac{1}{4}sin(4\alpha)
\end{equation}

Из (\ref{e4}) и (\ref{e6}) следует, что величина $C_{xxxy}(\alpha)$ достигает максимума при 
$\alpha = \frac{\pi}{8}$ и равена $\frac{1}{4}(C_{xxxx}(0) - C_{xxyy}(0) - 2 C_{xyxy}(0))$, а значит 
$\frac{1}{2}(\frac{C_{xxxx}(0) - C_{xxyy}(0)}{2} - C_{xyxy}(0))$.

%\includegraphics[width=1\linewidth]{Mxy_c_h}


\end{document}













