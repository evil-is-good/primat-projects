
\documentclass{beamer} 
\usepackage{beamerthemesplit} 

\usetheme{Madrid}%Frankfurt} 
%Boadilla}
\usecolortheme{default}

\setbeamertemplate{footline}{%
    \hspace{0.88\paperwidth}%
    \usebeamerfont{title in head/foot}%
    \large{\insertframenumber\,/\,\inserttotalframenumber}%
}

\setbeamertemplate{navigation symbols}{}

% Выпишем часть возможных стилей, некоторые из них могут содержать

% дополнительные опции

% default, Bergen, Madrid, AnnArbor,Pittsburg, Rochester, 

% Antiles, Montpellier, Berkley, Berlin, CambridgeUS

% 

% Далее пакеты, необходимые вам для создания презентации
\usepackage[T2A]{fontenc}
\usepackage[utf8]{inputenc}
\usepackage[english,russian]{babel}
\usepackage{amssymb,amsfonts,amsmath,mathtext}
\usepackage{cite,enumerate,float,indentfirst}
\usepackage{graphicx}

\usefonttheme[stillsansserifsmall]{serif}
%\usefonttheme[onlymath]{serif}

\title{Математическое моделирование упругих макрохарактеристиу
для композитных материалов с включениями в виде прямых периодически распределённых
параллельных волокон}
%МАТЕМАТИЧЕСКОЕ МОДЕЛИРОВАНИЕ УПРУГИХ МАКРОХАРАКТЕРИСТИК 
%ДЛЯ КОМПОЗИТНЫХ МАТЕРИАЛОВ С ВКЛЮЧЕНИЯМИ В ВИДЕ ПРЯМЫХ 
%ПЕРИОДИЧЕСКИ РАСПРЕДЕЛЁННЫХ ПАРАЛЛЕЛЬНЫХ ВОЛОКОН}
\author{А.Ф. Власко, аспирант}
\institute{Сургутский Государственный университет \\
Политехнический институт \\
Строительное отделение \\
\vspace{0.5cm}
Научный руководитель --- д-р.ф.-м.н. {\bf Г.Л. Горынин}}
\date{Новосибирск, 2013}


\begin{document} 

\maketitle

\frame{ \frametitle{2-периодическая среда} 

\begin{figure}
    \begin{center}
        \includegraphics[scale=0.2]{cube}
    \end{center}
\end{figure}

}

\frame{ \frametitle{Стационарное уравнение теплового равновесия} 

\begin{equation}
    \begin{array}{c}
        \frac{\partial q_x}{\partial x} +
        \frac{\partial q_y}{\partial y} + 
        \frac{\partial q_z}{\partial z} + Q = 0, \\ \\
        q_{\alpha} = \sum\limits_{\beta \in \{x,y\}} 
        \lambda_{\alpha \beta} 
        \frac{\partial T}{\beta} \\ \\ 
        \alpha \in \{x,y,z\}
    \end{array} 
\end{equation}

\begin{equation}
        [q_{n}] = 0, [T] = 0
\end{equation}

\begin{equation}
        [q_{n}] = 
        [q_{x}]n_x +
        [q_{y}]n_y +
        [q_{z}]n_z
\end{equation}

} 

\frame{ \frametitle{Стационарное уравнение теплового равновесия в безразмерных величинах} 

\begin{equation}
    \begin{array}{c}
        \frac{\partial q_x}{\partial x}\varepsilon +
        \frac{\partial q_y}{\partial y}\varepsilon + 
        \frac{\partial q_z}{\partial z}\varepsilon + Q = 0, \\ \\
        q_{\alpha} = \sum\limits_{\beta \in \{x,y\}} 
        \lambda_{\alpha \beta} 
        \frac{\partial T}{\beta}\varepsilon \\ \\ 
        \alpha \in \{x,y,z\}
    \end{array} 
\end{equation}

\begin{equation}
    \begin{array}{c}
    x \leftrightarrow \frac{x}{L},
    y \leftrightarrow \frac{y}{L},
    z \leftrightarrow \frac{z}{L},
    T \leftrightarrow \frac{T}{T^*}, \\ \\
    \lambda_{\alpha \beta} \leftrightarrow  
    \frac{\lambda_{\alpha \beta}}{\lambda^*},
    q_{\alpha} \leftrightarrow \frac{q_{\alpha}}{q^*},
    Q \leftrightarrow \frac{Qh}{q^*}, \\ \\
    q^*=\frac{\lambda^*T^*}{h}
    \end{array} 
\end{equation}

\begin{equation*}
\varepsilon = \frac{h}{L} \ll 1
\end{equation*}

}

\frame{ \frametitle{Верхняя и нижняя оценки Рейсса-Фойгта} 

\begin{equation}
    \frac{1}{\frac{\Theta^I}{\lambda_{\alpha\beta}^I} +
    \frac{\Theta^B}{\lambda_{\alpha\beta}^B}} \le
    \widetilde{\lambda}_{\alpha\beta} \le
    \Theta^I\lambda_{\alpha\beta}^I +
    \Theta^B\lambda_{\alpha\beta}^B
\end{equation}

\begin{equation}
    \begin{array}{c}
        \Theta^I - \text{доля включения (коэффициент армирования)} \\ \\
        \Theta^B - \text{доля связующего}
    \end{array} 
\end{equation}

}

\frame{ \frametitle{Формула Хашина-Штрикмана} 

\begin{equation}
    \begin{array}{c}
        \widetilde{\lambda}_{xx} = \widetilde{\lambda}_{yy} =
        \lambda^B\left[1+\frac{\Theta^I}
            {\lambda^B/(\lambda^I-\lambda^B)+\Theta^B/2}\right] \\ \\
    \widetilde{\lambda}_{zz} =
    \Theta^I\lambda^I +
    \Theta^B\lambda^B
    \end{array} 
\end{equation}

}

\frame{ \frametitle{Формула Ванина для включений цилиндрического сечения, 
тетрагональной упаковки} 

\begin{equation}
    \begin{array}{c}
        \widetilde{\lambda}_{xx} = \widetilde{\lambda}_{yy} = 
        \lambda_0\Bigg[1-n^2(n-1)\frac{\lambda_0}{\lambda^B}\left(
            \frac{1-\lambda^B/\lambda^I}{\Theta^B+(1+\Theta^I)\lambda^B/\lambda^I} 
            \right)^2 \times \\ \\ \times \frac{\sin^2\alpha}{\pi^2}
            \left( (\Theta^I)^2-(\Theta^I)^{2n}\left\{\frac{1-\lambda^B/\lambda^I}
            {1+\lambda^B/\lambda^I}\right\}^2\right)\Bigg] \\ \\
            \lambda_0 = \lambda^B\frac
            {1 + \Theta^I+\Theta^B\lambda^B/\lambda^I}
            {1 + \Theta^I+(1+\Theta^I)\lambda^B/\lambda^I} \\ \\
            \alpha = \frac{\pi}{2}, n = 4 - \text{тетрагональная упаковка}\\ \\ 
            \alpha = \frac{\pi}{3}, n = 6 - \text{гексагональная упаковка}\\ \\ 
            \widetilde{\lambda}_{zz} =
            \Theta^I\lambda^I +
            \Theta^B\lambda^B
    \end{array} 
\end{equation}

}

\frame{ \frametitle{Асимптотическое разложение} 

\begin{equation}
T^{(n)} = T_0^{(n)} + 
\sum\limits_{k=1}^{n}\sum\limits_{k_x+k_y+k_z=k}\left( 
\Psi^{\bar{k}}(\xi)
\frac{\partial^k T_0^{(n)}}{\partial r^{\bar{k}}}\varepsilon^k \right)
\end{equation}

\begin{equation}
q_{\alpha}^{(n)} = 
\sum\limits_{k=1}^{n}\sum\limits_{k_x+k_y+k_z=k}\left( 
K_{\alpha}^{\bar{k}}(\xi)
\frac{\partial^k \mu_0^{(n)}}{\partial r^{\bar{k}}}\varepsilon^k \right)
\end{equation}

\begin{equation}
    \begin{array}{c}
    \bar{k}=\{k_x,k_y,k_z\} = k_x\bar{\text{э}}_x+k_y\bar{\text{э}}_y+k_z\bar{\text{э}}_z, \\
    |\bar{k}|=k=k_x+k_y+k_z, \\ 
    \partial r^{\bar{k}}= \partial x^{k_x} \partial y^{k_y} \partial z^{k_z} \\
    k_{\alpha} \ge 0, k_{\alpha} \in \mathbb{Z}
    \end{array} 
\end{equation}

}

\frame{ \frametitle{Одна ячейка} 

\begin{figure}
    \begin{center}
        \includegraphics[scale=0.2]{cell}
    \end{center}
\end{figure}

\begin{equation}
\xi_x,\xi_y \in [0,1]
\end{equation}

\begin{equation}
x = x_i + \xi_x\varepsilon, \
y = y_i + \xi_y\varepsilon, \
z = z.
\end{equation}

}

\frame{ \frametitle{Краевые задачи на ячейке} 

\begin{equation}
    \begin{array}{c}
        \frac
        {\partial 
            K_x^{\overline{\text{э}}_{\beta}}}
        {\partial x} +
        \frac
        {\partial
            K_y^{\overline{\text{э}}_{\beta}}}
        {\partial y} = 0 \\ \\
        K_{\alpha}^{\overline{\text{э}}_{\beta}} =
        \lambda_{\alpha \beta} + 
        \sum\limits_{\varphi,\psi \in \{x,y\}} \lambda_{\varphi \psi} 
        \frac{\partial \Psi_{\varphi}^{\overline{\mu}_{\beta}}}
        {\partial \xi_{\psi}} + 
        \sum\limits_{\psi \in \{x,y\}} \lambda_{z \psi} 
        \frac{\partial \Psi_{z}^{\overline{\mu}_{\beta}}}
        {\partial \xi_{\psi}} \\ \\ 
        \alpha \in \{x,y\}, \beta \in \{x,y,z\}
    \end{array} 
\end{equation}

\begin{equation}
    \begin{array}{c}
        \left. \Psi^{\overline{\text{э}}_{\beta}} 
        \right|_{\xi_{\gamma} = 0} =
        \left. \Psi^{\overline{\text{э}}_{\beta}} 
        \right|_{\xi_{\gamma} = 1} \\ \\
        \left. K_{\alpha}^{\overline{\text{э}}_{\beta}}
        \right|_{\xi_{\gamma} = 0} =
        \left. K_{\alpha}^{\overline{\text{э}}_{\beta}}
        \right|_{\xi_{\gamma} = 1} \\ \\
        \gamma \in \{x, y\}
    \end{array} 
\end{equation}

}

\frame{ %\frametitle{Noname} 

\begin{equation}
        \left[ \Psi^{\overline{\text{э}}_{\beta}} \right] = 0,
        \left[ K_{\alpha}^{\overline{\text{э}}_{\beta}} \right] = 0
\end{equation}

\begin{equation}
        \langle \Psi^{\overline{\text{э}}_{\beta}} \rangle = 0
\end{equation}

\begin{equation}
        \langle \_ \rangle = \int\limits_0^1\int\limits_0^1\_d\xi_xd\xi_y
\end{equation}

}

\frame{ \frametitle{Вычисление коэффициентов теплопроводности макросреды}

\begin{equation}
    \begin{array}{c}
        \widetilde{\lambda}_{\alpha \beta} = 
        \langle 
        K_{\alpha}^{\overline{\text{э}}_{\beta}}
        \rangle = \\ \\
        \langle 
        \lambda_{\alpha \beta} 
        \rangle +
        \left\langle 
        \sum\limits_{\varphi,\psi \in \{x,y\}} \lambda_{\varphi \psi} 
        \frac{\partial \Psi_{\varphi}^{\overline{\mu}_{\beta}}}
        {\partial \xi_{\psi}} 
        \right\rangle + 
        \left\langle 
        \sum\limits_{\psi \in \{x,y\}} \lambda_{z \psi} 
        \frac{\partial \Psi_{z}^{\overline{\mu}_{\beta}}}
        {\partial \xi_{\psi}} 
        \right\rangle \\ \\
        \alpha, \beta \in \{x,y,z\}
    \end{array} 
\end{equation}

}

\frame{ \frametitle{Периодические ячейки с различными формами поперечных сечений
арматурных волокон} 

\begin{figure}
    \begin{center}
        \includegraphics[scale=0.3]{cells}
    \end{center}
\end{figure}

\begin{equation}
    \begin{array}{c}
        w = b - \sqrt{b^2 - S^I}, b \approx 0.75 \\ \\
        S^I - \text{Площадь включения}
    \end{array} 
\end{equation}

}

\end{document} 
\frame{ \frametitle{Модуль Юнга $E_x$} 

\begin{figure}
    \begin{center}
        \includegraphics[scale=0.3]
        {../elastic_test_on_cell/sources/i_10/article_nowosib/Ex}
    \end{center}
\end{figure}

}


\end{document} 
