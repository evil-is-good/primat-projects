\chapter{\ }

Она услышала перешептыванья и, подойдя к двери, увидела негров, испуганно толпившихся в холле у двери на задний двор:
Дилси стояла, с трудом держа на руках отяжелевшего спящего Во, дядюшка Питер плакал, а кухарка вытирала передником широкое мокрое лицо. Все трое посмотрели на Скарлетт, как бы молча спрашивая, что же делать теперь. Она посмотрела в направлении гостиной и увидела Индию и тетю Питти — они стояли молча, держась за руки, и вид у Индии впервые не был высокомерным. Как и негры, они умоляюще взглянули на Скарлетт, словно ожидая от нее указаний. Скарлетт прошла в гостиную, и обе женщины тотчас приблизились к ней.
— Ах, Скарлетт, что же… — начала было тетя Питти, ее по-детски пухлые губы тряслись.
— Не говорите со мной, или я закричу, — сказала Скарлетт. От чрезмерного напряжения голос ее прозвучал резко, она крепко прижала к бокам стиснутые кулаки. При мысли о том, что сейчас надо будет говорить о Мелани, давать необходимые распоряжения, сопутствующие смерти, она почувствовала, как в горле у нее встал ком. — Я не желаю слышать от вас ни слова.
Этот властный тон заставил их отступить, на лицах у обеих появилось беспомощное, обиженное выражение. «Я не должна плакать при них, — подумала Скарлетт. — Мне надо держаться, или они тоже расплачутся, а за ними начнут реветь черные, и тогда мы все с ума сойдем. Надо взять себя в руки. Мне предстоит столько всего сделать. Повидать гробовщика, и устроить похороны, и проследить за тем, чтобы в доме было чисто, и разговаривать с людьми, которые будут рыдать у меня на груди. Эшли всем этим заняться не может, Питти и Индия тоже не смогут. Значит, придется мне. Ох, как же это тяжело! Это всегда было тяжело, но всегда этим занимался кто-то другой!» Она посмотрела на удивленные, обиженные лица Индии и Питти, и искреннее раскаяние овладело ею. Мелани не понравилось бы, что она так резка с теми, кого та любила.
— Извините, что я нагрубила вам, — с трудом выговорила она. — Это просто оттого.., словом, извините, тетя, что я вам нагрубила. Я на минуту выйду на крыльцо. Мне надо побыть одной. А потом вернусь, и тогда мы…
Она потрепала тетю Питти по плечу и быстро прошла мимо нее к двери на крыльцо, чувствуя, что если еще хоть минуту пробудет в этой комнате, то уже не сумеет совладать с собой. Ей надо побыть одной. Надо выплакаться, иначе сердце у нее лопнет.
Она вышла на темное крыльцо и закрыла за собой дверь; влажный ночной воздух повеял прохладой ей в лицо. Дождь перестал, и вокруг стояла тишина — только время от времени капало с карниза крыши. Мир был окутан густым туманом, холодным туманом, который нес с собой запах умирающего года. Все дома на другой стороне улицы стояли темные, за исключением одного, и свет от лампы в окне этого дома, падая на улицу, боролся с туманом, окрашивая золотом клубившиеся в его лучах клочья. Казалось, будто весь мир накрыло одеялом серого дыма. И весь мир застыл.
Она прижалась головой к одному из столбов на крыльце и хотела заплакать, но слез не было. Слишком большая случилась беда — тут слезами не поможешь; Ее всю трясло. В мозгу снова и снова возникал образ двух неприступных крепостей, которые вдруг с грохотом рухнули у нее на глазах. Она стояла, пытаясь обрести опору в старом заклятии: «Я подумаю об этом потом, завтра, когда станет легче». Но заклятие потеряло свою силу. Она не могла не думать о двух людях: о Мелани, которую она, оказывается, так любила и которая, оказывается, так ей нужна, и об Эшли и своей безграничной слепоте, не позволявшей ей видеть его таким, каким он был. Скарлетт понимала, что думать об этом ей будет так же больно завтра, как и послезавтра, и все дни потом.
«Не могу я сейчас вернуться туда и говорить с ними, — думала она. — Не могу я сегодня вечером видеть Эшли и утешать его. Не сегодня! Завтра утром я приду пораньше, и все сделаю, что надо, и скажу все утешительные слова, какие должна сказать. Но не сегодня. Я не могу. Я пойду домой».
Дом был всего в пяти кварталах. Она не станет ждать, пока рыдающий Питер заложит кабриолет, не станет ждать, пока доктор Мид отвезет ее домой. Не в состоянии она вынести слезы одного и молчаливое осуждение другого. Она быстро — без шляпки и накидки — спустилась по темным ступеням крыльца и зашагала в туманной ночи. Завернув за угол, она пошла вверх, к Персиковой улице, ступая в этом застывшем мокром мире беззвучно, точно во сне.
Пока она шла вверх по холму, неся в себе груз непролитых слез, у нее возникло призрачное чувство, будто она уже раньше была в этом сумрачном холодном месте и при таких же обстоятельствах, причем была не раз и не два, а много раз. «Глупости какие, — подумала она, но ей стало не по себе, и она ускорила шаг. — Нервы разыгрались». Но возникшее ощущение не проходило, исподволь завладевая ее сознанием. Она боязливо озиралась вокруг, а ощущение все росло, нереальное, но такое знакомое, и она вдруг вскинула голову, точно животное, почуявшее опасность. «Я просто устала, — пыталась она успокоить себя. — И ночь такая странная, такой туман. Я никогда прежде не видела такого густого тумана, разве что.., разве что!..» И тут она поняла, и от страха у нее сжалось сердце. Теперь она знала. Сотни раз она видела этот кошмар, когда бежала сквозь такой же туман по призрачным местам, где не было ни указательных столбов, ни стрелок, пробираясь сквозь холодную липкую мглу, населенную цепкими тенями и привидениями. Во сне это с ней или просто сон стал явью?
На какое-то время ощущение реальности покинуло ее, и она окончательно растерялась. Чувство, испытанное во время кошмара, овладело ею с особою силой, сердце учащенно забилось. Она снова стояла среди смерти и гробовой тишины, как стояла когда-то в Таре. Все существенное, важное исчезло из жизни, жизнь лежала в обломках, и паника, словно холодный ветер, завывала в ее сердце. Ужас, который возникал из тумана и который был самим туманом, наложил на нее свою лапу, и она побежала. Она бежала сейчас, как бежала сотни раз во сне, — бежала вслепую, неизвестно куда, подгоняемая безымянными страхами, ища в сером тумане безопасное прибежище, которое должно же где-то быть!
Она мчалась по сумеречной улице, пригнув голову, с бешено колотящимся сердцем, ночной, сырой воздух прилипал к ее губам, деревья над головой угрожали ей чем-то. Но где-то, где-то в этом диком краю сырой тишины должен же быть приют! Она спешила задыхаясь вверх по холму, мокрые юбки холодом били по лодыжкам, легкие, казалось, вот-вот разорвутся, тугой корсет впивался в ребра, давил на сердце.
Внезапно перед глазами ее возник огонек, целый ряд огоньков, неясных, мерцающих, но настоящих огоньков. В ее кошмарах никогда не было огоньков — только серый туман. Сознание Скарлетт зацепилось за эти огоньки. Огоньки — это безопасность, это люди, это реальность, и она резко остановилась, сжимая руки, стараясь побороть панику, напряженно вглядываясь в ряд газовых фонарей, подсказывавших ее смятенному уму, что она — в Атланте, на Персиковой улице, а не в сером мире призраков и сна.
Тяжело дыша, она опустилась на каменную тумбу, к которой подъезжают кареты, стараясь совладеть с нервами, словно это были веревки, выскальзывавшие из рук.
«Я бежала… Бежала как сумасшедшая! — думала она, и хотя ее все еще трясло, но страх уже проходил, сердце же все еще колотилось до тошноты. — Но куда я бежала?» Ей стало легче дышать — она сидела, прижав руки к груди, и смотрела вдоль Персиковой улицы. Там, на самой вершине холма, стоит ее дом. Отсюда казалось, что во всех окнах — свет, свет, бросавший вызов туману. Дом! Вот это — реальность! Она смотрела на еще далекий, смутно очерченный силуэт дома с благодарностью, с жадностью, и что-то похожее на умиротворение снизошло на нее.
Домой! Вот куда ей хотелось. Вот куда она бежала. Домой — к Ретту!
Стоило Скарлетт осознать это, как с нее словно свалились путы, а вместе с ними — страх, преследовавший ее с той ночи, как она добралась до Тары и обнаружила, что ее мир рухнул. Тогда, вернувшись в Тару, она обнаружила, что никакой уверенности в завтрашнем дне нет и в помине и не осталось ни силы, ни мудрости, ни любви и нежности, ни понимания — ничего, что привносила в жизнь Эллин и что служило ей, Скарлетт, опорой в юности. И хотя с той ночи Скарлетт сумела материально обезопасить себя, в своих снах по-прежнему была испуганным ребенком, тщетно искавшим утраченную уверенность в завтрашнем дне, свой утраченный мир.
Сейчас она поняла, какое прибежище искала во сне, кто был источником тепла и безопасности, которых она никак не могла обрести среди тумана. Это был не Эшли — о, вовсе не Эшли! В нем нет тепла, как нет его в блуждающих огнях, с ним ты не чувствуешь себя в безопасности, как не чувствуешь себя в безопасности среди зыбучих песков. Это был Ретт — Ретт, который своими сильными руками может обнять ее, на чью широкую грудь она может положить свой усталую голову, чьи подтрунивания и смех помогают видеть вещи в их истинном свете. Ретт, который полностью понимает ее, потому что он, подобно ей, видит правду как она есть, а не затуманенную всякими так называемыми высокими представлениями о чести, самопожертвовании или вере в человеческую натуру. И он любит ее! Почему она до сих пор никак не могла понять, что он любит ее, несмотря на свои колкости, свидетельствовавшие вроде бы об обратном! Мелани вот поняла и уже на краю могилы сказала: «Будь подобрее к нему».
«О, — подумала она, — Эшли не единственный до глупости слепой человек. Мне бы следовало давно это понять».
На протяжении многих лет она спиною чувствовала каменную стену любви Ретта и считала ее чем-то само собою разумеющимся — Как и любовь Мелани, — теша себя мыслью, будто черпает силу только в себе. И так же, как она поняла сегодня, что Мелани всегда была рядом с нею в ее жестоких сражениях с жизнью, — поняла она сейчас и то, что в глубине, за нею всегда молча стоял Ретт, любивший ее, понимавший, готовый помочь. Ретт, который на благотворительном базаре, увидев нетерпение в ее глазах, пригласил ее на танец; Ретт, который помог ей сбросить с себя путы траура; Ретт, увезший ее сквозь пожары и взрывы в ту ночь, когда пала Атланта; Ретт, утешавший ее, когда она с криком просыпалась, испуганная кошмарами, — да ни один мужчина не станет такого делать, если он не любит женщину до самозабвения!
С деревьев на нее капало, но она этого даже не замечала. Вокруг клубился туман, но она не обращала на это внимания. Думая о Ретте, вспоминая его смуглое лицо, сверкающие белые зубы, черные живые глаза, она почувствовала, как дрожь пробежала у нее по телу.
«А ведь я люблю его, — подумала она и, как всегда, восприняла эту истину без удивления, словно ребенок, — принимающий подарок. — Я не знаю, как давно я его люблю, но это правда. И если бы не Эшли, я бы уже давно это поняла. Мне никогда не удавалось увидеть мир в подлинном свете, потому что между мною и миром всегда стоял Эшли».
Она любила его — негодяя, мерзавца, человека без чести и совести, во всяком случае в том смысле, в каком Эшли понимал честь. «Да ну его к черту, этого Эшли с его честью! — подумала она. — Из-за своего понятия о чести Эшли всегда предавал меня. Да, с самого начала, когда он ходил к нам, хотя знал, что его семья хочет, чтобы он женился на Мелани. А вот Ретт никогда меня не предавал — даже в ту страшную ночь, когда у Мелани был прием и он должен был бы свернуть мне шею. Даже в ту ночь, когда пала Атланта и он оставил меня на дороге, он знал, что я — в безопасности, он знал, что я как-нибудь выкручусь. Даже тогда, в тюрьме инки, когда он повел себя так, будто мне придется расплатиться с ним за те деньги, которые он мне даст. Он никогда бы насильно не овладел мною. Он просто испытывал меня. Он меня все время любил. А я как низко себя с ним вела! Сколько раз я его, обижала, а он был слишком горд, чтобы дать мне это понять. А когда умерла Бонни… Ах, как я могла?!» Она поднялась во весь рост и посмотрела на дом на холме. Всего полчаса тому назад она считала, что потеряла все, кроме денег, — все, что делало жизнь желанной: Эллин, Джералда, Бонни, Мамушку, Мелани и Эшли. Оказывается, ей надо было потерять их всех, чтобы понять, как она любит Ретта, — любит, потому что он сильный и беспринципный, страстный и земной, как она.
«Я все ему скажу, — думала она. — И он поймет. Он всегда донимал. Я скажу ему, какая я была дура, и как я люблю его, и как постараюсь все загладить». Она вдруг почувствовала себя сильной и такой счастливой. Она уже не боялась больше темноты и тумана, и сердце у нее запело от сознания, что никогда уже она не будет их бояться, какие бы туманы ни клубились вокруг нее в будущем, она теперь знает, где найти от них спасение. Она быстро пошла по улице к своему дому, и каждый квартал казался ей бесконечно длинным. Слишком, слишком длинным. Она подобрала юбки и побежала. Но она бежала не от страха. Она бежала, потому что в конце улицы ее ждали объятия Ретта.

