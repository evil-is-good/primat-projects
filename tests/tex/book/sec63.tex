\chapter{\ }

Парадная дверь была приоткрыта, и Скарлетт, задыхаясь, вбежала в холл и на мгновение остановилась в радужном сиянии люстры. Хотя дом был ярко освещен, в нем царила глубокая тишина — не мирная тишина сна, а тишина настороженная, усталая и слегка зловещая. Скарлетт сразу увидела, что Ретта нет ни в зале, ни в библиотеке, и сердце у нее упало. А что, если он не здесь, а где-нибудь с Красоткой или там, где он проводит вечера, когда не появляется к ужину? Этого она, как-то не учла.
Она стала было подниматься по лестнице, надеясь все-таки обнаружить его, когда увидела, что дверь в столовую закрыта. Сердце у нее чуть сжалось от стыда при виде этой закрытой двери — она вспомнила многие вечера прошедшего лета, когда Ретт сидел там один и пил до одурения, а потом Порк являлся и укладывал его в постель. Она виновата в том, что так было, но она все это изменит. Теперь все будет иначе… Только бы, великий боже, он не был сегодня слишком пьян. «Ведь если он сегодня будет слишком пьян, он мне не поверит и начнет смеяться, и это разобьет мне сердце».
Она тихонько раздвинула створки двери и заглянула в столовую. Он сидел за столом согнувшись, перед ним стоял графин, полный вина, но неоткупоренный, и в рюмке ничего не было. Слава богу, он трезв. Она потянула в стороны створки, стараясь удержаться, чтобы не броситься к нему. Но когда он посмотрел на нее, что-то в его взгляде остановило ее и слова замерли у нее на губах.
Он смотрел на нее в упор своими черными глазами из-под набрякших от усталости век, и в глазах его не загорелись огоньки. Хотя волосы у нее в беспорядке рассыпались по плечам, грудь тяжело вздымалась, а юбки были до колен забрызганы грязью, на лице его не отразилось удивления или вопроса, а губы не скривились в легкой усмешке. Он грузно развалился в кресле сюртук, обтягивавший пополневшее, некогда такое стройное и красивое тело, был смят, волевое лицо огрубело. Пьянство и разврат сотворили свое дело: его некогда четко очерченный профиль уже не напоминал профиля молодого языческого вождя на новой золотой монете, — это был профиль усталого, опустившегося Цезаря, выбитый на медяшке, стертой от долгого хождения. Он смотрел на Скарлетт, а она стояла у порога, прижав руку к сердцу, — смотрел долго, спокойно, чуть ли не ласково, и это испугало ее.
— Проходите, садитесь, — сказал. — Она умерла? Скарлетт кивнула и нерешительно пошла к нему, не зная, как действовать и что говорить при виде этого нового выражения на его лице. Он не встал — лишь ногой пододвинул ей кресло, и она опустилась в него. Жаль, что он заговорил сразу о Мелани. Ей не хотелось говорить сейчас о Мелани, не хотелось снова переживать боль минувшего часа. Еще будет время поговорить о Мелани. Ей так хотелось крикнуть: «Я люблю тебя!», и ей казалось, что именно теперь, в эту ночь, настал час, когда она должна сказать Ретту о том, что у нее на душе. Но что-то в его лице удержало ее, и внезапно ей стало стыдно говорить о любви в такую минуту — ведь Мелани еще не успела остыть.
— Упокой господи ее душу, — с трудом произнес он. — Она была единственным по-настоящему добрым человеком, какого я знал.
— Ох, Ретт! — жалобно воскликнула она, ибо его слова оживили в ее памяти все то доброе, что сделала для нее Мелани. — Почему вы не пошли со мной? Это было так ужасно.., и вы были мне так нужны!
— Я бы не мог этого вынести, — просто сказал он и на мгновение умолк. Потом с усилием, очень мягко произнес: — Это была настоящая леди.
Взгляд его темных глаз был устремлен куда-то мимо нее, и она увидела в них то же выражение, как тогда, при свете пожаров, полыхавших в Атланте, когда он сказал ей, что пойдет вместе с отступающей армией, — удивление человека, который, прекрасно зная себя, вдруг обнаруживает в своей душе верность чему-то неожиданному и неожиданные чувства и немного стесняется собственного открытия.
Он смотрел своим тяжелым взглядом поверх ее плеча, точно видел Мелани, тихо шедшую через комнату к двери. Он как бы прощался с ней, но в лице его не было ни горя, ни страдания, лишь удивление на самого себя, лишь болезненное пробуждение чувств, не заявлявших о себе с юности, и он повторил:
— Это была настоящая леди.
По телу Скарлетт прошла дрожь, и затеплившийся в ней было свет радости исчез, — исчезло тепло и упоение, которые побуждали ее мчаться как на крыльях домой. Она начала понимать, что происходило в душе Ретта, когда он говорил «прости» единственному человеку на свете, которого он уважал. И Скарлетт снова почувствовала отчаяние от невозместимости утраты, которую понесла не только она. Она, конечно, еще до конца не осознала и не могла разобраться в том, что чувствовал Ретт, но на секунду ей показалось, будто и ее коснулись шуршащие юбки мягкой лаской последнего «прости». Она смотрела глазами Ретта на то, как уходила из жизни не женщина, а легенда, — кроткая, незаметная, но несгибаемая женщина, которой Юг завещал хранить свой очаг во время войны и в чьи гордые, но любящие объятия он вернулся после поражения.
Взгляд Ретта снова обратился к Скарлетт.
— Значит, она умерла. — Он произнес это уже другим, небрежным и холодным тоном. — Видите, как хорошо все для вас устраивается.
— Ах, как вы можете говорить такое! — воскликнула она, глубоко уязвленная, чувствуя прихлынувшие к глазам слезы. — Вы же знаете, как я ее любила!
— Нет, не могу сказать, чтобы знал. Это для меня полнейшая неожиданность, и то, что при вашей любви к целому сброду вы сумели наконец оценить Мелани, делает вам честь.
— Да как вы можете так говорить? Конечно, я всегда ее ценила! А вы — нет. Вы не знали ее так, как я! Вы просто не способны понять ее.., не способны понять, какая она была хорошая!
— Вот как? Возможно.
— Она ведь обо всех заботилась, обо всех, кроме себя.., кстати, последние слова ее были о вас.
Он повернулся к ней — в глазах его вспыхнул живой интерес.
— Что же она сказала?
— Ах, только не сейчас, Ретт.
— Скажите мне.
Он произнес это спокойно, но так сжал ей руку, что причинил боль. Ей не хотелось говорить — не так она собиралась сказать ему о своей любви, но его пожатие требовало ответа.
— Она сказала.., она сказала… «Будь подобрее к капитану Батлеру. Он так тебя любит».
Он смотрел на нее не мигая, потом выпустил ее руку и прикрыл глаза — лицо приняло мрачное, замкнутое выражение. Внезапно он встал, подошел к окну и, отдернув портьеры, долго смотрел на улицу, точно видел там что-то еще, кроме все заволакивавшего тумана.
— А еще что она сказала? — спросил он, не поворачивая головы.
— Она просила меня позаботиться о маленьком Бо, и я сказала, что буду заботиться о нем, как о родном сыне.
— А еще что?
— Она сказала.., про Эшли.., просила меня позаботиться и об Эшли.
Он с минуту помолчал, потом тихо рассмеялся.
— Это так удобно, когда есть разрешение первой жены, верно?
— Что вы хотите этим сказать?
Он повернулся, и хотя Скарлетт была в смятении и соображала не очень четко, тем не менее она удивилась, увидев, что он смотрит на нее без насмешки. И безучастно — как человек, который досматривает последний акт не слишком забавной комедии.
— Мне кажется, я выразился достаточно ясно. Мисс Мелли умерла. Вы же располагаете всеми необходимыми данными, чтобы развестись со мной, а репутация у вас такая, что развод едва ли может вам повредить. От вашей религиозности тоже ведь ничего не осталось, так что мнение церкви для вас не имеет значения. А тогда — Эшли и осуществление мечты с благословения мисс Мелли.
— Развод? — воскликнула она. — Нет! Нет! — Она хотела было что-то сказать, запуталась, вскочила на ноги и, подбежав к Ретту, вцепилась ему в локоть: — Ох, до чего же вы не правы! Ужасно не правы! Я не хочу развода.., я… — И она умолкла, не находя слов. Он взял ее за подбородок и, осторожно приподняв лицо к свету, внимательно посмотрел ей в глаза. А она смотрела на него, вкладывая в свой взгляд всю душу, и губы у нее задрожали, когда она попыталась что-то произнести. Но слова не шли с, языка — она все пыталась найти в его лице ответные чувства, вспыхнувшую надежду, радость. Ведь теперь-то он, уж конечно, все понял! Но она видела перед собой лишь смуглое замкнутое лицо, которое так часто озадачивало ее. Он отпустил ее подбородок, повернулся, подошел к столу и сел, вытянув ноги, устало свесив голову на грудь, — глаза его задумчиво, безразлично смотрели на нее из-под черных бровей.
Она подошла к нему и, ломая пальцы, остановилась.
— Вы не правы, — начала было она, обретя наконец дар речи. — Ретт, сегодня, когда я поняла, я всю дорогу бежала — до самого дома, чтобы вам сказать. Любимый мой, я…
— Вы устали, — сказал он, продолжая наблюдать за ней, — Вам следовало бы лечь.
— Но я должна сказать вам!
— Скарлетт, — медленно произнес он, — я не желаю ничего слушать — ничего.
— Но вы же не знаете, что я собираюсь сказать!
— Кошечка моя, это все написано на вашем лице. Что-то или кто-то наконец заставил вас понять, что злополучный мистер Улкис — дохлая устрица, которую даже вам не разжевать. И это что-то неожиданно высветило для вас мои чары, которые предстали перед вами в новом, соблазнительном свете. — Он слегка пожал плечами. — Не стоит об этом говорить.
Она чуть не задохнулась от удивления. Конечно, он всегда читал в ней, как в раскрытой книге. До сих пор ее возмущала эта его способность, но сейчас, когда первое удивление оттого, что глубины ее души так легко просматриваются, прошло, она почувствовала огромную радость и облегчение. Он знает, он понимает — теперь ее задача казалась такой легкой. Ему ни о чем не надо говорить! Конечно, ему горько оттого, что она так долго им пренебрегала, конечно, он еще не верит этой внезапной перемене. Ей предстоит завоевать его, проявив доброту, убедить его, осыпав знаками любви, и как приятно будет ей все это!
— Любимый, я вам все расскажу, — сказала она, упираясь руками в подлокотники его кресла и пригибаясь к нему. — Я была так не права, я была такая идиотка…
— Скарлетт, прекратите. Не унижайтесь передо мной. Мне это невыносимо. Проявите немного достоинства, немного сдержанности, чтобы хоть это осталось, когда мы будем вспоминать о нашем браке. Избавьте нас от этого последнего объяснения.
Она резко выпрямилась. «Избавьте нас от этого последнего объяснения»? Что он имел в виду, говоря: «этого последнего»? Последнего? Да ведь это же их первое объяснение, это только начало.
— Но я все равно вам скажу, — быстро заговорила она, словно боясь, что он зажмет ей рот рукой. — Ах, Ретт, я так люблю вас, мой дорогой! И должно быть, я люблю вас уже многие годы и, такая идиотка, не понимала этого. Ретт, вы должны мне поверить!
Он с минуту смотрел на нее — это был долгий взгляд, проникший в самую глубину ее сознания. Она видела по его глазам, что он верит. Но его это мало интересовало. Ах, неужели он станет сводить с ней счеты — именно сейчас? Будет мучить ее, платя ей той же монетой?!
— О, я вам верю, — сказал он наконец. — А как же будет с Эшли Уилксом?
— Эшли! — произнесла она и нетерпеливо повела плечом. — Я.., по-моему, он уже многие годы мне безразличен. Просто.., ну, словом, это было что-то вроде привычки, за которую я цеплялась, потому что приобрела ее еще девочкой. Ретт, никогда в жизни я бы не считала, что он мне дорог, если бы понимала, что он представляет собой. А ведь он такое беспомощное, жалкое существо, несмотря на всю свою болтовню о правде, и о чести, и о…
— Нет, — сказал Ретт. — Если вы хотите видеть его таким, каков он есть, то он не такой. Он всего лишь благородный джентльмен, оказавшийся в мире, где он — чужой, и пытающийся худо-бедно жить в нем по законам мира, который отошел в прошлое.
— Ах, Ретт, не будем о нем говорить! Ну, какое нам сейчас до него дело? Неужели вы не рады, узнав.., я хочу сказать: теперь, когда я…
Встретившись с ним взглядом, — а глаза у него были такие усталые, — она смущенно умолкла, застеснявшись, словно девица, впервые оставшаяся наедине со своим ухажером. Хоть бы он немного ей помог! Хоть бы протянул ей руки, и она с благодарностью кинулась бы ему в объятия, свернулась бы калачиком у него на коленях и положила голову ему на грудь! Ее губы, прильнувшие к его губам, могли бы сказать ему все куда лучше, чем эти спотыкающиеся слова. Но глядя на него, она поняла, что он удерживает ее на расстоянии не из мстительности. Вид у него был опустошенный, такой, будто все, о чем она ему поведала, не имело никакого значения.
— Рад? — сказал он. — Было время, когда я отблагодарил бы бога постом, если бы услышал от вас то, что вы сейчас сказали. А сейчас это не имеет значения.
— Не имеет значения? О чем вы говорите? Конечно, имеет! Ретт, я же дорога вам, верно? Должна быть дорога. И Мелли так сказала.
— Да, и она была права, ибо это то, что она знала. Но, Скарлетт, вам никогда не приходило в голову, что даже самая бессмертная любовь может износиться?
Она смотрела на него, потеряв дар речи. Рот ее округлился буквой «о».
— Вот моя и износилась, — продолжал он, — износилась в борьбе с Эшли Уилксом и вашим безумным упрямством, которое побуждает вас вцепиться как бульдог в то, что, по вашим представлениям, вам желанно… Вот она и износилась.
— Но любовь не может износиться!
— Ваша любовь к Эшли тоже ведь износилась.
— На я никогда по-настоящему не любила Эшли! — В таком случае вы отлично имитировали любовь — до сегодняшнего вечера. Скарлетт, я не корю вас, не обвиняю, не упрекаю. Это время прошло. Так что избавьте меня от ваших оправдательных речей и объяснений. Если вы в состоянии послушать меня несколько минут не прерывая, я готов объяснить вам, что я имею в виду, хотя — бог свидетель — не вижу необходимости в объяснениях. Правда слишком уж очевидна.
Скарлетт села; резкий газовый свет падал на ее белое растерянное лицо. Она смотрела в глаза Ретта, которые знала так хорошо — и одновременно так плохо, — слушала его тихий голос, произносивший слова, которые сначала казались ей совсем непонятными. Впервые он говорил с ней так, по-человечески, как говорят обычно люди — без дерзостей, без насмешек, без загадок.
— Приходило ли вам когда-нибудь в голову, что я любил вас так, как только может мужчина любить женщину? Любил многие годы, прежде чем добился вас? Во время войны я уезжал, пытаясь забыть вас, но не Мог и снова возвращался. После войны, зная, что рискую попасть под арест, я все же вернулся, чтобы отыскать вас. Вы мне были так дороги, что мне казалось, я готов был убить Франка Кеннеди, и убил бы, если бы он не умер. Я любил вас, но не мог дать вам это понять. Вы так жестоки к тем, кто любит вас, Скарлетт. Вы принимаете любовь и держите ее как хлыст над головой человека.
Из всего сказанного им Скарлетт поняла лишь, что он ее любит. Уловив слабый отголосок страсти в его голове, она почувствовала, как радость и волнение потихоньку наполняют ее. Она сидела еле дыша, слушала и ждала.
— Я знал, что вы меня не любите, когда мы поженились. Понимаете, я знал про Эшли, но.., был настолько глуп, что считал, будто мне удастся стать для вас дороже. Смейтесь сколько хотите, но мне хотелось заботиться о вас, делать все, что бы вы ни пожелали. Я хотел жениться на вас, быть вам защитой, дать вам возможность делать все что пожелаете, лишь бы вы были счастливы, — так ведь было и с Бонни. Вам пришлось столько вытерпеть, Скарлетт. Никто лучше меня не понимал, через что вы прошли, и мне хотелось сделать так, чтобы вы перестали бороться, а чтобы я боролся вместо вас. Мне хотелось, чтобы вы играли как дитя. Потому что вы и есть дитя — храброе, испуганное, упрямое дитя. По-моему, вы так и остались ребенком. Ведь только ребенок может быть таким упорным и таким бесчувственным.
Голос его звучал спокойно, но устало, и было в нем что-то, что пробудило далекий отзвук в памяти Скарлетт. Она уже слышала такой голос — в другом месте, в один из решающих моментов своей жизни. Где же это было? Голос человека, смотрящего на себя и на свой мир без всяких чувств, без страха, без надежды.
Это же.., это же.., так говорил Эшли во фруктовом саду Тары, где гулял ветер, — говорил о жизни и театре теней с усталым спокойствием, свидетельствовавшим о неотвратимости конца в гораздо большей мере, чем если бы в словах его звучали горечь и отчаяние. И как тогда от интонаций в голосе Эшли она вся похолодела в ужасе перед тем, чего не могла понять, так и сейчас от тона Ретта сердце ее упало. Его голос, его манера держаться даже больше, чем слова, взволновали ее, заставив почувствовать, что радостное волнение, испытанное ею несколько минут тому назад, было преждевременным. Что-то не так, совсем не так. Она не знала, что именно, и в отчаянии ловила каждое слово Ретта, не спуская глаз с его смуглого лица, надеясь услышать слова, которые рассеют ее страхи.
— Все говорило о том, что мы предназначены друг для друга. Все, потому что я — единственный из ваших знакомых — способен был любить вас, даже узнав, что вы такое на самом деле: жестокая, алчная, беспринципная, как и я. Но я любил вас и решил рискнуть. Я надеялся, что Эшли исчезнет из ваших мыслей. Однако, — он пожал плечами, — я все перепробовал, и ничто не помогло. А я ведь так любил вас, Скарлетт. Если бы вы только мне позволили, я бы любил вас так нежно, так бережно, как ни один мужчина никогда еще не любил. Но я не мог дать вам это почувствовать, ибо я знал, что вы сочтете меня слабым и тотчас попытаетесь использовать мою любовь против меня же. И всегда, всегда рядом был Эшли. Это доводило меня до безумия. Я не мог сидеть каждый вечер напротив вас за столом, зная, что вы хотели бы, чтобы на моем месте сидел Эшли. Я не мог держать вас в объятиях ночью, зная, что.., ну, в общем, это не имеет сейчас значения. Сейчас я даже удивляюсь, почему мне было так больно. Это-то и привело меня к Красотке. Есть какое-то свинское удовлетворение в том, чтобы быть с женщиной — пусть даже она безграмотная шлюха, — которая безгранично любит тебя и уважает, потому что ты в ее глазах — безупречный джентльмен. Это было как бальзам для моего тщеславия. А вы ведь никогда не пытались быть для меня бальзамом, дорогая.
— Ах, Ретт… — начала было она, чувствуя себя глубоко несчастной от одного упоминания имени Красотки, но он жестом заставил ее умолкнуть и продолжал:
— А потом была та ночь, когда я унес вас наверх… Я думал.., я надеялся.., я так надеялся, что боялся встретиться с вами наутро и увидеть, что я ошибся и что вы не любите меня. Я так боялся, что вы будете надо мной смеяться, что ушел из дома и напился. А когда вернулся, то весь трясся от страха, и если бы вы сделали хотя бы шаг мне навстречу, мне кажется, я стал бы Целовать след ваших ног. Но вы этого не сделали.
— Ах, Ретт, но мне же тогда так хотелось быть с вами, а вы были таким омерзительным! В самом деле хотелось! По-моему.., да, именно тогда я впервые поняла, что вы мне дороги. Эшли… После этого дня Эшли меня больше не радовал. А вы были такой омерзительный, что я…
— Ах, да ладно, — сказал он. — Похоже, мы оба тогда не поняли друг друга, верно? Но сейчас это уже не имеет значения. Я все это говорю лишь затем, чтобы вы потом не ломали себе голову. Когда с вами было плохо, причем по моей вине, я стоял у вашей двери в надежде, что вы позовете меня, но вы не позвали, и тогда я понял, что я просто дурак и между нами все кончено.
Он умолк, глядя, сквозь нее на что-то за ней, как это часто делал Эшли, видя что-то, чего не могла видеть она. А Скарлетт лишь молча смотрела на его сумрачное лицо.
— Но у нас была Бонни, и я почувствовал, что не все еще кончено. Мне нравилось думать, что Бонни — это вы, снова ставшая маленькой девочкой, какой вы были до того, как война и бедность изменили вас. Она была так похожа на вас — такая же своенравная, храбрая, веселая, задорная, и я мог холить и баловать ее — как мне хотелось холить и баловать вас. Только она была не такая, как вы, — она меня любила. И я был счастлив отдать ей всю свою любовь, которая вам была не нужна… Когда ее не стало, с ней вместе ушло все.
Внезапно ей стало жалко его, так жалко, что она почти забыла и о собственном горе, и о страхе, рожденном его словами. Впервые в жизни ей было жалко кого-то, к кому она не чувствовала презрения, потому что впервые в жизни она приблизилась к подлинному пониманию другого человека. А она могла понять его упорное стремление оградить себя — столь похожее на ее собственное, его несгибаемую гордость, не позволявшую признаться в любви из боязни быть отвергнутым.
— Ах, любимый, — сказала она и шагнула к нему в надежде, что он сейчас раскроет ей объятия и привлечет к себе на колени. — Любимый мой, мне очень жаль, что так получилось, но я постараюсь все возместить вам сторицей. Теперь мы можем быть так счастливы, когда знаем правду. И… Ретт.., посмотрите на меня, Ретт! Ведь.., ведь у нас же могут быть другие дети.., ну, может, не такие, как Бонни, но…
— Нет уж, благодарю вас, — сказал Ретт, словно отказываясь от протянутого ему куска хлеба. — Я в третий раз своим сердцем рисковать не хочу.
— Ретт, не надо так говорить. Ну, что мне вам сказать, чтобы вы поняли. Я ведь уже говорила, что мне очень жаль и что я…
— Дорогая моя, вы такое дитя… Вам кажется, что если вы сказали: «мне очень жаль», все ошибки и вся боль прошедших лет могут быть перечеркнуты, стерты из памяти, что из старых ран уйдет весь яд… Возьмите мой платок, Скарлетт. Сколько я вас знаю, никогда в тяжелые минуты жизни у вас не бывает носового платка.
Она взяла у него платок, высморкалась и села. Было совершенно ясно, что он не раскроет ей объятий. И становилось ясно, что все эти его разговоры о любви ни к чему не ведут. Это был рассказ о временах давно прошедших, и смотрел он на все это как бы со стороны. Вот что было страшно. Он поглядел на нее задумчиво, чуть ли не добрыми глазами.
— Сколько лет вам, дорогая моя? Вы никогда мне этого не говорили.
— Двадцать восемь, — глухо ответила она в платок.
— Это еще не так много. Можно даже сказать, совсем юный возраст для человека, который завоевал мир и потерял собственную душу, верно? Не смотрите на меня так испуганно. Я не имею в виду, что вы будете гореть в адском пламени за этот ваш роман с Эшли. Образно говоря. С тех пор как я вас знаю, вы ставили перед собой две цели: Эшли и деньги, чтобы иметь возможность послать к черту всех и вся. Ну что ж, вы теперь достаточно богаты и можете со всеми разговаривать достаточно резко, и вы получили Эшли, если он вам нужен. Но вам и этого, видно, мало.
А Скарлетт было страшно, но не при мысли об адском пламени. Она думала: «Ведь Ретт — все для меня, а я его теряю. И если я потеряю его, ничто уже не будет иметь для меня значения! Нет, ни друзья, ни деньги.., ничто. Если бы он остался со мной, я бы даже готова была снова стать бедной. Я готова была бы снова мерзнуть и голодать. Не может же он… Ах, конечно, не может!» Она вытерла глаза и сказала в отчаянии:
— Ретт, если вы когда-то меня так любили, должно же что-то остаться от этого чувства!
— От всего этого осталось только два чувства, но вам они особенно ненавистны — это жалость и какая-то странная доброта.
«Жалость! Доброта! О боже!» — теряя последнюю надежду, подумала она. Все что угодно, только не жалость и не доброта. Когда она испытывала к кому-нибудь подобные чувства, это всегда сопровождалось презрением. Неужели он ее тоже презирает? Все что угодно, только не это! Даже циничная холодность дней войны, даже пьяное безумие, когда он в ту ночь нес ее наверх, так сжимая в объятиях, что ей было больно, даже эта его манера нарочно растягивать слова, говоря колкости, которыми, как она сейчас поняла, он прикрывал горькую свою любовь, — что угодно, лишь бы не эта безликая доброта, которая так отчетливо читалась на его лице.
— Значит.., значит, я все уничтожила.., и вы не любите меня больше?
— Совершенно верно.
— Но, — упрямо продолжала она, словно ребенок, считающий, что достаточно высказать желание, чтобы оно осуществилось, — но я же люблю вас!
— Это ваша беда.
Она быстро вскинула на него глаза, проверяя, нет ли в этих словах издевки, но издевки не было. Он просто констатировал факт. Но она все равно этому не верила — не могла поверить. Она смотрела на него, чуть прищурясь, в глазах ее горело упорство отчаяния, подбородок, совсем как у Джералда, вдруг резко выдвинулся, ломая мягкую линию щеки.
— Не глупите, Ретт! Ведь я же могу…
Он с наигранным ужасом поднял руку, и его черные брови поползли вверх, придавая лицу знакомое насмешливое выражение.
— Не принимайте такого решительного вида, Скарлетт! Вы меня пугаете. Я вижу, вы намерены перенести на меня ваши бурные чувства к Эшли. Я страшусь за свою свободу и душевный покой. Нет, Скарлетт, я не позволю вам преследовать меня, как вы преследовали злосчастного Эшли. А кроме того, я уезжаю.
Губы ее задрожали, прежде чем она успела сжать зубы и остановить дрожь. Уезжает? Нет, что угодно, только не это! Да как она сможет жить без него? Ведь все ее покинули. Все, кто что-то значил в ее жизни, кроме Ретта. Он не может уехать. Но как ей остановить его? Она бессильна, когда он что-то вот так холодно решил и говорит так бесстрастно.
— Я уезжаю. Я собирался сказать вам об этом после вашего возвращения из Мариетты.
— Вы бросаете меня?
— Не делайте из себя трагическую фигуру брошенной жены, Скарлетт. Эта роль вам не к лицу. Насколько я понимаю, вы не хотите разводиться и даже жить отдельно? Ну, в таком случае я буду часто приезжать, чтобы не давать повода для сплетен.
— К черту сплетни! — пылко воскликнула она. — Вы мне нужны. Возьмите меня с собой!
— Нет, — сказал он тоном, не терпящим возражений. Ей казалось, что она сейчас разрыдается, безудержно, как ребенок. Она готова была броситься на пол, сыпать проклятьями, кричать, бить ногами. Но какие-то остатки гордости и здравого смысла удержали ее. Она подумала: «Если я так поведу себя, он только посмеется или будет стоять и смотреть на меня. Я не должна рыдать, я не должна просить. Я не должна делать ничего такого, что может вызвать его презрение. Он должен меня уважать, даже.., даже если больше не любит меня».
Она подняла голову и постаралась спокойно спросить:
— Куда же вы едете?
В глазах его промелькнуло восхищение, и он ответил:
— Возможно, в Англию.., или в Париж. А возможно, в Чарльстон, чтобы наконец помириться с родными.
— Но вы же ненавидите их. Я часто слышала, как вы смеялись над ними и…
Он пожал плечами.
— Я по-прежнему смеюсь. Но хватит мне бродить по миру, Скарлетт. Мне сорок пять лет, и в этом возрасте человек начинает ценить то, что он так легко отбрасывал в юности: свой клан, свою семью, свою честь и безопасность, корни, уходящие глубоко… Ах, нет! Я вовсе не каюсь и не жалею о том, что делал. Я чертовски хорошо проводил время — так хорошо, что это начало приедаться. И сейчас мне захотелось чего-то другого. Нет, я не намерен ничего в себе менять, кроме своих пятен. Но мне хочется хотя бы внешне стать похожим на людей, которых я знал, обрести эту унылую респектабельность — респектабельность, какой обладают другие люди, моя кошечка, а не я, — спокойное достоинство, исконное изящество былых времен. В те времена я просто жил, не понимая их медленного очарования…
И снова Скарлетт очутилась в пронизанном ветром фруктовом саду Тары — в глазах Ретта было то же выражение, как в тот день в глазах Эшли. Слова Эшли отчетливо звучали в ее ушах, словно только что говорил он, а не Ретт. Что-то из этих слов всплыло в памяти, и она как попугай повторила их:
— и.., прелести — этого их совершенства, этой гармонии, как в греческом искусстве.
— Почему вы так сказали? — резко прервал ее Ретт. — Ведь именно эти слова и я хотел произнести.
— Так.., так однажды выразился Эшли, говоря о былом. Ретт передернул плечами, взгляд его потух.
— Вечно Эшли, — сказал он и умолк. — Скарлетт, когда вам будет сорок пять, возможно, вы поймете, о чем я говорю, и возможно, вам, как и мне, надоедят эти лжеаристократы, их дешевое жеманство и мелкие страстишки. Но я сомневаюсь. Я думаю, что вас всегда будет больше привлекать дешевый блеск, чем настоящее золото. Во всяком случае, ждать, пока это случится, я не могу. И У меня нет желания. Меня это просто не интересует. Я поеду в старые города и в древние края, где все еще сохранились черты былого. Вот такой я стал сентиментальный. Атланта для меня — слишком неотесанна, слишком молода — Прекратите, — внезапно сказала Скарлетт. Она едва ли слышала, о чем он говорил. Во всяком случае, в сознании у нее это не отложилось. Она знала лишь, что не в состоянии выносить дольше звук его голоса, в котором не было любви.
Он умолк и вопросительно взглянул на нее.
— Ну, вы поняли, что я хотел сказать, да? — спросил он, поднимаясь.
Она протянула к нему руки ладонями кверху в жесте мольбы, и все, что было у нее на сердце, отразилось на ее лице.
— Нет! — выкрикнула она. — Я знаю только, что вы меня разлюбили и что вы уезжаете! Ах, мой дорогой, если вы уедете, что же я буду делать?
Он помедлил, словно решая про себя, не будет ли в конечном счете великодушнее по-доброму солгать, чем сказать правду. Затем пожал плечами.
— Скарлетт, я никогда не принадлежал к числу тех, кто терпеливо собирает обломки, склеивает их, а потом говорит себе, что починенная вещь ничуть не хуже новой. Что разбито, то разбито. И уж лучше я буду вспоминать о том, как это выглядело, когда было целым, чем склею, а потом до конца жизни буду лицезреть трещины. Возможно, если бы я был моложе… — Он вздохнул. — Но мне не так мало лет, чтобы верить сентиментальному суждению, будто жизнь — как аспидная доска: с нее можно все стереть и начать сначала. Мне не так мало лет, чтобы я мог взвалить на себя бремя вечного обмана, который сопровождает жизнь без иллюзий. Я не мог бы жить с вами и лгать вам — и уж конечно, не мог бы лгать самому себе. Я даже вам теперь не могу лгать. Мне хотелось бы волноваться по поводу того, что вы делаете и куда едете, но я не могу. — Он перевел дух и сказал небрежно, но мягко: — Дорогая моя, мне теперь на это наплевать.
Скарлетт молча смотрела ему вслед, пока он поднимался по лестнице, и ей казалось, что она сейчас задохнется от боли, сжавшей грудь. Вот сейчас звук его шагов замрет наверху, и вместе с ним умрет все, что в ее жизни имело смысл. Теперь она знала, что нечего взывать к его чувствам или к разуму — ничто уже не способно заставить этот холодный мозг отказаться от вынесенного им приговора. Теперь она знала, что он действительно так думает — вплоть до последних сказанных им слов. Она знала это, потому что чувствовала в нем силу — несгибаемую и неумолимую, то, что искала в Эшли и так и не нашла.
Она не сумела понять ни одного из двух мужчин, которых любила, и вот теперь потеряла обоих. В сознании ее где-то таилась мысль, что если бы она поняла Эшли, она бы никогда его не любила, а вот если бы она поняла Ретта, то никогда не потеряла бы его. И она с тоской подумала, что, видимо, никогда никого в жизни по-настоящему не понимала.
Благодарение богу, на нее нашло отупение — отупение, которое, как она знала по опыту, скоро уступит место острой боли — так разрезанные ткани под ножом хирурга на мгновение утрачивают чувствительность, а потом начинается боль.
«Сейчас я не стану об этом думать, — мрачно решила она, призывая на помощь старое заклятье. — Я с ума сойду, если буду сейчас думать о том, что и его потеряла. Подумаю завтра».
«Но, — закричало сердце, отметая прочь испытанное заклятье и тут же заныв, — я не могу дать ему уйти! Должен же быть какой-то выход!» — Сейчас я не стану об этом думать, — повторила она уже вслух, стремясь отодвинуть свою беду подальше в глубь сознания, стремясь найти какую-то опору, ухватиться за что-то, чтобы не захлестнула нарастающая боль. — Я.., да, завтра же уеду домой в Тару. — И ей стало чуточку легче.
Однажды она уже возвращалась в Тару, гонимая страхом, познав поражение, и вышла из приютивших ее стен сильной, вооруженной для победы. Однажды так было — господи, хоть бы так случилось еще раз! Как этого добиться — она не знала. И не хотела думать об этом сейчас. Ей хотелось лишь передышки, чтобы утихла боль, — хотелось покоя, чтобы зализать свои раны, прибежища, где она могла бы продумать свою кампанию Она думала о Таре, и словно прохладная рука ложилась ей на сердце, успокаивая его учащенное биение. Она видела белый дом, приветливо просвечивающий сквозь красноватую осеннюю листву, ощущала тишину сельских сумерек, нисходившую на нее благодатью, самой кожей чувствовала, как увлажняет роса протянувшиеся на многие акры кусты хлопка с их коробочками, мерцающими среди зелени, как звезды, видела эту пронзительно красную землю и мрачную темную красоту сосен на холмах.
От этой картины ей стало немного легче, она даже почувствовала прилив сил, и боль и неистовое сожаление отодвинулись в глубь сознания. Она с минуту стояла, припоминая отдельные детали — аллею темных кедров, ведущую к Таре, раскидистые кусты жасмина, ярко-зеленые на фоне белых стен, колеблющиеся от ветра белые занавески. И Мамушка тоже там. И вдруг ей отчаянно захотелось увидеть Мамушку — так захотелось, словно она была еще девочкой, — захотелось положить голову на широкую грудь и чтобы узловатые пальцы гладили ее по голове. Мамушка — последнее звено, связывавшее ее с прошлым.
И сильная духом своего народа, не приемлющего поражения, даже когда оно очевидно, Скарлетт подняла голову. Она вернет Ретта. Она знает, что вернет. Нет такого человека, которого она не могла бы завоевать, если бы хотела.
«Я подумаю обо всем этом завтра, в Таре. Завтра я найду способ вернуть Ретта. Ведь завтра уже будет другой день».

