\chapter{\ }

Скарлетт заметила, что после ее болезни Ретт изменился, но не была уверена, нравится ли ей то, что с ним произошло. Он стал воздержан в выпивке, спокоен и всегда чем-то озабочен. Он чаще ужинал теперь дома, добрее относился к слугам, больше уделял внимания Уэйду и Элле. Он никогда не вспоминал прошлое, даже если это было что-то приятное, и, казалось, молча запрещал и. Скарлетт касаться этого в разговоре. Скарлетт вела себя мирно — от добра добра не ищут, — и жизнь текла довольно гладко — с внешней стороны. Ретт держался безразлично-вежливого тона, который принял в отношении нее, когда она начала выздоравливать, не говорил ей колкостей, мягко растягивая слова, и не ранил своим сарказмом.
Она поняла теперь, что, распаляя ее раньше своими ехидными замечаниями и вызывая на горячие споры, он поступал так потому, что его трогало все, что она делает и говорит. Сейчас же она не знала, трогает ли его хоть что-то. Он был вежлив и безразличен, и ей не хватало его заинтересованности — пусть даже ехидной, — не хватало былых пререканий и перепалок.
Он держался с ней мило, но так, точно она ему совсем чужая, и глаза его, раньше неотрывно следившие за ней, вот так же следили теперь за Бонни. Такое было впечатление, точно бурный поток его жизни направили в узкий канал. Порою Скарлетт казалось, что если бы Ретт уделял ей половину того внимания и нежности, которыми он окружал Бонни, жизнь стала бы иной. Порой ей было даже трудно улыбнуться, когда люди говорили: «До чего же капитан Батлер обожествляет свою девочку!» Но если она не улыбнется, людям это может показаться странным, а Скарлетт не хотелось признаваться даже себе в том, что она ревнует к маленькой девочке, тем более что эта девочка — ее любимое дитя. Скарлетт всегда хотелось занимать главное место в сердцах окружающих, а сейчас ей стало ясно, что Ретт и Бонни всегда будут занимать главное место в сердце друг друга.
Ретт часто возвращался домой поздно ночью, но всегда был трезв. Скарлетт нередко слышала; как он тихонько насвистывал, проходя по площадке мимо ее закрытой двери. Иногда вместе с ним в дом приходили какие-то люди и сидели в столовой, разговаривая за коньяком. Это были не те люди, с которыми он пил в первый год после их свадьбы. Богатые «саквояжники», подлипалы, республиканцы не появлялись больше в доме по его приглашению. Скарлетт, подкравшись на цыпочках к перилам лестничной площадки, прислушивалась и, к своему изумлению, часто слышала голоса Рене Пикара, Хью Элсинга, братьев Симмонсов и Энди Боннелла. И всегда там были дедушка Мерриуэзер и дядя Генри. Однажды, к своему удивлению, она услышала голос доктора Мида. А ведь все эти люди когда-то считали, что Ретта мало повесить!
Эта группа была неразрывно связана в ее сознании со смертью Фрэнка, а поздние возвращения Ретта еще больше приводили ей на память времена, предшествовавшие налету ку-клукс-клана, когда Фрэнк расстался с жизнью. Она с ужасом вспоминала слова Ретта о том, что он даже готов вступить в этот их чертов клан, лишь бы стать уважаемым гражданином, хотя он надеется, что господь бог не наложит на него столь тяжкого испытания. А что, если Ретт, как Фрэнк…
Однажды ночью, когда он отсутствовал дольше обычного, Скарлетт почувствовала, что больше не в состоянии выносить напряжение. Услышав звук ключа, поворачиваемого в замке, она накинула капот и, выйдя на освещенную газом площадку лестницы, встретила его у верхней ступеньки. При виде ее задумчивое и отрешенное выражение на его лице тотчас сменилось удивлением.
— Ретт, я должна знать! Я должна знать: вы, случайно.., не состоите ли в клане.., и потому задерживаетесь так долго! Вы в самом деле вступили…
При свете ярко горевшего газа он равнодушно посмотрел на нее, потом улыбнулся.
— Вы безнадежно отстали, — сказал он. — В Атланте нет ку-клукс-клана. В Джорджии вообще его, скорей всего, нет. Вы просто наслушались всяких россказней про клан, которые распространяют ваши друзья — подлипалы и «саквояжники».
— Нет ку-клукс-клана? Вы говорите неправду, чтобы успокоить меня?
— Дорогая моя, когда же я пытался вас успокаивать? Нет, ку-клукс-клана не существует. Мы решили, что от него больше вреда, чем пользы, потому что он только раздувал ненависть янки и лил воду на мельницу его превосходительства губернатора Баллока, давая повод для расправ. К тому же Баллок знает, что сумеет продержаться у власти, лишь пока федеральное правительство и газеты янки будут убеждены в том, что Джорджия полна бунтовщиков и за каждым кустом сидит куклуксклановец. И вот, чтобы не расставаться с креслом, он отчаянно изобретал всякие возмутительные истории про ку-клукс-клан, рассказывая направо и налево о том, как верных республиканцев подвешивают за большие пальцы, а честных негров линчуют за изнасилование. Он стреляет по несуществующим мишеням, и он это знает. Благодарю вас за то, что вы за меня волнуетесь, но ку-клукс-клан перестал существовать вскоре после того, как я отошел от подлипал и превратился в скромного демократа.
Почти все, что он говорил насчет губернатора Баллока, вошло у Скарлетт в одно ухо и вышло в другое, ибо мозг ее заполняла радостная мысль, что клана больше нет. Ретта не убьют, как убили Фрэнка; она не потеряет своей лавки и своих денег. Но одно слово, сказанное Реттом, всплыло в ее сознании. Он сказал:
«мы», как нечто само собою разумеющееся, связывая себя с теми, кого называл прежде «старой гвардией».
— Ретт, — неожиданно спросила она, — а вы имели какое-то отношение к роспуску клана?
Он посмотрел на нее долгим взглядом, и в глазах его затанцевали искорки.
— Да, любовь моя. Эшли Уилкс и я главным образом за это в ответе.
— Эшли.., и вы?
— Да, хотя это и звучит пошло, но политика укладывает очень разных людей в одну постель. А ни я, ни Эшли не жаждем очутиться в одной постели, однако… Эшли никогда не верил в ку-клукс-клан, потому что он вообще против насилия. А я никогда не верил, потому что все это просто идиотизм и таким путем ничего не добьешься. Это единственный способ заставить янки сидеть у нас на шее до второго пришествия. И вот мы с Эшли убедили горячие головы, что сумеем добиться куда большего, наблюдая, выжидая и кое-что делая, чем если наденем ночные рубашки и будем размахивать огненными крестами.
— Не хотите же вы сказать, что мужчины приняли ваш совет при том, что вы…
— Что я спекулянт? Подлипала? Человек, действовавший заодно с янки? Вы забываете, миссис Батлер, что я теперь добропорядочный демократ, преданный до последней капли крови благородной цели вытащить наш любимый штат из рук насильников! Я ал им хороший совет, и они его послушались. Не менее хорошие советы даю я и по другим вопросам, связанным с политикой. У нас в законодательном собрании теперь — демократическое большинство, верно? И очень скоро, любовь моя, мы засадим кое-кого из наших добрых друзей-республиканцев за решетку. Слишком они стали алчными, слишком в открытую повели игру.
— И вы поможете засадить их в тюрьму? Но ведь это же были ваши друзья! Они взяли вас в свою компанию, когда выпускали железнодорожные акции, на которых вы заработали тысячи.
Ретт вдруг усмехнулся, как бывало, — с легкой издевкой.
— О, я им за это не помню зла. Но я перешел на другую сторону, и если в моих силах помочь засадить их туда, где им место, я это сделаю. И вы еще увидите, какую пользу мне это принесет! Я достаточно осведомлен о подоплеке некоторых дел, и мои знания могут очень пригодиться, когда законодательное собрание начнет копаться в этих делах, а, судя по всему, это не за горами. Займется собрание и губернатором и, если сумеет, засадит и его в тюрьму. Предупредите-ка лучше ваших добрых друзей Гелертов и Хандонов, чтоб они готовились покинуть город, потому что если сцапают губернатора, им того же не миновать.
Но Скарлетт слишком долго наблюдала, как республиканцы, опираясь на армию янки, правили Джорджией, чтобы поверить небрежно брошенным словам Ретта. Слишком прочно сидел в своем кресле губернатор, чтобы какое-либо законодательное собрание могло сладить с ним, а тем более посадить его в тюрьму.
— Какую вы несете чушь, — заметила она.
— Если его и не посадят в тюрьму, то уж во всяком случае не переизберут. В следующий раз губернатором у нас будет демократ — для разнообразия.
— И я полагаю, это тоже будет делом ваших рук? — иронически спросила она.
— Конечно, моя кошечка. Именно этим я сейчас и занимаюсь. Потому так поздно и домой прихожу. Я работаю больше, чем работал лопатой во времена золотой лихорадки: пытаюсь помочь провести выборы. И я знаю, это причинит вам боль, миссис Батлер, — но я дал немало денег на их организацию. Помните, много лет тому назад вы сказали мне в лавке Фрэнка, что я поступаю бесчестно, держа у себя золото Конфедерации? Вот теперь, наконец, я согласился с вами, и золото Конфедерации идет на то, чтобы вернуть конфедератам власть.
— Да вы швыряете деньги в крысиную нору!
— Что? Это вы демократическую партию называете крысиной норой? — Он с насмешкой посмотрел на нее, и глаза его снова стали спокойными, ничего не выражающими. — Мне безразлично, кто победит на выборах. Но мне небезразлично то, что все теперь знают, сколько я положил на это сил и сколько потратил денег. Об этом будут помнить не один год, и это принесет свою пользу Бонни.
— Услышав ваши благочестивые речи, я даже испугалась, решив, что вы изменили свои взгляды, но вижу сейчас, что вы так же неискренни по отношению к демократам, как и ко всем остальным.
— Взглядов своих я не менял. Переменил только шкуру. Наверное, можно закрасить пятна у леопарда, но, сколько их ни крась, он все равно леопардом останется.
Бонни, проснувшаяся от звука голосов на лестнице, сонным голосом, но повелительно позвала: «Папочка!» И Ретт шагнул было к ее двери мимо Скарлетт.
— Ретт, обождите. Я еще хочу сказать вам кое-что. Перестаньте брать с собой Бонни на политические митинги. Это производит нехорошее впечатление — что за причуда брать маленькую девочку в такие места! Да и вы сами выглядите глупо. Мне в голову не приходило, что вы таскаете ее туда, пока дядя Генри не упомянул об этом, видимо считая, что я знаю, и…
Ретт круто повернулся к ней, лицо его стало жестким.
— А что тут плохого в том, что девочка сидит у отца на коленях, пока он разговаривает с друзьями? Вам, возможно, кажется, что это выглядит глупо, но это вовсе не так. Люди надолго запомнят, что Бонни сидела у меня на коленях, когда я помогал выкуривать республиканцев из этого штата. Люди надолго запомнят.. — Лицо его утратило жесткость, в глазах загорелся лукавый огонек. — А вы знаете, когда Бонни спрашивают, кого она больше всех любит, она говорит: «папочку и димикатов». А когда спрашивают, кого она больше всех не любит, она говорит: «подъипал». Люди, слава богу, помнят такое.
— И вы, очевидно, говорите ей, что я — подлипала! — не в силах сдержать ярость, громко воскликнула Скарлетт.
— Папочка! — прозвенел детский голосок, на этот раз с возмущением, и Ретт, смеясь, вошел в комнату дочери.




В октябре того года губернатор Баллок покинул свой пост и бежал из Джорджии. Разграбление общественных фондов, непомерные траты и коррупция достигли при его правлении таких размеров, что здание рухнуло под собственной тяжестью. Даже в партии Баллока произошел раскол — столь велико было негодование публики. Демократы имели теперь большинство мест в законодательном собрании, и это означало лишь одно. Баллок, узнав, что собрание намеревается расследовать его деятельность, и опасаясь, что ему предъявят импичмент, то есть отрешат от должности и посадят в тюрьму, — не стал ждать. Он тайком, поспешно удрал, позаботившись о том, чтобы его прошение об отставке было обнародовано лишь после того, как сам он благополучно достигнет Севера.
Когда весть об этом облетела город — через неделю после бегства Баллока, — Атланта себя не помнила от возбуждения и радости. Улицы заполнились народом, мужчины смеялись и, поздравляя друг друга, обменивались рукопожатиями; женщины целовались и плакали. Все праздновали и устраивали приемы, и пожарные были заняты по горло, воюя с пламенем, загоравшимся от шутих, которые пускали на радостях мальчишки.
Опасность уже позади! Реконструкция почти закончилась! На всякий случай в губернаторах оставили республиканца, но в декабре предстояли новые выборы, и никто не сомневался в том, каковы будут их результаты. А когда выборы наступили, несмотря на отчаянные усилия республиканцев, в Джорджии губернатором снова стал демократ.
И люди тоже радовались, тоже волновались, но иначе, чем тогда, когда удрал Баллок. Радость была менее буйная, более прочувствованная, люди от всей души благодарили бога, и все церкви были полны, и священники возносили хвалу господу за избавление штата. К восторгу и радости примешивалась гордость — гордость за то, что в Джорджии удалось восстановить прежнее правление, несмотря на все усилия Вашингтона, несмотря на присутствие армии, несмотря на наличие «саквояжников», подлипал и местных республиканцев. Семь раз конгресс принимал акты, направленные на то, чтобы раздавить штат, оставить его на положении завоеванной провинции, трижды армия отменяла гражданские законы. В законодательное собрание проникли негры, алчные чужеземцы правили штатом, частные лица обогащались за счет общественных фондов. Джорджия лежала раздавленная, беспомощная, измученная, оплеванная. А теперь, невзирая ни на что, она снова стала сама собой, и все это — благодаря усилиям своих граждан.
Столь неожиданный поворот в судьбе республиканцев не всеми был воспринят как великое счастье. Уныние воцарилось в рядах подлипал, «саквояжников» и самих республиканцев. Гелерты и Хандоны, явно узнав об отставке Баллока еще до того, как это стало широко известно, неожиданно покинули город и растворились в небытии, откуда они и появились. Оставшиеся в Атланте «саквояжники» и подлипалы чувствовали себя неуверенно — напуганные случившимся, они жались друг к другу в поисках взаимной поддержки, гадая, какие из их темных делишек выплывут на свет в связи с начавшимся расследованием. Они уже не держались с высоким пренебрежением. Они были потрясены, растеряны, испуганы. И дамы, посещавшие Скарлетт, повторяли снова и снова:
«Ну, кто бы мог подумать, что так все повернется? Мы все считали губернатора всемогущим. Мы считали, что он здесь — навеки. Мы считали…» Скарлетт была не меньше их потрясена поворотом событий, хотя Ретт и предупреждал ее о том, в каком направлении они будут развиваться. И она вовсе не жалела, что Баллока не стало, а демократы вернулись к власти. Хотя никто бы этому не поверил, но и она восприняла с мрачной радостью известие о том, что господству янки наступил конец. Слишком живо она помнила, как ей пришлось изворачиваться в первые дни Реконструкции, как она страшилась, что солдаты и «саквояжники» отберут у нее деньги и собственность. Она помнила, как была беспомощна, какой панический страх обуревал ее оттого, что она не в силах была ничего предпринять, какую питала ненависть к янки, навязавшим Югу свое жестокое правление. И эта ненависть к янки никогда у нее не иссякала. Но пытаясь наиболее достойно выйти из положения, пытаясь добиться полной безопасности и уверенности в завтрашнем дне, она шагала в ногу с победителями. При всей своей нелюбви к ним, она окружила себя ими, порвала узы, связывавшие ее со старыми друзьями и прежним образом жизни. А теперь власть победителей испарилась. Скарлетт поставила на то, что правлению Баллока не будет конца, — и проиграла.
Озираясь вокруг в то рождество 1871 года, самое счастливое рождество для штата за последние десять лет, Скарлетт испытывала чувство глубокого беспокойства. Она не могла не видеть, что Ретт, которого раньше все ненавидели в Атланте, стал теперь одним из самых популярных жителей города, ибо он смиренно отрекся от республиканской ереси и отдавал все свое время, деньги, труд и разум Джорджии, помогая ей вернуться к былому благополучию. Когда он ехал по улицам, улыбаясь, приподнимая шляпу в знак приветствия, с маленьким голубым комочком — Бонни, торчавшим впереди него в седле, все тоже улыбались ему, охотно с ним заговаривали и дружелюбно поглядывали на девочку. А она, Скарлетт…

