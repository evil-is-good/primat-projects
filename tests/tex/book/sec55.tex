\chapter{\ }

— Дорогая моя, я не приму от тебя никаких объяснений и не желаю их слышать, — решительно заявила Мелани, осторожно положив худенькие пальчики на дрожащие губы Скарлетт и заставляя ее умолкнуть. — Ты оскорбляешь самое себя, и Эшли, и меня, даже думая, что между нами необходимы какие-то объяснения. Ведь мы трое.., как солдаты сражались вместе на протяжении стольких лет, так что мне даже стыдно за тебя: ну, как ты могла подумать, что досужие сплетни могут возвести между нами стену. Да неужели ты считаешь, я поверю, будто ты и мой Эшли… Какие глупости! Ведь я же знаю тебя, как никто в целом свете! Ты думаешь, я забыла, как самозабвенно ты жертвовала собой ради Эшли, Бо и меня, — забыла все, что ты сделала, начиная с того, что спасла мне жизнь, и кончая тем, что не дала нам умереть с голоду! Ты думаешь, я не помню, как ты шла босая, со стертыми в кровь руками за плугом, в который была впряжена лошадь того янки, — шла, чтобы мне и моему ребенку было что есть, — а если я это помню, то как могу поверить всяким гадостям про тебя? Я не желаю больше слышать ни слова от тебя, Скарлетт О’Хара. Ни единого слова.
— Но… — Скарлетт запнулась и умолкла.
Ретт покинул город час тому назад с Бонни и Присси, и к позору и злости Скарлетт прибавилось отчаяние. А сейчас, когда Мелани стала ее защищать, тогда как сама она, Скарлетт, чувствовала себя глубоко виноватой, — этого она и вовсе не в состоянии была вынести. Если бы Мелани поверила Индии и Арчи, оскорбила бы ее на празднике или хотя бы холодно приняла, она могла бы еще высоко держать голову и обороняться всеми видами оружия, какие имелись в ее распоряжении. Но при воспоминании о том, как Мелани, стоя рядом с нею словно тонкий сверкающий клинок, помогла предотвратить ее падение в глазах общества, с каким достоинством, с каким вызовом смотрела она на всех, — Скарлетт понимала, что если быть честной, надо во всем признаться. Да, выплеснуть из себя все, начиная с тех далеких дней на освещенном солнцем крыльце Тары.
К такому шагу Скарлетт побуждала совесть, которая хоть и долго молчала, но все еще способна была поднять голос, — совесть истинно верующей католички. «Покайся в грехах своих и понеси наказание за них в горе и смирении». Эллин сотни раз повторяла ей это, и сейчас, в критическую минуту, религиозное воспитание Эллин дало себя знать. Она покается — да, покается во всем, в каждом взгляде, в каждом слове, в тех немногих ласках, которые были между ними, и тогда господь, возможно, облегчит ее муки и даст ей покой. А наказанием ей послужит изменившееся лицо Мелани, на котором вместо любви и доверия появятся ужас и отвращение. О, какое это будет тяжкое наказание, с болью подумала Скарлетт, до конца жизни помнить лицо Мелани, знать, что Мелани известно, какая она мелкая, низкая, двуличная и неверная, какая лгунья.
Когда-то мысль о том, чтобы швырнуть правду в лицо Мелани и увидеть, как рухнет рай, в котором живет эта дурочка, опьяняла Скарлетт, представлялась ей игрой, которая стоит свеч. Но сейчас, за один вечер, все изменилось, и ей меньше всего хотелось так поступить. Почему — она и сама не знала. Слишком много противоречивых мыслей теснилось в ее мозгу. И она не могла разобраться в них. Понимала лишь, что страстно хочет, чтобы Мелани сохранила о ней высокое мнение, так же как когда-то страстно хотела, чтобы мама считала ее скромной, доброй, чистой. Понимала, что ей глубоко безразлично мнение всего света, безразлично, что думают о ней Эшли или Ретт, а вот Мелани должна думать о ней так же, как думала всегда.
Скарлетт боялась сказать Мелани правду, но сейчас в ней заговорил инстинкт честности, который редко давал о себе знать, — инстинкт, не позволявший рядиться в сотканную из лжи одежду перед женщиной, которая встала на ее защиту. И она помчалась к Мелани, как только Ретт и Бонни покинули дом.
Но при первых же словах, которые она, заикаясь, произнесла:
«Мелли, я должна все объяснить насчет того дня…», Мелани повелительно остановила ее. И Скарлетт, со стыдом глядя в темные, сверкающие любовью и возмущением глаза, почувствовала, как у нее захолонуло сердце, ибо поняла, что никогда не узнает мира и покоя, следующих за признанием. Мелани раз и навсегда отрезала ей этот путь своими словами. Скарлетт же, не слишком часто мыслившая по-взрослому, понимала, что лишь чистый эгоизм побуждает ее излить то, что так мучило ее. Это избавило бы ее от тягостного бремени и переложило бы его на невинное и доверчивое существо. Но она была в долгу перед Мелани за заступничество и оплатить этот долг могла лишь своим молчанием. А как жестоко расплатилась бы она с Мелани, если бы сломала ей жизнь, сообщив, что муж был ей неверен и любимая подруга принимала участие в измене!
«Не могу я ей этого сказать, — с огорчением подумала Скарлетт. — Никогда не смогу, даже если совесть убьет меня». Ей почему-то вспомнились слова пьяного Ретта: «…не может она поверить в отсутствие благородства у тех, кого любит… Так что придется вам нести и этот крест».
Да, этот крест она будет нести до самой смерти — будет молча терпеть свою муку, никому не скажет о том, как больно колет ее власяница стыда при каждом нежном взгляде и жесте Мелани, будет вечно подавлять в себе желание крикнуть: «Не будь такой доброй! Не сражайся за меня! Я этого недостойна!» «Если бы ты не была такой дурочкой, такой милой доверчивой, простодушной дурочкой, мне было бы легче, — в отчаянии думала она. — В своей жизни я несла не один тяжкий груз, но этот будет самым тяжким и наиболее неприятным из всех, какие когда-либо выпадали мне на долю».
Мелани сидела напротив нее в низком кресле, поставив ноги на высокий пуфик, так что колени у нее торчали, как у ребенка, — она бы никогда не приняла такой позы, если бы не забылась во гневе. В руке она держала кружевное плетение и так стремительно двигала блестящей иглой, словно это была рапира, которой она дралась на дуэли.
Будь Скарлетт в таком гневе, она бы топала ногами и орала, как некогда Джералд, громко призывая бога стать свидетелем проклятого двоедушия и подлости человеческой и клянясь так отомстить, что кровь будет стынуть в жилах. А у Мелани лишь стремительное мелькание иглы да сдвинутые на переносице тонкие брови указывали на то, что она вся кипит. Голос же ее звучал спокойно — вот только она отрывистее обычного произносила слова. Однако эта энергичная манера выражаться была чужда Мелани, которая вообще редко высказывала мнение вслух, а тем более никогда не злобствовала. Только тут Скарлетт поняла, что Уилксы и Гамильтоны способны распаляться не меньше, а наоборот — даже сильнее, чем О’Хара.
— Мне и так уже надоело слушать, как люди критикуют тебя, дорогая, — сказала Мелани, — а эта капля переполнила чашу, и я намерена кое-что предпринять. А ведь все потому, что люди завидуют тебе из-за твоего ума и успеха. Ты сумела преуспеть там, где даже многие мужчины потерпели крах. Только не обижайся на меня, дорогая. Я ведь вовсе не хочу сказать, что ты в чем-то перестала быть женщиной или утратила женскую прелесть, хотя многие это утверждают. Ничего подобного. Просто люди не понимают тебя и к тому же не выносят умных женщин. Однако то, что ты — умная и так преуспела в делах, не дает людям права говорить, будто вы с Эллин… Силы небесные!
Она произнесла это так пылко, что в устах мужчины это звучало бы как богохульство. Скарлетт смотрела на нее во все глаза, напуганная столь неожиданным взрывом.
— И еще являются ко мне со своей грязной ложью — и Арчи, и Индия, и миссис Элсинг! Да как они посмели? Миссис Элсинг, конечно, тут не было! У нее действительно не хватило мужества.
Но она всегда ненавидела тебя, дорогая, потому что ты пользовалась большим успехом, чем Фэнни. И потом она так взбесилась, когда ты отстранила Хью от управления лесопилкой. Но ты была абсолютно права. Он совершенно никчемный, неповоротливый, ни на что не годный человек! — Так Мелани одной фразой расправилась с товарищем детских игр и ухажером дней юности. — А вот за Арчи я виню себя. Не следовало мне давать этому старому негодяю приют. Все мне об этом говорили, но я не слушала. Ему, видите ли, не нравится, дорогая, что ты пользуешься трудом каторжников, но кто он такой, чтобы критиковать тебя? Убийца — да к тому же убил-то он женщину! И после всего, что я для него сделала, он является ко мне и говорит… Да я бы нисколько не пожалела, если бы Эшли пристрелил его. Ну, словом, я его выпроводила с такой отповедью, что уж можешь мне поверить! И он уехал из города.
Что же до Индии, этого подлого существа! Дорогая моя, я, конечно, сразу заметила, как только увидела вас вместе, что она завидует тебе и ненавидит, потому что ты красивее ее и у тебя столько поклонников. А особенно она возненавидела тебя из-за Стюарта Тарлтона. Она ведь так сокрушалась по Стюарту.., словом, неприятно говорить такое о своей золовке, но мне кажется, у нее помутилось в голове, потому что она все время только о Стюарте и думает! Другого объяснения ее поступкам я не нахожу… Я сказала ей, чтобы она никогда больше не смела переступать порог этого дома, и если я услышу, что она хотя бы шепотом намекнет на подобную гнусность, я.., я при всех назову ее лгуньей!
Мелани умолкла, гневное выражение сразу сошло с ее лица, уступив место скорби. Как все уроженцы Джорджии, Мелани была страстно предана своему клану, и мысль о ссоре в семье разрывала ей сердце. Она секунду поколебалась, но Скарлетт была дороже ей, Скарлетт была первой в ее сердце, и, верная своим привязанностям, она продолжала:
— Индия ревновала меня к тебе, потому что тебя, дорогая, я всегда любила больше. Но она теперь никогда не переступит порога этого дома, моей же ноги не будет в том доме, где принимают ее. Эшли полностью согласен со мной, правда, то, что родная сестра сказала такое, чуть не разбило ему сердце…
При упоминании имени Эшли натянутые нервы Скарлетт сдали, и она разразилась слезами. Когда же она перестанет причинять ему боль? Ведь она-то думала лишь о том, чтобы сделать его счастливым, обезопасить, а всякий раз только ранила. Она разбила ему жизнь, сломала его гордость и чувство самоуважения, разрушила внутренний мир, его спокойствие, проистекавшее от цельности натуры. А теперь она еще и отторгла его от сестры, которую он так любит. Чтобы спасти ее, Скарлетт, репутацию и не разрушать счастья своей жены, ему пришлось принести в жертву Индию, выставить ее лгуньей, полубезумной ревнивой старой девой, — Индию, которая была абсолютно права в своих подозрениях и своем осуждении: ведь ни одного лживого слова она не произнесла. Всякий раз, когда Эшли смотрел Индии в глаза, он видел в них правду — правду, укор и холодное презрение, на какое Уилксы были мастера.
Зная, что Эшли ставит честь выше жизни, Скарлетт понимала, что он, должно быть, кипит от ярости. Он тоже, как и Скарлетт, вынужден теперь прятаться за юбками Мелани. И хотя Скарлетт понимала, что это необходимо, и знала, что вина за ложное положение, в какое она поставила Эшли, лежит прежде всего на ней, тем не менее.., тем не менее… Как женщина, она больше уважала бы Эшли, если бы он пристрелил Арчи и честно повинился перед Мелани и перед всем светом. Она знала, что несправедлива к нему, но слишком она была сама несчастна, чтобы обращать внимание на такие мелочи. На память ей пришли едкие слова презрения, сказанные Реттом, и она подумала: «А в самом деле — так ли уж по-мужски вел себя в этой истории Эшли?» И впервые сияние, неизменно окружавшее Эшли с того первого дня, когда она в него влюбилась, начало немного тускнеть. Черное пятно позора и вины, лежавшее на ней, переползло и на него. Она решительно попыталась подавить в себе эту мысль, но лишь громче заплакала.
— Не надо так! Не надо! — воскликнула Мелани, бросая плетение, и, пересев на диван, притянула к своему плечу голову Скарлетт. — Не следовало мне говорить об этом и так тебя расстраивать. Я знаю, каково тебе; мы больше никогда не будем об этом говорить. Нет, нет, ни друг с другом, ни с кем-либо еще, словно ничего и не было. Только… — добавила она, и в ее тихом голосе почувствовался яд, — я уж проучу Индию и миссис Элсинг. Пусть не думают, что могут безнаказанно распространять клевету про моего мужа и мою невестку. Я так устрою, что ни одна из них не сможет больше ходить по Атланте с высоко поднятой головой. И всякий, кто будет им верить или будет их принимать, — отныне мой враг. Скарлетт, с грустью представив себе долгую череду грядущих лет, поняла, что становится отныне причиной вражды, которая на протяжении жизни многих поколений будет раскалывать город и семью.




Слово свое Мелани сдержала. Она никогда больше не упоминала о случившемся при Скарлетт или Эшли. Да и вообще ни с кем этого не обсуждала. Она вела себя с холодным безразличием, которое мгновенно превращалось в ледяную официальность, если кто-либо хотя бы намеком смел напомнить при ней о случившемся. На протяжении недель, последовавших за приемом, который она устроила в честь Эшли, когда город лихорадило от сплетен и перешептываний, а Ретт продолжал таинственно отсутствовать и не было человека, который стоял бы от всего этого в стороне, — Мелани не щадила клеветников, поносивших Скарлетт, будь то ее давние друзья или родня. Причем она не говорила, а действовала.
Она прилепилась к Скарлетт, точно моллюск — к раковине. Она заставила Скарлетт по утрам, как всегда, ездить в лавку и на лесной склад, и сама отправлялась с ней. Она настояла на том, чтобы Скарлетт днем разъезжала по городу, хотя той и не очень хотелось выставлять себя на обозрение своих любопытствующих сограждан. И всякий раз Мелани сидела в коляске рядом со Скарлетт. Мелани брала ее с собой, когда ездила днем к кому-нибудь с визитом, и мягко, но настойчиво вводила в гостиные, в которых Скарлетт не бывала уже больше двух лет. Беседуя с потрясенными хозяйками, Мелани всем своим видом решительно давала понять: «Если любишь меня, люби и моего пса».
Она заставляла Скарлетт рано приезжать с визитом и сидеть до тех пор, пока не уйдет последний гость, тем самым лишая дам возможности всласть понаслаждаться пересудами и домыслами и вызывая немалое их возмущение. Эти визиты были особенно мучительны для Скарлетт, но она не могла отказать Мелани. Ей ненавистно было сидеть среди женщин, гадавших в душе, в самом ли деле ее уличили в адюльтере. Ей ненавистно было сознание, что эти женщины никогда бы не заговорили с ней, если бы не любили так Мелани и не боялись потерять ее дружбу. Но Скарлетт знала: раз уж они приняли ее, то больше не смогут закрыть перед ней двери своего дома.
Любопытно, что лишь немногие, беря под защиту или критикуя Скарлетт, ссылались на ее порядочность. «Я считаю ее на все способной» — таково было общее мнение. Слишком много Скарлетт нажила себе врагов, чтобы теперь иметь защитников. Ее слова и действия не выходили у многих из ума, и потому людям было безразлично, причинит эта скандальная история ей боль или нет. Однако не нашлось бы человека, которому было бы безразлично, пострадают ли Мелани и Индия, и страсти бушевали прежде всего вокруг них, а не вокруг Скарлетт — все хотели знать:
«Солгала ли Индия?» Те, кто стоял на стороне Мелани, торжествующе отмечали, что все эти дни постоянно видели Мелани со Скарлетт. Неужели такая женщина, как Мелани, исповедующая столь высокие принципы, станет защищать женщину согрешившую, особенно если та согрешила с ее мужем? Конечно же, нет! Индия — просто свихнувшаяся старая дева, которая, ненавидя Скарлетт, налгала на нее так, что Арчи и миссис Элсинг поверили.
Но, спрашивали сторонники Индии, если Скарлетт ни в чем не повинна, где же тогда капитан Батлер? Почему он не рядом с женой, почему не поддерживает ее своим присутствием? На этот вопрос ответа не было, и по мере того, как шли недели, а по городу поползли слухи, что Скарлетт в положении, группа, поддерживавшая Индию, закивала с удовлетворением. Ребенок-то это не капитана Батлера, утверждали они. Слишком давно все знают, что супруги живут врозь. Слишком давно уже город потрясла скандальная весть об отдельных спальнях.
Сплетни ползли, разъединяя жителей города, — разъединяя и тесно спаянный клан Гамильтонов, Уилксов, Бэрров, Уитменов и Уинфилдов. Каждый в этой обширной родне вынужден был принять ту или иную сторону. Держаться нейтралитета было невозможно. Об этом уж позаботились и Мелани — с холодным достоинством, и Индия — с едкой горечью. Но на чьей бы стороне ни стояли родственники, всех злило то, что раскол в семье произошел из-за Скарлетт. И каждый считал, что такой огород городить из-за нее не стоило. А кроме того, на чьей бы стороне ни стояли родственники, все были искренне возмущены тем, что Индия решила стирать грязное белье семьи у всех на глазах и втянула Эшли в столь отвратительный скандал. Однако стоило ей заговорить, и многие поспешили на ее защиту и приняли ее сторону против Скарлетт, тогда как другие, любившие Мелани, встали на сторону Мелани и Скарлетт.
Добрая половина Атланты была в родстве или считала себя в родстве с Мелани и Индией. Разветвленная сеть кузенов, троюродных братьев, сестер, племянниц и всяких дальних родственников была столь сложной и запутанной, что никто, кроме уроженца Джорджии, не мог бы в этом разобраться. Они всегда представляли собой единый клан, который, стоило прийти беде, смыкал щиты и образовывал непробиваемую фалангу, каковы бы ни были мнения родственников друг о друге и о поведении тех или иных из них. Если не считать партизанской войны тети Питти против дяди Генри, над которой многие годы весело потешались все родственники, между членами клана внешне всегда поддерживались добрые отношения. Как большинство семей в Атланте, это были милые, уравновешенные, сдержанные люди, не склонные даже мягко журить друг друга.
Но сейчас клан раскололся надвое, и город мог наслаждаться зрелищем кузенов и кузин в пятом или шестом колене, которые принимали разные стороны в этом скандале, потрясшем всю Атланту. Это принесло немало трудностей и осложнений и той половине города, которая не состояла с ними в родстве, ибо требовало от людей предельного такта и долготерпения: ведь война между Индией и Мелани произвела раскол почти в каждом кружке или собрании. Поклонники Талии, Кружок шитья для вдов и сирот Конфедерации, Ассоциация по благоустройству могил наших доблестных воинов. Субботний музыкальный кружок. Дамское вечернее общество котильона, Кружок по устройству библиотек для юношества — все были вовлечены в эту вражду, как и четыре церкви, вместе с Дамами-попечительницами и миссионерскими обществами. И все зорко следили за тем, чтобы члены враждующих групп не оказались в одном и том же комитете.
В дни приемов у себя дома от четырех до шести матроны Атланты пребывали в непрерывном трепыхании и страхе, как бы Мелани и Скарлетт не явились и не застали у них в гостиной Индию и преданных ей родственников. Из всей семьи больше всего страдала тетя Питти. Именно Питти, всегда стремившаяся к тому, чтобы уютно жить в окружении любящих родственников, и сейчас готовая служить и вашим и нашим. Но ни те, ни другие не допускали этого.
Индия жила с тетей Питти, и если Питти встанет на сторону Мелани, как ей бы хотелось, Индия тут же уедет. А если Индия уедет, что станется тогда с бедненькой Питти? Она же не может жить одна. Ей пришлось бы тогда поселить у себя чужого человека, либо заколотить дом и переехать к Скарлетт. А тетя Питти смутно догадывалась, что капитану Батлеру это может не понравиться. Или же ей пришлось бы поселиться у Мелани и спать в крошечной каморке, которая служила детской для Бо.
Питти не слишком обожала Индию — она робела перед ней: уж очень Индия была сухая, жесткая, непреклонная в суждениях. Но она позволяла Питти сохранять уютный образ жизни, а для Питти соображения личного комфорта всегда больше значили, чем проблемы морали. Так что Индия продолжала жить у Питти.
Однако ее присутствие в доме превратило тетю Питти в центр бури, ибо и Скарлетт и Мелани считали, что она стала на сторону Индии. Скарлетт решительно отказалась увеличить содержание Питти, пока Индия находится под одной с ней крышей. Эшли же каждую неделю посылал Индии деньги, и каждую неделю Индия молча гордо возвращала чек — к великому огорчению и испугу Питти. С деньгами в красном кирпичном доме было бы совсем плохо, если бы не дядя Генри, хотя Питти и унижало то, что приходится брать деньги у него.
Питти любила Мелани больше всех на свете, за исключением себя самой, а теперь Мелани держалась с ней холодно и вежливо, как чужой человек. Хотя она жила, по сути дела, на заднем дворе тети Питти, она ни разу не прошла через изгородь, а в свое время бегала туда-сюда по десять раз в день. Питти заходила к ней, и плакала, и изливалась в любви и преданности, но Мелани всегда отказывалась что-либо с ней обсуждать и никогда не отдавала визитов. Питти прекрасно понимала, что она обязана Скарлетт, — по сути дела, жизнью. В те черные дни после войны, когда Питти встала перед выбором — либо принять помощь брата Генри, либо голодать, именно Скарлетт взялась вести ее дом, кормить ее, одевать, именно благодаря ей могла тетя Питти высоко держать голову в атлантском обществе. Да и после того как Скарлетт вышла замуж и перебралась в собственный дом, она была — сама щедрость. А этот страшный таинственный капитан Батлер.., после того как он заходил к ней со Скарлетт, Питти не раз обнаруживала у себя на столике новенький кошелек, набитый банкнотами, или в шкатулке для шитья — кружевной платочек, завязанный в узелок, а в нем — золотые монеты. Ретт всякий раз клялся, что понятия ни о чем не имеет, и без обиняков заявлял, что это наверняка от тайного поклонника — не иначе как от шаловливого дедушки Мерриуэзера.
Да, Мелани дарила Питти свою любовь, Скарлетт дарила обеспеченную жизнь, а что дарила ей Индия? Ничего, кроме своего присутствия, которое избавляло Питти от необходимости нарушить приятное течение жизни и самой принимать решения. Все это было очень грустно и так бесконечно вульгарно, и Питти, которая в жизни не приняла ни одного решения, махнула на все рукой — пусть идет, как идет, однако же проводила немало времени в безутешных рыданиях.
В конце концов, кое-кто искренне поверил в то, что Скарлетт ни в чем не виновата, — поверил не из-за ее личных достоинств, а потому, что этому верила Мелани. Иные поверили с оговорками, но были любезны со Скарлетт и посещали ее, потому что любили Мелани и хотели сохранить ее любовь. Сторонники же Индии лишь холодно с нею раскланивались, а некоторые даже открыто грубили. Эти последние ставили Скарлетт в неловкое положение, раздражали, но она понимала, что если бы Мелани так быстро не пришла ей на помощь, весь город был бы сейчас против нее, она бы стала изгоем.

