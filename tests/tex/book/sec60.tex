\chapter{\ }

Что-то в мире было не так, в воздухе чувствовалось что-то мрачное, зловещее, окутавшее все непроницаемым туманом, который кольцом окружал Скарлетт. И это ощущение, что в мире что-то не так, было связано не только со смертью Бонни, потому что боль от невыносимого горя постепенно сменилась смирением перед понесенной утратой. И однако же призрачное ощущение беды не проходило — точно кто-то черный, зловещий стоял у плеча Скарлетт, точно земля под ее ногами в любой момент могла превратиться в зыбучие пески.
Никогда прежде не знала Скарлетт такого страха. Всю свою жизнь она строила на надежной основе здравого смысла и боялась лишь зримого и ощутимого — увечья, голода, бедности, утраты любви Эшли. Никогда прежде не занимаясь анализом, она пыталась сейчас анализировать свое состояние, но безуспешно. Она потеряла любимое дитя, но в общем-то могла это вынести, как вынесла другие сокрушительные потери. Она была здорова, денег у нее было сколько душе угодно, и у нее по-прежнему был Эшли, хотя последнее время она все реже и реже видела его. Даже то отчуждение, которое возникло между ними после незадавшегося праздника, устроенного Мелани, не тревожило ее больше, ибо она знала, что это пройдет. Нет, она боялась не боли, и не голода, и не утраты любви. Эти страхи никогда не угнетали ее в такой мере, как угнетало сейчас ощущение, что в мире что-то не так, — это был страх, затмевавший все остальные чувства, очень похожий на то, что она испытывала раньше во время кошмаров, когда бежала сквозь густой клубящийся туман так, что казалось, сердце лопнет, — потерявшееся дитя в поисках приюта.
Она вспоминала, как Ретт умел утешить ее и смехом разгонял ее страхи. Вспоминала, как успокаивалась на его сильных руках, прижавшись к смуглой груди. И всем своим существом потянувшись к нему, она впервые за многие недели по-настоящему его увидела. Он так переменился, что она пришла в ужас. Этот человек уже никогда не будет смеяться, никогда не будет ее утешать.
Какое-то время после смерти Бонни она была слишком на него зла, слишком поглощена собственным горем и лишь вежливо разговаривала с ним при слугах. Слишком ушла она в себя, вспоминая быстрый топот Бонниных ножек, ее заливчатый смех, и потому даже не подумала, что и он, наверное, вспоминает все это, только ему вспоминать еще больней, чем ей. На протяжении всех этих недель они встречались и разговаривали — вежливо, как чужие люди, которые встречаются в безликих стенах отеля, живут под одной крышей, сидят за одним столом, но не обмениваются друг с другом сокровенными мыслями.
Теперь, когда ей стало страшно и одиноко, она сломала бы этот барьер, если бы могла, но она обнаружила, что Ретт удерживает ее на расстоянии, словно не желает говорить с ней ни о чем, кроме самого необходимого. Теперь, когда злость ее стала проходить, ей хотелось сказать ему, что она не винит его в смерти Бонни. Ей хотелось поплакать в его объятиях и признаться, что она ведь тоже невероятно гордилась тем, что девочка так хорошо скачет на своем пони, тоже была невероятно снисходительна к капризам дочурки. Сейчас Скарлетт готова была унизиться перед Реттом и признать, что напрасно обвинила его в смерти дочери — слишком она была несчастна и надеялась, причинив ему боль, облегчить себе душу. Но ей все не удавалось найти подходящий момент для разговора. Ретт смотрел на нее своими черными непроницаемыми глазами, и слова не шли у нее с языка. А когда слишком долго откладываешь признание, его все труднее и труднее сделать, и наконец наступает такой момент, когда оно просто становится невозможным.
Скарлетт не могла понять, почему так получается. Ведь Ретт же — ее муж, и они неразрывно связаны тем, что делили одну постель, были близки, и зачали любимое дитя, и слишком скоро увидели, как их дитя поглотила темная яма. Только в объятиях отца этого ребенка, делясь с ним воспоминаниями и горюя, могла бы она обрести утешение — сначала было бы больно, а потом именно боль и помогла бы залечить рану. Но при нынешнем положении дел она готова была бы скорее упасть в объятия постороннего человека.
Ретт редко бывал дома. А когда они вместе садились ужинать, он, как правило, был пьян. Вино действовало на него теперь иначе, чем прежде, когда он постепенно становился все более любезным и язвительным, говорил забавные колкости, и в конце концов она, помимо воли, начинала смеяться. Теперь, выпив, он был молчалив и мрачен, а под конец вечера сидел совсем отупевший. Иной раз на заре она слышала, как он въезжал на задний двор и колотил в дверь домика для слуг, чтобы Порк помог ему подняться по лестнице и уложил в постель. Его — укладывать в постель! Это Ретта-то, который всегда был трезв, когда другие уже валялись под столом, и сам укладывал всех спать.
Он перестал следить за собой, тогда как раньше всегда был тщательно выбрит и причесан, и Порку приходилось долго возмущаться и упрашивать, чтобы он хотя бы сменил рубашку перед ужином. Пристрастие к виски начало сказываться и на внешности Ретта, и его еще недавно четко очерченное лицо расплылось, щеки нездорово опухли, а под вечно налитыми кровью глазами образовались мешки. Его большое мускулистое тело обмякло, и талия стала исчезать.
Он часто вообще не приезжал домой ночевать и даже не сообщал, что не приедет. Вполне возможно, что он храпел пьяный в комнате над каким-нибудь салуном, но Скарлетт всегда казалось, что в таких случаях он — у Красотки Уотлинг. Однажды она увидела Красотку в магазине — теперь это была вульгарная располневшая женщина, от красоты которой почти ничего не осталось. Накрашенная и одетая во все яркое, она тем не менее выглядела дебелой матроной. Вместо того чтобы опустить глаза или с вызовом посмотреть на Скарлетт, как это делали, встречаясь с дамами, другие особы легкого поведения. Красотка ответила ей внимательным взглядом, в котором была чуть ли не жалость, так что Скарлетт даже вспыхнула.
Но она уже не могла обвинять Ретта, не могла устраивать ему бурные сцены, требовать верности или стыдить его — как не могла заставить себя просить у него прощения за то, что обвинила его в смерти Бонни. Ее сковывала какая-то непостижимая апатия. Она сама на понимала, почему она несчастна — такой несчастной она еще никогда себя не чувствовала. И она была одинока — бесконечно одинока. Возможно, до сих пор у нее просто не было времени почувствовать себя одинокой. А сейчас одиночество и страх навалились на нее и ей не к кому было обратиться — разве что к Мелани. Ибо даже Мамушка, ее извечная опора, отбыла в Тару. Уехала навсегда.
Мамушка не объяснила причины своего отъезда. Она с грустью посмотрела на Скарлетт усталыми старыми глазами и попросила дать ей денег на дорогу домой. В ответ на слезы Скарлетт и уговоры остаться Мамушка сказала лишь:
— Похоже, мисс Эллин зовет меня: «Мамушка, возвращайся домой. Поработала — и хватит». Вот я и еду домой.
Ретт, услышав этот разговор, дал Мамушке денег и похлопал по плечу.
— Ты права, Мамушка, и мисс Эллин права. Ты здесь действительно поработала, и хватит. Поезжай домой. И дай мне знать, если тебе когда-нибудь что-то понадобится. — И когда Скарлетт попыталась было возмутиться и начать командовать, он прикрикнул на нее: — Замолчите вы, идиотка! Отпустите ее! Кому охота оставаться в этом доме — теперь?
При этих словах в глазах его загорелся такой дикий огонь, что Скарлетт в испуге, отступила.
— Доктор Мид, вам не кажется, что, пожалуй.., что он, возможно, потерял рассудок? — спросила она, решив через некоторое время поехать к доктору посоветоваться, ибо чувствовала себя совершенно беспомощной.
— Нет, — сказал доктор, — но он пьет как лошадь и убьет себя, если не остановится. Он ведь очень любил девочку и, мне кажется, Скарлетт, пьет, чтобы забыться. Мой совет вам, мисс: подарите ему ребенка — да побыстрее.
«Ха! — с горечью подумала Скарлетт, выходя из его кабинета. — Легко сказать». Она бы с радостью родила ему еще ребенка — и не одного, — если бы это помогло прогнать отчужденность из глаз Ретта, затянуть раны его сердца. Мальчика, который был бы такой же смуглый и красивый, как Ретт, и маленькую девочку. Ах, еще одну девочку — хорошенькую, веселую, своенравную, смешливую, не как эта пустоголовая Элла. Почему, ну, почему не мог господь прибрать Эллу, если уж надо было отнимать у нее ребенка? Элла не утешала ее — она не могла заменить матери Бонни. Но Ретт, казалось, не хотел больше иметь детей. Во всяком случае, он никогда не заходил к ней в спальню, хотя дверь теперь не запиралась, а была, наоборот, зазывно приотворена. Ему, казалось, это было глубоко безразлично. Видно, ему все было безразлично, кроме виски и этой толстой рыжей бабы.
Теперь он злился, тогда как раньше мило подшучивал над ней, и грубил, тогда как раньше его уколы смягчались юмором. После смерти Бонни многие из добропорядочных дам, которые жили по соседству и которых он сумел завоевать своим чудесным отношением к дочке, старались проявить к нему доброту. Они останавливали его на улице, выражали сочувствие, заговаривали с ним поверх изгороди и выказывали свое понимание. Но теперь, когда Бонни не стало, для него исчезла необходимость держаться благовоспитанно, а вместе с необходимостью исчезла и благовоспитанность. И он грубо обрывал искренние соболезнования дам.
Но, как ни странно, дамы не оскорблялись. Они понимали или считали, что понимают. Когда он ехал домой в сумерки, пьяный, еле держась в седле, хмуро глядя на тех, кто с ним заговаривал, дамы говорили: «Бедняга!» — и удваивали усилия, стараясь проявить мягкость и доброту. Они очень его жалели — этого человека с разбитым сердцем, который возвращался домой к такому утешителю, как Скарлетт.
А все знали, до чего она холодная и бессердечная. Все были потрясены тем, как, казалось, легко и быстро она оправилась после смерти Бонни, и никто так и не понял, да и не пытался понять, скольких усилий ей это стоило. Всеобщее искреннее сочувствие было на стороне Ретта, а он не знал этого и не ценил. Скарлетт же все в городе просто не выносили, в то время как именно сейчас она так нуждалась в сочувствии старых друзей.
Теперь никто из старых друзей не появлялся у нее в доме, за исключением тети Питти, Мелани и Эшли. Только новые друзья приезжали в блестящих лакированных колясках, стремясь выразить ей свое сочувствие, стараясь отвлечь ее сплетнями о каких-то других друзьях, которые нисколько ее не интересовали. Все это «пришлые», чужаки — все до одного! Они не знают ее. И никогда не узнают. Они понятия не имеют, как складывалась ее жизнь, пока она не обрела нынешнее прочное положение и не поселилась в этом дворце на Персиковой улице. Да и они не стремились говорить о том, какова была их жизнь до того, как у них появились эти шуршащие шелка и виктории, запряженные отличными лошадьми. Они не знали, какую она вынесла борьбу, какие лишения, сколько было в ее жизни такого, что позволило ей по-настоящему ценить и этот большой дом, и красивые наряды, и серебро, и приемы. Они всего этого не знали. Да и не стремились узнать — что им до нее, этим людям, приехавшим неизвестно откуда, как бы скользившим по поверхности, не связанным общими воспоминаниями о войне, и голоде, и борьбе с захватчиками, людям, не имеющим общих корней, которые уходят вглубь, все в ту же красную землю.
Теперь, оставшись одна, Скарлетт предпочла бы коротать дни с Мейбелл, или Фэнни, или миссис Элсинг, или миссис Уайтинг, или с миссис Тоннелл, или даже с этой грозной воинственной старухой миссис Мерриуэзер, или.., или с кем угодно из старых друзей и соседей. Потому что они знали. Они знали войну, и ужасы, и пожары, видели, как преждевременно погибли их близкие, они голодали и ходили в старье, и нужда стояла у их порога. И из ничего снова создали себе состояние.
Каким утешением было бы посидеть с Мейбелл, помня о том, что Мейбелл похоронила ребенка, погибшего во время отчаянного бегства перед наступавшей армией Шермана. Какое утешение нашла бы она в обществе Фэнни, помня, что обе они потеряли мужей в те черные дни, когда их край был объявлен на военном положении. Как горько посмеялись бы они с миссис Элсинг, вспоминая, как пожилая дама, нахлестывая лошадь, мчалась в день падения Атланты мимо Пяти Углов и захваченные на военном складе припасы вылетали у нее из повозки. Как было бы приятно излить душу миссис Мерриуэзер, теперь такой преуспевающей благодаря своей булочной, и сказать ей: «А помните, как туго было после поражения? Помните, как мы не знали, откуда взять новую пару обуви? А посмотрите на нас теперь!» Да, это было бы приятно. Теперь Скарлетт понимала, почему бывшие конфедераты при встрече с таким упоением, с такой гордостью, с такой ностальгией принимались говорить о войне. То были дни серьезных испытаний, но они прошли через них. И стали ветеранами. Она ведь тоже ветеран, но только нет у нее товарищей, с которыми она могла бы возродить в памяти былые битвы. Ах, как бы ей хотелось снова быть с людьми своего круга, с теми, кто пережил то же, что и она, и знает, как это было больно и, однако же, стало неотъемлемой частью тебя самого!
Но каким-то образом все эти люди отошли от нее. Скарлетт понимала, что никто, кроме нее, тут не виноват. Она никогда не нуждалась в них до сегодняшнего дня — дня, когда Бонни уже нет на свете, а она так одинока и испугана и сидит за сверкающим обеденным столом напротив пьяного, отупевшего, совсем чужого человека, превратившегося в животное у нее на глазах.

