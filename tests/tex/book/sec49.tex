\chapter{\ }

Миссис Элсинг прислушалась к звукам, доносившимся из холла, и, убедившись, что Мелани прошла на кухню, откуда донесся стук посуды и звон серебра, обещавшие скорое подкрепление, повернулась и тихо заговорила, обращаясь к дамам, которые сидели полукругом в гостиной, держа на коленях корзиночки с шитьем.
— Лично я не намерена посещать Скарлетт ни сейчас, ни когда-либо впредь, — заявила она, и ее тонкое холодное лицо стало еще холоднее.
Остальные члены Кружка шитья для вдов и сирот Конфедерации быстро воткнули иголки в материю и сдвинули свои качалки. Всех дам буквально распирало от желания поговорить о Скарлетт и Ретте, но мешало присутствие Мелани. Как раз накануне эта парочка вернулась из Нового Орлеана и поселилась в свадебном номере в отеле «Нейшнл».
— Хью говорит, что вежливость требует, чтобы я нанесла им визит: ведь капитан Батлер спас ему жизнь, — продолжала миссис Элсинг. — И бедняжка Фэнни приняла его сторону и сказала, что тоже пойдет к ним. А я сказала ей: «Фэнни, — сказала я, — если бы не Скарлетт, Томми был бы сейчас жив. Это оскорбительно для его памяти — идти туда». А у Фэнни хватило ума заявить мне: «Мама, я же пойду не к Скарлетт. Я пойду к капитану Батлеру. А он все сделал, чтобы спасти Томми, и не его вина, что ему это не удалось».
— До чего же они глупые, эти молодые люди! — сказала миссис Мерриуэзер. — Визиты наносить — как же! — Грудь у нее поднялась горой от возмущения при воспоминании о том, как грубо отклонила Скарлетт ее совет не выходить замуж за Ретта. — И моя Мейбелл такая же глупая, как ваша Фэнни. Заявила, что они с Рене отправятся с визитом: ведь капитан Батлер спас-де Рене от виселицы. А я сказала, что, если бы Скарлетт не раскатывала всем напоказ, Рене никогда бы не оказался в опасности. И папаша Мерриуэзер тоже намерен идти с визитом — послушать его, так он от старости, видно, совсем ума лишился: говорит, что если я не чувствую благодарности, то он безусловно благодарен этому мерзавцу. Клянусь, с тех пор как папаша Мерриуэзер побывал в доме этой Уотлинг, он ведет себя самым постыдным образом. Визиты наносить — как же! Я-то уж, конечно, не пойду. Скарлетт изгнала себя из нашего общества, выйдя замуж за такого человека. Он был уже достаточно мерзок, когда спекулировал во время войны и наживался на нашем горе, а сейчас, когда его с «саквояжниками» и подлипалами водой не разольешь, да еще он в друзьях-приятелях — да-да, в ближайших друзьях — с этим отъявленным мерзавцем губернатором Баллоном… Визиты им наносить — как же!
Миссис Боннелл вздохнула. Это была тучная смуглая клуша с веселым лицом.
— Они ведь пойдут к ним только раз — с визитом вежливости, Долли. Не знаю, можно ли людей за это винить. Я слышала, все мужчины, которые в ту ночь не были дома, хотят идти к ним с визитом, и я думаю, они должны пойти. Правда, мне как-то трудно представить себе, что Скарлетт — дочь своей матери. Я ходила в школу с Эллин Робийяр в Саванне, и это была прелестнейшая девушка, я ее очень любила. И почему только ее отец не захотел, чтоб она вышла замуж за своего кузена Филиппа Робийяра! Ничего по-настоящему плохого в этом молодом человеке не было — ведь всем молодым людям надо перебеситься. А Эллин зачем-то поспешила и выскочила замуж за этого О’Хара, который был намного старше нее, — вот у нее и получилась такая дочь, как Скарлетт. И все же я считаю, что один раз должна нанести им визит — в память об Эллин.
— Сентиментальная чепуха! — решительно фыркнула миссис Мерриуэзер. — Китти Боннел, неужели вы пойдете с визитом к женщине, которая вышла замуж меньше чем через год после смерти мужа? К женщине…
— Из-за которой к тому же погиб мистер Кеннеди, — вмешалась Индия. Она произнесла эта спокойным, но таким язвительным тоном. Стоило зайти речи о Скарлетт, как она забывала о вежливости и помнила одно — вечно помнила о Стюарте Тарлтоне. — Да к тому же я всегда считала, что между нею и этим Батлером было что-то до того, как погиб мистер Кеннеди, о чем мало кто подозревает.
Не успели дамы прийти в себя от изумления, вызванного ее словами, — к тому же говорила-то это девица, — как в дверях появилась Мелани. А они были настолько поглощены пересудами, что не слышали ее легких шагов, и сейчас, застигнутые врасплох хозяйкой дома, походили на перешептывающихся школьниц, пойманных с поличным учительницей. Они не только оцепенели, но и перепугались при виде того, как изменилось лицо Мелани. Она стояла вся красная от праведного гнева, ласковые глаза ее метали молнии, ноздри трепетали. Никто из присутствующих ни разу не видел Мелани разгневанной. Ни одна из дам даже и предположить не могла, что Мелани способна на такую ярость. Они все любили ее, но считали необычайно мягкой, уступчивой молодой женщиной, с уважением относящейся к старшим и не имеющей собственного мнения.
— Да как ты смеешь, Индия! — произнесла она тихо, дрожащим от гнева голосом. — Куда способна завести тебя ревность! Постыдилась бы.
Индия побелела, но не опустила головы.
— Я ни одного своего слова не возьму назад, — заявила она. Но ум ее тем временем усиленно работал. «Неужели я действительно ревную?» — думала она. Разве воспоминания о Стюарте Тарлтоне, о Милочке и Чарльзе не давали ей достаточно оснований завидовать Скарлетт? Разве не было у нее достаточно оснований ненавидеть Скарлетт, особенно сейчас, когда у нее зародилось подозрение, что Скарлетт каким-то образом опутала своей паутиной и Эшли? Она подумала: «Я бы многое могла рассказать тебе, Мелани, об Эшли и вашей драгоценной Скарлетт». Индию раздирали противоречивые желания: оберечь Эшли своим молчанием или показать, каков он на самом деле, поделившись своими подозрениями с Мелани и всем миром. Это заставило бы Скарлетт выпустить Эшли из своих цепких рук. Но время для разговора еще не настало. У нее ведь не было никаких доказательств — ничего, кроме подозрений.
— Я ни одного слова не возьму назад, — повторила она.
— В таком случае твое счастье, что ты не живешь больше под моей крышей, — сказала Мелани холодным, как лед, тоном. Индия вскочила, впалые щеки ее залила яркая краска.
— Мелани, ты.., ты же моя невестка.., не станешь же ты ссориться со мной из-за этой беспутной…
— Скарлетт тоже моя невестка, — сказала Мелани, глядя в упор на Индию, точно они были чужими друг другу. — И она мне дороже родной сестры. Если ты забыла о том, скольким я ей обязана, то я не забыла. Она пробыла со мной всю осаду, хотя могла уехать домой, — даже тетя Питти и та сбежала в Мейкон. Она помогла мне родить, когда янки были у самой Атланты, и взвалила на себя такую обузу — повезла нас с Бо в Тару, хотя могла оставить меня в больнице на милость янки. Она ухаживала за мной и кормила меня, а сама едва держалась на ногах от усталости и недоедания. В Таре мне дали лучший матрац, потому что я была больная и слабая. А когда я смогла ходить, мне дали единственную целую пару туфель. Ты, Индия, возможно, забыла все, что она сделала для меня, но я не могу этого забыть. А когда Эшли вернулся больной, несчастный, без крова над головой, без денег в кармане, она приняла его в свой дом, как сестра. А когда мы думали, что придется уехать на Север, и у нас сердце разрывалось при мысли, что мы расстанемся с Джорджией, Скарлетт вмешалась и поставила Эшли управлять лесопилкой. А капитан Батлер спас Эшли жизнь исключительно по доброте. И конечно же, у Эшли нет никаких к нему претензий! Я благодарна, благодарна и Скарлетт, и капитану Батлеру. Что же до тебя, Индия!.. Как ты можешь забыть то, что Скарлетт сделала для меня и для Эшли?! Как ты можешь так мало ценить жизнь брата, чтобы порочить человека, который его спас?! Да встань ты на колени перед капитаном Батлером и Скарлетт — даже этого было бы недостаточно.
— Ну, вот что, Мелли, — решительно вмешалась миссис Мерриуэзер, к которой вернулось самообладание, — не надо так говорить с Индией.
— Я ведь слышала и то, что вы говорили про Скарлетт, — воскликнула Мелани, поворачиваясь к дородной пожилой даме словно дуэлянт, который, вытащив клинок из распростертого тела противника, яростно набрасывается на другого. — И вы тоже, миссис Элсинг. Как вы относитесь к Скарлетт в глубине ваших мелких душонок, мне безразлично, это меня не касается. Но то, что вы говорите о ней в моем доме или в моем присутствии, это уже меня касается. Вот только как вы можете хотя бы думать такие гадости, а тем более их говорить? Неужели вы так мало цените своих мужчин, что хотели бы видеть их мертвыми, а не живыми? Неужели у вас нет ни капли благодарности к человеку, который спас их, причем спас, рискуя собственной жизнью? Ведь янки легко могли заподозрить, что он — член ку-клукс-клана, если бы вся правда вышла наружу! Они могли бы повесить его. И тем не менее он пошел на риск ради ваших мужчин. Ради вашего свекра, миссис Мерриуэзер, и вашего зятя, и ваших двух племянников в придачу. И ради вашего брата, миссис Боннелл, и ради вашего сына и вашего зятя, миссис Элсинг. Неблагодарные — вот вы кто! И я требую, чтобы все вы извинились.
Миссис Элсинг вскочила на ноги и, крепко сжав губы, принялась засовывать шитье в корзиночку.
— Если бы кто-нибудь когда-нибудь сказал мне, что ты можешь быть такой невоспитанной, Мелли… Нет, я не стану перед тобой извиняться. Индия права. Скарлетт — легкомысленная, беспутная женщина. Я не могу забыть, как она вела себя во время войны. И не могу забыть, как она повела себя точно последняя голодранка, когда у нее завелось немного денег…
— Вы не можете забыть, — перебила ее Мелани, крепко прижав кулачки к бокам, — что она выставила Хью, потому что у него не хватало ума управлять лесопилкой.
— Мелли! — хором взмолились дамы. Миссис Элсинг вскинула голову и направилась к выходу. Уже взявшись за ручку парадной двери, она остановилась и обернулась.
— Мелли, — сказала она, и голос ее потеплел, — деточка, ты разбиваешь мне сердце. Я ведь была лучшей подругой твоей мамы и помогала доктору Миду принимать тебя, и я любила тебя, точно собственное дитя. Если бы речь шла о чем-то важном, было бы не так тяжело выслушивать все это от тебя. Но когда речь идет о такой женщине, как Скарлетт О’Хара, которая способна сделать и тебе гадость, как и любой из нас…
Слезы вновь выступили на глазах Мелани при первых же словах миссис Элсинг, но к концу ее тирады лицо Мелани стало жестким.
— Я хочу, чтобы вы все знали, — сказала она, — та из вас, кто не пойдет с визитом к Скарлетт, может никогда, никогда больше не приходить ко мне.
Раздался гул голосов, и дамы в смятении поднялись. Миссис Элсинг швырнула на пол корзиночку с шитьем и вернулась в гостиную, ее фальшивая челка съехала набок.
— Я не могу с этим примириться! — воскликнула она. — Не могу! Ты не в своем уме, Мелли: по-моему, ты сама не знаешь, что говоришь. Ты по-прежнему останешься моим другом, а я по-прежнему останусь твоим. Я не допущу, чтобы мы из-за этого поссорились.
Она расплакалась, и Мелани неожиданно очутилась в ее объятиях — она тоже плакала, но и всхлипывая, продолжала твердить, что не откажется ни от одного своего слова. Еще две-три дамы разрыдались, а миссис Мерриуэзер, громко сморкаясь в платок, принялась целовать миссис Элсинг и Мелани. Тетя Питти, в ужасе наблюдавшая за всем этим, вдруг опустилась на пол, и на сей раз — что случалось с ней нечасто — действительно лишилась чувств. Среди этих слез, суматохи, поцелуев, поисков нюхательных солей и коньяка лишь у одной женщины лицо оставалось бесстрастным, лишь у одной были сухие глаза. Индия Уилкс вышла из дома, не замеченная никем.
Дедушка Мерриуэзер, встретившись несколькими часами позже с дядей Генри Гамильтоном в салуне «Наша славная девчонка», рассказал со слов миссис Мерриуэзер о том, что произошло утром. Поведал он об этом с превеликим удовольствием, ибо был в восторге от того, что у кого-то хватило мужества осадить его грозную сноху. Ему самому мужества на это никогда, конечно, не хватало.
— Ну и что же эта свора идиоток наконец решила? — раздраженно осведомился дядя Генри.
— Я, право, не знаю, — сказал дедушка, — но похоже, что Мелли в этом забеге шутя одержала победу. Могу поклясться, все они нанесут визит Скарлетт — хотя бы раз. Люди очень высоко ставят вашу племянницу, Генри.
— Мелли — дурочка, а дамы правы. Скарлетт — ловкая штучка, и я просто не понимаю, зачем Чарли понадобилось в свое время жениться на ней, — мрачно заметил дядя Генри. — Но и Мелли по-своему права. Приличия требуют, чтобы все, кого спас капитан Батлер, нанесли ему визит. Если на то пошло, я лично ничего против Батлера не имею. Он достойно вел себя в ту ночь, спасая наши шкуры. А вот Скарлетт сидит у меня как заноза под хвостом. Слишком она шустра — ей же от этого только хуже. Но с визитом мне пойти к ней придется. Стала Скарлетт подлипалой или не стала, а она, как-никак, моя родственница. И пойду я к ним сегодня же.
— И я пойду с вами. Генри. Долли, пронюхай она об этом, пришлось бы, наверно, связать. Подождите, вот только пропущу еще стаканчик.
— Нет, пить мы будем у капитана Батлера. Что ни говорите, у него хорошего вина всегда вдоволь.




Ретт сказал, что «старая гвардия» никогда не сдастся, и был прав. Он понимал, что никакого значения этим нескольким визитам придавать нельзя, как понимал и то, почему они были нанесены. Хотя родственники мужчин, участвовавших в том злополучном налете ку-клукс-клана, и пришли к ним первые с визитом, но больше почти не появлялись. И к себе Ретта Батлера не приглашали.
Ретт, заметил, что они и вовсе бы не пришли, если бы не боялись Мелани. Скарлетт понятия не имела, откуда у него возникла такая мысль, но она тотчас с презрением ее отвергла. Ну, какое влияние могла иметь Мелани на таких людей, как миссис Элсинг и миссис Мерриуэзер? То, что они больше не заходили, мало волновало ее, — собственно, их отсутствия она почти и не замечала, поскольку в номере у нее полно было гостей другого рода. «Пришлые» — называли в Атланте таких людей, а то употребляли и менее вежливое слово.
А в отеле «Нейшнл» нашло себе пристанище немало «пришлых», которые, как и Ретт со Скарлетт, жили там в ожидании, пока будут выстроены их дома. Это были веселые богатые люди, очень похожие на новоорлеанских друзей Ретта, — элегантно одетые, легко сорящие деньгами, не слишком распространяющиеся о своем прошлом. Все мужчины были республиканцами и «находились в Атланте по делам, которые вели с правительством штата». Что это были за дела, Скарлетт не знала и не трудилась узнавать.
Правда, Ретт мог бы со всею достоверностью рассказать ей, что эти люди занимались тем же, что и канюки, обгладывающие падаль. Они издали чуют смерть и безошибочно находят то место, где можно нажраться до отвала. А в правительстве Джорджии, по сути дела, уже не осталось коренных жителей, штат был совершенно беспомощен, и авантюристы стаями слетались сюда.
Жены приятелей Ретта из подлипал и «саквояжников» гуртом валили к Скарлетт, как и «пришлые», с которыми она познакомилась, когда продавала лес для их новых домов. Ретт сказал, что если она может вести с этими людьми дела, то должна принимать их, а однажды приняв их, она поняла, что с ними весело. Они были хорошо одеты и никогда не говорили о войне или о тяжелых временах, а беседовали лишь о модах, скандалах и висте. Скарлетт никогда раньше не играла в карты и, научившись за короткое время хорошо играть, с увлечением предалась висту.
В номере у нее всегда собиралась компания игроков. Но она редко бывала у себя в эти дни, ибо была слишком занята строительством дома и не могла уделять время гостям. Сейчас ее не слишком занимало, есть у нее визитеры или нет. Ей хотелось отложить все светские развлечения до той поры, когда будет закончен дом и она сможет предстать перед обществом в роли хозяйки самого большого особняка в Атланте, где будут устраиваться изысканнейшие приемы.
Долгими жаркими днями следила она за тем, как растет ее красный кирпичный дом под крышей из серой дранки, — дом, возвышавшийся надо всеми домами на Персиковой улице. Забыв о лавке и о лесопилках, она проводила все время на участке — препиралась с плотниками, спорила со штукатурами, изводила подрядчика. Глядя на то, как быстро поднимаются стены, она с удовлетворением думала о том, что когда дом будет закончен, он станет самым большим и самым красивым в городе. Даже более внушительным, чем соседний особняк Джеймса, который только что купили для официальной резиденции губернатора Баллока.
Особняк губернатора мог гордиться своими кружевными перилами и карнизами, но все это в подметки не годилось затейливому орнаменту на доме Скарлетт. У губернатора была бальная зала, но она казалась не больше бильярдного стола по сравнению с огромным помещением, отведенным для этой цели у Скарлетт и занимавшим весь четвертый этаж. Вообще в ее доме было все, и даже в больших количествах, чем в любом особняке или любом другом городском доме, — больше куполов, и башен, и башенок, и балконов, и громоотводов, и окон с цветными стеклами.
Вокруг дома шла веранда, и с каждой стороны к ней вели четыре ступени. Двор был большой, зеленый, со старинными чугунными скамьями, расставленными тут и там, чугунной беседкой, названной модным словечком «бельведер» и, как заверили Скарлетт, построенной по готическому образцу, а также двумя большими чугунными статуями — одна изображала оленя, другая — бульдога величиной с шотландского пони. Для Уэйда и Эллы, несколько испуганных размерами, роскошью и модным в ту пору полумраком их нового дома, эти два чугунных зверя были единственной утехой.
Внутри дом был обставлен сообразно желаниям Скарлетт: толстые красные ковры покрывали пол от стены до стены, на дверях висели портьеры темно-красного бархата и повсюду стояла самая новомодная мебель из черного полированного ореха с затейливой резьбой, не оставлявшей и дюйма гладкой поверхности, обитая такой скользкой тканью из конского волоса, что дамам приходилось садиться крайне осторожно, дабы не соскользнуть на пол. Повсюду висели зеркала в золоченых рамах и стояли трюмо — такое множество только в заведении Красотки Уотлинг и можно увидеть, небрежно заметил как-то Ретт. Промежутки между ними заполняли офорты в тяжелых рамах — иные футов восемь длиной, — которые Скарлетт специально выписала из Нью-Йорка. Стены были оклеены дорогими темными обоями, и при высоких потолках и вишневых плюшевых портьерах на окнах, загораживавших солнечный свет, в комнатах всегда царил полумрак.
Так или иначе, дом производил ошеломляющее впечатление, и Скарлетт, ступая по мягким коврам, погружаясь в объятия пуховиков на кровати, вспоминала холодный пол и соломенные матрацы в Таре и чувствовала несказанное удовлетворение. Дом казался ей самым красивым и самым элегантно обставленным из всех, что она видела на своем веку, Ретт же сказал, что это какой-то кошмар. Однако если она счастлива — пусть радуется.
— Теперь всякий, кто о нас слова худого не слышал, войдя в этот дом, сразу поймет, что он построен на сомнительные доходы, — сказал Ретт. — Знаешь, Скарлетт, говорят, что деньги, добытые сомнительным путем, никогда ничего хорошего не приносят, так вот наш дом — подтверждение этой истины. Типичный дом спекулянта.
Но Скарлетт, переполненная гордостью и счастьем, заранее мечтая о том, какие они будут устраивать приемы, когда тут поселятся, лишь игриво ущипнула его за ухо и сказала:
— Че-пу-ха! Ишь, куда тебя понесло.
К этому времени она уже успела понять, что Ретт очень любит сбивать с нее спесь и только рад будет испортить ей удовольствие, а потому не надо обращать внимания на его колкости. Если принимать его всерьез, надо ссориться, а она вовсе не стремилась скрещивать с ним шпаги в словесных поединках, ибо никогда в таком споре не одерживала верх. Поэтому она пропускала его слова мимо ушей или старалась обратить все в шутку. Во всяком случае, какое-то время старалась.
Пока длился их медовый месяц, да и потом — почти все время, пока они жили в отеле «Нейшнл», — отношения у них были вполне дружеские. Но не успели они переехать в свой дом, как Скарлетт окружила себя новыми друзьями и между ней и Реттом начались страшные ссоры. Ссоры эти были краткими, да они и не могли долго длиться, ибо Ретт с холодным безразличием выслушивал ее запальчивые слова и, дождавшись удобного момента, наносил ей удар по самому слабому месту. Ссоры затевала Скарлетт, Ретт — никогда. Он только излагал ей без обиняков свое мнение — о ней самой, об ее действиях, об ее доме и ее новых друзьях. И некоторые его оценки были таковы, что она не могла ими пренебречь или обратить их в шутку.
Так, например, решив изменить вывеску «Универсальная лавка Кеннеди» на что-то более громкое, она попросила Ретта придумать другое название, в котором было бы слово «эмпориум». И Ретт предложил «Cаveаt Emрtorium», утверждая, что такое название соответствовало бы товарам, продаваемым в лавке. Скарлетт решила, что это звучит внушительно, и даже заказала вывеску, которую и повесила бы, не переведи ей Эшли Уилкс не без некоторого смущения, что это значит. Она пришла в ярость, а Ретт хохотал как безумный.
А потом была проблема Мамушки. Мамушка не желала отрекаться от своего мнения, что Ретт — это мул в лошадиной сбруе. Она была с Реттом учтива, но холодна. Называла его всегда «капитан Батлер» и никогда — «мистер Ретт». Она даже не присела в реверансе, когда Ретт подарил ей красную нижнюю юбку, и ни разу ее не надела. Она старалась, насколько могла, держать Эллу и Уэйда подальше от Ретта, хотя Уэйд обожал дядю Ретта и Ретт явно любил мальчика. Но вместо того чтобы убрать Мамушку из дома или обращаться с ней сухо и сурово, Ретт относился к ней с предельным уважением и был куда вежливее, чем с любой из недавних знакомых Скарлетт. Даже вежливее, чем с самой Скарлетт. Он всегда спрашивал у Мамушки разрешение взять Уэйда на прогулку, чтобы покатать на лошади, и советовался с ней, прежде чем купить Элле куклу. А Мамушка была лишь сухо вежлива с ним.
Скарлетт считала, что Ретт, как хозяин дома, должен быть требовательнее к Мамушке, но Ретт лишь смеялся и говорил, что настоящий хозяин в доме — Мамушка.
Однажды он довел Скарлетт до белого каления, холодно заметив, что ему будет очень жаль ее, когда республиканцы через несколько лет перестанут править в Джорджии и к власти снова вернутся демократы.
— Стоит демократам посадить своего губернатора и выбрать свое законодательное собрание, и все твои новые вульгарные республиканские дружки кубарем полетят отсюда — придется им снова прислуживать в барах и чистить выгребные ямы, как им на роду написано. А ты останешься ни при чем — не будет у тебя ни друзей-демократов, ни твоих дружков-республиканцев. Вот и не думай о будущем.
Скарлетт рассмеялась, и не без основания, ибо в это время Баллок надежно сидел в губернаторском кресле, двадцать семь негров заседали в законодательном собрании, а тысячи избирателей-демократов в Джорджии были лишены права голоса.
— Демократы никогда не вернутся к власти. Они только злят янки и тем самым все больше отдаляют тот день, когда могли бы вернуться. Они только попусту мелют языком да разъезжают по ночам в балахонах ку-клукс-клана.
— Вернутся. Я знаю южан. И я знаю уроженцев Джорджии. Они люди упорные и упрямые. И если для того, чтобы они могли вернуться к власти, потребуется война, они станут воевать. И если им придется покупать голоса черных, как это делали янки, они будут их покупать. И если им придется поднять из могилы десять тысяч покойников, чтобы они проголосовали, как это сделали янки, то все покойники со всех кладбищ Джорджии явятся к избирательным урнам. Дела пойдут так скверно при милостивом правлении нашего доброго друга Руфуса Баллока, что Джорджия быстро выхаркнет это вместе с блевотиной.
— Ретт, не смей так вульгарно выражаться! — воскликнула Скарлетт. — Ты так говорить, точно я буду не рада, если демократы вернутся к власти! Ты же знаешь, что это неправда! Я буду очень рада, если они вернутся. Неужели ты думаешь, мне нравится смотреть па этих солдат, которыми кишмя псе кишит и которые напоминают мне.., неужели ты думаешь, мне это нравится.., как-никак, я уроженка Джорджии! Я очень хочу, чтобы демократы вернулись к власти. Да только они но вернутся. Никогда. Ну, а если даже и вернутся, то как это может отразиться на моих друзьях? Деньги-то все равно при них останутся, верно?
— Останутся, если они сумеют их удержать. Но я очень сомневаюсь, чтобы кому-либо из них удалось больше пяти лет удержать деньги, если они будут так их тратить. Легко досталось — легко и спускается. Их деньги никогда не принесут им счастья. Как и мои деньги — тебе. Они же не сделали из тебя скаковой лошади, мой прелестнейший мул, не так ли?
Ссора, последовавшая за этими словами, длилась не один день. Скарлетт четыре дня дулась и своим молчанием явно намекала на то, что Ретт должен перед ней извиниться, а он взял и отбыл в Новый Орлеан, прихватив с собой Уэйда, несмотря на все возражения Мамушки, и пробыл там, пока приступ раздражения у Скарлетт не прошел. Она надолго запомнила то, что не сумела заставить его приползти к ней.
Однако когда он вернулся из Нового Орлеана как ни в чем не бывало, спокойный и уравновешенный, она постаралась подавить в себе злость и отодвинуть подальше все мысли об отмщении, решив, что подумает об этом потом. Ей не хотелось сейчас забивать себе голову чем-то неприятным. Хотелось радоваться в предвкушении первого приема в новом доме. Она собиралась дать большой бал — от зари до зари: уставить зимний сад пальмами, пригласить оркестр, веранды превратить в шатры и угостить таким ужином, при одной мысли о котором у нее текли слюнки. Она намеревалась пригласить на этот прием всех, кого знала в Атланте, — и всех своих старых друзей, и всех новых, и прелестных людей, с которыми познакомилась уже после возвращения из свадебного путешествия. Волнение, связанное с предстоящим приемом, изгнало из ее памяти колкости Ретта, и она была счастлива — счастлива, как никогда на протяжении многих лет.
Ах, какое это удовольствие — быть богатой! Устраивать приемы — не считать денег! Покупать самую дорогую мебель, и одежду, и еду — и не думать о счетах! До чего приятно отправлять чеки тете Полин и тете Евлалии в Чарльстон и Уиллу в Тару! Ах, до чего же завистливы и глупы люди, которые твердят, что деньги — это еще не все! И как не прав Ретт, утверждая, что деньги нисколько ее не изменили!




Скарлетт разослала приглашения всем своим друзьям и знакомым, старым и новым, даже тем, кого она не любила. Не исключила она и миссис Мерриуэзер, хотя та держалась весьма неучтиво, когда явилась к ней с визитом в отель «Нейшнл»; не исключила и миссис Элсинг, хотя та была с ней предельно холодна. Пригласила Скарлетт и миссис Мид с миссис Уайтинг, зная, что они не любят ее и она поставит их в сложное положение: ведь надеть-то им на столь изысканный вечер будет нечего. Дело в том, что новоселье у Скарлетт, или «толкучка», как это было модно тогда называть — полуприем-полубал, — намного превосходило все светские развлечения, когда-либо виденные в Атланте.
В тот вечер и в доме и на верандах, над которыми натянули полотно, полно было гостей — они пили ее пунш из шампанского, и поглощали ее пирожки и устрицы под майонезом, и танцевали под музыку оркестра, тщательно замаскированного пальмами и каучуковыми деревьями. Но не было здесь тех, кого Ретт называл «старой гвардией», никого, кроме Мелани и Эшли, тети Питти и дяди Генри, доктора Мида с супругой и дедушки Мерриуэзера.
Многие из «старой гвардии» решили было пойти на «толкучку», хоть им и не очень хотелось. Одни приняли приглашение из уважения к Мелани, другие — потому что считали себя обязанными Ретту жизнью, своей собственной или жизнью своих близких. Но за два дня до приема по Атланте прошел слух, что к Скарлетт приглашен губернатор Баллок. И «старая гвардия» тотчас поспешила высказать свое порицание: на Скарлетт посыпались карточки с выражением сожаления и вежливым отказом присутствовать на празднестве. А небольшая группа старых друзей, которые все же пришли, тотчас отбыла, весьма решительно, хоть и смущенно, как только губернатор вступил в дом.
Скарлетт была столь поражена и взбешена этими оскорблениями, что праздник уже нисколько не радовал ее. Эту изысканную «толкучку» она с такой любовью продумала, а старых друзей, которые могли бы оценить прием, пришло совсем мало, и ни одного не пришло старого врага! Когда последний гость отбыл на заре домой, она бы, наверное, закричала и заплакала, если бы не боялась, что Ретт разразится хохотом, если бы не боялась прочесть в его смеющихся черных глазах: «А ведь я тебе говорил», пусть даже он бы и не произнес ни слова Поэтому она кое-как подавила гнев и изобразила безразличие Она позволила себе взорваться лишь на другое утро при Мелани.
— Ты оскорбила меня, Мелли Уилкс, и сделала так, что Эшли и все другие оскорбили меня! Ты же знаешь, они никогда не ушли бы так рано домой, если бы не ты А я все видела! Я как раз вела к тебе губернатора Баллока, когда ты, точно заяц, кинулась вон из дома!
— Я не верила.., я просто не могла поверить, что он будет у тебя, — с удрученным видом проговорила Мелани. — Хотя все вокруг говорили…
— Все? Так, значит, все мололи языком и судачили обо мне? — с яростью воскликнула Скарлетт. — Ты, что же, хочешь сказать, что если б знала, что губернатор будет у меня, ты бы тоже не пришла?
— Да, — тихо произнесла Мелани, глядя в пол. — Дорогая моя, я просто не могла бы прийти.
— Чтоб им сгореть! Значит, ты тоже оскорбила бы меня, как все прочие?
— О, не порицай меня! — воскликнула Мелли с искренним огорчением. — Я вовсе не хотела тебя оскорбить. Ты мне все равно как сестра, дорогая моя, ты же вдова моего Чарли, и я…
Она робко положила руку на плечо Скарлетт, но Скарлетт сбросила ее, от души жалея, что не может наорать на Мелли, как в свое время орал во гневе Джералд. А Мелани спокойно выдержала ее гнев. Распрямив худенькие плечики, она смотрела в сверкающие зеленые глаза Скарлетт, и, по мере того как бежали секунды, она все больше преисполнялась чувства собственного достоинства, столь не вязавшегося с ее по-детски наивным личиком и детской фигуркой.
— Мне очень жаль, если я обидела тебя, моя дорогая, но я не считаю возможным встречаться ни с губернатором Баллоком, ни с кем-либо из республиканцев или этих подлипал. Я не стану встречаться с ними ни в твоем доме, ни в чьем-либо другом. Нет, даже если бы мне пришлось.., пришлось… — Мелани отчаянно подыскивала самое сильное слово, — …даже если бы мне пришлось проявить грубость.
— Ты что, осуждаешь моих друзей?
— Нет, дорогая, но это твои друзья, а не мои.
— Значит, ты осуждаешь меня за то, что я пригласила к себе в дом губернатора?
Хоть и загнанная в угол, Мелани твердо встретила взгляд Скарлетт.
— Дорогая моя, все, что ты делаешь, ты делаешь всегда с достаточно вескими основаниями, и я люблю тебя и верю тебе, и не мне тебя осуждать. Да и никому другому я не позволю осуждать тебя при мне. Но, Скарлетт! — Слова внезапно вырвались стремительным потоком, подгоняя друг друга, — резкие слова, а в тихом голосе зазвучала неукротимая ненависть. — Неужели ты можешь забыть, сколько горя эти люди причинили нам? Неужели можешь забыть, что дорогой наш Чарли мертв, а у Эшли подорвано здоровье и Двенадцать Дубов сожжены? Ах, Скарлетт, ты же не можешь забыть того ужасного человека со шкатулкой твоей матушки в руках, которого ты тогда застрелила! Ты не можешь забыть солдат Шермана в Таре и как они грабили — украли даже твое белье! И хотели все сжечь и даже забрать саблю моего отца! Ах, Скарлетт, ведь это же люди, которые грабили нас, и мучили, и морили голодом, а ты пригласила их на свой прием! Тех самых, что грабят нас, не дают нашим мужчинам голосовать и теперь поставили чернокожих командовать нами! Я не могу это забыть. И не забуду. И не позволю, чтобы мой Бо забыл. Я и внукам моим внушу ненависть к этим людям — и детям моих внуков, если господь позволит, чтобы я столько прожила! Да как же можешь ты, Скарлетт, такое забыть?!
Мелани умолкла, переводя дух, а Скарлетт смотрела на нее во все глаза, и гнев ее постепенно утихал — до того она была потрясена дрожавшим от возмущения голосом Мелани.
— Ты что, считаешь, что я совсем уж идиотка? — бросила она. — Конечно, я все помню! Но это уже в прошлом, Мелли. От нас зависит попытаться извлечь из жизни как можно больше — все это я и стараюсь делать. Губернатор Баллок и некоторые милые люди из числа республиканцев могут оказать нам немалую помощь, если найти к ним верный подход.
— Среди республиканцев нет милых людей, — отрезала Мелани. — И мне не нужна их помощь. И я не собираюсь извлекать из жизни как можно больше.., если дело упирается в янки.
— Силы небесные, Мелли, зачем так злиться?
— О! — воскликнула Мелли, засовестившись. — До чего же я разошлась! Скарлетт, я вовсе не хотела оскорбить твои чувства или осудить тебя. Каждый человек думает по-своему, и каждый имеет право на свое мнение. Я же люблю тебя, дорогая моя, и ты знаешь, что я тебя люблю и ничто никогда не заставит меня изменить моим чувствам. И ты тоже по-прежнему меня любишь, верно? Я не вызвала у тебя ненависти, нет? Скарлетт, я просто не вынесу, если что-то встанет между нами, — в конце-то концов, мы столько вместе вынесли! Скажи же, что все в порядке.
— Чепуха все это, Мелли, и ни к чему устраивать такую бурю в стакане воды, — пробурчала Скарлетт, но не сбросила с талии обнявшей ее руки.
— Ну вот, теперь все снова хорошо, — с довольным видом заметила Мелани и добавила мягко: — Я хочу, чтобы мы снова бывали друг у друга — как всегда, дорогая. Ты только говори, в какие дни республиканцы и подлипалы приходят к тебе, и я в эти дни буду сидеть дома.
— Мне, глубоко безразлично, придешь ты ко мне или нет, — заявила Скарлетт и, вдруг собравшись домой, поспешно надела шляпку. И тут — словно бальзам пролился на ее уязвленное тщеславие — увидела, как огорчилась Мелани.




Недели, последовавшие за первым приемом, оказались нелегкими, и Скарлетт не так-то просто было делать вид, будто ей глубоко безразлично общественное мнение. Когда никто из старых друзей, кроме Мелани, тети Питти, дяди Генри и Эшли, не появился больше у нее и она не получила приглашения ни на один из скромных приемов, которые они устраивали, — это вызвало у нее искреннее удивление и огорчение. Разве она не сделала все, чтобы забыть прошлые обиды и показать этим людям, что не питает к ним зла за их сплетни и подкусывания? Не могут же они не знать, что она, как и они, вовсе не любит губернатора Баллока, но обстоятельства требуют любезно относиться к нему. Идиоты! Если бы все постарались любезно вести себя с республиканцами, Джорджия очень быстро вышла бы из своего тяжелого положения.
Скарлетт не поняла тогда, что одним ударом навсегда разорвала ту тонкую нить, которая еще связывала ее с былыми днями, с былыми друзьями. Даже влияния Мелани было недостаточно, чтобы вновь связать эту нить. К тому же Мелани, растерянная, глубоко огорченная, хоть и по-прежнему преданная Скарлетт, и не пыталась ее связать. Даже если бы Скарлетт захотела вернуться к былым традициям, былым друзьям, теперь путь назад для нее уже был заказан. Город обратил к ней высеченное из гранита лицо. Ненависть, окружавшая правление Баллока, окружила и ее, — ненависть, в которой было мало кипения страстей, но зато много холодной непримиримости! Скарлетт перешла на сторону врага и теперь, несмотря на свое происхождение и семейные связи, попала в категорию перевертышей, поборников прав негров, предателей, республиканцев и — подлипал.
Помучившись немного, Скарлетт почувствовала, что напускное безразличие сменяется у нее безразличием подлинным. Она никогда подолгу не задумывалась над причудами человеческого поведения и никогда не позволяла себе подолгу унывать, если что-то не получалось. Вскоре она перестала тревожиться по поводу того, что думают о ней Мерриуэзеры, Элсинги, Уайтинги, Боннеллы, Миды и прочие. Главное, что Мелани заходила к ней и приводила с собой Эшли, а Эшли интересовал Скарлетт превыше всего. Найдутся и другие люди в Атланте, которые станут посещать ее вечера, другие люди, гораздо более близкие ей по своим вкусам, чем эти ограниченные старые курицы. Да стоит ей захотеть, и дом ее наполнится гостями, и эти гости будут куда интереснее, куда лучше одеты, чем чопорные, нетерпимые старые дуры, которые с таким неодобрением относятся к ней.
Все это были новички в Атланте. Одни знали Ретта, другие участвовали вместе с ним в каких-то таинственных аферах, о которых он говорил: «дела, моя кошечка». Были тут и супружеские пары, с которыми Скарлетт познакомилась, когда жила в отеле «Нейшнл», а также чиновники губернатора Баллока.
Общество, в котором вращалась теперь Скарлетт, было весьма пестрым.
Некие Гелерты, побывавшие уже в десятке разных штатов и, судя по всему, поспешно покидавшие каждый, когда выяснялось, в каких мошенничествах они были замешаны; некие Коннингтоны, неплохо нажившиеся в Бюро вольных людей одного отдаленного штата за счет невежественных черных, чьи интересы они, судя по всему, должны были защищать; Дилы, продававшие сапоги на картонной подошве правительству конфедератов и вынужденные потом провести последний год войны в Европе; Хандоны, на которых были заведены досье полицией многих городов и которые тем не менее с успехом не раз получали контракты от штата; Караханы, заложившие основу своего состояния в игорном доме, а теперь рассчитывавшие на более крупный куш, затеяв на бумаге строительство несуществующей железной дороги на деньги штата; Флэгерти, закупившие в 1861 году соль по центу за фунт и нажившие состояние, продавая ее в 1863 году по пятьдесят центов за фунт; и Барты, владевшие самым крупным домом терпимости в северной столице во время войны, а сейчас вращавшиеся в высших кругах «саквояжников».
Такими друзьями окружила себя теперь Скарлетт, но среди тех, кто посещал ее большие приемы, были и люди интеллигентные, утонченные, многие — из превосходных семей. Помимо сливок «саквояжников», в Атланту переселялись с Севера и люди более солидные, привлеченные городом, в котором не прекращалась бурная деловая жизнь в этот период восстановления и переустройства. Богатые семьи янки посылали своих сыновей на Юг для освоения новых мест, а офицеры-янки после выхода в отставку навсегда поселялись в городе, которым они с таким трудом сумели овладеть. Чужие в чужом городе, они поначалу охотно принимали приглашения на роскошные балы богатой и гостеприимной миссис Батлер, но очень скоро покинули круг ее друзей. Это были в общем-то люди порядочные, и им достаточно было короткого знакомства с «саквояжниками» и их нравами, чтобы относиться к ним так же, как уроженцы Джорджии. Многие из этих пришельцев стали демократами и в большей мере южанами, чем сами южане.
Другие переселенцы остались среди друзей Скарлетт только потому, что их нигде больше не принимали. Они бы охотно предпочли тихие гостиные «старой гвардии», но «старая гвардия» не желала с ними знаться. К числу таких людей относились наставницы-янки, отправившиеся на Юг, горя желанием просветить негров, а также подлипалы, родившиеся добрыми демократами, но перешедшие на сторону республиканцев после поражения.
Трудно сказать, кого больше ненавидели коренные горожане — непрактичных наставниц-янки или подлипал, но пожалуй, последние перетягивали чашу весов. Наставниц можно было сбросить со счета: «Ну, чего можно ждать от этих янки, которые обожают негров? Они, конечно, считают, что негры ничуть не хуже их самих!» А вот тем уроженцам Джорджии, которые стали республиканцами выгоды ради, уже не было оправдания.
«Мы ведь смирились с голодом. Вы тоже могли бы смириться» — так считала «старая гвардия». Многие же бывшие солдаты Конфедерации, видевшие, как страдают люди, сознавая, что их семьи нуждаются, куда терпимее относились к бывшим товарищам по оружию, сменившим политические симпатии, чтобы прокормить семью. Но ни одна дама из «старой гвардии» не могла этого простить — то была неумолимая и непреклонная сила, являвшаяся опорой определенного порядка вещей. Идеи Правого Дела были для них сейчас важнее и дороже, чем в пору его торжества. Эти идеи превратились в фетиш. Все связанное с ними было священно: могилы тех, кто отдал Делу жизнь; поля сражений; разодранные знамена; висящие в холлах крест-накрест сабли; выцветшие письма с фронта; ветераны. «Старая гвардия» не оказывала помощи бывшим врагам, не проявляла к ним сочувствия и не давала им приюта, а теперь к этим врагам причислили и Скарлетт.
В разношерстном обществе, образовавшемся под влиянием политической обстановки, всех объединяло лишь одно. Деньги» У многих до войны ни разу не было и двадцати пяти долларов в кармане, и теперь они пустились в такое расточительство, какого Атланта еще не знала.
С приходом к власти республиканцев город вступил в эру неслыханного мотовства и бахвальства своим богатством, когда внешняя благопристойность поведения лишь слабо прикрывала пороки и пошлость. Никогда еще граница между очень богатыми и очень бедными не пролегала так четко. Те, кто был наверху, нимало не заботились о тех, кому меньше повезло в жизни. Исключение составляли лишь негры. Вот им старались дать что получше. Хорошие школы, и жилища, и одежду, и развлечения, ибо негры представляли собой политическую силу и каждый негритянский голос был на учете. Что же до недавно обедневших жителей Атланты, они могли падать на улице от голода — недавно разбогатевшим республиканцам было вес равно.
На волне этой пошлости победоносно плыла и Скарлетт, молодая жена Ретта, прочно обеспеченная его деньгами, ослепительно хорошенькая в своих красивых нарядах. Настали времена, отвечавшие духу Скарлетт, — времена разнузданной, кричащей безвкусицы, пышно разодетых женщин, пышно обставленных домов, изобилия драгоценностей, лошадей, еды, виски. Когда Скарлетт — что случалось нечасто — задумывалась над этим, она понимала, что ни одна из ее новых знакомых не могла бы называться «леди» по строгим критериям Эллин. Но она уже не раз нарушала принципы Эллин после того далекого дня, когда, стоя в гостиной Тары, решила стать любовницей Ретта, и нельзя сказать, чтобы теперь ее часто мучила из-за этого совесть.
Возможно, эти ее новые друзья и не были, строго говоря, леди и джентльменами, но, как и с новоорлеанскими друзьями Ретта, с ними было так весело! Намного веселее, чем со смиренными, богобоязненными поклонниками Шекспира — ее прежними друзьями в Атланте. А если не считать краткого медового месяца, она ведь так давно не веселилась. И так давно не чувствовала себя в безопасности. Теперь же, когда она познала это чувство, ей хотелось танцевать, играть, вдоволь есть и пить, одеваться в шелка и атлас, спать на пуховой постели, сидеть на мягких диванах. И всему этому она отдавала дань. Поощряемая снисходительностью Ретта, — а он только забавлялся, глядя на нее, — освободившись от запретов, сковывавших ее в юности, освободившись даже от недавно владевшего ею страха перед бедностью, она позволяла себе роскошь, о которой давно мечтала, — роскошь поступать так, как хочется, и посылать к черту всех, кому это не по душе.
Она познала приятное опьянение, какое бывает у того, кто своим образом жизни бросает вызов благопристойному обществу, — у игрока, мошенника, авантюриста, — словом, у всех, кто процветает за счет хитрости и изворотливости ума. Она говорила и делала что хотела и скоро в своей наглости переступила все границы.
Она, не задумываясь, дерзила своим новым друзьям — республиканцам и подлипалам, но ни с кем не держалась так грубо или так вызывающе, как с гарнизонными офицерами-янки и их семьями. Из всей разнородной массы, прихлынувшей в Атланту, она не желала терпеть и принимать у себя лишь военных. Она даже всячески изощрялась, чтобы попренебрежительнее обойтись с ними. Не одна Мелани не могла забыть, что значил синий мундир. Этот мундир с золотыми пуговицами всегда воскрешал в памяти Скарлетт страхи, пережитые во время осады, ужасы бегства, грабежи и пожары, страшную бедность и невероятно тяжелый труд в Таре. Теперь, став богатой, сознавая, что ей многое позволено благодаря дружбе с губернатором и разными влиятельными республиканцами, она могла вести себя резко и грубо с любым синим мундиром, который встречался на ее пути. И она была резка и груба.
Однажды Ретт как бы между прочим заметил, что большинство мужчин, которые приходят к ней в гости, еще совсем недавно носили те же синие мундиры, но она возразила, что янки для нее лишь тогда янки, когда на них синий мундир. Ретт сказал: «Последовательность — редкая драгоценность», — и пожал плечами.
Ненавидя синие мундиры, Скарлетт любила задирать тех, кто их носил, и получала тем больше удовольствия, чем больше озадачивала своим поведением янки. Офицеры гарнизона и их семьи имели право удивляться, ибо это были, как правило, спокойные, воспитанные люди, которые жили одиноко во враждебном краю, жаждали вернуться к себе на Север и немного стыдились того, что вынуждены поддерживать правление всяких подонков, — словом, это были люди куда более достойные, чем те, с кем общалась Скарлетт. Жен офицеров, естественно, озадачивало то, что ослепительная миссис Батлер пригрела у себя эту вульгарную рыжую Бриджет Флэгерти, а их всячески оскорбляла.
Впрочем, даже и тем, кого привечала Скарлетт, приходилось немало от нее терпеть. Однако они охотно терпели. Для них он, была олицетворением не только богатства и элегантности, но и старого мира с его старинными именами, старинными семьями, старинными традициями, — мира, к которому они так жаждали приобщиться. Старинные семьи, с которыми они мечтали познакомиться, возможно, и знаться со Скарлетт не желали, но дамы из новой аристократии понятия об этом не имели. Они знали лишь, что отец Скарлетт владел большим количеством рабов, ее мать была из саваннских Робийяров, а ее муж — Ретт Батлер из Чарльстона. И этого было для них достаточно. Скарлетт открывала им путь в старое общество, куда они стремились проникнуть, — общество тех, кто их презирал, не отдавал визитов и сухо раскланивался в церкви. В сущности, Скарлетт не только открывала им путь в общество. Для них, делавших лишь первые шаги из безвестности, она уже была обществом. Дутые аристократки, они не видели — как, кстати, и сама Скарлетт, — что она такая же дутая аристократка. Они мерили ее той меркой, какой она сама мерила себя, и немало от нее терпели, смиряясь с ее высокомерием, ее манерами, ее вспышками раздражения, с ее наглостью и с откровенной, неприкрытой грубостью ее замечаний, если они совершали оплошность.
Они так недавно стали кем-то из ничего и были еще так неуверенны в себе, что отчаянно боялись показаться недостаточно рафинированными, дать волю своему нраву или резко ответить: а вдруг подумают, что они вовсе и не леди. Им же во что бы то ни стало хотелось быть леди. Вот они и строили из себя этаких деликатных, скромных наивных дам. Послушать их, можно было подумать, что они бесплотны, не отправляют естественных нужд и понятия не имеют об этом порочном мире. Никому бы в голову не пришло, что рыжая Бриджет Флэгерти, чья белая кожа оставалась белой, невзирая на яркое солнце, а ирландский акцент был густым, как патока, украла сбережения своего отца, чтобы приехать в Америку, где стала горничной в нью-йоркском отеле. А глядя на хрупкую восторженную Сильвию Коннингтон (бывшую Красотку Сэйди) и на Мэйми Барт, никто бы не заподозрил, что первая выросла на Бауэри над салуном своего отца и во время наплыва клиентов помогала в баре, а вторая, судя по слухам, подвизалась прежде в одном из публичных домов своего мужа. О нет, теперь это были нежные, хрупкие создания.
Мужчины же хоть и нажили деньги, однако не так быстро обучились манерам, а возможно, просто поплевывали на требования новой знати. Они много пили на вечерах у Скарлетт — даже слишком много, — и в результате после приема гость-другой неизменно оставался на ночь. Пили они совсем иначе, чем те мужчины, которых знала Скарлетт в юности. Они становились отталкивающими, глупыми, отвратительными сквернословами. А кроме того, сколько бы она ни ставила плевательниц у всех на виду, наутро после приема ковры неизменно изобиловали следами от табачной жвачки.
Скарлетт презирала этих людей и в то же время получала удовольствие от общения с ними. И поскольку она получала удовольствие, то и наполняла ими дом. А поскольку она их презирала, то без стеснения посылала к черту, как только они начинали ее раздражать. Но они со всем мирились.
Они мирились даже с ее супругом, что было куда труднее, ибо Ретт видел их насквозь и они это знали. Он, не задумываясь, мог, что называется, раздеть их догола даже в своем доме — да так, что им и отвечать было нечего. Не стесняясь того, какими путями он сам пришел к богатству, он делал вид, будто думает, что и они не стесняются своих корней, и потому при любой возможности касался таких предметов, которые, по общему мнению, лучше было вежливо обходить молчанием.
Никто не мог предвидеть, когда ему вздумается весело бросить за кружкой пунша: «Ральф, будь я поумнее, я нажил бы состояние, как ты, — продавая акции золотых приисков вдовам и сиротам, вместо того чтобы прорывать блокаду. Оно куда безопаснее».
Или: «Эй, Билл, я смотрю, у тебя новые лошади появились. Продал еще несколько тысчонок акций несуществующих железных дорог? Хорошо работаешь, мальчик!» Или: «Поздравляю, Эймос, с получением контракта от штата. Жаль только, что тебе пришлось столько народу подмазать, чтоб добиться его».
Дамы считали Ретта отвратительно, невыносимо вульгарным. Мужчины за его спиной говорили, что он свинья и мерзавец. Словом, новая Атланта любила Ретта не больше, чем старая, а он, как и прежде, даже не пытался наладить с ней отношения. Он следовал своим путем, забавляясь, всех презирая, глухой к претензиям окружающих, настолько подчеркнуто любезный, что сама любезность его выглядела как вызов. Для Скарлетт он по-прежнему являлся загадкой, но загадкой, над которой она больше не ломала голову. Она была убеждена, что ему ничем и никогда не потрафить: он либо очень чего-то хотел, но не мог получить, либо вообще ничего не хотел и плевал на все. Он смеялся над любыми ее начинаниями, поощрял ее расточительность и высокомерие, глумился над ее претензиями и… платил по счетам.

