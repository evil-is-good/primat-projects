\chapter{\ }

Ретт всегда держался со Скарлетт спокойно, бесстрастно — даже в самые интимные минуты. Но Скарлетт никогда не покидало давно укоренившееся чувство, что он исподтишка наблюдает за ней: она знала, что стоит ей внезапно повернуть голову, и она обнаружит в его глазах это задумчивое, настороженно-выжидательное выражение, которое она не могла объяснить. Выражение, исполненное поистине безграничного терпения.
Порой ей было очень уютно с ним, несмотря на одно злополучное свойство его характера — он не терпел в своем присутствии никакой лжи, никакой претенциозности или бахвальства. Он слушал ее рассказы о лавке, о лесопилках, о салуне, о каторжниках и о том, сколько стоит их прокормить, и давал ей дельные практические советы. Он с неутомимой энергией участвовал в танцах и в вечеринках, которые она так любила, и располагал бесконечным запасом не слишком пристойных историй, которыми угощал ее, когда они изредка проводили вечера вдвоем, после того, как со стола была убрана еда и перед ними появлялся кофе с коньяком. Она обнаружила, что если быть с ним прямой и откровенной, он даст ей все, чего бы она ни пожелала, ответит на любой ее вопрос, но окольным путем и женскими хитростями она ничего от него не добьется. Он обезоруживал ее тем, что видел насквозь и, разгадав ее уловки, открыто смеялся.
Наблюдая это мягкое безразличие, с каким он обычно относился к ней, Скарлетт нередко удивлялась — впрочем, без особого любопытства, — почему он женился на ней. Мужчины женятся по любви, или ради того, чтобы завести дом и детей, или ради денег, но она знала, что Ретт женился на ней не поэтому. Он, конечно, ее не любил. Ее прелестный дом он называл архитектурным кошмаром и говорил, что предпочел бы жить в хорошо обставленной гостинице. И ни разу не намекнул, что хотел бы иметь от нее ребенка, как делали в свое время Чарльз и Фрэнк. Однажды она спросила кокетства ради, почему он женился на ней, и пришла в ярость, услышав ответ да еще увидев в его глазах веселые искорки: «Я женился на вас, чтобы держать вместо кошки, моя дорогая».
Да, он женился на ней вовсе не по тем причинам, которые обычно побуждают мужчину жениться. Он женился только потому, что хотел обладать ею, а другим путем не мог этого добиться. Ведь так он и сказал в тот вечер, когда сделал ей предложение. Он хотел обладать ею, как в свое время хотел обладать Красоткой Уотлинг. Эта мысль была ей не очень приятна. Собственно, она была просто оскорбительна. Но Скарлетт быстро выкинула ее из головы, как научилась выкидывать из головы все неприятное. Они заключили сделку, и она со своей стороны была вполне довольна этой сделкой. Она надеялась, что доволен и он, а в общем-то ей это было безразлично.
Но вот однажды, советуясь с доктором Мидом по поводу расстройства желудка, она услышала неприятное известие, от которого было уже не отмахнуться. Вернувшись в сумерки домой, она ворвалась к себе в спальню и, глядя на Ретта ненавидящими глазами, сообщила, что у нее будет ребенок.
Он сидел в шелковом халате, окруженный облаком табачного дыма, и тотчас вскинул па нее глаза. Однако не произнес ни слова. Он молча смотрел на нее, ожидая, что она скажет дальше, — только поза его стала напряженной, но она этого не заметила. Ею владело такое возмущение и отчаяние, что она ни о чем другом и думать не могла.
— Ты же знаешь, что я не хочу больше иметь детей! Я вообще их не хотела. Стоит моей жизни наладиться, как у меня появляется ребенок. Ах, да не сиди ты так и не смейся надо мной: ты ведь тоже не хочешь ребенка. Ах, мать пресвятая богородица!
Если он ждал от нее каких-то слов, то, во всяком случае, не этих. В лице его появилась жесткость, глаза стали пустыми.
— Ну, в таком случае почему бы не отдать его мисс Мелли? Разве ты не говорила мне: она такая непрактичная, хочет еще одного ребенка?
— Ох, так бы и убила тебя! Я не хочу его, говорю тебе: не хочу!
— Нет? Прошу, продолжай дальше.
— Ах, но есть же способы избавиться. Я ведь уже не та глупая деревенская девчонка, какой была когда-то. Теперь я знаю, что женщине вовсе не обязательно иметь детей, если она не хочет! Есть способы…
Он вскочил и схватил ее за руку — в лице его был неприкрытый всепоглощающий страх.
— Скарлетт, дурочка ты этакая, скажи мне правду! Ты ничего с собой не сделала?
— Нет, не сделала, но сделаю. Ты что думаешь, я допущу, чтоб у меня снова испортилась фигура — как рая когда я добилась, что талия у меня стала топкая, и я так весело провожу время, и…
— Откуда ты узнала, что это возможно? Кто тебе сказал?
— Мэйми Барт.., она…
— Еще бы владелице борделя не знать про всякие такие штуки. Ноги этой женщины больше не будет в нашем доме, ясно? В конце концов, это мой дом, и я в нем хозяин. Я не желаю, чтобы ты когда-либо упоминала о ней.
— Я буду делать то, что хочу. Отпусти меня. Да и вообще — тебе-то не все ли равно?
— Мне все равно, будет у тебя один ребенок или двадцать, но мне не все равно, если ты умрешь.
— Умру? Я?
— Да, умрешь. Мэйми Карт, видимо, не рассказала тебе, чем рискует женщина, идя на такое?
— Нет, — нехотя призналась Скарлетт. — Она просто сказала, что все отлично устроится.
— Клянусь богом, я убью ее! — воскликнул Ретт, и лицо его потемнело от гнева. Он посмотрел сверху на заплаканную Скарлетт, я гнев его поутих, но лицо по-прежнему оставалось жестким и замкнутым. Внезапно он подхватил Скарлетт па руки и, опустившись со своей ношей в кресло, крепко прижал к себе, словно боялся, что она убежит.
— Послушай, детка, я не позволю тебе распоряжаться твоей жизнью. Слышишь? Боже правый, я тоже, как и ты, не хочу иметь детей, но я могу их вырастить. Так что хватит болтать о всяких глупостях — я не хочу об этом слышать, и если ты только попытаешься… Скарлетт, я видел, как одна женщина умерла от этого. Она была всего лишь.., ну, в общем, вовсе неплохая была женщина. И смерть это нелегкая. Я…
— Боже мой, Ретт! — воскликнула она, забыв о своем горе под влиянием волнения, звучавшего в его голосе. Она никогда еще не видела его столь взволнованным. — Где же.., кто…
— Это было в Новом Орлеане, много лет назад. Я был молод и впечатлителен. — Он вдруг пригнул голову и зарылся губами в ее полосы. — Ты родишь этого ребенка, Скарлетт, даже если бы мне пришлось надеть на тебя наручники и приковать к себе на ближайшие девять месяцев. Не слезая с его колен, она выпрямилась и с откровенным любопытством уставилась па него. Под се взглядом лицо его, словно по мановению волшебной палочки, разгладилось и стало непроницаемым. Брови приподнялись, уголки губ поползли вниз.
— Неужели я так много для тебя значу? — спросила она, опуская ресницы.
Он внимательно посмотрел на псе, словно хотел разгадать, что таится под этим вопросом — кокетство или что-то большее. И, отыскав ключ к ее поведению, небрежно ответил:
— В общем, да. Ведь я вложил в тебя столько денег — мне не хотелось бы их потерять.
Мелани вышла из комнаты Скарлетт усталая и в то же время до слез счастливая: у Скарлетт родилась дочь. Ретт, ни жив ни мертв, стоял в холле, у ног его валялись окурки сигар, прожегшие дырочки в дорогом ковре.
— Теперь вы можете войти, капитан Батлер, — застенчиво сказала Мелани.
Ретт быстро прошел мимо нее в комнату, и Мелани, прежде чем доктор Мид закрыл дверь, успела увидеть, как он склонился над голенькой малюткой, которая лежала на коленях у Мамушки.
Невольно став свидетельницей столь интимной сцены, Мелани покраснела от смущения.
«О, какой же он славный, капитан Батлер! — подумала она, опускаясь в кресло. — Как он беспокоился, бедняга! И за все время капли не выпил! До чего же это мило с его стороны. Ведь многие джентльмены к тому времени, когда рождается ребенок, уже еле на ногах держатся. А ему, думается, очень не мешало бы выпить. Может быть, предложить? Нет, это было бы слишком невоспитанно с моей стороны».
Она с наслаждением откинулась в кресле: последние дни у нее все время болела спина, так что казалось, будто она вот-вот переломится в пояснице. Какая счастливица Скарлетт: ведь капитан Батлер все время стоял под дверью спальни, пока она рожала! Если бы в тот страшный день, когда она производила на свет Бо, рядом был Эшли, она наверняка бы меньше страдала. Как было бы хорошо, если бы крошечная девочка, лежавшая за этими закрытыми дверями, была ее дочерью, а не дочерью Скарлетт! «Ах, какая же я скверная, — виновато подумала Мелани. — Я завидую Скарлетт, а ведь она всегда была так добра ко мне. Прости меня, господи. Я, право же, вовсе не хочу отбирать у Скарлетт дочку, но.., но мне бы так хотелось иметь собственную!» Она подложила подушечку под нывшую спину и принялась думать о том, как было бы хорошо иметь дочку. Но доктор Мид на этот счет продолжал держаться прежнего мнения. И хотя она сама готова была рисковать жизнью, лишь бы родить еще ребенка, Эшли и слышать об этом не хотел. Дочка… Как порадовался бы Эшли дочке!
«Дочка!.. Смилуйся, господи! — Мелани в волнении выпрямилась. — Я ведь не сказала капитану Батлеру, что это девочка! А он, конечно, ждет мальчика. Ах, какая незадача!» Мелани знала, что мать радуется появлению любого ребенка, но для мужчины, особенно для такого самолюбивого человека, как капитан Батлер, девочка — это удар, ставящий под сомнение его мужественность. О, как она благодарна была господу за то, что ее единственное дитя — мальчик! Она знала, что, будь она женой грозного капитана Батлера, она предпочла бы умереть в родах, лишь бы не дарить ему первой дочь.
Но Мамушка вышла враскачку из спальни, улыбаясь во весь рот, и Мелани успокоилась, однако в то же время и подивилась: что за странный человек этот капитан Батлер.
— Я сейчас, как стала купать дите-то, — принялась рассказывать Мамушка, — ну, и сказала мистеру Ретту — жаль, мол, что не мальчик у вас. И господи, знаете, мисс Мелли, что он сказал? Говорит: «Перестань болтать. Мамушка! Кому нужен мальчик? С мальчиками — никакого интереса. Одни только хлопоты. А с девочками — оно интересно. Да я эту девочку на десяток мальчишек не променяю». И тут хотел было выхватить у меня крошку-то, а девочка-то голенькая, ну я и ударила его по руке и говорю: «Ведите себя прилично, мистер Ретт! А уж я доживу до того времени, как у вас сынок-то родится, и тогда вдоволь посмеюсь над вами — ведь заголосите от радости-то». А он эдак усмехнулся, покачал головой и говорит: «Мамушка, ты совсем глупая. Никому мальчишки не нужны. Разве я тому не доказательство?» Так что вот, мисс Мелли, вел он себя как настоящий жентмун, — снизошла до похвалы Мамушка.
И Мелани, естественно, не могла не заметить, что такое поведение Ретта существенно обелило его в глазах Мамушки.
— Может, я была чуток и не права насчет мистера Ретта-то. Очень это для меня, мисс Мелли, сегодня счастливый день. Я ведь три поколения робийяровских девочек вынянчила, так что очень это для меня счастливый день.
— Конечно, это счастливый день, Мамушка! Когда родятся дети, это самые счастливые дни.
Но было в доме существо, которому этот день вовсе не представлялся счастливым. Уэйд Хэмптон, которого все ругали, а по большей части не замечали, с несчастным видом бродил по столовой. Утром Мамушка разбудила его очень рано, быстро одела и отослала вместе с Эллой к тете Питти завтракать. Ему сказали только, что мама заболела, а когда он играет и шумит, это ее нервирует. Однако в Доме у тети Питти все было вверх дном, ибо известие о болезни Скарлетт тотчас уложило старушку в постель, кухарка танцевала возле нее, и завтраком детей кормил Питер, так что поели они плохо. Время стало приближаться к полудню, и в душу Уэйда начал закрадываться страх. А что, если мама умрет? Ведь у других мальчиков умирали мамы. Он видел, как от домов отъезжали катафалки, слышал, как рыдали его маленькие приятели. Что, если и его мама умрет? Уэйд очень любил свою маму — почти так же сильно, как и боялся, — и при мысли о том, что ее повезут на черном катафалке, запряженном черными лошадьми с перьями на голове, его маленькая грудка разрывалась от боли, так что ему даже трудно было дышать.
И когда настал полдень, л Питер был занят на кухне, Уэйд выскользнул из парадной двери и побежал домой со всей быстротой, на какую были способны его короткие ножки, — страх подстегивал его. Дядя Ретт, или тетя Мелли, или Мамушка, уж конечно, скажут ему правду. Но дяди Ретта и тети Мелли нигде не было видно, а Мамушка и Дилси бегали вверх и вниз по лестнице с полотенцами и тазами с горячей водой и не заметили его в холле. Сверху, когда открывалась дверь в комнату мамы, до мальчика долетали отрывистые слова доктора Мида. В какой-то момент он услышал как застонала мама, и разрыдался так, что у него началась икота. Теперь от твердо знал, что она умрет. Чтобы немножко утешиться, он принялся гладить медово-желтого кота, который лежал на залитом солнцем подоконнике в холле. Но Том, отягощенный годами и не любивший, чтобы его беспокоили, махнул хвостом и фыркнул на мальчика.
Наконец появилась Мамушка — спускаясь по парадной лестнице в мятом, перепачканном переднике и съехавшем набок платке, она увидела Уэйда и насупилась. Мамушка всегда была главной опорой Уэйда, и он задрожал, увидев ее хмурое лицо.
— Вот уж отродясь не видала таких плохих мальчиков, как вы, — сказала Мамушка. — Я же отослала вас к мисс Питти! Сейчас же отправляйтесь назад!
— А мама.., мама умрет?
— Вот уж отродясь не видала таких настырных детей. Умрет?! Господи, господи, нет, конечно! Ну, и докука эти мальчишки. И зачем только господь бог посылает людям мальчишек! А ну, уходите отсюда.
Но Уэйд не ушел. Он спрятался за портьерами в холле, потому что заверение Мамушки лишь наполовину успокоило его. А ее слова про плохих мальчишек показались обидными, ибо он всегда старался быть хорошим мальчиком. Через полчаса тетя Мелли сбежала по лестнице, бледная и усталая, но улыбающаяся. Она чуть не упала в обморок, увидев в складках портьеры скорбное личико Уэйда. Обычно у тети Мелли всегда находилось для него время. Она никогда не говорила, как мама: «Не докучай мне сейчас. Я спешу». Или: «Беги, беги, Уэйд. Я занята».
Но на этот раз тетя Мелли сказала:
— Какой ты непослушный, Уэйд. Почему ты не остался у тети Питти?
— А мама умрет?
— Великий боже, нет, Уэйд! Не будь глупым мальчиком. — И смягчившись, добавила: — Доктор Мид только что принес ей хорошенького маленького ребеночка — прелестную сестричку, с которой тебе разрешат играть, и если ты будешь хорошо себя вести, то тебе покажут ее сегодня вечером. А сейчас беги играй и не шуми.
Уэйд проскользнул в тихую столовую — его маленький ненадежный мирок зашатался и вот-вот готов был рухнуть. Неужели в этот солнечный день, когда взрослые ведут себя так странно, семилетнему мальчику негде укрыться, чтобы пережить свои тревоги? Он сел на подоконник в нише и принялся жевать бегонию, которая росла в ящике на солнце. Бегония оказалась такой горькой, что у него на глазах выступили слезы и он заплакал. Мама, наверное, умирает, никто не обращает на него внимания, а все только бегают туда-сюда, потому что появился новый ребенок — какая-то девочка. А Уэйда не интересовали младенцы, тем более девчонки. Единственной девочкой, которую он знал, была Элла, а она пока ничем не заслужила ни его уважения, ни любви.
Он долго сидел так один; потом доктор Мид и дядя Ретт спустились по лестнице, остановились в холле и тихо о чем-то заговорили. Когда дверь за доктором закрылась, дядя Ретт быстро вошел в столовую, налил себе большую рюмку из графина и только тут увидел Уэйда. Уэйд юркнул было за портьеру, ожидая, что ему сейчас снова скажут, что он плохо себя ведет, и велят возвращаться к тете Питти, но дядя Ретт ничего такого не сказал, а, наоборот, улыбнулся. Уэйд никогда еще не видел, чтобы дядя Регт так улыбался или выглядел таким счастливым, а потому, расхрабрившись, соскочил с подоконника и кинулся к нему.
— У тебя теперь будет сестренка, — сказал Ретт, подхватывая его на руки. — Ей-богу, необыкновенная красотка! Ну, а почему же ты плачешь?
— Мама…
— Твоя мама сейчас ужинает — уплетает за обе щеки: и курицу с рисом и с подливкой, и кофе, а немного погодя мы ей сделаем мороженое и тебе дадим двойную порцию, если захочешь. И я покажу тебе сестричку.
Уэйду сразу стало легко-легко, и он решил быть вежливым и хотя бы спросить про сестричку, но слова не шли с языка. Все интересуются только этой девчонкой. А о нем никто и не думает — ни тетя Мелли, ни дядя Ретт.
— Дядя Ретт, — спросил он тогда, — а что, девочек больше любят, чем мальчиков?
Ретт поставил рюмку, внимательно вгляделся в маленькое личико и сразу все понял.
— Нет, я бы этого не сказал, — с самым серьезным видом ответил он, словно тщательно взвесил вопрос Уэйда. — Просто с девочками больше хлопот, чем с мальчиками, а люди склонны больше волноваться о тех, кто доставляет им заботы, чем об остальных.
— А вот Мамушка сказала, что мальчики доставляют много хлопот.
— Ну, Мамушка была просто не в себе. На самом деле она вовсе так не думает.
— Дядя Ретт, а вы бы не хотели иметь мальчика вместо девочки? — с надеждой спросил Уэйд.
— Нет, — поспешил ответить Ретт и, видя, как сразу сник Уэйд, добавил: — Ну, зачем же мне нужен мальчик, если у меня уже есть один?
— Есть? — воскликнул Уэйд и от изумления раскрыл рот. — А где же он?
— Да вот тут, — ответил Ретт и, подхватив ребенка, посадил к себе на колено. — Ты же ведь мой мальчик, сынок.
На мгновение сознание, что его оберегают, что он кому-то нужен, овладело Уэйдом с такою силой, что он чуть снова не заплакал. Он судорожно глотнул и уткнулся головой Ретту в жилет.
— Ты же мой мальчик, верно?
— А разве можно быть.., ну, сыном двух людей сразу? — спросил Уэйд: преданность отцу, которого он никогда не знал, боролась в нем с любовью к человеку, с таким пониманием относившемуся к нему.
— Да, — твердо сказал Ретт. — Так же, как ты можешь быть сыном своей мамы и тети Мелли.
Уэйд обдумал эти слова. Они были ему понятны, и, улыбнувшись, он робко потерся спиной об лежавшую на ней руку Ретта.
— Вы понимаете маленьких мальчиков, верно, дядя Ретт? Смуглое лицо Ретта снова прорезали резкие морщины, губы изогнулись в усмешке.
— Да, — с горечью молвил он, — я понимаю маленьких мальчиков. На секунду к Уэйду вернулся страх — страх и непонятное чувство ревности. Дядя Ретт думал сейчас не о нем, а о ком-то другом.
— Но у вас ведь нет других мальчиков, верно? Ретт спустил его на пол.
— Я сейчас выпью, и ты тоже, Уэйд, поднимешь свой первый тост за сестричку.
— У вас ведь нет других… — не отступался Уэйд, но, увидев, что Ретт протянул руку к графину с кларетом, не докончил фразы, возбужденный перспективой приобщения к миру взрослых. — Ох, нет, не могу я, дядя Ретт! Я обещал тете Мелли, что не буду пить, пока не кончу университета, и если я сдержу слово, она подарит мне часы.
— А я подарю тебе для них цепочку — вот эту, которая на мне сейчас, если она тебе понравится, — сказал Ретт и снова улыбнулся. — Тетя Мелли совершенно права. Но она говорила о водке, не о вине. Ты же должен научиться пить вино, как подобает джентльмену, сынок, и сейчас самое время начать обучение.
Он разбавил кларет водой из графина, пока жидкость не стала светло-розовая, и протянул рюмку Уэйду. В этот момент в столовую вошла Мамушка. На ней было ее лучшее воскресное черное платье и передник; на голове — свеженакрахмаленная косынка. Мамушка шла, покачивая бедрами, сопровождаемая шепотом и шорохом шелковых юбок. С лица ее исчезло встревоженное выражение, почти беззубый рот широко улыбался.
— С новорожденной вас, мистер Ретт! — сказала она. Уэйд замер, не донеся рюмки до рта. Он знал, что Мамушка не любит его отчима. Она никогда не называла его иначе как «капитан Батлер» и держалась с ним вежливо, но холодно. А тут она улыбалась ему во весь рот, пританцовывала да к тому же назвала «мистер Ретт»! Ну и день — все вверх дном!
— Я думаю, вам лучше налить рому, чем кларету, — сказал Ретт и, приоткрыв погребец, вытащил оттуда квадратную бутылку. — Хорошенькая она у нас, верно, Мамушка?
— Да уж куда лучше. — ответила Мамушка и, причмокнув, взяла рюмку.
— Вы когда-нибудь видели ребенка красивее?
— Ну, мисс Скарлетт, когда появилась на свет, уж конечно, была прехорошенькая, а все не такая.
— Выпейте еще рюмочку. Мамушка. Кстати, Мамушка, — это было произнесено самым серьезным тоном, но в глазах Ретта плясали бесенята, — что это так шуршит?
— О господи, мистер Ретт, да что же еще, как не моя красная шелковая юбка! — Мамушка хихикнула и качнула бедрами, так что заколыхался весь ее могучий торс.
— Значит, это ваша нижняя юбка! Вот уж никогда бы не поверил. Так шуршит, будто ворох сухих листьев переворачивают. Ну-ка, дайте взглянуть. Приподнимите подол.
— Нехорошо это, мистер Ретт! О господи! — слегка взвизгнула Мамушка и, отступив на ярд, скромно приподняла на несколько дюймов подол и показала оборку нижней юбки из красной тафты.
— Долго же вы раздумывали, прежде чем ее надеть, — буркнул Ретт, но черные глаза его смеялись и в них поблескивали огоньки.
— Да уж, сэр, слишком даже долго.
Тут Ретт сказал нечто такое, чего Уэйд не понял:
— Значит, с мулом в лошадиной сбруе покончено?
— Мистер Ретт, негоже это, что мисс Скарлетт сказала вам! Но вы не станете сердиться на бедную старую негритянку?
— Нет, не стану. Я просто хотел знать. Выпейте еще, Мамушка. Берите хоть всю бутылку. И ты тоже пей, Уэйд! Ну-ка, произнеси тост.
— За сестренку! — воскликнул Уэйд и одним духом осушил спою рюмку. И задохнулся, закашлялся, начал икать, а Ретт и Мамушка смеялись и шлепали его по спине.




С той минуты, как у Ретта родилась дочь, он повел себя настолько неожиданно для всех, кто имел возможность его наблюдать, что перевернул все установившиеся о нем представления, — представления, от которых ни городу, ни Скарлетт не хотелось отказываться. Ну, кто бы мог подумать, что из всех людей именно он будет столь открыто, столь бесстыдно гордиться своим отцовством? Особенно если учесть то весьма щекотливое обстоятельство, что его первенцем была девочка, а не мальчик.
И новизна отцовства не стиралась. Это вызывало тайную зависть у женщин, чьи мужья считали появление потомства вещью естественной и забывали об этом событии, прежде чем ребенка окрестят. Ретт же останавливал людей на улице и рассказывал во всех подробностях, как на диво быстро развивается малышка, даже не предваряя это — хотя бы из вежливости — ханжеской фразой: «Я знаю, все считают своего ребенка самым умным, но…» Он считал свою дочку чудом, — разве можно ее сравнить с другими детьми, и плевать он хотел на тех, кто думал иначе. Когда новая няня дала малышке пососать кусочек сала и тем вызвала первые желудочные колики, Ретт повел себя так, что видавшие виды отцы и матери хохотали до упаду. Он спешно вызвал доктора Мида и двух других врачей, рвался побить хлыстом злополучную няньку, так что его еле удержали. Однако ее выгнали, после чего в доме Ретта перебывало много нянь — каждая держалась не больше недели, ибо ни одна не в состоянии была удовлетворить требованиям Ретта.
Мамушка тоже без удовольствия смотрела на появлявшихся и исчезавших нянек, ибо не хотела, чтобы в доме была еще одна негритянка: она вполне может заботиться и о малышке, и об Уэйде с Эллой. Но годы уже начали серьезно сказываться на Мамушке, и ревматизм сделал ее медлительной и неповоротливой. У Ретта не хватало духу сказать ей об этом и объяснить, почему нужна вторая няня. Вместо этого он говорил ей, что человеку с его положением не пристало иметь всего одну няню. Это плохо выглядит. Он намерен нанять еще двоих, чтобы они занимались тяжелой работой, а она, Мамушка, командовала ими. Вот такие рассуждения Мамушке были понятны. Чем больше слуг, тем выше ее положение, как и положение Ретта. Тем не менее она решительно заявила ему, что не потерпит в детской никаких вольных негров. Тогда Ретт послал в Тару за Присси. Он знал ее недостатки, но в конце концов она все-таки выросла в доме Скарлетт. А дядюшка Питер предложил свою внучатую племянницу по имени Лу, которая жила у Бэрров, кузенов мисс Питти.
Еще не успев окончательно оправиться, Скарлетт заметила, насколько Ретт поглощен малышкой, и даже чувствовала себя неловко и злилась, когда он хвастался ею перед гостями. Да, конечно, хорошо, если мужчина любит ребенка, но проявлять свою любовь на людях — это казалось ей немужественным. Лучше бы он держался небрежнее, безразличнее, как другие мужчины.
— Ты выставляешь себя на посмешище, — раздраженно сказала она как-то ему, — и я просто не понимаю, зачем тебе это.
— В самом деле? Ну, и не поймешь. А объясняется это тем, что малышка — первый человечек на свете, который всецело и полностью принадлежит мне.
— Но она и мне принадлежит!
— Нет, у тебя еще двое. Она — моя.
— Чтоб ты сгорел! — воскликнула Скарлетт. — Ведь это я родила ее, да или нет? К тому же, дружок, я тоже твоя. Ретт посмотрел на нее и как-то странно улыбнулся.
— В самом деле, моя прелесть?
Только появление Мелли помешало возникновению одной из тех жарких ссор, которые в последние дни так легко вспыхивали между супругами. Скарлетт подавила в себе гнев и отвернулась, глядя на то, как Мелани берет малышку. Решено было назвать се Юджини-Виктория, но в тот день Мелани невольно нарекла ее так, как потом все и стали звать девочку, — это имя прочно прилепилось к ней, заставив забыть о другом, как в свое время «Питтипэт» начисто перечеркнуло Сару-Джейн.
Ретт, склонившись над малюткой, сказал в ту минуту:
— Глаза у нее будут зеленые, как горох.
— Ничего подобного! — возмущенно воскликнула Мелани, забывая, что глаза у Скарлетт были почти такого оттенка. — У нее глаза будут голубые, как у мистера О’Хара, как.., как наш бывший голубой флаг: Бонни Блу.
— Бонни-Блу Батлер — рассмеялся Ретт, взял у Мелани девочку и внимательно вгляделся в ее глазки.
Так она и стала Бонни, и потом даже родители не могли вспомнить, что в свое время окрестили дочку двойным именем, состоявшим из имен двух королев.

