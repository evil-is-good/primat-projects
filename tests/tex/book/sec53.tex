\chapter{\ }

Наступил день рождения Эшли, и Мелани решила в тот вечер устроить сюрпризом ему торжество. Все об этом знали, за исключением Эшли. Даже Уэйд и крошка Бо, которых заставили поклясться, что они будут хранить тайну, и малыши ходили страшно гордые. Все благопристойные обитатели Атланты были приглашены и дали согласие прийти. Генерал Гордон и его семья любезно приняли приглашение; Александр Стефенс обещал быть, если ему позволит не слишком крепкое здоровье; ожидали даже Боба Тумбса, буревестника Конфедерации.
Все утро Скарлетт, Мелани, Индия и тетя Питти бегали по дому, руководя неграми, пока те вешали свежевыглаженные занавеси, чистили серебро, натирали полы, жарили, парили, помешивали и пробовали яства. Скарлетт никогда еще не видела Мелани такой счастливой и возбужденной.
— Ты понимаешь, дорогая, у Эшли ведь не было дня рождения со времени.., со времени, помнишь, того пикника в Двенадцати Дубах? Еще в тот день мистер Линкольн призвал добровольцев? Ну, так вот, у Эшли с тех пор не было дня рождения. А он так много работает и так устает, когда вечером приходит домой, что даже и не вспомнил про свой день рождения. Вот он удивится, когда после ужина столько народу явится к нам!
— А куда же вы фонарики-то над лужайкой спрячете — ведь мистер Уилкс наверняка увидит их, когда придет ужинать домой? — ворчливо спросил Арчи.
Старик просидел все утро, с интересом наблюдая за приготовлениями, хотя и не признавался в этом. Он еще ни разу не видел, как в городе готовятся к большому празднику, и это было ему внове. Больно уж разбегались женщины, — не стесняясь, говорил он: суетятся, будто в доме пожар, а всего-то навсего ждут гостей; но даже дикие кони не могли бы утащить его с места события. Особенно заинтересовали его цветные бумажные фонарики, которые собственноручно смастерили и покрасили миссис Элсинг с Фэнни: он никогда не видал «таких штуковин». Они хранились в его комнатушке в подвале, и он тщательно их обследовал.
— Бог ты мой! Я об этом и не подумала! — воскликнула Мелани. — Арчи, как хорошо, что ты вспомнил. Ну и ну! Что же делать? Ведь их надо повесить на кусты и деревья, воткнуть в них свечечки и зажечь, когда начнут появляться гости. Скарлетт, а ты не могла бы прислать Порка, чтобы он это сделал, пока мы будем ужинать?
— Мисс Уилкс, вы женщина умная — таких мало встретишь, а уж больно быстро начинаете волноваться, — сказал Арчи. — Да этому дураку Порку такие штуковины и в руки-то давать нельзя, Он их мигом сожжет. А они.., они красивенькие, — снисходительно заметил он. — Я уж сам их повешу, пока вы с мистером Уилксом кушать будете.
— Ах, Арчи, какой же ты добрый. — Мелани обратила к нему взгляд, исполненный детской благодарности и доверия. — Право, не знаю, что бы я без тебя делала. Может быть, тебе сейчас вставить в них свечи, чтобы хоть это было уже сделано?
— Что ж, можно, — сказал не очень-то любезно Арчи и заковылял к лестнице, которая вела в погреб.
— Оказывается, есть много способов убить кошку — не обязательно закармливать ее маслом, — весело заметила Мелани, когда старик, громко стуча деревяшкой, уже спустился вниз. — Я ведь все время хотела, чтобы именно Арчи развесил фонарики, но ты знаешь, какой он. Ни за что не станет делать то, о чем его просят. Да к тому же мы хоть убрали его на время с дороги. Чернокожие так его боятся, что просто ничего не делают, пока он стоит у них над душой.
— Я бы, Мелли, ни за что не держала такого отпетого человека у себя в доме, — раздраженно заявила Скарлетт. Она ненавидела Арчи так же пылко, как он ненавидел ее, и они едва разговаривали. Только из одного дома — дома Мелани — он не уходил, когда появлялась Скарлетт. Но даже и здесь он смотрел на нее подозрительно, с холодным презрением, — Ты еще с ним не оберешься хлопот — помяни мое слово.
— Он совершенно безвредный, если его хвалить и делать вид, будто зависишь от него, — сказала Мелани. — К тому же он так предан Эшли и Бо, что мне спокойнее, когда он тут.
— Ты хочешь сказать, что он предан тебе, Мелли, — заметила Индия, и холодное лицо ее потеплело от легкой улыбки, а взгляд с любовью остановился на жене брата. — По-моему, ты первая, кого этот старый грубиян полюбил с тех пор, как.., м-м.., словом, после своей жены. Мне кажется, он бы даже хотел, чтобы кто-то оскорбил тебя — тогда он убил бы этого человека, чтобы доказать, как он тебя высоко ставит.
— Бог ты мой! Что это ты говоришь, Индия! — воскликнула краснея Мелани. — Он считает меня страшной простофилей, и ты это знаешь.
— Ну, а я просто не понимаю, какое может иметь значение, что думает или считает этот вонючий старик, — вставила Скарлетт. Стоило ей вспомнить, как Арчи отчитывал ее за каторжников, она тут же выходила из себя. — Мне, кстати, пора идти. Еще надо распорядиться насчет ужина, потом съездить в лавку и заплатить приказчикам, а потом — на лесной склад и расплатиться с возчиками и Хью Элсингом.
— Ах, ты едешь на лесной склад? — переспросила Мелани. — Эшли собирался зайти на склад во второй половине дня, чтобы повидать Хью. Ты не могла бы задержать его там до пяти часов? Если он явится домой раньше, то, конечно, застигнет нас врасплох: мы еще наверняка будем возиться с тортом или каким-нибудь блюдом, и тогда никакого сюрприза не получится.
Скарлетт улыбнулась про себя — у нее сразу улучшилось настроение.
— Я задержу его, — сказала она.
Еще произнося это, она почувствовала на себе пронизывающий взгляд светлых, почти лишенных ресниц, глаз Индии. «Она всегда так странно смотрит на меня, когда я говорю об Эшли», — подумала Скарлетт.
— Тогда задержи его, сколько сможешь, и после пяти, — сказала Мелани. — А Индия приедет и заберет его… Только сама приходи сегодня вечером пораньше, Скарлетт. Я хочу, чтобы ты была с самого начала торжества.
«Она, значит, хочет, — мрачно размышляла по дороге домой Скарлетт, — чтобы я была с самого начала торжества, да? Почему же она не предложила мне принимать гостей вместе с нею, Индией и тетей Питти?» Вообще Скарлетт вовсе не жаждала принимать гостей на жалких торжествах у Мелани. Но это было необычно большое торжество, да к тому же день рождения Эшли, и Скарлетт очень хотелось бы стоять рядом с Эшли и принимать с ним гостей. Но она понимала, почему ей этого не предложили. А если бы не понимала, то ей стало бы все ясно из достаточно откровенного высказывания Ретта на сей счет:
«Чтобы какая-то подлипала принимала гостей — а там ведь будут все видные конфедераты и демократы?! Ваши представления о жизни столь же прелестны, сколь и нелепы. Да вас пригласили туда только благодаря доброму отношению мисс Мелли».
В тот день Скарлетт одевалась для поездки в лавку и на лесной склад тщательнее обычного. Она надела новое тускло-зеленое платье из переливчатой тафты, которая при определенном освещении казалась сиреневой, и новую бледно-зеленую шляпку с темно-зелеными перьями. Если бы только Ретт позволил ей сделать челку и завить ее, насколько лучше выглядела бы на ней эта шляпка! Но он заявил, что обреет ее наголо, если только она посмеет отрезать себе челку. А последнее время он был такой злющий, что и в самом деле мог это сделать.
День стоял чудесный — солнечный, но не слишком жаркий, свет был яркий, но не слепящий, и от теплого ветерка, шелестевшего листвой деревьев вдоль Персиковой улицы, танцевали перья на шляпке Скарлетт. Сердце ее тоже танцевало, как, впрочем, всегда перед встречей с Эшли. Быть может, если она расплатится пораньше с возчиками и Хью, они отправятся домой и оставят ее с Эшли наедине в маленькой квадратной конторке посреди лесного склада. Последнее время возможность увидеться с Эшли наедине появлялась все реже и реже. И подумать только, что сама Мелани предложила ей задержать его! Вот смешно-то!
Сердце Скарлетт пело, когда она подъехала к лавке, и она расплатилась с Уиллом и другими приказчиками, даже не спросив, как шли сегодня дела. А была суббота — день, когда в лавке идет самая оживленная торговля, так как все фермеры приезжают в город за покупками, но она ни о чем не спросила.
По дороге на лесной склад она раз десять останавливалась поговорить с женами «саквояжников», разъезжавшими в роскошных экипажах, — правда, менее роскошных, чем ее собственный, не без удовольствия подумала Скарлетт, — и со многими мужчинами, которые подходили к ее коляске и, стоя в красной пыли со шляпой к руке, сыпали комплиментами. День был прелестный, она была счастлива, она хорошо выглядела и ехала по городу, как принцесса. Из-за этих задержек в пути она приехала на склад позже, чем предполагала, и увидела, что Хью и возчики сидят на невысокой поленнице, дожидаясь ее.
— А Эшли здесь?
— Да, он в конторе, — ответил Хью, и озабоченное выражение исчезло с его лица при виде ее веселых, смеющихся глаз. — Он пытается.., то есть я хочу сказать, он просматривает книги.
— А ему сегодня не следовало бы этим заниматься, — заметила она и, понизив голос, добавила: — Мелли послала меня задержать его здесь подольше, пока они там дома готовятся к торжеству.
Хью улыбнулся: он был приглашен в гости. Он любил праздники и полагал, что Скарлетт тоже любит — судя по тому, как она сегодня выглядела. Расплатившись с возчиками и с Хью, Скарлетт круто повернулась и направилась к конторе, явно давая понять, что не желает, чтоб ее сопровождали. Эшли встретил ее на пороге — он стоял, озаренный предзакатным солнцем, отчего волосы его казались светлыми-светлыми; губы Эшли раздвинулись в улыбке, похожей на усмешку.
— Ба-а, Скарлетт, что вы делаете в городе в такое время дня? Почему вы не у меня дома и не помогаете Мелли готовить мне сюрприз?
— Ба-а, Эшли Уилкс! — возмущенно воскликнула она, — Вы же не должны знать об этом. Мелли будет так разочарована, если вы не удивитесь.
— О, я и виду не подам, что знаю. Буду удивлен больше всех в Атланте, — сказал Эшли, и глаза его смеялись.
— А теперь признайтесь, у кого хватило низости все вам рассказать?
— Так поступили почти все мужчины, которых пригласила Мелли. Первым был генерал Гордон. Он сказал, что на своем опыте знает: если женщины решают устроить сюрпризом торжество, они обычно назначают его на тот самый вечер, когда мужчина решил почистить и смазать все ружья в доме. А потом дедушка Мерриуэзер решил меня предупредить. Рассказал мне, как миссис Мерриуэзер надумала устроить ему однажды сюрпризом торжество, а сюрприз-то ждал ее, потому что дедушка, лечивший втихомолку ревматизм с помощью бутылочки виски, оказался слишком пьян и к началу торжества не мог вылезти из постели… Словом, все, кому домашние устраивали сюрпризом торжество, сказали, что меня ждет.
— Вот низкие люди! — воскликнула Скарлетт, не в силах, однако, сдержать улыбку.
Он был совсем прежним Эшли, когда улыбался, — таким, каким она знала его в Двенадцати Дубах. Но он теперь редко улыбался. Воздух был такой бархатный, солнце светило так мягко, у Эшли было такое веселое лицо, и он говорил так непринужденно, что сердце Скарлетт подпрыгнуло от радости. В груди ее что-то росло, росло, ей скоро стало больно от счастья — больно, как бывает от непролившихся горячих радостных слез. Она вдруг почувствовала себя снова шестнадцатилетней, и дух у нее перехватило от счастья и волнения. Ей неудержимо захотелось сорвать с головы шляпку, подкинуть в воздух и крикнуть: «Ур-ра!» Но она представила себе, как ошарашен будет Эшли, если она такое вытворит, и рассмеялась — рассмеялась безудержно, до слез. Он тоже рассмеялся, откинув голову, наслаждаясь смехом: он, видимо, считал, что она потешается над дружеским предательством мужчин, выдавших секрет Мелли.
— Заходите, Скарлетт. Я как раз просматриваю книги. Она вошла в маленькую комнатку, залитую солнцем, и опустилась на стул у бюро с убирающейся крышкой. Эшли вошел следом за ней и присел на край грубо отесанного стола, свесив длинные ноги.
— Ах, давайте не будем сегодня, Эшли, возиться с бухгалтерией! Не хочу я утруждать себя. Когда я надеваю новую шляпку, у меня все цифры вылетают из головы.
— Какие там могут быть цифры, когда такая красивая шляпка на голове, — заметил он. — Скарлетт, вы с каждым днем становитесь все красивее! — Он соскользнул со стола и, взяв с улыбкой ее руки, развел их, чтобы рассмотреть платье. — Вы такая красивая! И ничуть не меняетесь.
А Скарлетт, когда он взял ее за руки, поняла, что все время надеялась, — хотя и не отдавала себе в этом отчета, — что это произойдет. Весь этот счастливый день она надеялась почувствовать тепло его рук, увидеть нежность в глазах, услышать пусть слово, которое открыло бы ей, что она ему дорога. Сейчас они впервые были одни с того холодного дня, когда стояли во фруктовом саду Тары, и руки их впервые встретились не в обычном формальном пожатии, — она уже столько месяцев жадно ждала такой встречи с ним. Однако на сей раз…
Как ни странно, но его прикосновение нисколько не взволновало ее! Раньше одна близость Эшли уже вызывала в ней дрожь. А сейчас она чувствовала лишь дружеское тепло и ублаготворенность. От прикосновения его рук ее не опалило огнем, и сердце билось тихо и ровно. Это удивило Скарлетт, даже несколько разочаровало. Ведь это же ее Эшли, ее яркий, сверкающий герой, и она любит его больше жизни. Тогда почему же…
Она поспешила выкинуть эту мысль из головы. Достаточно того, что она с ним и что он держит ее руки и улыбается ей по-дружески, непринужденно, без напряжения. Это просто какое-то чудо, думала она: ведь между ними столько невысказанного. Его глаза, ясные и сияющие, смотрели в ее глаза, он улыбался как прежде, — а она так любила его улыбку, — улыбался, словно ничто никогда не омрачало их счастья. Между ними сейчас ничего не стояло, ничто их друг от друга не отбрасывало. Она рассмеялась.
— Ах, Эшли, конечно же, я старею, скоро стану совсем развалиной.
— Ну, это сразу видно! Нет, Скарлетт, даже когда вам исполнится шестьдесят, вы для меня останетесь прежней. Я всегда буду помнить вас такой, какой увидел в тот день на нашем последнем пикнике, когда вы сидели под дубом в окружении десятка юнцов. Я даже могу сказать вам, как вы были одеты: на вас было белое платье в мелкий зеленый цветочек и белая кружевная косынка на плечах. На ногах у вас были крошечные зеленые туфельки с черной шнуровкой, а на голове — огромная шляпа с ниспадавшими на спину длинными зелеными лентами. Я этот ваш туалет запомнил во всех подробностях, потому что в тюрьме, когда мне было худо, я извлекал из памяти картины прошлого и перебирал их, припоминая каждую мелочь… — Эшли внезапно умолк, и возбужденное выражение исчезло с его лица. Он осторожно выпустил ее руки, а она сидела и ждала, ждала, что он скажет дальше. — Мы с вами оба проделали с того дня долгий путь, верно, Скарлетт? Мы шли дорогами, которыми никогда не предполагали идти. Вы шли быстро, прямо, а я — медленно, нехотя. — Он снова присел на стол и посмотрел на Скарлетт, и снова на лице его появилась улыбка. Но это была не та улыбка, которая только что наполнила ее сердце счастьем. Улыбка была печальная, — Да, вы шли быстро и тянули еще меня, привязав к своей колеснице. Знаете, Скарлетт, я иногда смотрю на себя со стороны и думаю: что было бы со мной, если б не вы.
Скарлетт тотчас ринулась защищать Эшли от него самого — тем более что в ее мозгу предательски возникли слова Ретта.
— Но я же ничего для вас не сделала, Эшли. Вы и без меня были бы тем, что вы есть. В один прекрасный день вы стали бы богатым человеком, большим человеком, каким вы и станете.
— Нет, Скарлетт, семена величия никогда не сидели во мне. Я думаю, если бы не вы, все уже давно бы обо мне забыли — как о бедной Кэтлин Калверт и о многих других, чьи имена, старинные имена, когда-то гремели.
— Ах, Эшли, не надо так говорить. В ваших словах столько грусти.
— Да нет, я не грущу. Больше не грущу. Когда-то.., когда-то мне было грустно. А сейчас всего лишь…
Он умолк, и внезапно она поняла, о чем он думает. Она впервые поняла, о чем думает Эшли, заметив, как его взгляд устремился куда-то вдаль и глаза стали кристально прозрачными, отсутствующими. Когда любовь бушевала в ее сердце, она не способна была его понять. Сейчас же в атмосфере установившейся между ними спокойной дружбы она сумела чуть-чуть проникнуть в его мысли, чуть-чуть его понять. Нет, ему больше не грустно. Ему было грустно после падения Юга, грустно, когда она упрашивала его переехать в Атланту. Сейчас же он смирился.
— Не хочется мне слушать, когда вы говорите такое, Эшли, — вспылила она. — Совсем как Ретт. Он вот вечно твердит о таких же вещах да о выживании сильных, как он это называет, и мне до того надоело слушать, что просто кричать хочется, когда он заводит свою музыку. Эшли усмехнулся.
— А вы никогда не задумывались, Скарлетт, что я и Ретт в чем-то главном очень схожи?
— Ах, нет! Вы — такой тонкий, такой благородный, а он… — И она умолкла, смутившись.
— А ведь мы похожи. Мы произошли от людей одной породы, воспитывались по одинаковому образцу, были приучены одинаково думать. Но где-то по дороге повернули в разные стороны. Мы по-прежнему думаем одинаково, а воспринимаем вещи по-разному. К примеру, ни один из нас не верил в войну, но я пошел добровольцем и сражался, а он принял участие в войне лишь к самому концу. Мы оба понимали, что не надо было начинать эту войну. Мы оба понимали, что проиграем ее. Я готов был сражаться, зная, что мы проиграем. А он — нет. Иной раз я думаю, что он был прав, и тогда опять-таки…
— Ах, Эшли, когда вы перестанете поворачивать любой вопрос и так и эдак? — воскликнула она. Но в голосе ее уже не звучало, как прежде, нетерпение. — Ничего это не дает — смотреть на дело с двух сторон.
— Это верно, но… Скарлетт, чего вы все-таки добиваетесь? Я часто удивлялся вам. Понимаете, я, к примеру, никогда ничего не добивался. Единственное, чего мне хотелось, — это остаться самим собой.
Чего она добивается? Какой глупый вопрос. Денег и уверенности в завтрашнем дне, конечно. И однако же… Скарлетт почувствовала, что стала в тупик. Ведь у нее же есть деньги, и она уверена в завтрашнем дне, насколько можно быть в чем-то уверенной в этом ненадежном мире. Но вот сейчас, думая об этом, она почувствовала, что этого недостаточно. Сейчас, думая об этом, она почувствовала, что не стала счастливее, хотя, конечно, жилось ей спокойнее, она меньше боялась завтрашнего дня. «Если бы у меня были деньги, уверенность в завтрашнем дне и вы, — я считала бы, что добилась всего», — подумала она, с любовью глядя на него. Но она не произнесла этих слов, боясь нарушить то, что возникло между ними, боясь, что снова перестанет понимать его.
— Вам хочется всего лишь остаться самим собой? — немного печально рассмеялась она. — Моей большой бедой было то, что я никогда не была сама собою! А чего я хочу добиться — что ж, по-моему, я этого уже добилась. Мне хотелось быть богатой, чувствовать уверенность в завтрашнем дне и…
— Но, Скарлетт, неужели вам никогда не приходило в голову, что мне безразлично, богат я или нет?
Нет, ей никогда не приходило в голову, что есть люди, которые не хотят быть богатыми.
— Тогда чего же вы хотите?
— Сейчас — не знаю. Когда-то знал, но уже почти забыл. Главным образом — чтобы меня оставили в покое, чтобы меня не донимали люди, которых я не люблю, чтобы меня не заставляли делать то, чего мне не хочется. Пожалуй.., мне хотелось бы, чтобы вернулись былые дни, а они никогда не вернутся, меня же все время преследуют воспоминания о них и о том, как вокруг меня рухнул мир.
Скарлетт с решительным видом сжала губы. Она прекрасно понимала, о чем он говорил. Само звучание его голоса вызывало к жизни те далекие времена, рождало ноющую боль в сердце. Но с того дня, когда она в отчаянии кинулась на землю в огороде в Двенадцати Дубах и сказала себе: «Я не буду оглядываться назад», она запретила себе думать о прошлом.
— А мне больше нравится сегодняшняя жизнь, — сказала Скарлетт. Но произнесла она это, не глядя на Эшли. — Сейчас столько всего интересного — приемы, разные торжества. Все так пышно. А раньше было так уныло. — (О, эти лениво-неспешные дни и тихие теплые сельские сумерки! Приглушенный женский смех в службах! Какой золотисто-теплой была тогда жизнь, как грела спокойная уверенность, что и завтра будет так же! Да разве можно все это зачеркнуть?) — Мне, право, больше нравится сегодняшняя жизнь, — повторила она, но голос ее при этом дрогнул.
Эшли соскользнул со стола и недоверчиво рассмеялся. Взяв Скарлетт за подбородок, он приподнял ее лицо.
— Ах, Скарлетт, как же вы не умеете лгать! Да, конечно, жизнь стала более пышной — в определенном смысле. В том-то вся и беда. В прошлом не было пышности, но дни тогда были окрашены очарованием, они имели свою прелесть, свою медлительную красоту.
Раздираемая противоречивыми чувствами, она опустила глаза. Звук его голоса, прикосновение его руки мягко открывали двери, которые она для себя навсегда заперла. За этими дверями лежала красота былых дней, и грусть и тоска по ним наполнили ее душу. Но Скарлетт знала, что сколько бы красоты ни таилось в прошлом, она в прошлом и должна остаться. Человек не может двигаться вперед, если душу его разъедает боль воспоминаний.
Эшли выпустил ее подбородок, взял ее руку в свои ладони и нежно сжал.
— Помните… — сказал он, и в ее мозгу тотчас Предупреждающе зазвенело: «Не оглядывайся назад! Не оглядывайся!» Но она не обратила на это внимания, увлекаемая волной счастья. Наконец-то она понимала его, наконец-то они одинаково мыслили. Это мгновение было слишком бесценно — нельзя его потерять, какую бы оно ни повлекло за собой боль.
— Помните… — повторил он, и от звука его голоса, словно по волшебству, рухнули голые стены конторы и все эти годы куда-то ушли, и они, Скарлетт и Эшли, снова ехали верхом по сельским проселкам той далекой, давно минувшей весной. Он все говорил и крепче сжимал ее руку, и в голосе его звучала грусть и колдовские чары старых, полузабытых песен. Скарлетт слышала веселое позвякиванье уздечки, когда они ехали под кизиловыми деревьями на пикник к Тарлтонам, слышала свой беззаботный смех, видела, как солнце блестит на золотистых, отливающих серебром волосах Эшли, любовалась горделивой грацией, с какой он сидит в седле. В голосе его звучала музыка — музыка скрипок и банджо; под эти звуки они танцевали тогда в белом доме, которого больше нет. Где-то вдали, в темных болотах, кричали опоссумы под холодной осенней луной, а на рождество от чаш, увитых остролистом, пахло ромовым пуншем, и кругом сияли улыбками лица черных слуг и белых господ. И друзья былых дней, смеясь, вдруг собрались вокруг, словно и не лежали в могилах уже многие годы: длинноногие рыжеволосые Стюарт и Брент с их вечными остротами; Том и Бойд, похожие на молодых, необузданных коней; черноглазый пылкий Джо Фонтейн, Кэйд и Рейфорд Калверты, двигавшиеся с такой ленивой грацией. Были тут и Джон Уилкс, и Джералд, раскрасневшийся от коньяка; и шорох юбок и аромат Эллин. И над всем этим царило чувство уверенности, сознание, что завтрашний день может быть лишь таким же счастливым, как сегодняшний.
Эшли умолк, и они долго смотрели друг другу в глаза, и между ними лежала навсегда утраченная золотая юность, которую они в свое время так бездумно провели.
«Теперь я знаю, почему ты не можешь быть счастливым, — с грустью подумала она. — Раньше я этого не понимала. Не понимала я раньше и того, почему сама не моту быть счастлива. Но.., впрочем, почему это мы говорим, будто два старика! — подумала она с тоской и удивлением. — Два старика, которые оглядываются на то, что было пятьдесят лет назад. А мы не такие уж старые! Просто столько всего произошло за это время. Все так изменилось, точно пролетело пятьдесят лет. Но мы же не такие ведь старые!» Однако взглянув на Эшли, она поняла, что он уже не тот молодой и блестящий юноша, каким был когда-то. Он стоял, нагнув голову, и отсутствующим взглядом смотрел на ее руку, которую продолжал держать в ладонях, и Скарлетт увидела, что его некогда золотистые волосы стали серыми, серебристо-серыми, как лунный свет на недвижной воде. И все сияние, вся красота апрельского дня вдруг исчезла — она перестала их ощущать, — и грустная сладость воспоминаний опалила горечью.
«Не надо было мне поддаваться ему — дать увлечь меня в прошлое, — в отчаянии подумала она. — Я была права, сказав тогда, что никогда не буду оглядываться. Слишком это больно, слишком терзает сердце, так что потом ты уже ни на что не способен — все и будешь смотреть назад. Вот в чем беда Эшли. Он не может больше смотреть вперед. Он не видит настоящего, он боится будущего и потому все время оглядывается назад. Я прежде этого не понимала. Я не понимала Эшли. Ах, Эшли, дорогой мой, не надо оглядываться назад! Какой от этого прок? Не следовало мне допускать этого разговора о прошлом. Вот что получается, когда оглядываешься назад — на то время, когда ты был счастлив, — одна боль, душевная мука и досада».
Она поднялась, не отнимая у него руки. Надо ехать. Не может она больше здесь оставаться, думать о былом и видеть его лицо, усталое, грустное и такое замкнутое.
— Мы прошли длинный путь с той поры, Эшли, — сказала она, стараясь, чтобы голос звучал твердо, и пытаясь проглотить стоявший в горле комок. — Прекрасные у нас тогда были представления обо всем, верно? — И вдруг у нее вырвалось: — Ах, Эшли, все получилось совсем не так, как мы ждали.
— Так было и будет, — сказал он. — Жизнь не обязана давать нам то, чего мы ждем. Надо брать то, что она дает, и быть благодарным уже за то, что это так, а не хуже.
Сердце у Скарлетт вдруг заныло от боли и усталости, когда она подумала о том, какой длинный путь прошла с тех пор. В памяти ее возник образ Скарлетт О’Хара, которая любила ухажеров и красивые платья и намеревалась когда-нибудь — когда будет время — стать такой же настоящей леди, как Эллин.
Слезы вдруг навернулись ей на глаза и медленно покатились по щекам — она стояла и смотрела на Эшли тупо, как растерявшийся ребенок, которому ни с того ни с сего причинили боль. Эшли не произнес ни слова — только нежно обнял Скарлетт, прижал ее голову к своему плечу и, пригнувшись, коснулся щекою ее щеки. Она вся обмякла и обхватила Эшли руками. В его объятиях было так покойно, что неожиданно набежавшие слезы сразу высохли. Ах, как хорошо, когда тебя вот так обнимают — без страсти, без напряжения, словно любимого друга. Только Эшли, которого роднили с ней воспоминания юности, который знал, с чего она начинала и к чему пришла, мог ее понять.
Она услышала на улице шаги, но не придала этому значения, решив, что, должно быть, возчики расходятся по домам. Она стояла, застыв, и слушала, как медленно бьется сердце Эшли. Внезапно он резко оттолкнул ее, так что она не сразу пришла в себя от неожиданности. Она в изумлении подняла на него глаза, но он смотрел не на нее. Он смотрел поверх ее плеча на дверь.
Она обернулась. В дверях стояла Индия, смертельно бледная, с горящими, белыми от ярости глазами, а рядом Арчи, ехидный, как одноглазый попугай. Позади них стояла миссис Элсинг.




Она не помнила, как вышла из конторы. Но вышла она стремительно, тотчас же — по приказанию Эшли; Эшли и Арчи остались в конторе, а Индия с миссис Элсинг стояли на улице, повернувшись к Скарлетт спиной. Скарлетт помчалась домой, подгоняемая стыдом и страхом, и перед ее мысленным взором возник Арчи — Арчи со своей бородой патриарха, превратившийся в ангела-мстителя из Ветхого завета.
Дом стоял пустой и тихий в лучах апрельского заката. Все слуги отправились на чьи-то похороны, а дети играли на заднем дворе у Мелани. Мелани…
Мелани! Поднимаясь к себе в комнату, Скарлетт похолодела при мысли о ней. Мелани обо всем узнает. Индия ведь говорила, что скажет ей. О, Индия с наслаждением ей все расскажет, не заботясь о том, что может очернить имя Эшли, не заботясь о том, что может ранить Мелани, — лишь бы уязвить Скарлетт! Да и миссис Элсинг не заставишь молчать, хотя на самом деле она ничего не видела, потому что стояла позади Индии и Арчи, за порогом конторы. Но болтать языком она будет все равно. К ужину сплетня обежит город. А наутро к завтраку уже все будут об этом знать, даже негры. Сегодня вечером на торжестве женщины будут собираться группками в уголках и со злорадным удовольствием перешептываться: Скарлетт Батлер слетела ее своего высокого пьедестала! И сплетня будет расти, расти. Ничем ее не остановишь. Не остановишь, даже если рассказать, как все было на самом деле, а ведь Эшли обнял ее только потому, что она заплакала. Еще до наступления ночи люди станут говорить, что ее застигли в чужой постели. А ведь все было так невинно, так хорошо! Вне себя от досады Скарлетт думала: «Если бы нас застигли в то рождество, когда он приезжал на побывку из армии и я целовала его на прощание.., если бы нас застигли в саду Тары, когда я умоляла его бежать со мной.., ох, если бы нас застигли в любое другое время, когда мы были в самом деле виноваты, это не было бы так обидно! Но сейчас! Сейчас! Когда он держал меня в объятиях, как друг…» Но ведь никто этому не поверит. Никто из друзей не станет на ее сторону, никто не возвысит голос и не скажет: «Не верю, что она вела себя дурно». Слишком долго она оскорбляла старых друзей, чтобы среди них нашелся теперь человек, который стал бы за нее сражаться. А новые друзья, молча переносившие ее выходки, будут рады случаю пошантажировать ее. Нет, все поверят любой сплетне, хотя, возможно, многие и будут жалеть, что такой прекрасный человек, как Эшли Уилкс, замешан в столь грязной истории. По обыкновению, всю вину свалят на женщину, а по поводу мужчины лишь пожмут плечами. И в данном случае они будут правы. Ведь это она кинулась к нему в объятия.
О, она все вытерпит: уколы, оскорбления, улыбки исподтишка, все, что может сказать о ней город, — но только не Мелани! Ох, нет, только не Мелани! Она сама не понимала, почему ей так важно, чтобы не узнала Мелани. Слишком она была испугана и подавлена сознанием вины за прошлое, чтобы пытаться это понять. Тем не менее она залилась слезами при одной мысли о том, какое выражение появится в глазах Мелани, когда Индия скажет ей, что застала Скарлетт в объятиях Эшли. И как поведет себя Мелани, когда узнает? Бросит Эшли? А что еще ей останется делать, если она не захочет потерять достоинство? «И что тогда будем делать мы с Эшли? — Мысли бешено крутились в голове Скарлетт, слезы текли по лицу. — О, Эшли просто умрет со стыда и возненавидит меня за то, что я навлекла на него такое». Внезапно слезы ее иссякли, смертельный страх сковал сердце. А Ретт? Как поступит он?
Быть может, он никогда об этом и не узнает. Как это говорится в старой циничной поговорке? «Муж всегда узнает все последним». Быть может, никто ему не расскажет. Надо быть большим храбрецом, чтобы рассказать такое Ретту, ибо у Ретта репутация человека, который сначала стреляет, а потом задает вопрос. «Господи, смилуйся, сделай так, чтобы ни у кого не хватило мужества сказать ему!» Но тут она вспомнила лицо Арчи в конторе лесного склада, его холодные, светлые, безжалостные глаза, полные ненависти к ней и ко всем женщинам на свете. Арчи не боится ни бога, ни человека и ненавидит беспутных женщин. Так люто ненавидит, что одну даже убил. И он ведь говорил, что расскажет все Ретту. И расскажет, сколько бы ни пытался Эшли его разубедить. Разве что Эшли убьет его — в противном случае Арчи все расскажет Ретту, считая это своим долгом христианина.
Скарлетт стянула с себя платье и легла в постель — мысли ее кружились, кружились. Если бы только она могла запереть дверь и просидеть всю жизнь здесь, в безопасности, и никогда больше никого не видеть. Быть может, Ретт сегодня еще ничего и не узнает. Она скажет, что у нее болит голова и что ей не хочется идти на прием. А к утру она, быть может, придумает какое-то объяснение, хоть что-то в свою защиту, подо что так просто не подкопаешься.
«Сейчас я об этом не буду думать, — в отчаянии сказала она себе, зарываясь лицом в подушку. — Сейчас я об этом не буду думать. Подумаю потом, когда соберусь с силами».
Она услышала, как с наступлением сумерек вернулись слуги, и ей показалось, что они как-то особенно тихо готовят ужин. Или, быть может, так ей казалось из-за нечистой совести? К двери подошла Мамушка и постучала, но Скарлетт отослала ее прочь, сказав, что не хочет ужинать. Время шло, и наконец на лестнице раздались шаги Ретта. Она вся напряглась, когда он поднялся на верхнюю площадку, и, готовясь к встрече с ним, призвала на помощь все свои силы, но он прошел прямиком к себе в комнату. Скарлетт облегченно вздохнула. Значит, он ничего не слышал. Слава богу, он пока еще считается с ее ледяным требованием никогда не переступать порога ее спальни, ибо если бы он увидел ее сейчас, то сразу бы все понял. Она должна взять себя в руки и сказать ему, что плохо себя чувствует и не в состоянии пойти на прием. Что ж, у нее есть время успокоиться. Впрочем, есть ли? С той страшной минуты время как бы перестало существовать в ее жизни. Она слышала, как Ретт долго ходил по своей комнате, слышала, как он обменивался какими-то фразами с Порком. Но она все не могла найти в себе мужество окликнуть его. Она лежала неподвижно на постели в темноте и дрожала.
Прошло много времени; наконец он постучал к ней в дверь, и она сказала, стараясь голосом не выдать волнения:
— Войдите.
— Неужели меня приглашают в святилище? — спросил он, открывая дверь. Было темно, и Скарлетт не могла видеть его лицо. Не могла она ничего понять и по его тону. Он вошел и закрыл за собой дверь, — Вы готовы идти на прием?
— Мне очень жаль, но у меня болит голова, — Как странно, что голос у нее звучит вполне естественно! Благодарение богу, в комнате темно! — Не думаю, чтобы я смогла пойти. А вы, Ретт, идите, и передайте Мелани мои сожаления.
Долго длилось молчание, наконец в темноте протяжно прозвучали язвительные слова:
— Какая же вы малодушная трусливая сучка. Он знает! Скарлетт лежала и тряслась, не в силах произнести ни слова. Она услышала, как он что-то ищет в темноте, чиркнула спичка, и комната озарилась светом. Ретт подошел к кровати и посмотрел на нее. Она увидела, что он во фраке.
— Вставайте, — сказал он ровным голосом. — Мы идем на прием. И извольте поторопиться.
— Ох, Ретт, я не могу. Видите ли…
— Я все вижу. Вставайте.
— Ретт, неужели Арчи посмел…
— Арчи посмел. Он очень храбрый человек, этот Арчи.
— Вам следовало пристрелить его, чтоб он не врал…
— Такая уж у меня странная привычка: я не убиваю тех, кто говорит правду. Сейчас не время для препирательств. Вставайте.
Она села, стянув на груди халат, внимательно глядя ему в лицо. Смуглое лицо Ретта было бесстрастно.
— Я не пойду, Ретт. Я не могу, пока.., пока это недоразумение не прояснится.
— Если вы не покажетесь сегодня вечером, то вы уже до конца дней своих никогда и нигде не сможете в этом городе показаться. И если я еще готов терпеть то, что у меня жена — проститутка, трусихи я не потерплю. Вы пойдете сегодня на прием, даже если все, начиная с Алекса Стефенса и кончая последним гостем, будут оскорблять вас, а миссис Уилкс потребует, чтобы мы покинули ее дом.
— Ретт, позвольте, я все вам объясню.
— Я не желаю ничего слышать. И времени нет. Одевайтесь.
— Они неверно поняли — и Индия, и миссис Элсинг, и Арчи. И потом, они все меня так ненавидят. Индия до того ненавидит меня, что готова наговорить на собственного брата, лишь бы выставить меня в дурном свете. Если бы вы только позволили мне объяснить…
«О, мать пресвятая богородица, — в отчаянии подумала она, — а что, если он скажет: „Пожалуйста, объясните!“ Что я буду говорить? Как я это объясню?» — Они, должно быть, всем наговорили кучу лжи. Не могу я идти сегодня.
— Пойдете, — сказал он. — Вы пойдете, даже если мне придется тащить вас за шею и при каждом шаге сапогом подталкивать под ваш прелестный зад.
Глаза его холодно блестели. Рывком поставив Скарлетт на ноги, он взял корсет и швырнул ей его.
— Надевайте. Я сам вас затяну. О да, я прекрасно знаю, как затягивают. Нет, я не стану звать на помощь Мамушку, а то вы еще запрете дверь и сядете тут, как последняя трусиха.
— Я не трусиха! — воскликнула она, от обиды забывая о своем страхе.
— О, избавьте меня от необходимости слушать вашу сагу о том, как вы пристрелили янки и выстояли перед всей армией Шермана. Все равно вы трусиха. Так вот: если не ради себя самой, то ради Бонни вы пойдете сегодня на прием. Да как вы можете так портить ее будущее?! Надевайте корсет, и быстро.
Она поспешно сбросила с себя халат и осталась в одной ночной рубашке. Если бы он только взглянул на нее и увидел, какая она хорошенькая в своей рубашке, быть может, это страшное выражение исчезло бы с его лица. Ведь, в конце концов, он не видел ее в ночной рубашке так давно, бесконечно давно. Но он не смотрел на нее. Он стоял лицом к шкафу и быстро перебирал ее платья. Пошарив немного, он вытащил ее новое, нефритово-зеленое муаровое платье. Оно было низко вырезано на груди; обтягивающая живот юбка лежала на турнюре пышными складками, и на складках красовался большой букет бархатных роз.
— Наденьте вот это, — сказал он, бросив платье на постель и направляясь к ней. — Сегодня никаких скромных, приличествующих замужней даме серо-сиреневых тонов. Придется прибить флаг гвоздями к мачте, иначе вы его живо спустите. И побольше румян. Уверен, что та женщина, которую фарисеи застигли, когда она изменяла мужу, была далеко не такой бледной. Повернитесь-ка.
Он взялся обеими руками за тесемки ее корсета и так их дернул, что она закричала, испуганная, приниженная, смущенная столь непривычной ситуацией.
— Больно, да? — Он отрывисто рассмеялся, но она не видела его лица. — Жаль, что эта тесемка не на вашей шее.
Все комнаты в доме Мелани были ярко освещены, и звуки музыки разносились далеко по улице. Когда коляска, в которой ехали Скарлетт и Ретт, остановилась у крыльца, до них долетел многоголосый шум и приятно возбуждающий гомон пирующих людей. В доме было полно гостей. Многие вышли на веранды, другие сидели на скамьях в окутанном сумерками, увешанном фонариками саду.
«Не могу я туда войти… Не могу, — подумала Скарлетт, сидя в коляске, комкая в руке носовой платок. — Не могу. Не пойду.
Выскочу сейчас и убегу куда глаза глядят, назад, домой, в Тару. Зачем Ретт заставил меня приехать сюда? Как поведут себя люди? Как поведет себя Мелани? Какой у нее будет вид? Ох, не могу я показаться ей на глаза. Я сейчас сбегу».
Словно прочитав ее мысли, Ретт с такою силой схватил ее за руку, что наверняка потом будет синяк, — схватил грубо, как чужой человек.
— Никогда еще не встречал трусов среди ирландцев. Где же ваша знаменитая храбрость?
— Ретт, пожалуйста, отпустите меня домой, я все вам объясню.
— У вас будет целая вечность для объяснений, но всего одна ночь, чтобы выступить как мученица на арене. Вылезайте, моя дорогая, и я посмотрю, как набросятся на вас львы. Вылезайте же.
Она не помнила, как прошла по аллее, опираясь на руку Ретта, крепкую и твердую, как гранит, — рука эта придавала ей храбрости. Честное слово, она может предстать перед ними всеми и предстанет. Ну, что они такое — свора мяукающих, царапающихся кошек, завидующих ей! Она им всем покажет. Плевать, что они о ней думают. Вот только Мелани.., только Мелани.
Они поднялись на крыльцо, и Ретт, держа в руке шляпу, уже раскланивался направо и налево, голос его звучал мягко, спокойно. Когда они вошли, музыка как раз умолкла, и в смятенном сознании Скарлетт гул толпы вдруг возрос, обрушился на нее словно грохот прибоя и отступил, замирая, все дальше и дальше. Неужели сейчас все набросятся на нее? Ну, чтоб вам всем пропасть — попробуйте! Она вздернула подбородок, изобразила улыбку, прищурила глаза.
Но не успела она повернуться к тем, кто стоял у двери, и сказать хоть слово, как почувствовала, что толпа раздается, пропуская кого-то. Наступила странная тишина, и сердце у Скарлетт остановилось. Она увидела Мелани — маленькие ножки быстро-быстро шагали по проходу: она спешила встретить Скарлетт у двери, первой приветствовать ее. Мелани шагала, распрямив узенькие плечики, чуть выдвинув вперед подбородок и всем своим видом показывая возмущение, — точно для нее существовала одна Скарлетт, а других гостей вовсе не было. Она подошла к Скарлетт и обняла ее за талию.
— Какое прелестное платье, дорогая, — сказала она звонким, тоненьким голоском. — Будь ангелом! Индия не смогла прийти, чтобы помочь мне. Ты не согласилась бы принимать со мной гостей?

