\chapter{\ }

Месяц спустя Ретт посадил в поезд, шедший в Джонсборо, бледную худую женщину. Уэйд и Элла, отправлявшиеся с нею в путь, молчали и не знали, как себя вести при этой женщине с застывшим, белым как мел лицом. Они жались к Присси, потому что даже их детскому уму казалось страшным холодное отчуждение, установившееся между их матерью и отчимом.
Скарлетт решила поехать к себе в Тару, хотя еще и была очень слаба. Ей казалось, что она задохнется, если пробудет в Атланте еще один день; голова ее раскалывалась от мыслей, которые она снова и снова гоняла по протоптанной дорожке, тщетно пытаясь разобраться в создавшемся положении. Она была нездорова и душевно надломлена; ей казалось, что она, словно потерявшийся ребенок, забрела в некий страшный край, где нет ни одного знакомого столба или знака, который указал бы дорогу.
Однажды она уже бежала из Атланты, спасаясь от наступавшей армии, а теперь бежала снова, отодвинув заботы в глубину сознания с помощью старой уловки: «Сейчас я не стану об этом думать. Я не вынесу. Я подумаю об этом завтра, в Таре. Завтра будет уже новый день». Ей казалось, что если только она доберется до дома и очутится среди тишины и зеленых хлопковых полей, все ее беды сразу отпадут, и она сможет каким-то чудом собрать раздробленные мысли, построить из обломков что-то такое, чем можно жить.
Ретт смотрел вслед поезду, пока он не исчез из виду, и на лице его читались озадаченность и горечь, отчего оно выглядело не очень приятным. Ретт вздохнул, отпустил карету и, вскочив в седло, поехал по Плющовой улице к дому Мелани.
Утро было теплое, и Мелани сидела на затененном виноградом крыльце, держа на коленях корзину с шитьем, полную носков. Она смутилась и растерялась, увидев, как Ретт соскочил с лошади и перекинул поводья через руку чугунного негритенка, стоявшего у дорожки. Они не виделись наедине с того страшного дня, когда Скарлетт была так больна, а он был.., ну, словом.., так ужасно пьян. Мелани неприятно было даже мысленно произносить это слово. Пока Скарлетт поправлялась, они лишь изредка переговаривались, причем Мелани всякий раз обнаруживала, что ей трудно встретиться с ним взглядом. Он же в таких случаях всегда держался со своим неизменно непроницаемым видом и никогда ни взглядом, ни намеком не дал понять, что помнит ту сцену между ними. Эшли как-то говорил Мелани, что мужчины часто не помнят, что они делали и говорили спьяну, и Мелани молилась в душе, чтобы память на этот раз изменила капитану Батлеру. Ей казалось, что она умрет, если узнает, что он помнит, о чем он тогда ей говорил. Она погибала от чувства неловкости и смущения, и вся залилась краской, пока он шел к ней по дорожке. Но наверно, он пришел лишь затем, чтобы спросить, не может ли Бо провести день с Бонни. Едва ли он столь плохо воспитан, чтобы явиться к ней с благодарностью за то, что она тогда сделала?
Она поднялась навстречу ему, лишний раз не без удивления подметив, как легко он движется для такого высокого крупного мужчины.
— Скарлетт уехала?
— Да. Тара пойдет ей на пользу, — с улыбкой сказал он. — Иной раз я думаю, что Скарлетт — вроде этого гиганта Антея, которому придавало силу прикосновение к матери-земле. Скарлетт нельзя долго расставаться со своей красной глиной, которую она так любит. А вид растущего хлопка куда больше поможет ей, чем все укрепляющие средства доктора Мила.
— Не хотите ли присесть? — предложила Мелани, дрожащей от волнения рукой указывая на кресло. Он был такой большой и в нем так сильно чувствовался мужчина, а это всегда выводило Мелани из равновесия. В присутствии людей, от которых исходила подобная сила и жизнестойкость, она ощущала себя как бы меньше и даже слабее, чем на самом деле. Мистер Батлер был такой смуглый, могучий, под белым полотняным его пиджаком угадывались такие мускулы, что, взглянув на него, она немного испугалась. Сейчас ей казалось невероятным, что она видела эту силу, эту самонадеянность сломленными. И держала эту черноволосую голову на своих коленях!
«О господи!» — подумала она в смятении и еще больше покраснела.
— Мисс Мелли, — мягко сказал Ретт, — мое присутствие раздражает вас? Может, вы хотите, чтобы я ушел? Прошу вас, будьте со мной откровенны.
«О!» — подумала она. — Значит, он помнит! И понимает, как я растеряна!» Она умоляюще подняла на него глаза, и вдруг все ее смущение и смятение исчезли. Он смотрел на нее таким спокойным, таким добрым, таким понимающим взглядом, что она просто уразуметь не могла, как можно быть такой глупой и волноваться. Лицо у него было усталое и, не без удивления подумала Мелани, очень печальное. Да как могло ей прийти в голову, что он столь плохо воспитан и может затеять разговор о том, о чем оба они хотели бы забыть? «Бедняга, он так переволновался из-за Скарлетт», — подумала она и, заставив себя улыбнуться, сказала:
— Садитесь же, пожалуйста, капитан Батлер. Он тяжело опустился в кресло, глядя на нее, а она снова взялась за штопку носков. — Мисс Мелли, я пришел просить вас о большом одолжении, — он улыбнулся, и уголки его губ поползли вниз, — и о содействии в обмане, хоть я и знаю, что вам это не по нутру.
— В обмане?
— Да. Я, в общем-то, пришел поговорить с вами об одном деле.
— О господи! В таком случае вам надо бы повидать мистера Уилкса. Я такая гусыня во всем, что касается дел. Я ведь не такая шустрая, как Скарлетт.
— Боюсь, что Скарлетт слишком шустрая во вред себе, — сказал он, — и как раз об этом я и намерен с вами говорить. Вы знаете, она снова точно бешеная возьмется за свою лавку и за эти свои лесопилки — признаюсь, я от всей души желаю, чтобы обе они как-нибудь ночью взлетели на воздух. Я боюсь за ее здоровье, мисс Мелли.
— Да, она слишком много взвалила на себя. Вы должны заставить ее отойти от дел и заняться собой. Он рассмеялся.
— Вы знаете, какая она упрямая. Я не пытаюсь даже спорить с ней. Она как своенравный ребенок. Она не разрешает мне помогать ей — не только мне, но вообще никому. Я пытался убедить ее продать свою долю в лесопилках, но она не желает. А теперь, мисс Мелли, я и подхожу к тому делу, по поводу которого пришел к вам. Я знаю, что Скарлетт продала бы свою долю в лесопилках мистеру Уилксу — и только ему, и я хочу, чтобы мистер Уилкс выкупил у нее эти лесопилки.
— О, боже ты мой! Это было бы, конечно, очень славно, но… — Мелани умолкла, прикусив губу. Не могла же она говорить с посторонним о деньгах. Так уж получалось, что хоть Эшли и зарабатывал кое-что на лесопилке, но им почему-то всегда не хватало. Мелани тревожило то, что они почти ничего не откладывают. Она сама не понимала, куда уходят деньги. Эшли давал ей достаточно, чтобы вести дом, но когда дело доходило до каких-то дополнительных трат, им всегда бывало трудно. Конечно, счета от ее врачей складывались в изрядную сумму, да и книги, и мебель, которую Эшли заказал в Нью-Йорке, тоже немало стоили. И они кормили и одевали всех бесприютных, которые спали у них в подвале. И Эшли ни разу не отказал в деньгах бывшим конфедератам. И…
— Мисс Мелли, я хочу одолжить вам денег, — сказал Ретт.
— Это очень любезно с вашей стороны, но ведь мы, возможно, не сумеем расплатиться.
— Я вовсе не хочу, чтобы вы со мной расплачивались. Не сердитесь на меня, мисс Мелли! Пожалуйста, дослушайте до конца. Вы сполна расплатитесь со мной, если я буду знать, что Скарлетт больше не изнуряет себя, ездя на свои лесопилки, которые ведь так далеко от города. Ей вполне хватит и лавки, чтобы не сидеть без дела и чувствовать себя счастливой… Вы со мной согласны?
— М-м.., да, — неуверенно сказала Мелани.
— Вы хотите, чтобы у вашего мальчика был пони? И вы хотите, чтобы он пошел в университет, причем в Гарвардский, и чтобы он поехал в Европу?
— Ах, конечно! — воскликнула Мелани, и лицо ее, как всегда при упоминании о Бо, просветлело. — Я хочу, чтобы у него все было, но.., все вокруг сейчас такие бедные, что…
— Со временем мистер Уилкс сможет нажить кучу денег на лесопилках, — сказал Ретт. — А мне бы хотелось, чтобы Бо имел все, чего он заслуживает.
— Ах, капитан Батлер, какой вы хитрый бесстыдник! — с улыбкой воскликнула она. — Играете на моих материнских чувствах! Я ведь читаю ваши мысли, как раскрытую книгу.
— Надеюсь, что нет, — сказал Ретт, и впервые в глазах его что-то сверкнуло. — Ну, так как? Разрешаете вы мне одолжить вам деньги?
— А при чем же тут обман?
— Мы с вами будем конспираторами и обманем и Скарлетт и мистера Уилкса.
— О господи! Я не могу!
— Если Скарлетт узнает, что я замыслил что-то за ее спиной — даже для ее же блага.., ну, вы знаете нрав Скарлетт! Что же до мистера Уилкса, то боюсь, он откажется принять от меня любой заем. Так что ни один из них не должен знать, откуда деньги.
— Ах, я уверена, что мистер Уилкс не откажется, если поймет в чем дело. Он так любит Скарлетт.
— Да, я в этом не сомневаюсь, — ровным тоном произнес Ретт. — И все равно откажется. Вы же знаете, какие гордецы все эти Уилксы.
— О, господи! — воскликнула несчастная Мелани. — Хотела бы я… Право же, капитан Батлер, я не могу обманывать мужа.
— Даже чтобы помочь Скарлетт? — Вид у Ретта был очень обиженный. — А ведь она так любит вас! Слезы задрожали на ресницах Мелани.
— Вы же знаете, я все на свете готова для нее сделать. Я никогда-никогда не смогу расплатиться с ней за то, что она сделала для меня. Вы же знаете.
— Да, — коротко сказал он, — я знаю, что она для вас сделала. А не могли бы вы сказать мистеру Уилксу, что получили деньги по наследству от какого-нибудь родственника?
— Ах, капитан Батлер, у меня нет родственников, у которых был бы хоть пении в кармане.
— Ну, а если я пошлю деньги мистеру Уилксу по почте — так, чтобы он не узнал, от кого они пришли? Проследите вы за тем, чтобы он приобрел на них лесопилки, а не.., ну, словом, не раздал бы их всяким обнищавшим бывшим конфедератам?
Сначала Мелани обиделась на его последние слова, усмотрев в них порицание Эшли, но Ретт так понимающе улыбался, что она улыбнулась в ответ.
— Конечно, прослежу.
— Значит, договорились? Это будет нашей тайной?
— Но я никогда не имела тайн от мужа!
— Уверен в этом, мисс Мелли.
Глядя сейчас на него, она подумала, что всегда правильно о нем судила. А вот многие другие судили неправильно. Люди говорили, что он грубиян, и насмешник, и плохо воспитан, и даже бесчестен. Правда, многие вполне приличные люди признавали сейчас, что были не правы. Ну, а вот она с самого начала знала, что он отличный человек. Она всегда видела от него только добро, заботу, величайшее уважение и удивительное понимание! А как он любит Скарлетт! Как это мило с его стороны — найти такой обходной путь, чтобы снять со Скарлетт одну из ее забот!
И в порыве чувств Мелани воскликнула:
— Какая же Скарлетт счастливица, что у нее такой муж, который столь добр к ней!
— Вы так думаете? Боюсь, она не согласилась бы с вами, если бы услышала. А кроме того, я хочу быть добрым и к вам, мисс Мелли. Вам я даю больше, чем даю Скарлетт.
— Мне? — с удивлением переспросила она. — Ах, вы хотите сказать — для Бо?
Он нагнулся, взял свою шляпу и встал. С минуту он стоял и смотрел вниз на некрасивое личико сердечком с длинным мысиком волос на лбу, на темные серьезные глаза. Какое неземное лицо, лицо человека, совсем не защищенного от жизни.
— Нет, не для Бо. Я пытаюсь дать вам нечто большее, чем Бо, если вы можете представить себе такое.
— Нет, не могу, — сказала она, снова растерявшись, — На всем свете для меня нет ничего дороже Бо, кроме Эшли… То есть мистера Уилкса.
Ретт молчал и только смотрел на нее, смуглое лицо его было непроницаемо.
— Вы такой милый, что хотите что-то сделать для меня, капитан Батлер, но право же, я совершенно счастлива. У меня есть все, чего может пожелать женщина.
— Вот и прекрасное — сказал Ретт, вдруг помрачнев. — И уж я позабочусь о том, чтобы так оно и оставалось.




Когда Скарлетт вернулась из Тары, нездоровая бледность исчезла с ее лица, а щеки округлились и были розовые. В зеленых глазах ее снова появилась жизнь, они сверкали, как прежде, и впервые за многие недели она громко рассмеялась при виде Ретта и Бонни, которые встречали ее, Уэйда и Эллу на вокзале, — рассмеялась, потому что уж больно нелепо и смешно они выглядели. У Ретта из-за ленточки шляпы торчали два растрепанных индюшачьих пера, а у Бонни, чье воскресное платье было основательно порвано, на обеих щеках виднелись полосы синей краски и в кудрях торчало петушиное перо, свисавшее чуть не до пят. Они явно играли в индейцев, когда подошло время ехать к поезду, и по озадаченно беспомощному виду Ретта и возмущенному виду Мамушки ясно было, что Бонни отказалась переодеваться — даже чтобы встречать маму.
Скарлетт заметила: «Что за сорванец!» — и поцеловала малышку, а Ретту подставила щеку для поцелуя. На вокзале было много народу, иначе она не стала бы напрашиваться на эту ласку.
Она не могла не заметить, хоть и была смущена видом Бонни, что все улыбаются, глядя на отца и дочку, — улыбаются не с издевкой, а искренне, по-доброму. Все знали, что младшее дитя Скарлетт держит отца в кулачке, и Атланта, забавляясь, одобрительно на это, взирала. Великая любовь к дочери существенно помогла Ретту восстановить свою репутацию в глазах общества.
По пути домой Скарлетт делилась новостями сельской жизни. Погода стояла сухая, жаркая, и хлопок рос не по дням, а по часам, но Уилл говорит, что цены на него осенью все равно будут низкими. Сьюлин снова ждет ребенка — Скарлетт так это сообщила, чтобы дети не поняли, — а Элла проявила неожиданный норов: взяла и укусила старшую дочку Сьюлин. Правда, заметила Скарлетт, и поделом маленькой Сьюзи: она вся Пошла в мать. Но Сьюлин вскипела, и между ними произошла сильная ссора — совсем как в старые времена. Уэйд собственноручно убил водяную змею. Рэнда и Камилла Тарлтон учительствуют в школе — ну, не смех? Ведь ни один из Тарлтонов никогда не мог написать даже слова «корова»! Бетси Тарлтон вышла замуж за какого-то однорукого толстяка из Лавджоя, и они с Хэтти и Джимом Тарлтоном выращивают хороший хлопок в Прекрасных Холмах. Миссис Тарлтон завела себе племенную кобылу с жеребенком и счастлива так, будто получила миллион долларов. А в бывшем доме Калвертов живут негры! Целый выводок, причем дом-то теперь — их собственный! Они купили его с торгов. Дом совсем разваливается — смотреть больно. Куда девалась Кэтлин и ее никудышный муженек — никто не знает. А Алекс собирается жениться на Салли, вдове собственного брата! Подумать только, после того, как они прожили в одном доме столько лет! Все говорят, что они решили обвенчаться для удобства, потому что пошли сплетни: ведь они жили там одни с тех пор, как Старая Хозяйка и Молодая Хозяйка умерли. Известие об их свадьбе чуть не разбило сердце Димити Манро. Но так ей и надо. Будь она чуточку порасторопнее, она бы давно подцепила себе другого, а не ждала бы, пока Алекс накопит денег, чтобы жениться на ней.
Скарлетт весело болтала, выплескивая новости, но было много такого, что она оставила при себе, — такого, о чем было больно даже думать. Она ездила по округе с Уиллом, стараясь не вспоминать то время, когда эти тысячи акров плодородной земли стояли в зелени кустов хлопчатника. А теперь плантацию за плантацией пожирал лес, и унылый ракитник, чахлые дубки и низкорослые сосны исподволь выросли вокруг молчаливых развалин, завладели бывшими хлопковыми плантациями. Там, где прежде сотня акров была под плугом, сейчас хорошо, если хоть один обрабатывался. Казалось, будто едешь по мертвой земле.
«В этих краях если все назад и вернется, так не раньше, чем лет через пятьдесят, — заметил Уилл. — Тара — лучшая ферма в округе, благодаря вам и мне, Скарлетт, но это только ферма, ферма, которую обрабатывают два мула, а вовсе не плантация. За нами идут Фонтейны, а потом Тарлтоны. Больших денег они не делают, но перебиваются, и у них есть сноровка. А почти все остальные, остальные фермы…» Нет, Скарлетт не хотелось вспоминать, как выглядит тот пустынный край. Сейчас же, оглядываясь назад из шумной, процветающей Атланты, она и вовсе загрустила.
— А какие здесь новости? — поинтересовалась она, когда они, наконец, прибыли домой и уселись на парадном крыльце. Всю дорогу она без умолку болтала, боясь, что может наступить гнетущее молчание. Она ни разу не разговаривала с Реттом наедине с того дня, когда упала с лестницы, и не слишком стремилась оказаться с ним наедине теперь. Она не знала, как он к ней относится. Он был сама доброта во время ее затянувшегося выздоровления, но это была доброта безликая, доброта чужого человека. Он предупреждал малейшее ее желание, удерживал детей на расстоянии, чтобы они не беспокоили ее, и вел дела в лавке и на лесопилках. Но он ни разу не сказал: «Мне жаль, что так получилось». Что ж, возможно, он ни о чем и не жалел. Возможно, он до сих пор считает, что это неродившееся дитя было не от него. Откуда ей знать, какие мысли сокрыты за этой любезной улыбкой на смуглом лице? Но он впервые за их супружескую жизнь старался держаться обходительно и выказывал желание продолжать совместную жизнь, как если бы ничего неприятного не стояло между ними, — как если бы, невесело подумала Скарлетт, как если бы между ними вообще никогда ничего не было. Что ж, если он так хочет, она поведет игру по его правилам.
— Так все у нас в порядке? — повторила она. — А вы достали новую дранку для лавки? Поменяли мулов? Ради всего святого, Ретт, выньте вы эти перья из шляпы. У вас вид шута — вы можете забыть про них и еще поедете так в город.
— Нет, — заявила Бонни и на всякий случай забрала шляпу у отца.
— Все у нас здесь шло преотлично, — сказал Ретт. — Мы с Бонни очень мило проводили время и, по-моему, с тех пор как вы уехали, ни разу не расчесывали ей волосы. Не надо сосать перья, детка, они могут быть грязные. Да, крыша покрыта новой дранкой, и я хорошо продал мулов. В общем-то, особых новостей нет. Все довольно уныло. — И, словно спохватившись, добавил: — Кстати, вчера вечером у нас тут был многоуважаемый Эшли. Хотел выяснить, не знаю ли я, не согласитесь ли вы продать ему свою лесопилку и свою долю в той лесопилке, которой управляет он.
Скарлетт, покачивавшаяся в качалке, обмахиваясь веером из индюшачьих перьев, резко выпрямилась.
— Продать? Откуда, черт побери, у Эшли появились деньги? Вы же знаете, у них никогда не было ни цента. Мелани сразу тратит все, что он ни заработает.
Ретт пожал плечами.
— Я всегда считал ее экономной маленькой особой, но я, конечно, не столь хорошо информирован об интимных подробностях жизни семьи Уилксов, как, видимо, вы.
Это уже был почти прежний Ретт, и Скарлетт почувствовала нарастающее раздражение.
— Беги, поиграй, детка, — сказала она Бонни. — Мама хочет; поговорить с папой.
— Нет, — решительно заявила Бонни и залезла к Ретту на колени.
Скарлетт насупилась, и Бонни в ответ состроила ей рожицу, столь напоминавшую Скарлетт Джералда, что она чуть не прыснула со смеху.
— Пусть сидит, — примирительно сказал Ретт. — Что же до денег, то их прислал ему как будто кто-то, кого он помог выходить во время эпидемии оспы в Рок-Айленде. Я начинаю вновь верить в человека, когда узнаю, что благодарность еще существует.
— Кто же это? Кто-то из знакомых?
— Письмо не подписано, прибыло оно из Вашингтона. Эшли терялся в догадках, кто бы мог ему эти деньги послать. Но когда человек, отличающийся такой жертвенностью, как Эшли, разъезжая по свету, направо и налево творит добро, разве может он помнить обо всех своих деяниях.
Не будь Скарлетт столь удивлена этим неожиданным счастьем, свалившимся на Эшли, она подняла бы перчатку, хотя и решила в Таре, что никогда больше не станет ссориться с Реттом из-за Эшли. Слишком она была сейчас не уверена в своих отношениях с обоими мужчинами, и до тех пор, пока они не прояснятся, ей не хотелось ни во что ввязываться.
— Значит, он хочет выкупить у меня лесопилки?
— Да. Но я, конечно, сказал ему, что вы не продадите.
— А я бы попросила позволить мне самой вести свои дела.
— Ну, вы же знаете, что не расстанетесь с лесопилками. Я сказал, что ему, как и мне, известно ваше стремление верховодить всеми и вся, а если вы продадите ему лесопилки, то вы же не сможете больше его наставлять.
— И вы посмели сказать ему про меня такое?
— А почему бы и нет? Ведь это же правда! По-моему, он искренне со мною согласился, но он, конечно, слишком джентльмен, чтобы взять и прямо мне об этом сказать.
— Все это ложь! Я продам ему лесопилки! — возмущенно воскликнула Скарлетт.
До этого момента ей и в голову не приходило расставаться с лесопилками. У нее было несколько причин сохранять их, причем наименее существенное были деньги. В последние два-три года она могла бы их продать, когда только ей заблагорассудится, причем за довольно крупную сумму, но она отклоняла все предложения.
Эти лесопилки были реальным доказательством того, чего она сумела достичь сама, вопреки всему, и она гордилась ими и собой. Но главным образом ей не хотелось продавать их потому, что они были единственным связующим звеном между нею и Эшли. Если лесопилки уйдут из ее рук, она будет редко видеться с Эшли, а наедине, по всей вероятности, и вовсе никогда. А она должна увидеться с ним наедине. Не может она дольше так жить, не зная, что он теперь к ней испытывает, не зная, сгорела ли его любовь от стыда после того страшного вечера, когда Мелани устроила прием. Во время деловых свиданий она могла найти немало поводов для разговора — так, что никто бы не догадался, что она специально ищет с ним встречи. А со временем, Скарлетт знала, она бы полностью вернула себе то место, которое прежде занимала в его сердце. Но если она продаст лесопилки…
Нет, она не собиралась их продавать, но мысль о том, что Ретт выставил ее перед Эшли в столь правдивом и столь неблаговидном свете, мгновенно заставила ее передумать. Надо отдать эти лесопилки Эшли — и по такой низкой цене, что ее великодушие сразу бросится ему в глаза.
— Я продам их ему! — разозлившись, воскликнула она. — Ну, что вы теперь скажете?
Глаза Ретта еле заметно торжествующе сверкнули, и он наклонился, чтобы завязать Бонни шнурок.
— Я скажу, что вы об этом будете жалеть, — заметил он. И она уже жалела, что произнесла эти поспешно вылетевшие слова. Скажи она их кому угодно, кроме Ретта, она бы без всякого стеснения взяла их назад. И зачем ей понадобилось так спешить? Она насупилась от злости и посмотрела на Ретта, а он смотрел на нее с этим своим обычным настороженным выражением — так кот наблюдает за мышиной норой. Увидев, что она нахмурилась, он вдруг рассмеялся, обнажив белые зубы. И у Скарлетт возникло смутное чувство, что он ловко ее провел.
— Вы что, имеете к этому какое-то отношение? — резко спросила она.
— Я? — Брови его поднялись в насмешливом удивлении. — Вам бы следовало лучше меня знать. Я не разъезжаю по свету, направо и налево творя добро — без крайней необходимости.




В тот вечер она продала Эшли обе лесопилки. Она ничего на этом не прогадала, ибо Эшли не принял ее предложения и купил их по самой высокой цене, какую ей когда-либо предлагали. Когда бумаги были подписаны и лесопилки навсегда ушли из ее рук, а Мелани подавала Эшли и Ретту рюмки с вином, чтобы отпраздновать это событие, Скарлетт почувствовала себя обездоленной, словно продала одного из детей.
Лесопилки были ее любимым детищем, ее гордостью, плодом труда ее маленьких цепких рук. Она начала с небольшой лесопилки в те черные дни, когда Атланта только поднималась из пепла и развалин и не было спасения от нужды. Скарлетт сражалась, интриговала, оберегая свои лесопилки в те мрачные времена, когда янки грозили все конфисковать, когда денег было мало, а ловких людей расстреливали. И вот у Атланты стали зарубцовываться раны, повсюду росли дома, и в город каждый день стекались пришельцы, а у Скарлетт было две отличных лесопилки, два лесных склада, несколько десятков мулов и команды каторжников, работавшие за сущую ерунду. Теперь, прощаясь со всем этим, она как бы навеки запирала дверь, за которой оставалась та часть ее жизни, когда было много горечи и забот, он она вспоминала эти годы с тоской и удовлетворением.
Она ведь создала целое дело, а теперь продала его, и ее угнетала уверенность в том, что если ее не будет у кормила, Эшли все потеряет — все, что ей стоило таких трудов создать. Эшли всем верит и до сих пор не может отличить доски два на четыре от доски шесть на восемь. А теперь она уже не сможет помочь ему своими советами — и только потому, что Ретт изволил сказать Эшли, как она-де любит верховодить.
«О, черт бы подрал этого Ретта», — подумала она и, наблюдая за ним, все больше убеждалась, что вся, эта затея исходит от его. Как это произошло и почему — она не знала. Он в эту минуту беседовал с Эшли, и одно его замечание заставило ее насторожиться.
— Я полагаю, вы тотчас откажетесь от каторжников, — говорил он.
Откажется от каторжников? Почему, собственно, надо от них отказываться? Ретт прекрасно знал, что лесопилки приносили такие больше доходы только потому, что она пользовалась дешевым трудом каторжников. И почему это Ретт уверен, что Эшли будет поступать именно так, а не иначе? Что он знает о нем?
— Да, я тотчас отправлю их назад, — ответил Эшли, стараясь не смотреть на потрясенную Скарлетт.
— Вы что, потеряли рассудок? — воскликнула она. — Вам же не вернут денег, которые заплачены за них по договору, да и кого вы сумеете потом нанять?
— Вольных негров, — сказал Эшли.
— Вольных негров! Чепуха! Вы же знаете, сколько вам придется им платить, а кроме того, вы посадите себе на шею янки, которые будут ежеминутно проверять, кормите ли вы их курицей три раза в день и спят ли они под стеганым одеялом. Если же какому-нибудь лентяю вы дадите кнута, чтобы его подогнать, янки так разорутся, что их будет слышно в Далтоне, а, вы очутитесь в тюрьме. Да ведь каторжники — единственные…
Мелани сидела, уставив взгляд в сплетенные на коленях руки. Вид у Эшли был несчастный, но решительный. Какое-то время он молчал, потом глаза его встретились с глазами Ретта, и он увидел в них понимание и поощрение — Скарлетт это заметила.
— Я не буду пользоваться трудом каторжников, Скарлетт, — спокойно сказал Эшли.
— Ну, скажу я вам, сэр! — Скарлетт даже задохнулась. — А почему нет? Или вы что, боитесь, что люди станут говорить о вас так же, как говорят обо мне?
Эшли поднял голову.
— Я не боюсь того, что скажут люди, если я поступаю как надо. А я всегда считал, что пользоваться трудом каторжников — не надо.
— Но почему…
— Я не могу наживать деньги На принудительном труде и несчастье других…
— Но у вас же были рабы!
— Они жили вполне пристойно. А кроме того, после смерти отца я бы всех их освободил, но война освободила их раньше. А каторжники — это совсем другое дело, Скарлетт. Сама система их найма дает немало возможностей для надругательства над ними. Вы, возможно, этого не знаете, а я знаю. Я прекрасно знаю, что Джонни Гэллегер по крайней мере одного человека в лагере убил. А может быть, и больше — кто станет волноваться по поводу того, что одним каторжником стало меньше? Джонни говорит, то тот человек был убит при попытке к бегству, но я слышал другое. И я знаю, что он заставляет работать больных людей. Можете называть это суеверием, но я неубежден, что деньги, нажитые на страданиях, могут принести счастье.
— Чтоб вам пропасть! Вы что же, хотите сказать.., господи, Эшли, неужели вы купились на эти разглагольствования преподобного Уоллеса насчет грязных денег?
— Мне не надо было покупаться. Я был убежден в этом задолго до того, как Уоллес начал произносить свои проповеди.
— Тогда, значит, вы считаете, что все мои деньги — грязные? — воскликнула Скарлетт, начиная злиться. — Потому что на меня работали каторжники, и у меня есть салун, и…
Она вдруг умолкла. Вид у обоих Уилксов был смущенный, а Ретт широко улыбался. «Черт бы его подрал, — в пылу гнева подумала Скарлетт. — Он, как и Эшли, считает, что я опять суюсь не в свои дела. Так бы взяла и стукнула их головами, чтобы лбы затрещали!..» Она постаралась проглотить свой гнев и принять вид оскорбленного достоинства, но это ей не слишком удалось.
— В общем-то, меня ведь это не касается, — промолвила она.
— Скарлетт, только не считайте, что я осуждаю вас! Ничего подобного! Просто мы по-разному смотрим на многое и то, что хорошо для вас, может быть совсем не хорошо для меня.
Ей вдруг захотелось остаться с ним наедине, отчаянно захотелось, чтобы Ретт и Мелани были на другом конце света, и тогда она могла бы крикнуть ему: «Но я хочу смотреть на все так же, как ты! Скажи мне только — как, чтобы я поняла и стала такой же!» Но в присутствии Мелани, которую от огорчения била дрожь, и Ретта, стоявшего прислонясь к стене и с усмешкой глядевшего на нее, она могла лишь сказать как можно более холодно, оскорбленным тоном:
— Конечно, это ваше дело, Эшли, и я и не помышляю учить вас, как и что делать. Но я все же должна сказать, что не понимаю вашей позиции и ваших суждений.
Ах, если они были одни и она не была вынуждена говорить с ним так холодно, произносить эти слова, которые огорчали его!
— Я обидел вас, Скарлетт, хотя вовсе этого не хотел. Поверьте и простите меня. В том, что я сказал, нет ничего непонятного. Просто я действительно верю, что деньги, нажитые определенным путем, редко приносят счастье.
— Но вы не правы! — воскликнула она, не в силах больше сдерживаться. — Посмотрите на меня! Вы же знаете, откуда у меня деньги. И вы знаете, как обстояло дело до того, как они у меня появились! Вы же помните ту зиму в Таре, когда в доме стоял такой холод, и мы резали ковры, чтобы сделать подметки для туфель, и нечего было есть, и мы ломали голову, не зная, где будем брать деньги на обучение Бо и Уэйда! Пом…
— Я все помню, — устало сказал Эшли, — но я предпочел бы это забыть.
— Ну, вы едва ли можете сказать, что кто-либо из нас был тогда счастлив, верно? А посмотрите на нас сейчас! У вас милый дом и хорошее будущее. А есть ли у кого-нибудь более красивый дом, чем у меня, или более нарядные платья, или лучшие лошади? Ни у кого нет такого стола, как у меня, никто не устраивает лучших приемов, и у моих детей есть все, что они хотят. Ну, а откуда я взяла деньги, чтобы все это стало возможным? Сорвала с дерева? Нет, сэр! Каторжники и арендная плата с салуна, и…
— Не забудьте про того убитого янки, — вставил Ретт. — Ведь это с него все началось.
Скарлетт стремительно повернулась к нему, резкие слова уже готовы были сорваться у нее с языка.
— И эти деньги сделали вас очень, очень счастливой, верно, дорогая? — спросил он этаким сладким, ядовитым тоном.
Скарлетт поперхнулась, открыла было рот, быстро оглядела всех троих. Мелани чуть не плакала от неловкости, Эшли вдруг помрачнел и замкнулся, а Ретт наблюдал за ней поверх своей сигары и явно забавлялся. Она хотела было крикнуть: «Ну конечно же, они сделали меня счастливой!» Но почему-то не смогла этого произнести.

