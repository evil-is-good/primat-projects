\chapter{\ }

Снова очутившись в безопасности и уединении своей комнаты, Скарлетт бросилась на постель как была — в муаровом платье, не заботясь о его турнюре и розах. Какое-то время она лежала неподвижно, не в силах думать ни о чем, кроме того, как она стояла между Мелани и Эшли и принимала гостей. Какой ужас! Да она готова скорее встретиться лицом к лицу со всей армией Шермана, чем повторить такое! Наконец она поднялась и нервно зашагала по комнате, сбрасывая с себя на ходу одежду.
После напряжения наступила реакция, и ее затрясло. Шпильки вываливались у нее из пальцев и со звоном падали на пол, а когда она попыталась по обыкновению расчесать волосы, то больно ударила себя щеткой по виску. Раз десять она подходила на цыпочках к двери, чтобы послушать, что происходит внизу, но в холле, точно в бездонном колодце, царила тишина.
Когда праздник окончился, Ретт отослал ее домой в коляске, и она благодарила бога за эту передышку. Сам Ретт еще не вернулся. Слава богу, еще нет. Она просто не в силах предстать перед ним сегодня — опозоренная, испуганная, трясущаяся. Но где все-таки он? Скорее всего у этой твари. Впервые Скарлетт была рада, что на свете существует Красотка Уотлинг, была рада, что, кроме этого дома, у Ретта есть другое прибежище, где он может побыть, пока у него не пройдет этот приступ холодной светскости, граничащей с жестокостью. Плохо, конечно, радоваться тому, что твой муж находится у проститутки, но она ничего не могла с собой поделать. Она бы предпочла видеть его мертвым, лишь бы это избавило ее от сегодняшней встречи с ним.
Завтра — ну, завтра это уже другое дело. Завтра она сможет придумать какое-то оправдание, какие-то ответные обвинения, какой-то способ свалить на него всю вину. Завтра воспоминания об этом жутком вечере уже не будут вызывать у нее такой дрожи. Завтра ее уже не будет преследовать лицо Эшли, воспоминание об его сломленной гордости и позоре — позоре, который навлекла на него она, тогда как он был ни в чем не повинен. Неужели он теперь возненавидит ее — он, ее дорогой благородный Эшли, — за то, что она опозорила его? Конечно, возненавидит — тем более что спасла их Мелани своими возмущенно распрямленными плечиками, любовью и доверием, какие звучали в ее голосе, когда она, скользнув к Скарлетт по натертому полу, обняла ее и стала с ней рядом, лицом к любопытно-ехидной, скрыто враждебной толпе. Как она точно заклеила прорезь, сквозь которую мог прорваться скандал, продержав возле себя Скарлетт весь этот страшный вечер. Люди были холодны с ней, несколько ошарашены, но — вежливы.
Ах, как это унизительно — спасаться за юбками Мелани от тех, кто так ненавидит ее, кто разорвал бы ее на куски своим перешептыванием! Спасаться с помощью слепой веры Мелани — именно Мелани!
При мысли об этом по телу Скарлетт пробежал озноб. Она должна выпить, выпить как следует, прежде чем сможет лечь и попытается уснуть. Она накинула поверх капота на плечи шаль и поспешила вниз, в темный холл, — ее ночные туфли без пяток громко хлопали в тишине. Она была уже на середине лестницы, когда! увидела тоненькую полоску света, пробивавшуюся из-под закрытой двери в столовую. Сердце у нее на секунду перестало биться. Свет уже горел там, когда она вернулась домой, а она была слишком расстроена и не заметила? Или же Ретт все-таки дома? Он ведь мог войти потихоньку, через кухонную дверь. Если Ретт дома, она тотчас же на цыпочках вернется к себе и ляжет в постель без коньяка, хоть ей и очень нужно было бы выпить. Тогда ей не придется встречаться с Реттом. У себя в комнате она будет в безопасности: можно ведь запереть дверь.
Но только она нагнулась, чтобы снять ночные туфли и тихонько вернуться назад, как дверь в столовую распахнулась и при неверном свете свечи в проеме возник силуэт Ретта. Он казался огромным, необычайно широким — жуткая черная безликая фигура, которая стояла и слегка покачивалась.
— Прошу вас, составьте мне компанию, миссис Батлер, — сказал он, и голос его звучал чуть хрипло.
Он был пьян, и это бросалось в глаза, а она никогда еще не видела, чтобы Ретт был так пьян, что это бросалось в глаза. Она в нерешительности медлила, и, поскольку не говорила ни «да», ни «нет», он повелительно взмахнул рукой.
— Да идите же сюда, черт бы вас побрал! — грубо рявкнул он.
«Должно быть, он очень пьян», — подумала она, и сердце ее отчаянно заколотилось. Обычно чем больше он пил, тем вежливее становился. Чаще язвил, больнее жалил словами, но держался при этом всегда церемонно — подчеркнуто церемонно.
«Я не должна показывать ему, что боюсь», — подумала она и, плотнее закутавшись в шаль, пошла вниз по лестнице, высоко подняв голову, громко стуча каблуками.
Он отступил в сторону и с поклоном пропустил ее в дверь, — в этом поклоне была такая издевка, что она внутренне содрогнулась. Она увидела, что он снял фрак и развязал галстук — концы его болтались по обеим сторонам распахнутого воротничка. Из-под расстегнутой на груди рубашки торчала густая черная шерсть. Волосы у него были взъерошены, налитые кровью глаза прищурены. На столе горела свеча — маленькая точка света, громоздившая тени в высокой комнате, превращая массивные шкафы и буфет в застывшие, притаившиеся чудовища. На столе стоял серебряный поднос; на нем — хрустальный графин с лежавшей рядом пробкой и рюмки.
— Садитесь, — отрывисто приказал Ретт, проходя следом за ней в комнату.
Новый, неведомый дотоле страх овладел Скарлетт, — страх, по сравнению с которым боязнь встретиться с Реттом лицом к лицу казалась ерундой. Он выглядел и говорил и вел себя сейчас, как чужой человек. Перед ней был Ретт-грубиян — таким она прежде никогда его не видела. Никогда, даже в самые интимные минуты, он не был таким — в худшем случае проявлял к ней небрежение. Даже в гневе он был мягок и ехиден, а виски обычно лишь обостряло его ехидство. Сначала Скарлетт злилась и пыталась сломить его небрежение, но вскоре смирилась — ее это даже устраивало. Долгое время она считала, что ему все безразлично и что ко всему в жизни, включая ее, он относится не всерьез, а как к шутке. Но сейчас, глядя на него через стол, она поняла — и у нее засосало под ложечкой, — что наконец появилось что-то ему небезразличное, далеко не безразличное.
— Не вижу оснований, почему бы вам не выпить на ночь, даже если я плохо воспитан и сегодня явился ночевать домой, — сказал он. — Налить?
— Я вовсе не собиралась пить, — сухо ответила она. — Просто услышала шум и спустилась.
— Ничего вы не слышали. И не стали бы вы спускаться, если б знали, что я дома. А я сидел здесь и слушал, как вы бегаете у себя по комнате. Вам, видно, очень нужно выпить. Так выпейте.
— Я вовсе не…
Он взял графин и плеснул в рюмку коньяку, так что перелилось через край.
— Держите, — сказал он, всовывая ей рюмку в руку. — Вас всю трясет. Да перестаньте прикидываться. Я знаю, что вы втихую пьете, и знаю сколько. Я уже давно собирался вам сказать, чтоб вы перестали притворяться и пили в открытую, если вам охота. Вы что, думаете, меня хоть сколько-нибудь занимает то, что вы пристрастились к коньячку?
Она взяла мокрую рюмку, честя его про себя на чем свет стоит. Он читает ее мысли, как раскрытую книгу. Он всегда читал ее мысли, а как раз от него-то она и хотела их скрыть.
— Пейте же, говорю вам.
Она подняла рюмку и резким движением руки, не сгибая запястья, опрокинула содержимое себе в рот — совсем как это делал Джералд, когда пил чистое виски, — опрокинула, не думая о том, каким привычным и неженским выглядит этот жест. Ретт не преминул это отметить, и уголок его рта пополз вниз.
— Присядьте, и давайте мило, по-домашнему поговорим об изысканном приеме, на котором мы только что побывали.
— Вы пьяны, — холодно сказала она, — а я хочу лечь.
— Я очень пьян и намерен надраться еще больше до конца вечера. А вы никуда не пойдете и не ляжете — пока. Садитесь же.
Голос его звучал все так же подчеркнуто холодно и тягуче, но она почувствовала за этими словами рвущуюся наружу ярость — ярость безжалостную, как удар хлыста. Она колебалась, не зная, на что решиться, но он уже стоял рядом и крепко схватил ее за руку, причинив ей боль. Он слегка вывернул ей руку, и, вскрикнув от боли, Скарлетт поспешила сесть. Вот теперь ей стало страшно — так страшно, как еще никогда в жизни. Когда он нагнулся к ней, она увидела, что лицо у него темно-багровое, а глаза по-прежнему угрожающе сверкают. И было что-то в их глубине, чего она прежде не видела, не могла понять, что-то более сильное, чем гнев, более сильное, чем боль, владело им, отчего глаза его сейчас горели красноватым огнем, как два раскаленных угля. Он долго смотрел на нее — сверху вниз, — так долго, что, не в силах сохранить вызывающий вид, она вынуждена была опустить глаза; тогда он тяжело рухнул в кресло напротив нее и налил себе еще коньяку. Мозг Скарлетт лихорадочно работал, придумывая систему обороны. Но пока Ретт не заговорит, ей ведь трудно что-то сказать, ибо она в точности не знала, в чем он ее обвиняет.
Он медленно пил, наблюдая за ней поверх края рюмки, и она вся напряглась, стараясь сдержать дрожь. Какое-то время лицо его оставалось застывшим, потом он рассмеялся, продолжая на нее смотреть, и при звуке его смеха ее снова затрясло.
— Забавная была комедия сегодня вечером, верно? Она молчала, лишь поджала пальцы в свободных туфлях, надеясь, что, быть может, это уймет ее дрожь.
— Прелестная комедия со всеми необходимыми действующими лицами. Селяне, собравшиеся, чтобы закидать камнями падшую женщину; опозоренный муж, поддерживающий свою жену, как и подобает джентльмену; опозоренная жена, в порыве христианского милосердия широко раскинувшая крылья своей безупречной репутации, чтобы все ими прикрыть. И любовник…
— Прошу вас.
— Не просите. Во всяком случае, сегодня. Слишком уж все это занимательно. И любовник, выглядящий абсолютным идиотом и мечтающий уж лучше умереть. Как вы чувствовали себя, моя дорогая, когда женщина, которую вы ненавидите, стояла рядом с вами и прикрывала ваши грехи? Садитесь же.
Она села.
— Вы, я полагаю, едва ли больше ее за это полюбили? Вы сейчас раздумываете, знает ли она все про вас и Эшли.., раздумываете, почему она так поступила, если знает.., и, быть может, она это сделала только для того, чтобы спасти собственное лицо. И вы считаете, что она дурочка, хотя это спасло вашу шкуру. Тем не менее…
— Я не желаю больше слушать…
— Нет, вы меня выслушаете. И я скажу вам это, чтобы избавить вас от лишних волнений. Мисс Мелли действительно дурочка, но не в том смысле, как вы думаете. Ей, конечно же, кто-то все рассказал, но она этому не поверила. Даже если бы она увидела своими глазами, все равно бы не поверила. Слишком много в ней благородства, чтобы она могла поверить в отсутствие благородства у тех, кого любит. Я не знаю, какую ложь сказал ей Эшли Уилкс.., но любая самая неуклюжая ложь сойдет, ибо она любит Эшли и любит вас. Честно говоря, не понимаю, почему она вас любит, но факт остается фактом. Так что придется вам нести и этот крест.
— Если бы вы не были так пьяны и не вели себя так оскорбительно, я бы все вам объяснила, — сказала Скарлетт, к которой в какой-то мере вернулось самообладание. — Но сейчас…
— А меня не интересуют ваши объяснения. Я знаю правду лучше, чем вы. Клянусь богом, если вы еще хоть раз встанете с этого стула… Однако куда больше, чем сегодняшняя комедия, меня забавляет то, что вы из высокоцеломудренных соображений отказывали мне в радостях супружеского ложа из-за моих многочисленных грехов, а сами вожделели в душе Эшли Уилкса. «Вожделели в душе» — хорошее выражение, верно? Сколько хороших выражений в этой книжице, правда?
«Какой книжице? Какой?» — мысли ее метались, глупо, бессмысленно, а глаза испуганно озирали комнату, машинально отмечая тусклый блеск тяжелого серебра в неверном свете свечи, пугающую темноту в углах.
— Я был изгнан потому, что моя грубая страсть оскорбляла ваши утонченные чувства.., и потому, что вы не хотели больше иметь детей. Бог ты мой, до чего же мне было больно! Как меня это ранило! Что ж, я ушел из дома, нашел себе милые утехи, а вас предоставил вашим утонченным чувствам. Вы же все это время гонялись за многострадальным мистером Уилксом. Ну, а он-то, черт бы его подрал, чего мучается? В мыслях не может сохранять верность жене и не может физически быть ей неверным. Неужели нельзя на что-то решиться — раз и навсегда? Вы бы не возражали рожать ему детей, верно.., и выдавать их за моих?
Она с криком вскочила на ноги, но он мгновенно поднялся с места и с легким смешком, от которого у Скарлетт кровь застыла в жилах, крепко прижал ее плечи к спинке стула своими большими смуглыми руками и заставил снова сесть.
— Посмотрите на мои руки, дорогая моя, — сказал он, сгибая их перед ней и разгибая. — Я мог бы запросто разорвать вас на куски и разорвал бы, лишь бы изгнать Эшли из ваших мыслей. Но это не поможет. Значит, придется убрать его из ваших мыслей иначе. Вот я сейчас возьму этими руками вашу голову и раздавлю как орех, так что никаких мыслей в ней не останется.
Он взял ее голову в руки, пальцы его погрузились в ее распущенные волосы, они ласкали — твердые, сильные, потом приподняли ее лицо. На нее смотрел совсем чужой человек — пьяный незнакомец, гнусаво растягивающий слова. Ей всегда была присуща этакая звериная смелость, и сейчас, перед лицом опасности, она почувствовала, как в ней забурлила кровь, и спина ее выпрямилась, глаза сузились, — Вы пьяный идиот, — сказала она. — Уберите прочь руки. К ее великому изумлению, он послушался и, присев на край стола, налил себе еще коньяку.
— Я всегда восхищался силой вашего духа, моя дорогая, а сейчас, когда вы загнаны в угол, — особенно.
Она плотнее запахнула на себе капот. Ах, если бы только она могла добраться до своей комнаты, повернуть ключ в замке и остаться одна за толстыми дверями! Она как-то должна удержать его на расстоянии, подчинить себе этого нового Ретта, какого она прежде не видела. Она не спеша поднялась, хотя у нее тряслись колени, крепче стянула полы капота на бедрах и отбросила волосы с лица.
— Ни в какой угол вы меня не загнали, — колко сказала она. — Вам никогда не загнать меня в угол, Ретт Батлер, и не напугать. Вы всего лишь пьяное животное и так долго общались с дурными женщинами, что все меряете их меркой. Вам не понять Эшли или меня. Слишком долго вы жили в грязи, чтобы иметь представление о чем-то другом. И вы ревнуете к тому, чего не в состоянии понять. Спокойной ночи!
Она повернулась и направилась к двери — оглушительный хохот остановил ее. Она повернула голову — Ретт шел, пошатываясь, за ней. О боже правый, только бы он перестал хохотать! Да и вообще — что тут смешного? Он почти настиг ее — она попятилась к двери и почувствовала, что спиной уперлась в стену. Ретт тяжело положил руки ей на плечи и прижал к стене.
— Перестаньте смеяться.
— Я смеюсь потому, что мне жаль вас.
— Жаль — меня? Жалели бы себя.
— Да, клянусь богом, мне жаль вас, моя дорогая, моя прелестная маленькая дурочка. Больно, да? Вы не выносите ни смеха, ни жалости?
Он перестал смеяться и всей своей тяжестью навалился на нее, так что ей стало больно. Лицо его изменилось, он находился так близко, что тяжелый запах коньяка заставил ее отвернуть голову.
— Значит, я ревную? — сказал он. — А почему бы и нет? О да, я ревную к Эшли Уилксу. Почему бы и нет? Только не говорите и не старайтесь что-то мне объяснить. Я знаю, физически вы были верны мне. Это вы и пытались сказать? О, я все время это знал. Все эти годы. Каким образом знал? Просто потому, что я знаю Эшли Уилкса и людей его сорта. Знаю, что он человек порядочный и благородный. А вот о вас, моя дорогая, я так сказать не могу. Да и о себе тоже. Мы с вами благородством не отличаемся, и у нас нет понятия чести, верно? Вот потому мы и цветем, как вечнозеленый лавр.
— Отпустите меня. Я не желаю стоять здесь и подвергаться оскорблениям.
— Я вас не оскорбляю. Я превозношу вашу физическую добродетель. Вам ведь ни разу не удалось меня провести. Вы считаете, Скарлетт, что мужчины — круглые дураки. А никогда не стоит недооценивать силу и ум противника. Так что я не дурак. Вы думаете, я не знаю, что, лежа в моих объятиях, вы представляли себе, будто я — Эшли Уилкс?
Она невольно раскрыла рот — лицо ее выражало страх и неподдельное удивление.
— Приятная это штука. Немного, правда, похоже на игру в призраки. Все равно как если бы в кровати вдруг оказалось трое вместо двоих. — Он слегка встряхнул ее за плечи, икнул и насмешливо улыбнулся. — О да, вы были верны мне, потому что Эшли вас не брал. Но, черт подери, я бы не стал на него злиться, овладей он вашим телом. Я знаю, сколь мало значит тело — особенно тело женщины. Но я злюсь на него за то, что он овладел вашим сердцем и вашей бесценной, жестокой, бессовестной, упрямой душой. А ему, этому идиоту, не нужна ваша душа, мне же не нужно ваше тело. Я могу купить любую женщину задешево. А вот вашей душой и вашим сердцем я хочу владеть, но они никогда не будут моими, так же как и душа Эшли никогда не будет вашей. Вот потому-то мне и жаль вас.
Несмотря на обуревавшие ее страх и смятение, Скарлетт больно ранили его издевки.
— Вам жаль — меня?
— Да, жаль, потому что вы такое дитя, Скарлетт. Дитя, которое плачет оттого, что не может получить луну. А что бы стало дитя делать с луной? И что бы вы стали делать с Эшли? Да, мне жаль вас — жаль, что вы обеими руками отталкиваете от себя счастье и тянетесь к чему-то, что никогда не сделает вас счастливой. Жаль, что вы такая дурочка и не понимаете, что счастье возможно лишь там, где схожие люди любят друг друга. Если б я умер и если бы мисс Мелли умерла и вы получили бы своего бесценного благородного возлюбленного, вы думаете, что были бы счастливы с ним? Черта с два — нет! Вы никогда бы так и не узнали его, никогда бы не узнали, о чем он думает, никогда бы не поняли его, как не понимаете музыку, поэзию, прозу, — вы же ни в чем не разбираетесь, кроме долларов и центов. А вот мы с вами, дражайшая моя супруга, могли бы быть идеально счастливы, если бы вы дали мне малейшую возможность сделать вас счастливой, потому что мы с вами — одного поля ягода. Мы оба мерзавцы, Скарлетт, и ни перед чем не остановимся, когда чего-то хотим. Мы могли бы быть счастливы, потому что я любил вас и знаю вас, Скарлетт, до мозга костей — так, как Эшли никогда вас не узнает. А если узнает, то будет презирать… Но нет, вы всю жизнь прогоняетесь за человеком, которого вы не можете понять. А я, дорогая моя, буду гоняться за шлюхами. И все же смею надеяться, жизнь у нас сложится лучше, чем у многих других пар.
Он внезапно отпустил ее и сделал несколько нетвердых шагов к графину. Какое-то мгновение Скарлетт стояла неподвижно, точно приросла к месту, — мысли так стремительно проносились у нее в мозгу, что она не могла сосредоточиться ни на одной. Ретт сказал, что любил ее. Это действительно так? Или он сболтнул спьяну? Или это просто одна из его отвратительных шуточек? А Эшли — недостижимый, как луна… И она плачет, потому что не может получить луну. Она выскочила в темный холл и помчалась, точно демоны гнались за ней. Ах, если бы только добраться до своей комнаты! Она подвернула ногу, и ночная туфля соскочила. Она приостановилась, чтобы скинуть туфлю совсем, и тут в темноте ее настиг Ретт — он налетел бесшумно, как индеец. Она почувствовала на лице его горячее дыхание, руки его резко распахнули капот, обхватили ее нагое тело.
— Меня вы заставили уехать из города, а сами принялись гоняться за этим своим Эшли. Клянусь богом, сегодня ночью в моей постели нас будет только двое.
Он подхватил ее на руки и понес вверх по лестнице. Голова ее была крепко прижата к его груди — Скарлетт слышала тяжелые удары его сердца. Ей было больно, и она вскрикнула, приглушенно, испуганно. А он шел все вверх и вверх в полнейшей тьме, и Скарлетт не помнила себя от страха. Она — на руках у чужого, обезумевшего человека, а вокруг — неведомая кромешная тьма, темнее смерти. И сам он точно смерть, которая несла ее, до боли сжимая в объятиях. Скарлетт снова глухо вскрикнула — он вдруг остановился, повернул ей голову и впился в нее таким неистовым поцелуем, что она забыла обо всем, — осталась лишь тьма, в которую она погружалась. Да его губы на ее губах. Он покачивался, точно под порывами сильного ветра, и губы его, оторвавшись от ее рта, скользнули вниз — туда, где распахнутый капот обнажал нежную кожу. Он шептал какие-то слова, которые она не могла разобрать, губы его рождали чувства, прежде ей неведомые. Тьма владела ею, тьма владела им, и его губы на ее теле. Она попыталась что-то сказать, и он тотчас снова закрыл ей рот поцелуем. И вдруг дотоле не познанный дикий вихрь восторга закружил ее — радость, страх, волнение, безумие, желание раствориться в этих сильных руках, под этими испепеляющими поцелуями, отдаться судьбе, которая стремительно несла ее куда-то. Впервые в жизни она встретила человека, который оказался сильнее ее, человека, которого она не смогла ни запугать, ни сломить, человека, который сумел запугать и сломить ее. И она вдруг почувствовала, что руки ее сами собой обвились вокруг его шеи и губы трепещут под его губами, и они снова поднимаются — все выше, выше, в темноте, темноте мягкой, кружащей голову, обволакивающей.




Когда на другое утро она проснулась, Ретта возле нее уже не было, и если бы не смятая подушка рядом, она могла бы подумать, что все происшедшее ночью приснилось ей в диком, нелепом сне. Она покраснела при одном воспоминании и, натянув до подбородка одеяло, продолжала лежать в солнечном свете, пытаясь разобраться в своих беспорядочных ощущениях.
Два обстоятельства обращали на себя внимание. Она прожила не один год с Реттом, спала с ним, ела с ним, ссорилась, родила ему ребенка — и, однако же, не знала его. Человек, поднявшийся с нею на руках по темным ступеням, был ей незнаком, о существовании его она даже не подозревала. И сейчас, сколько она ни старалась возненавидеть его, возмутиться, — она ничего не могла с собой поделать. Он унизил ее, причинил ей боль, делал с ней что хотел на протяжении всей этой дикой, безумной ночи, и она лишь упивалась этим.
А ведь должна была бы устыдиться, должна была бы бежать даже воспоминаний о жаркой, кружащей голову тьме! Леди, настоящая леди, не могла бы людям в глаза глядеть после такой ночи. Но над чувством стыда торжествовала память о наслаждении, об экстазе, охватившем ее, когда она уступила его ласкам. Впервые она почувствовала, что живет полной жизнью, почувствовала страсть столь же всеобъемлющую и первобытную, как страх, который владел ею в ту ночь, когда она бежала из Атланты, столь же головокружительно сладкую, как холодная ненависть, с какою она пристрелила того янки.
Ретт любит ее! Во всяком случае, он сказал, что любит, и разве может она сомневаться теперь? Как это странно и удивительно, как невероятно то, что он любит ее, — этот незнакомый дикарь, рядом с которым она прожила столько лет, не чувствуя ничего, кроме холода. Она не была вполне уверена в своем отношении к этому открытию, но тут ей в голову пришла одна мысль, и она громко рассмеялась. Он любит ее — значит, теперь наконец-то она держит его в руках. А она ведь почти забыла о своем желании завлечь его, заставить полюбить себя, чтобы потом поднять кнут над этой своевольной черной башкой. И вот сейчас она об этом вспомнила и почувствовала огромное удовлетворение. Одну ночь она была полностью в его власти, зато теперь узнала, где брешь в его броне. Отныне он будет плясать под ее дудку. Слишком долго она терпела его издевки, теперь он у нее попрыгает, как тигр в цирке, а она все выше будет поднимать обруч.
При мысли о том, что ей предстоит встретиться с ним при трезвом свете дня, ее охватило легкое смущение, смешанное с волнением и удовольствием.
«Я волнуюсь, как невеста, — подумала она. — Из-за кого — из-за Ретта!» И, подумав так, она захихикала, как дурочка.
Но к обеду Ретт не появился, не было его за столом и во время ужина. Прошла ночь, долгая ночь, когда Скарлетт лежала без сна до зари, прислушиваясь, не раздастся ли звук поворачиваемого в замке ключа. Но Ретт не пришел. Когда и на второй день от него не было ни слова, Скарлетт уже не могла себе места найти от огорчения и страха. Она поехала в банк, но его там не оказалось. Она поехала в лавку и устроила всем разнос, но всякий раз, как открывалась дверь, она с бьющимся сердцем поднимала глаза на вошедшего, надеясь, что это — Ретт. Она отправилась на лесной склад и так распушила Хью, что он спрятался за грудой досок. Но и здесь Ретт не настиг ее.
Унизиться до того, чтобы расспрашивать друзей, не видел ли кто Ретта, — она не могла. Не могла она расспрашивать и слуг. Но она чувствовала: они знают что-то такое, чего не знает она. Негры всегда все знают. Мамушка эти два дня была как-то особенно молчалива. Она наблюдала за Скарлетт краешком глаза, но не говорила ничего. Когда прошла и вторая ночь, Скарлетт решила отправиться в полицию. Может быть, с Реттом случилось несчастье. Может быть, его сбросила лошадь и он лежит где-нибудь в канаве, без помощи… О, какая страшная мысль!.. Может быть, он мертв…
На другое утро, когда она, покончив с завтраком, надевала у себя в комнате шляпку, на лестнице раздались быстрые шаги. Сразу ослабев, благодаря в душе бога, она опустилась на кровать, и в эту минуту вошел Ретт. Он явно приехал от парикмахера — был подстрижен, выбрит, ухожен и — трезв, но с красными, налитыми кровью глазами и опухшим от вина лицом. Небрежно взмахнув рукой, он сказал:
— О, приветствую вас!
Как это можно сказать всего лишь «О, приветствую вас!» после того, как он отсутствовал целых двое суток? Неужели он совсем не помнит той ночи, которую они провели вместе? Нет, не может не помнить, если.., если… Страшная мысль мелькнула у Скарлетт: если проводить так ночи — не самое обычное для него дело. На мгновение она лишилась дара речи, забыв про все свои заготовленные уловки и улыбки. Он даже не подошел к ней, чтобы по установившейся привычке небрежно поцеловать ее, — он стоял с дымящейся сигарой в руке и насмешливо смотрел на Скарлетт.
— Где.., где же вы были?
— Только не говорите, что не знаете! Я-то считал, что, уж конечно, весь город знает теперь. А может быть, все знают, кроме вас. Вам ведь известна старая поговорка: «Жена всегда узнает последней».
— Что вы хотите этим сказать?
— Я считал, что, после того как полиция появилась позапрошлой ночью в доме Красотки Уотлинг…
— Красотки.., этой.., этой женщины! Вы были.., вы были с…
— Конечно. А где же еще мне быть? Надеюсь, вы за меня не тревожились.
— Значит, прямо от меня вы направились.., о-о!
— Полно, полно, Скарлетт! Нечего разыгрывать обманутую жену. Вы, конечно же, давно знали о существовании Красотки Уотлинг. — И вы отправились к ней от меня, после.., после…
— Ах., это. — Он небрежно повел рукой. — Я забылся. Прошу прощения за то, что так вел себя во время нашей последней встречи. Я был очень пьян, как вы, конечно, не преминули заметить, и совсем потерял голову под влиянием ваших чар — перечислить каких?
Глаза у нее вдруг наполнились слезами — ей хотелось кинуться на постель и рыдать, безудержно рыдать. Он не изменился, ничто не изменилось, а она-то, дурочка, глупая, самовлюбленная дурочка, подумала, что он любит ее. На самом же деле это было лишь очередной его отвратительной пьяной выходкой. Он взял ее и использовал спьяну, как использовал бы любую женщину в доме Красотки Уотлинг. А теперь явился — наглый, насмешливый, недосягаемый. Она проглотила слезы и призвала на помощь всю свою волю. Он никогда-никогда не узнает, что она думает. Как бы он хохотал, если бы узнал! Ну так вот, никогда он об этом не узнает. Она метнула на него взгляд и уловила в его глазах уже знакомое озадаченное, вопрощающее выражение — взволнованное, настороженное, точно его жизнь зависела от ее слов, точно он хотел, чтобы она сказала… Чего он, собственно, от нее хотел? Чтобы она выставила себя в глупом свете, заголосила, дала ему повод посмеяться? Ну, уж нет! Брови Скарлетт стремительно сдвинулись, лицо приняло холодное выражение.
— Я, естественно, подозревала о ваших отношениях с этой тварью.
— Только подозревали? Почему же вы не спросили меня и не удовлетворили свое любопытство? Я бы вам сказал. Я живу с ней с того дня, как вы и Эшли Уилкс решили, что у нас с вами должны быть отдельные спальни.
— И вы имеете наглость стоять тут и хвалиться передо мною, вашей женой, что вы…
— О, избавьте меня от вашего высоконравственного возмущения. Вам всегда было наплевать, что я делаю, — лишь бы я оплачивал ваши счета. Ну, а насчет того, что вы моя жена.., нельзя сказать, что вы были моей женой с тех пор, как появилась Бонни, верно ведь? Вы оказались-, Скарлетт, неприбыльным предприятием: я только зря вкладывал в вас капитал. А вот в Красотку Уотлинг капитал вкладывать можно.
— Капитал? Вы хотите сказать, что дали ей?..
— Правильнее было бы, наверно, выразиться так: «поставил на ноги», Красотка — ловкая женщина. Мне захотелось помочь ей, а ей недоставало лишь денег, чтобы открыть собственное заведение. Вы ведь знаете, какие чудеса способна совершить женщина, у которой есть немного денег. Взгляните на себя.
— Вы сравниваете меня с…
— Видите ли, вы обе практичные, деловые женщины, и обе преуспели. Но у Красотки несомненно есть перед вами одно преимущество: она доброе, благожелательное существо…
— Извольте выйти из моей комнаты!
Он не спеша направился к двери, иронически приподняв бровь. «Да как он может так меня оскорблять», — думала она с болью и яростью. Он только и делает, что принижает ее и оскорбляет: ее затрясло, когда она вспомнила, как ждала его домой, а он в это время пьянствовал и задирался с полицией в борделе.
— Убирайтесь из этой комнаты и не смейте больше входить сюда. Я уже раз сказала вам это, а вы — не джентльмен и не способны понять. Отныне я буду запирать дверь.
— Можете не утруждать себя.
— Буду запирать. После того, как вы вели себя той ночью — такой пьяный, такой омерзительный…
— Зачем уж так-то, дорогая! Конечно же, я не был вам омерзителен!
— Убирайтесь вон!
— Не волнуйтесь. Я ухожу. И обещаю, что никогда больше не потревожу вас. Это окончательно и бесповоротно. Кстати, я как раз думал, что если вам слишком тяжело выносить мое безнравственное поведение, мы можем разойтись. Отдайте мне Бонни, и я не буду возражать против развода.
— Я никогда в жизни не опозорю свою семью и не стану разводиться.
— Вы довольно быстро опозорили бы ее, если бы мисс Мелли умерла. У меня даже дух захватывает при мысли о том, как быстро вы бы со мной расстались.
— Уйдете вы или нет?
— Да, я ухожу. Я и домой-то вернулся, только чтобы сказать вам это. Я уезжаю в Чарльстон и Новый Орлеан и.., словом, это будет долгая поездка. Я уезжаю сегодня.
— О-о!
— И забираю Бонни с собой. Велите этой дуре Присси уложить ее вещички. Присси я забираю тоже.
— Вы не увезете моего ребенка из этого дома.
— Но это и мой ребенок, миссис Батлер. Едва ли вы станете возражать, чтобы я отвез ее в Чарльстон повидаться с бабушкой!
— Нечего сказать, с бабушкой! Да неужели вы думаете, я позволю увезти ребенка, чтобы вы там каждый вечер напивались и еще, может, даже брали ее с собой во всякие дома вроде дома Красотки Уотлинг…
Он с такою силой швырнул сигару, что она вонзилась горящим концом в ковер и в комнате запахло паленым. В мгновение ока он пересек разделявшее их расстояние и остановился рядом со Скарлетт — лицо его почернело от гнева.
— Будь вы мужчиной, я бы свернул вам за это шею. Сейчас же могу лишь сказать вам — заткните свой чертов рот. Вы что же, считаете, что я не люблю Бонни, что я стану водить ее к.., мою-то дочь?! Боже правый, да вы настоящая идиотка! Вот вы тут изображаете из себя добродетельную маменьку, а ведь кошка — и то лучшая мать, чем вы. Ну, что вы когда-либо сделали для своих детей? Уэйд и Элла боятся вас до смерти, и если бы не Мелани Уилкс, они бы так и не узнали, что такое любовь и нежность. А Бонни, моя Бонни! Вы что, думаете, я хуже буду заботиться о ней, чем вы? Думаете, я позволю вам застращать ее и сломить ей волю, как вы сломили волю Уэйда и Эллы? Черта с два, нет! Упакуйте ее вещи, и чтоб она была готова через час, иначе — предупреждаю: то, что произошло прошлой ночью, — сущий пустяк по сравнению с тем, что я сделаю сейчас. Я всегда считал, что хорошая порка кнутом очень пошла бы вам на пользу.
Он повернулся на каблуках и, прежде чем она успела вымолвить хоть слово, стремительно вышел из комнаты. Она услышала, как он пересек площадку и открыл дверь в детскую. Раздался веселый, радостный щебет детских голосов, и до Скарлетт донесся голосок Бонни, перекрывавший голос Эллы:
— Папочка, где ты был?
— Охотился на зайчика, чтобы из его шкурки сделать своей маленькой Бонни шубку. Ну-ка, поцелуй своего любимого папку, Бонни, и ты тоже, Элла.

