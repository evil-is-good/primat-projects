\chapter{\ }
%\addcontentsline{toc}{chapter}{Глава 48}

И Скарлетт действительно развлекалась — она не веселилась так с довоенной весны. Новый Орлеан был город необычный, шикарный, и Скарлетт, точно узник, приговоренный к пожизненному заключению и получивший помилование, с головой погрузилась в удовольствия. «Саквояжники» выкачивали из города все соки; многие честные люди вынуждены были покидать свои дома, не зная, где и когда удастся в очередной раз поесть; в кресле вице-губернатора сидел негр. И все же Новый Орлеан, который Ретт показал Скарлетт, был самым веселым местом из всех, какие ей доводилось видеть. У людей, с которыми она встречалась, казалось, было сколько угодно денег и никаких забот. Ретт познакомил ее с десятком женщин, хорошеньких женщин в ярких платьях, женщин с нежными руками, не знавшими тяжелого труда, женщин, которые над всем смеялись и никогда не говорили ни о чем серьезном или тяжелом. А мужчины — с какими интереснейшими мужчинами она была теперь знакома! И как они были не похожи на тех, кого она знала в Атланте, и как стремились потанцевать с нею, какие пышные комплименты ей отпускали, точно она все еще была юной красоткой.
У этих мужчин были жесткие и дерзкие лица, как у Ретта. И взгляд всегда настороженный, как у людей, которые слишком долго жили рядом с опасностью и не могли позволить себе расслабиться. Казалось, у них не было ни прошлого, ни будущего, и они вежливо помалкивали, когда Скарлетт — беседы ради — спрашивала, чем они занимались до того, как приехать в Новый Орлеан, или где жили раньше. Это уже само по себе было странно, ибо в Атланте каждый уважающий себя пришелец спешил представить свои верительные грамоты и с гордостью рассказывал о доме и семье, вычерчивая сложную сеть семейных отношений, охватывавшую весь Юг. Это же люди предпочитали молчать, а если говорили, то осторожно, тщательно подбирая слова. Иной раз, когда Ретт был с ними, а Скарлетт находилась в соседней комнате, она слышала их смех и обрывки разговоров, которые для нее ничего не значили: отдельные слова, странные названия — Куба и Нассау в дни блокады, золотая лихорадка и захват участков, торговля оружием и морской разбой, Никарагуа и Уильям Уокер, и как он погиб, расстрелянный у стены в Трухильо. Однажды, когда она неожиданно вошла в комнату, они тотчас прервали разговор о том, что произошло с повстанцами Квонтрилла, — она успела услышать лишь имена Франка и Джесси Джеймса.
Но у всех ее новых знакомых были хорошие манеры, они были прекрасно одеты и явно восхищались ею, а потому — не все ли ей равно, если они хотят жить только настоящим. Зато ей было далеко не все равно то, что они — друзья Ретта, что у них большие дома и красивые коляски, что они брали ее с мужем кататься, приглашали на ужины, устраивали приемы в их честь. И Скарлетт они очень нравились. Когда она сказала свое мнение Ретту, это немало его позабавило.
— Я так и думал, что они тебе понравятся, — сказал он и рассмеялся.
— А почему, собственно, они не должны были мне понравиться? — Всякий раз, как Ретт смеялся, у нее возникали подозрения.
— Все это люди второсортные, мерзавцы, паршивые овцы. Они все — авантюристы или аристократы из «саквояжников». Они все нажили состояние, спекулируя продуктами, как твой любящий супруг, или получив сомнительные правительственные контракты, или занимаясь всякими темными делами, в которые лучше не вникать.
— Я этому не верю. Ты дразнишь меня. Это же приличнейшие люди…
— Самые приличные люди в этом городе голодают, — сказал Ретт. — И благородно прозябают в развалюхах, причем я сомневаюсь, чтобы меня в этих развалюхах приняли. Видишь ли, прелесть моя, я во время войны участвовал здесь в некоторых гнусных делишках, а у людей чертовски хорошая память! Скарлетт, ты не перестаешь меня забавлять. У тебя какая-то удивительная способность выбирать не тех людей и не те вещи.
— Но они же твои друзья!
— Дело в том, что я люблю мерзавцев. Юность свою я провел на речном пароходике — играл в карты, и я понимаю таких людей, как они. Но я знаю им цену. А вот ты, — и он снова рассмеялся, — ты совсем не разбираешься в людях, ты не отличаешь настоящее от дешевки. Иной раз мне кажется, что единственными настоящими леди, с которыми тебе приходилось общаться, были твоя матушка и мисс Мелли, но, видимо, это не оставило на тебе никакого следа.
— Мелли! Да она такая невзрачная, как старая туфля; и платья у нее всегда какие-то нелепые, и мыслей своих нет!
— Избавьте меня от проявлений свой зависти, мадам! Красота еще не делает из женщины леди, а платье — настоящую леди!
— Ах, вот как?! Ну, погоди, Ретт Батлер, я тебе еще кое-что покажу. Теперь, когда у меня.., когда у нас есть деньги, я стану самой благородной леди, каких ты когда-либо видел!
— С интересом буду ждать этого превращения, — сказал он. Еще больше, чем люди, Скарлетт занимали наряды, которые покупал ей Ретт, выбирая сам цвета, материи и рисунок. Кринолинов уже не носили, и новую моду Скарлетт находила прелестной — перед у юбки был гладкий, материя вся стянута назад, где она складками ниспадала с турнюра, и на этих складках покоились гирлянды из цветов, и банты, и каскады кружев. Вспоминая скромные кринолины военных лет, Скарлетт немного смущалась, когда надевала эти новые юбки, обрисовывающие живот. А какие прелестные были шляпки — собственно, даже и не шляпки, а этакие плоские тарелочки, которые носили надвинув на один глаз, а на них — и фрукты, и цветы, и гибкие перья, и развевающиеся ленты. (Ах, если бы Ретт не был таким глупым и не сжег накладные букли, которые она купила, чтобы увеличить пучок, торчавший у нее сзади из-под маленькой шляпки!) А тонкое, сшитое монахинями белье! Какое оно прелестное и сколько у нее этого белья! Сорочки, и ночные рубашки, и панталоны из тончайшего льна, отделанные изящнейшей вышивкой, со множеством складочек. А атласные туфли, которые Ретт ей купил! Каблуки у них были в три дюйма высотой, а спереди — большие сверкающие пряжки. А шелковые чулки — целая дюжина, и ни одной пары — с бумажным верхом! Какое богатство!
Скарлетт без оглядки покупала подарки для родных. Пушистого щенка сенбернара — для Уэйда, которому всегда хотелось иметь щенка; персидского котенка — для Бо коралловый браслет — для маленькой Эллы; литое колье с подвесками из лунных камней — для тети Питти; полное собрание сочинений Шекспира — для Мелани и Эшли; расшитую ливрею, включая кучерской цилиндр с кисточкой, — для дядюшки Питера; материю на платья — для Дилси и кухарки; дорогие подарки для всех в Таре.
— А что ты купила Мамушке? — спросил ее Ретт, глядя на груду подарков, лежавших на постели в их гостиничном номере, и извлекая оттуда щенка и котенка, чтобы отнести в гардеробную.
— Ничего. Она вела себя отвратительно. С чего это я стану привозить ей подарки, когда она обзывает нас мулами?
— Почему ты так не любишь, когда тебе говорят правду, моя кошечка? Мамушке нужно привезти подарок. Ты разобьешь ей сердце, если этого не сделаешь, а такое сердце, как у нее, слишком ценно, чтобы взять его и разбить.
— Ничего я ей не повезу. Она этого не заслуживает.
— Тогда я сам куплю ей подарок. Я помню, моя нянюшка всегда говорила — хорошо бы, если бы на ней, когда она отправится на небо, была нижняя юбка из тафты, такой жесткой, чтоб она торчком стояла и шуршала при малейшем движении, а господь бог подумал бы, что она сшита из крыльев ангелов. Я куплю Мамушке красной тафты и велю сшить ей элегантную нижнюю юбку.
— В жизни Мамушка от тебя ее не примет. Да она скорее умрет, чем наденет такую.
— Не сомневаюсь. И все-таки я это сделаю.
Магазины в Новом Орлеане были такие богатые, и ходить по ним с Реттом было так увлекательно. Обедать с ним было тоже увлекательным занятием — даже еще более волнующим, чем посещение магазинов, ибо Ретт знал, что заказать и как это должно быть приготовлено. И вина, и ликеры, и шампанское в Новом Орлеане — было для Скарлетт в новинку, все возбуждало: ведь она знала лишь домашнюю наливку из ежевики, да мускатное вино, да «обморочный» коньяк тети Питти. А какую Ретт заказывал еду! Лучше всего в Новом Орлеане была еда. Когда Скарлетт вспоминала о страшных голодных днях в Таре и о совсем еще недавней поре воздержания, ей казалось, что она никогда вдоволь не наестся этих вкусных блюд. Суп из стручков бамии, коктейль из креветок, голуби в вине и устрицы, запеченные в хрустящем тесте со сметанным соусом; грибы, сладкое мясо и индюшачья печенка; рыба, хитро испеченная в промасленной бумаге, и к ней — лаймы. У Скарлетт никогда не было недостатка в аппетите, ибо стоило ей вспомнить о неизменных земляных орехах, сушеном горохе и сладком картофеле в Таре, как она готова была проглотить буквально все блюда креольской кухни.
— Всякий раз ты ешь так, словно больше тебе не дадут, — говорил Ретт. — Не вылизывай тарелку, Скарлетт. Я уверен, что на кухне еще сколько угодно еды. Стоит только попросить официанта. Если ты не перестанешь предаваться обжорству, будешь толстой, как кубинские матроны, и тогда я с тобой разведусь.
Но она лишь показывала ему язык и просила принести еще кусок торта с шоколадом и взбитыми сливками.
До чего же было приятно тратить деньги — сколько хочешь, а не считать каждое пенни и не думать о том, что надо экономить, чтобы заплатить налоги или купить мулов. До чего же было приятно находиться в обществе людей веселых и богатых, а не благородных, но бедных, как в Атланте. До чего же было приятно носить шуршащие парчовые платья, которые подчеркивали твою талию, оставляя открытыми для всеобщего обозрения твою шею, и руки, и в значительной мере грудь, и знать, что мужчины любуются тобой. И как приятно было есть все что захочется и не слышать при этом менторских наставлений о том, что, мол, леди так не едят. И как приятно было пить шампанское вволю! В первый раз, когда она выпила лишку, она проснулась на другое утро с раскалывающейся головой и не без смущения вспомнила, что ехала назад, в отель, по улицам Нового Орлеана в открытой коляске и во весь голос распевала «Славный голубой наш флаг». Она ни разу не видела ни одной хотя бы чуть подвыпившей леди; если же говорить о простых смертных, то она видела пьяную женщину только раз, в день падения Атланты, — это была та тварь, Уотлинг. Скарлетт не знала, как показаться на глаза Ретту после такого позора, но вся эта история, казалось, лишь позабавила его. Все, что бы Скарлетт ни делала, забавляло его, словно она была игривым котенком.
Ей доставляло удовольствие появляться с ним на людях — до того он был хорош. Она почему-то никогда раньше не обращала внимания на его внешность, а в Атланте все были слишком заняты обсуждением его недостатков и никто не обращал внимания на то, как он выглядит. Но здесь, в Новом Орлеане, она видела, как женщины провожают его взглядом и как они трепещут, когда он склоняется, целуя им руку. Сознание, что другие женщины находят ее мужа привлекательным и, возможно, даже завидуют ей, неожиданно преисполнило Скарлетт гордостью за то, что у нее такой спутник в жизни.
«А мы, оказывается, красивая пара», — не без удовольствия подумала она.
Да, как и предсказывал Ретт, от брака действительно можно получать уйму удовольствия. И не только получать удовольствие, но и научиться кое-чему. Это само по себе было странным, ибо Скарлетт считала, что жизнь уже ничему не может ее научить. А сейчас она чувствовала себя ребенком, которому каждый день сулил новое открытие.
Прежде всего она поняла, что брак с Реттом совсем не был похож на брак с Чарльзом или Франком. Они уважали ее и боялись ее нрава. Они выклянчивали ее расположение, и если это ее устраивало, она снисходила до них. Ретт же не боялся ее, и она часто думала о том, что он не слишком ее и уважает. Он делал то, что хотел, и лишь посмеивался, если ей это не нравилось. Она его не любила, но с ним, конечно, было очень интересно. И самое интересное было то, что даже во время вспышек страсти, которая порой была окрашена жестокостью, а порой — раздражающей издевкой, он, казалось, всегда держал себя в узде, всегда был хозяином своих чувств. «Это, наверное, потому, что на самом-то деле он вовсе не влюблен в меня, — думала она, и такое положение дел вполне ее устраивало. — Мне бы не хотелось, чтобы он когда-либо в чем-то дал себе полную волю». И однако же мысль о такой возможности щекотала ее, вызывая любопытство.
Живя с Реттом, Скарлетт обнаружила в нем много нового, а ведь ей казалось, что она хорошо его знает. Голос его, оказывается, мог быть мягким и ласковым, как кошачья шерсть, а через секунду — жестким, хрипло выкрикивающим проклятья. Он мог вроде бы откровенно и с одобрением рассказывать о мужественных, благородных, добродетельных, подсказанных любовью поступках, коим он был свидетелем в тех странных местах, куда его заносило, и тут же с холодным цинизмом добавлять скабрезнейшие истории. Она понимала, что муж не должен рассказывать жене таких историй, но они развлекали ее, воздействуя на какие-то грубые земные струны ее натуры. Он мог быть страстным, почти нежным любовником, но это длилось недолго, а через мгновение перед ней был хохочущий дьявол, который срывал крышку с пороховой бочки ее чувств, поджигал запал и наслаждался взрывом. Она узнала, что его комплименты всегда двояки, а самым нежным выражениям его чувств не всегда можно верить. Словом, за эти две недели в Новом Орлеане она узнала о нем все — кроме того, каков он был на самом деле.
Случалось, он утром отпускал горничную, сам приносил Скарлетт завтрак на подносе и кормил ее, точно она — дитя, брал щетку и расчесывал ее длинные черные волосы, так что они начинали потрескивать и искрить. В другое же утро он грубо пробуждал ее от глубокого сна, сбрасывал с нее одеяло и щекотал ее голые ноги… Бывало, он с почтительным интересом и достоинством слушал ее рассказы о том, как она вела дела, выражая одобрение ее мудрости, а в другой раз называл ее весьма сомнительные сделки продувным мошенничеством, грабежом на большой дороге и вымогательством. Он водил Скарлетт на спектакли и, желая ей досадить, нашептывал на ухо, что бог наверняка не Одобрил бы такого рода развлечений, а когда водил в церковь, тихонько рассказывал всякие смешные непристойности и потом корил за то, что она смеялась. Он учил ее быть откровенной, дерзкой и смелой. Следуя его примеру, она стала употреблять больно ранящие слова, иронические фразы и научилась пользоваться ими, ибо они давали ей власть над другими людьми. Но она не обладала чувством юмора Ретта, которое смягчало его ехидство, и не умела так улыбаться, чтобы казалось, будто, издеваясь над другими, она издевается сама над собой.
Он научил ее играть, а она почти забыла, как это делается. Жизнь для нее была такой серьезной, такой горькой. А он умел играть и втягивал ее в игру. Но он никогда не играл как мальчишка — это был мужчина, и что бы он ни делал, Скарлетт всегда об этом помнила. Она не могла смотреть на него сверху вниз — с высоты своего женского превосходства — и улыбаться, как улыбаются женщины, глядя на взрослых мужчин, оставшихся мальчишками в душе.
Это вызывало у нее досаду. Было бы так приятно чувствовать свое превосходство над Реттом. На всех других мужчин она могла махнуть рукой, не без презрения подумав: «Что за дитя!» На своего отца, на близнецов Тарлтонов с их любовью к поддразниванию и ко всяким хитрым проделкам, на необузданных младших Фонтейнов с их чисто детскими приступами ярости, на Чарльза, на Фрэнка, на всех мужчин, которые увивались за ней во время войны, — на всех, за исключением Эшли. Только Эшли и Ретт не поддавались ее пониманию и не позволяли помыкать собой, потому что оба были люди взрослые, ни в том, ни в другом не было мальчишества.
Она не понимала Ретта и не давала себе труда понять, в нем было порой нечто такое, что озадачивало ее. К примеру, то, как он на нее смотрел, когда думал, что она этого не видит. Внезапно обернувшись, она вдруг замечала, что он наблюдает за ней, и в глазах его — напряженное, взволнованное ожидание.
— Почему ты на меня так смотришь? — однажды раздраженно спросила она его. — Точно кошка, подстерегающая мышь!
Но выражение его лица мгновенно изменилось, и он лишь рассмеялся. Вскоре она об этом забыла и перестала ломать над этим голову, как и вообще над всем, что касалось Ретта. Слишком уж он вел себя непредсказуемо, а жизнь была так приятна — за исключением тех минут, когда она вспоминала об Эшли.
Правда, Ретт занимал все ее время, и она не могла часто думать об Эшли. Днем Эшли почти не появлялся в ее мыслях, зато по ночам, когда она чувствовала себя совсем разбитой от танцев, а голова кружилась от выпитого шампанского, — она думала об Эшли. Часто, засыпая в объятиях Ретта при свете луны, струившемся на постель, она думала о том, как прекрасна была бы жизнь, если бы это руки Эшли обнимали ее, если бы это Эшли прятал лицо в ее черные волосы, обматывал ими себе шею.
Однажды, когда эта мысль в очередной раз пришла ей в голову, она вздохнула и отвернулась к окну — и почти тотчас почувствовала, как могучая рука Ретта у нее под головой напряглась и словно окаменела, и голос Ретта произнес в тишине:
— Хоть бы бог отправил твою мелкую лживую душонку на веки вечные в ад!
Он встал, оделся и вышел из комнаты, несмотря на ее испуганные вопросы и протесты. Вернулся он лишь на следующее утро, когда она завтракала у себя в номере, — вернулся пьяный, мятый, в самом саркастическом настроении; он не извинился перед ней и никак не объяснил своего отсутствия.
Скарлетт ни о чем его не спрашивала и держалась с ним холодно, как и подобает оскорбленной жене, а покончив с завтраком, оделась под взглядом его налитых кровью глаз и отправилась по магазинам. Когда она вернулась, его уже не было; и появился он снова лишь к ужину.
За столом царило молчание, Скарлетт старалась не дать воли гневу — ведь это был их последний ужин в Новом Орлеане, а ей хотелось поесть крабов. Но она не получала удовольствия от еды под взглядом Ретта. Тем не менее она съела большущего краба и выпила немало шампанского. Быть может, это сочетание возродило старый кошмар, ибо проснулась она в холодном поту, отчаянно рыдая. Она была снова в Таре — разграбленной, опустошенной. Мама умерла, и вместе с ней ушла вся сила и вся мудрость мира. В целом свете не было больше никого, к кому Скарлетт могла бы воззвать, на кого могла бы положиться. А что-то страшное преследовало ее, и она бежала, бежала, так что сердце, казалось, вот-вот разорвется, — бежала в густом, клубящемся тумане и кричала, бежала, слепо ища неведомое безымянное пристанище, которое находилось где-то в этом тумане, окружавшем ее.
Проснувшись, она увидела склонившегося над ней Ретта; он молча взял ее, как маленькую девочку, на руки и прижал к себе — в его крепких мускулах чувствовалась надежная сила, его шепот успокаивал, и она перестала рыдать.
— Ох, Ретт, мне было так холодно, я была такая голодная, такая усталая. Я никак не могла это найти. Я все бежала и бежала сквозь туман и не могла найти.
— Что найти, дружок?
— Не знаю. Хотелось бы знать, но не знаю.
— Это все тот старый сон?
— Да, да!
Он осторожно опустил ее на постель, пошарил в темноте и зажег свечу. Она озарила его лицо, резко очерченное, с налитыми кровью глазами, — непроницаемое, точно каменное. Распахнутая до пояса рубашка обнажала смуглую, покрытую густыми черными волосами грудь. Скарлетт, все еще дрожа от страха, подумала — какой он сильный, какой крепкий, и прошептала:
— Обними меня, Ретт.
— Хорошая моя! — пробормотал он, подхватил ее на руки, прижал к себе и опустился со своей ношей в глубокое кресло.
— Ах, Ретт, это так страшно, когда ты голодная.
— Конечно, страшно умирать во сне от голода после того, как съеден ужин из семи блюд, включая того огромного краба. — Он улыбнулся, и глаза у него были добрые.
— Ах, Ретт, я все бежала и бежала, и что-то искала и не могла найти то, что искала. Оно все время пряталось от меня в тумане. А я знала, что если это найду, то навсегда буду спасена и никогда-никогда не буду больше страдать от холода или голода.
— Что же ты искала — человека или вещь?
— Сама не знаю. Я об этом не думала. А как ты думаешь, Ретт, мне когда-нибудь приснится, что я добралась до такого места, где я буду в полной безопасности?
— Нет, — сказал он, приглаживая ее растрепанные волосы, — думаю, что нет. Сны нам неподвластны. Но я думаю, что если ты привыкнешь жить в безопасности и в тепле и каждый день хорошо питаться, то перестанешь видеть этот сон. А уж я, Скарлетт, позабочусь, чтобы ничто тебе не угрожало.
— Ретт, ты такой милый.
— Благодарю вас за крошки с вашего стола, госпожа Богачка. Скарлетт, я хочу, чтобы каждое утро, просыпаясь, ты говорила себе: «Я никогда больше не буду голодать и ничего со мной не случится, пока Ретт рядом и правительство Соединенных Штатов — у власти».
— Правительство Соединенных Штатов? — переспросила она и в испуге выпрямилась, не успев смахнуть слезы со щек.
— Бывшая казна конфедератов стала честной женщиной. Я вложил большую часть этих денег в правительственные займы.
— Мать пресвятая богородица! — воскликнула Скарлетт, отстраняясь от него, забыв о недавно владевшем ею ужасе. — Неужели ты хочешь сказать, что ссудил свои деньги этим янки?
— Под весьма приличный процент.
— Да мне плевать, если даже под сто процентов! Немедленно продай эти облигации! Подумать только — чтобы янки пользовались твоими деньгами!
— А что мне с ними делать? — с улыбкой спросил он, подмечая, что в глазах ее уже нет страха.
— Ну.., ну, например, купить землю у Пяти Углов. Уверена, что при твоих деньгах ты мог бы все участки там скупить.
— Премного благодарен, но не нужны мне эти Пять Углов. Теперь, когда правительство «саквояжников» по-настоящему прибрало Джорджию к рукам, никто не знает, что может случиться. Я не поручусь за этих стервятников, что налетели в Джорджию с севера, востока, юга и запада. Я, как ты понимаешь, будучи добросовестным подлипалой, действую с ними заодно, но я им не верю. И деньги свои в недвижимость вкладывать не стану. Я предпочитаю государственные займы. Их можно спрятать. А недвижимость спрятать не так-то легко.
— Ты, что же, считаешь… — начала было она, похолодев при мысли о лесопилках и лавке.
— Я ничего не знаю. Но не надо так пугаться, Скарлетт. Наш обаятельный новый губернатор — добрый мой друг. Просто времена сейчас очень уж неверные, и я не хочу замораживать большие деньги в недвижимости.
Он пересадил ее на одно колено, потянулся за сигарой и закурил. Она сидела, болтая босыми ногами, глядя на игру мускулов на его смуглой груди, забыв все страхи.
— И раз уж мы заговорили о недвижимости, Скарлетт, — сказал он, — я намерен построить дом. Ты могла заставить Фрэнка жить в доме мисс Питти, но не меня. Не думаю, что я сумею вынести ее причитания по три раза в день, а кроме того, мне кажется, дядюшка Питер скорее прикончит меня, чем допустит, чтоб я поселился под священной крышей Гамильтонов. А мисс Питти может предложить мисс Индии Уилкс пожить с нею, чтобы отпугивать привидения. Мы же, вернувшись в Атланту, поселимся в свадебном номере отеля «Нейшнл», пока наш дом не будет построен. Еще до отъезда из Атланты мне удалось сторговаться насчет того большого участка на Персиковой улице — что близ дома Лейденов. Ты знаешь, о каком я говорю?
— Ах, Ретт, какая прелесть! Мне так хочется иметь свой дом. Большой-большой.
— Ну вот, наконец-то мы хоть в чем-то согласны. Что, если построить белый оштукатуренный дом и украсить его чугунным литьем, как эти креольские дома здесь?
— О нет, Ретт. Только не такой старомодный, как эти новоорлеанские дома. Я знаю, чего бы мне хотелось. Я хочу совсем новый дом — я видела картинку.., стой, стой… — в «Харперс уикли». Что-то вроде швейцарского шале.
— Швейцарского — чего?
— Шале.
— Скажи по буквам.
Она выполнила его просьбу.
— Вот как! — произнес он и пригладил усы.
— Дом прелестный. У него высокая остроугольная крыша, под ней — мансарда, поверху идет как бы частокол, а по углам — башенки, крытые цветной черепицей. И в этих башенках — окна с синими и красными стеклами. Все — по моде.
— И перила крыльца, должно быть, с переплетом?
— Да.
— А с крыши над крыльцом свешиваются этакие деревянные кружева?
— Да. Ты, должно быть, видел такой.
— Видел.., но не в Швейцарии. Швейцарцы очень умный народ и остро чувствуют красоту в архитектуре. Ты в самом деле хочешь такой дом?
— Ах, конечно!
— А я-то надеялся, что общение со мной улучшит твой вкус. Ну, почему ты не хочешь дом в креольском или колониальном стиле, с шестью белыми колоннами?
— Я же сказала, что не хочу ничего старомодного. А внутри чтобы были красные обои и красные бархатные портьеры и чтоб все двери раздвигались. И конечно, много дорогой ореховой мебели и роскошные толстые ковры, и… Ах, Ретт, все позеленеют от зависти, когда увидят наш дом!
— А так ли уж необходимо, чтобы все нам завидовали? Впрочем, если тебе так хочется, пусть зеленеют. Только тебе не приходило в голову, Скарлетт, что не очень это хороший вкус — обставлять свой дом с такой роскошью, когда вокруг все так бедны.
— А я хочу, — упрямо заявила она. — Я хочу, чтобы всем, кто плохо ко мне относится, стало не по себе. И мы будем устраивать большие приемы, чтобы все в городе жалели, что говорили обо мне разные гадости.
— Но кто же в таком случае будет приходить на наши приемы?
— Как кто — все, конечно.
— Сомневаюсь. «Старая гвардия» умирает, но не сдается.
— Ах, Ретт, какую ты несешь чушь. У кого есть деньги, того люди всегда будут любить.
— Только не южане. Спекулянту с деньгами куда труднее проникнуть в лучшие гостиные города, чем верблюду пройти сквозь игольное ушко. А уж про подлипал, — а это мы с тобой, моя кошечка, — и говорить нечего: нам повезет, если нас не оплюют. Но если ты, моя дорогая, хочешь все же попытаться, что ж, я тебя поддержу и, уверен, получу немало удовольствия от твоей кампании. А теперь, раз уж мы заговорили о деньгах, мне хотелось бы поставить точки над «i». Я дам тебе на дом столько денег, сколько ты захочешь, и сколько ты захочешь — на всякую мишуру. И если ты любишь драгоценности — пожалуйста, но только я сам буду их выбирать. У тебя, моя кошечка, преотвратительный вкус. Ну, и конечно, ты получишь все, что требуется для Уэйда или Эллы. И если Уиллу Бентину не удастся сбыть весь хлопок, я готов оказать ему помощь и покрыть расходы на содержание этого белого слона в графстве Клейтон, который вам так дорог. Это будет справедливо, как ты считаешь?
— Конечно. Ты очень щедр.
— Но теперь слушай меня внимательно. Я не дам ни цента на твою лавку и ни цента на эту твою дохлую лесопилку.
— О-о, — вырвалось у Скарлетт, и лицо ее вытянулось. Весь медовый месяц она думала о том, как бы попросить у него тысячу долларов, которые нужны ей были, чтобы прикупить еще пятьдесят футов земли и расширить свой лесной склад. — По-моему, ты всегда похвалялся широтой взглядов: тебе, мол, все равно, пусть болтают про то, что у меня свое дело, а ты, оказывается, как все мужчины — до смерти боишься, как бы люди не сказали, что это я ношу брюки в нашей семье.
— Вот уж ни у кого никогда не возникнет сомнения насчет того, кто в семье Батлеров носит брюки, — протянул Ретт. — И мне безразлично, что болтают дураки. Видишь ли, я настолько плохо воспитан, что горжусь своей красавицей женой. И я не возражаю: продолжай держать лавку и лесопилки. Они принадлежат твоим детям. Когда Уэйд подрастет, ему, возможно, не захочется жить на содержании отчима и он сам возьмет на себя управление делами. Но ни единого цента ни на одно из этих предприятий я не дам.
— Почему?
— Потому что не желаю содержать Эшли Уилкса.
— Ты что, начинаешь все сначала?
— Нет. Но ты спросила — почему, и я объяснил. И еще одно. Не думай, что тебе удастся провести меня, и не лги и не придумывай, сколько стоят твои туалеты и как дорого вести дом, чтобы на сэкономленные деньги иметь потом возможность накупить мулов или еще одну лесопилку для Эшли. Я намерен тщательно проверять твои расходы, а что сколько стоит, я знаю. Ах, пожалуйста, не напускай на себя оскорбленный вид. Ты бы на все это пошла. Я считаю, что ты вполне на это способна. Я вообще считаю, что ты на что угодно пойдешь, если дело касается Тары или Эшли. Против Тары я не возражаю. Но что касается Эшли — тут я против. Я не натягиваю поводья, моя кошечка, но не забудь, что у меня есть уздечка и шпоры.

