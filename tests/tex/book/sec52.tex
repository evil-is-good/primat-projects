\chapter{\ }

Как-то раз дождливым днем, когда Бонни только что исполнился годик, Уэйд уныло бродил по гостиной, время от времени подходя к окошку и прижимаясь носом к стеклу, исполосованному дождем. Мальчик был худенький, хрупкий, маленький для своих восьми лет, застенчивый и тихий — рта не раскроет, пока его не спросят. Ему было скучно, и он явно не знал, чем заняться, ибо Элла возилась в углу с куклами, Скарлетт сидела у секретера и, что-то бормоча себе под нос, подсчитывала длинную колонку цифр, а Ретт, лежа на полу возле Бонни, развлекал ее, раскачивая часы на цепочке, но так, чтобы она не могла до них дотянуться.
Уэйд взял было несколько книг, потом с грохотом уронил их и глубоко вздохнул.
— О господи, Уэйд! — раздраженно воскликнула, поворачиваясь к нему, Скарлетт. — Пошел бы куда-нибудь поиграл.
— Не могу. На дворе дождик.
— В самом деле? Я не заметила. Ну, займись чем-нибудь. Ты действуешь мне на нервы, когда вертишься без толку. Пойди скажи Порку, чтобы он запряг карету и отвез тебя к Бо поиграть.
— Он же не дома, — вздохнул Уэйд. — Он на дне рождения у Рауля Пикара.
Рауль, сынишка Мейбелл и Репе Пикара, был, по мнению Скарлетт, преотвратительным существом, больше похожим на обезьяну, чем на ребенка.
— Ну, можешь поехать к кому хочешь. Пойди скажи Порку.
— Никого нет дома, — возразил Уэйд. — Все на дне рождения. Хотя к слову «все» и не было прибавлено: «кроме меня», однако эти слова повисли в воздухе, но Скарлетт, вся ушедшая в свои подсчеты, не обратила на это внимания.
Ретт же приподнялся и спросил:
— А ты почему не на дне рождения, сынок? Уэйд подошел к нему совсем близко и остановился, шаркая ногой по ковру. Вид у него был глубоко несчастный.
— Меня не пригласили, сэр.
Ретт отдал Бонни часы на растерзание и легко вскочил на ноги.
— Да бросьте вы эти проклятые цифры, Скарлетт. Почему Уэйда не пригласили на день рождения?
— Ах, ради всего святого, Ретт! Оставьте меня сейчас в покос. У Эшли тут такая неразбериха в цифрах… А-а, вы про день рождения? Что ж, тут нет ничего необычного в том, что Уэйда не пригласили, да к тому же я все равно не пустила бы его. Не забудьте, что Рауль — внук миссис Мерриуэзер, а миссис Мерриуэзер скорее впустит вольного негра в свою драгоценную гостиную, чем кого-либо из нас.
Ретт, задумчиво наблюдавший за лицом Уэйда, увидел, как тот сморщился.
— Пойди-ка сюда, сынок, — сказал он, привлекая к себе мальчика. — А тебе хочется быть на этом дне рождения?
— Нет, сэр, — храбро ответил Уэйд, но глаза опустил.
— М-м… Скажи-ка мне, Уэйд, а ты бываешь на дне рождения у Джо Уайтинга или Фрэнка Боннелла.., или у кого-нибудь из твоих приятелей?
— Нет, сэр. Меня очень редко приглашают.
— Ты лжешь, Уэйд! — воскликнула, оборачиваясь, Скарлетт. — Ты же на прошлой неделе был на трех детских праздниках у Бартов, у Гелертов и Хандонов. — Более отборную коллекцию мулов в лошадиной сбруе трудно себе представить, — заметил Ретт, с легкой издевкой растягивая слова. — И ты хорошо провел время на этих праздниках? Ну, говори же.
— Нет, сэр.
— А почему нет?
— Я.., я не знаю, сэр. Мамушка.., она говорит, что все это белая рвань.
— Я с Мамушки шкуру спущу — сию же минуту! — воскликнула, вскакивая, Скарлетт. — А ты, Уэйд, если будешь так говорить о друзьях своей мамы…
— Мальчик верно говорит, как и Мамушка, — сказал Ретт. — Но где же вам знать правду, если вы отворачиваетесь от нее… А ты, сынок, не волнуйся. Можешь больше не ходить на праздники, если тебе не хочется. Вот, — добавил он, вытаскивая из кармана банкноту, — скажи Порку, чтобы запряг карету и повозил тебя по городу. Купи себе чего-нибудь сладкого — да побольше, чтоб разболелся живот.
Лицо Уэйда расцвело в улыбке. Он сунул в карман бумажку и с тревогой посмотрел на мать — одобрит ли она такую затею. Но Скарлетт, сдвинув брови, глядела на Ретта. Он поднял с пола Бонни и прижал к себе — крошечное личико уткнулось ому в щеку. Скарлетт не могла понять, о чем он думает, но в глазах увидела что-то похожее на страх — страх и чувство вины.
Приободренный щедростью отчима, Уэйд застенчиво подошел к нему.
— Дядя Ретт, а можно мне у вас что-то спросить?
— Конечно — Лицо у Ретта было напряженное, отсутствующее, он крепче прижал к себе головку Бонни. — О чем же ты хочешь спросить меня, Уэйд?
— Дядя Ретт, вы.., а вы воевали?
Взгляд Ретта мгновенно обратился на мальчика — глаза смотрели пронзительно, но голос, задавший вопрос, звучал небрежно:
— А почему ты спрашиваешь, сынок?
— Да вот Джо Уайтинг говорит — вы не воевали, и Франк Боннелл тоже.
— А-а, — проронил Ретт, — а ты им что сказал? Уэйд стоял с несчастным видом.
— Я., я сказал.., я говорю, что не знаю. — И на одном дыхании выпалил: — Но мне все равно, воевали вы или нет. Я поколотил их. А вы были на войне, дядя Ретт?
— Да, — с неожиданной резкостью сказал Ретт. — Я был на войне. Я был в армии восемь месяцев. Я прошел с боями весь путь от Лавджоя до Франклина, штат Теннесси. И я был с Джонстоном, когда он сдался.
Уэйд запрыгал от восторга, а Скарлетт рассмеялась.
— А я-то полагала, что вы стыдитесь своего участия в войне, — сказала она. — Разве вы не просили меня помалкивать об этом?
— Прекратите! — оборвал он ее. — Ну как, ты доволен, Уэйд?
— О да, сэр! Я знал, что вы воевали. Я знал, что вы не трус, как они говорят. Вот только.., почему вы не воевали вместе с отцами других мальчиков?
— Да потому, что отцы других мальчиков были люди глупые и их направили в пехоту. Я же закончил Вест-Пойнт и был в артиллерии. В полевой артиллерии, Уэйд, а не в войсках внутреннего охранения. А чтобы служить в артиллерии, Уэйд, нужно иметь голову.
— Еще бы! — сказал Уэйд, весь сияя. — А вы были ранены, дядя Ретт?
Ретт ответил не сразу.
— Вы расскажите ему про свою дизентерию, — с издевкой заметила Скарлетт.
Ретт осторожно опустил малышку на пол и вытянул сорочку и нижнюю рубашку из брюк.
— Подойди сюда, Уэйд, я покажу тебе, куда я был ранен. Уэйд, возбужденный происходящим, подошел и уставился на то место, на которое указывал пальцем Ретт. Смуглую грудь пересекал длинный рубец, спускавшийся вниз на мускулистый живот. То была память о ножевой драке на калифорнийских золотых приисках, но Уэйд не мог этого знать. Он глубоко вздохнул от счастья.
— А вы, видно, такой же храбрый, как мой папа, дядя Ретт.
— Почти, но не совсем, — сказал Ретт, засовывая сорочку в брюки. — А теперь иди, накупи себе сластей на доллар и бей всех мальчишек, которые посмеют сказать, что я не был в армии.
Уэйд, приплясывая и громко зовя Порка, выбежал из комнаты, а Ретт снова подхватил на руки малышку.
— Ну, к чему вся эта ложь, мой доблестный вояка? — спросила Скарлетт.
— Мальчик должен гордиться отцом.., или отчимом. Я не хочу, чтобы ему было стыдно перед другими маленькими зверюгами. Дети ведь жестокие существа.
— Какая чепуха!
— Я никогда не задумывался над тем, что это может значить для Уэйда, — медленно произнес Ретт. — Я никогда не думал, что он страдает. С Бонни так не будет.
— Как — так?
— Вы думаете, я допущу, чтобы моя Бонни стыдилась своего отца? Допущу, чтобы ее не приглашали на дни рождения, когда ей будет девять или десять лет? Думаете, допущу, чтобы ее унижали, как Уэйда, за то, в чем виновата не она, а мы с вами?
— Подумаешь — детские дни рождения!
— Детские дни рождения превращаются потом в балы для барышень и молодых людей. Вы думаете, я позволю, чтобы моя дочь росла вне благовоспитанного общества Атланты? Я не намерен посылать ее на Север в школу, чтобы она приезжала сюда лишь на каникулы, потому что с ней не желают знаться здесь, или в Чарльстоне, или в Саванне, или в Новом Орлеане. Я не хочу, чтобы она вынуждена была выйти замуж за янки или за иностранца, потому что никакая приличная семья южан не захочет принять ее в свое лоно.., так как мать ее была дура, а отец — мерзавец.
Уэйд, вернувшийся за чем-то, стоял на пороге и озадаченно, но с интересом слушал.
— Бонни может выйти замуж за Бо, дядя Ретт. На лице Ретта не было и следа ярости, когда он повернулся к мальчику; казалось, он со всей серьезностью обдумывал его слова — он всегда был серьезен, разговаривая с детьми.
— А ведь и правда, Уэйд. Бонни может выйти замуж за Бо Уилкса. А вот на ком ты женишься?
— О, я — ни на ком, — доверительно сообщил Уэйд, наслаждаясь этой беседой на равных, какую он мог вести только с Реттом да еще разве с тетей Мелли, никогда не корившими его и всегда поощрявшими. — Я поеду учиться в Гарвард и стану юристом, как мой отец, а потом буду храбрым солдатом — тоже, как он.
— Ну, почему Мелли не может держать рот на замке! — воскликнула Скарлетт. — Нив какой Гарвард ты, Уэйд, не поедешь. Это заведение для янки. Ты пойдешь в университет Джорджии, а когда окончишь его, будешь управлять лавкой вместо меня. Ну, а что касается того, что твой отец был храбрым солдатом…
— Прекратите! — коротко приказал Ретт: от него не укрылось, как заблестели глаза Уэйда, когда речь зашла об его отце, которого он никогда не знал. — Ты вырастешь и будешь таким же храбрым, как твой отец, Уэйд. Постарайся быть таким, как он, потому что он был героем, и никому не позволяй говорить о нем иначе. Он ведь женился на твоей матери, верно? Ну, так вот это уже достаточное доказательство его героизма. А уж я прослежу за тем, чтобы ты пошел в Гарвард и стал юристом. А теперь, беги и скажи Порку, чтобы он повозил тебя по городу.
— Я была бы вам очень признательна, если бы вы позволили мне самой заниматься воспитанием моих детей! — воскликнула Скарлетт, когда Уэйд послушно выбежал из комнаты.
— Очень плохо вы ими занимаетесь. Вы сделали все возможное, чтобы испортить будущее Эллы и Уэйда, но я не допущу, чтобы то же повторилось и с Бонни. Она будет расти как принцесса, и на всем свете не найдется человека, которому не захотелось бы общаться с ней. Ни один дом не будет для нее закрыт. Великий боже, да неужели вы думаете, я позволю, чтобы она, когда вырастет, общалась с тем сбродом, который заполняет этот дом?
— Однако этот сброд вполне устраивает вас…
— И более чем устраивает вас, моя кошечка. Но это не для Бонни. Да неужели вы думаете, я позволю, чтобы она вышла замуж за кого-либо из этих беглых каторжников, с которыми вы проводите время? Выскочки-ирландцы, янки, белая рвань, парвеню-«саквояжники»… Чтобы моя Бонни, в жилах которой течет кровь Батлеров и Робийяров…
— Кровь О’Хара…
— Возможно, в свое время О’Хара были королями Ирландии, но ваш отец был всего лишь ловким ирландским выскочкой. Да и вы не лучше… Но я тоже, конечно, хорош. Я мчался по жизни, точно летучая мышь, выпущенная из ада, не задумываясь над тем, что я делаю, так как все и вся было мне безразлично. А вот Бонни не безразлична. Боже, каким я был дураком! Теперь Бонни ни за что не примут в Чарльстоне, сколько бы ни старались моя мать, или ваши тетя Евлалия или тетя Полин.., ясно, что не примут ее и здесь, если мы чего-то не придумаем — и быстро…
— Ах, Ретт, вы относитесь к этому так трагически, что даже смешно. При наших-то деньгах…
— К черту наши деньги! Никакие наши деньги не могут купить то, чего я хочу для Бонни. Я бы предпочел, чтоб ее приглашали на черствый хлеб в жалкий дом Пикаров или в этот прохудившийся сарай, в котором живет миссис Элсинг, чем на республиканские балы, где она была бы первой красавицей. Скарлетт, вы вели себя как последняя дура. Вам следовало обеспечить своим детям место в обществе много лет назад, а вы этого не сделали. Вы даже не позаботились о том, чтобы удержать то место, которое сами там занимали. И сейчас едва ли можно надеяться, что вы вдруг изменитесь. Слишком вы стремитесь к наживе и слишком любите принимать людей.
— Я считаю, это буря в стакане воды, — холодно заметила Скарлетт, перебирая бумаги и тем самым показывая, что она, во всяком случае, разговор окончила.
— У нас теперь осталась одна миссис Уилкс, которая способна нам помочь, а вы все делаете, чтобы оттолкнуть ее и оскорбить. О, избавьте меня, пожалуйста, от ваших умозаключений по поводу ее бедности и убогой одежды. Она — душа всего, что есть в Атланте неподкупного. Слава богу, что она существует. И она поможет мне что-то предпринять.
— И что же вы намерены предпринять?
— Что я намерен предпринять? Буду обихаживать всех драконов «старой гвардии» в женском обличий, какие есть в этом городе, — и миссис Мерриуэзер и миссис Элсинг, и миссис Уайтинг, и миссис Мид. И если мне придется ползти на животе к каждой толстой старой кошке, которая ненавидит меня, я поползу. Я буду кроток, как бы холодно они меня ни встретили, и буду каяться в своих прегрешениях. Я дам денег на их дурацкие благотворительные затеи и буду ходить в их чертовы церкви. Я признаюсь и даже стану хвастать, что оказывал услуги Конфедерации; в худшем случае, войду даже в этот их чертов ку-клукс-клан, хотя надеюсь, всемилостивый бог не подвергнет меня столь тяжкому испытанию. И я, не колеблясь, напомню этим идиотам, чьи головы я спас, что они кое-чем мне обязаны. А вы, мадам, будьте любезны, не портите мне дело, продавая им гнилой лес, или накладывая лапу на имущество должников из числа тех, кого я буду обихаживать, или как-либо иначе оскорбляя их. И еще одно: отныне ноги губернатора Баллока не будет в этом доме. Вы меня слышите? Как и никого из этих элегантных грабителей, с которыми вы свели компанию. Если, несмотря на мою просьбу, вы их все же пригласите, то окажетесь в щекотливом положении, так как хозяина в доме не будет. Стоит им появиться у нас, как я тотчас отправлюсь к Красотке Уотлинг, засяду у нее в баре и буду говорить всем и каждому, что нисколько не желаю находиться под одной с ними крышей.
Скарлетт, кипя от гнева, выслушала эту тираду и расхохоталась.
— Значит, шулер с речных пароходов и спекулянт хочет стать уважаемым господином! В таком случае, чтобы добиться уважения, вам следовало бы для начала хотя бы продать дом Красотки Уотлинг. Это был удар наугад. Скарлетт ведь никогда не была вполне уверена, что дом принадлежит Ретту. Он вдруг рассмеялся, точно прочел ее мысли.
— Спасибо за совет.




Ретт едва ли мог выбрать более неподходящее время для возвращения в ряды уважаемых людей. Никогда еще слова «республиканец» и «подлипала» не вызывали такой ненависти, ибо коррупция среди пришедших к власти «саквояжников» достигла апогея. А со времени поражения Юга имя Ретта было неразрывно связано с янки, республиканцами и подлипалами.
В 1866 году обитатели Атланты, пылая бессильным гневом, считали, что хуже навязанного им жесткого военного режима уже ничего быть не может, но только сейчас, при Баллоке, они поняли, почем фунт лиха. Благодаря голосам негров республиканцы и их союзники твердо закрепились на Юге и железной рукой правили бессильным что-либо изменить, но по-прежнему бунтующим коренным белым населением.
Неграм внушали, что в Библии говорится только о двух политических партиях — мытарях и грешниках. А поскольку ни один негр не желал вступить в партию, состоящую из грешников, все они спешили пополнить ряды республиканцев. Новые хозяева заставляли их снова и снова голосовать и выбирать белых подонков и подлипал, а порой даже и кое-кого из негров в органы управления. Негры, которым посчастливилось быть избранными, заседали в законодательном собрании, где большую часть времени грызли земляные орехи и пытались освоиться с непривычной для них обувью, то и дело вытаскивая из башмаков ноги. Лишь немногие из них умели читать или писать. Еще совсем недавно они работали на хлопковых полях или сахарных плантациях, а теперь голосовали за налоги и выпуск займов, а также утверждали огромные сметы расходов на собственные нужды и нужды своих республиканских покровителей. Штат совсем придавило налогами, которые население выплачивало, скрипя зубами от ярости, так как налогоплательщики знали, что большая часть денег, якобы предназначаемых для общественных нужд, оседала у частных лиц.
Капитолий штата прочно окружила орда прожектеров, спекулянтов, подрядчиков и всех прочих, кто надеялся поживиться на этой оргии расточительства; многим это удавалось, и они беспардонно сколачивали себе состояния. Они без труда получали от штата деньги на строительство железных дорог, которые так никогда и не были построены, на приобретение вагонов и локомотивов, которые так никогда и не были куплены, на сооружение общественных зданий, которые существовали лишь в уме прожектеров. Займы выпускались миллионами. По большей части они выпускались незаконно и были явным мошенничеством — и все равно выпускались. Казначей штата — хоть и республиканец, но человек честный — возражал против незаконного выпуска займов и отказывался подписывать соответствующие бумаги, но ни он, ни те, кто пытался поставить преграду злоупотреблениям, не могли противостоять этой все сметающей жажде обогащения.
Железная дорога, принадлежащая штату, когда-то приносила изрядный доход, сейчас же она висела на бюджете штата мертвым грузом и задолженность ее превышала миллион. Да, собственно, железной дорогой это уже и назвать-то было нельзя. Рельсы пролегали по огромной глубокой котловине, где валялись в грязи свиньи. Многие чиновники были назначены по политическим соображениям, а не потому, что они знали, как управлять железной дорогой, да и вообще работало там в три раза больше народу, чем требовалось, республиканцы ездили бесплатно по пропускам, негров целыми вагонами бесплатно развозили по всему штату, чтобы они могли по несколько раз голосовать во время одних и тех же выборов. Плохое управление дорогой приводило в ярость налогоплательщиков еще и потому, что на доходы с нее предполагалось открыть бесплатные школы. Но доходов не было, были только долги, и потому бесплатных школ тоже не было. Лишь немногие имели возможность посылать своих детей в платные школы, так что росло целое поколение невежественных, неграмотных людей.
Жители, конечно, возмущались растратами, неумелым хозяйствованием и казнокрадством, но больше всего их возмущало то, что губернатор в плохом свете выставляет их перед Севером. Когда в Джорджии стали громко возмущаться коррупцией, губернатор поспешно отправился на Север, предстал перед конгрессом и заявил, что белые безобразно ведут себя по отношению к неграм, что в Джорджии готовится новое восстание, а потому необходимо-де ввести в штате военное положение. На самом же деле в Джорджии все старательно избегали осложнений с неграми. Никто не хотел новой войны, никто не хотел, чтобы в штате правила сила штыка, — этого не требовалось. В Джорджии хотели лишь одного: чтобы их оставили в покое и дали возможность штату залечить раны. Но акция, предпринятая губернатором и ставшая впоследствии известной как «сотворение клеветы», представила Джорджию Северу лишь как бунтующий штат, на который требовалось надеть узду, и эта узда была надета.
Это вызвало великое ликование в банде, державшей Джорджию за горло. Началась настоящая оргия хищений и холодно-циничного, беззастенчивого воровства на высоких постах, которое больно было наблюдать. Все протесты и попытки сопротивляться кончались крахом, так как правительство штата подпирали штыки армии Соединенных Штатов. Атланта проклинала Баллока, его подлипал и всех республиканцев вообще, равно как и тех, кто был с ними связан. А Ретт был с ними связан. Он действовал с ними заодно — так говорили все вокруг — и участвовал во всех их начинаниях. Теперь же он решительно повернулся и вместо того, чтобы плыть по течению в потоке, который еще недавно нес его вперед, изо всех сил поплыл в противоположном направлении.
Он повел свою кампанию медленно, исподволь, чтобы не вызвать подозрений в Атланте своим превращением за одну ночь из леопарда в лань. Он стал теперь избегать своих подозрительных дружков — никто больше не видел его в обществе офицеров-янки, подлипал и республиканцев. Он стал посещать сборища демократов и демонстративно голосовал за них. Он перестал играть в карты на большие ставки и относительно мало пил. Если он и заходил к Красотке Уотлинг, то вечером и исподтишка, как большинство уважаемых горожан, а не днем, оставив для всеобщего обозрения свою лошадь у коновязи возле ее дома.
И прихожане епископальной церкви чуть не упали со своих скамей, когда он, осторожно ступая и ведя за руку Уэйда, вошел в храм. Немало удивило прихожан и появление Уэйда, ибо они считали мальчика католиком. Во всяком случае, Скарлетт-то ведь была католичкой. Или считалась таковой. Правда, она уже многие годы не бывала в церкви, религиозность слетела с нее, как и многое другое, чему учила ее Эллин. По мнению всех, Скарлетт пренебрегала религиозным воспитанием мальчика, и тем выше в глазах «старой гвардии» поднялся Ретт, когда он решил исправить дело и Привел мальчика в церковь — пусть в епископальную вместо католической.
Ретт умел держаться серьезно и бывал обаятелен, если задавался целью не распускать язык и гасить лукавый блеск в черных глазах. Многие годы он не считал нужным это делать, но сейчас надел на себя маску серьезности и обаяния, как стал надевать жилеты более темных тонов. И добиться благорасположения тех, кто был обязан ему жизнью, не составило особого труда. Они бы уже давно проявили к нему дружелюбие, не поведи себя Ретт так, будто оно мало значит для него. А теперь Хью Элсинг, Рене, Симмонсы, Энди Боннелл и другие вдруг обнаружили, что Ретт человек приятный, не любящий выдвигать себя на передний план и смущающийся, когда при нем говорят, сколь многим ему обязаны.
— Пустяки! — возражал он. — Вы бы все на моем месте поступили точно так же.
Он пожертвовал кругленькую сумму в фонд обновления епископальной церкви и сделал весомый — но в меру весомый — дар Ассоциации по благоустройству могил наших доблестных воинов. Он специально отыскал миссис Элсинг, которой и вручил свой дар, смущено попросив, чтобы она держала его пожертвование в тайне, и прекрасно зная, что тем лишь подстегивает ее желание всем об этом рассказать. Миссис Элсинг очень не хотелось брать у него деньги — «деньги спекулянта», — но Ассоциация так нуждалась в средствах.
— Не понимаю, с чего это вы вдруг решили сделать нам пожертвование, — колко заметила она.
И когда Ретт сообщил ей с приличествующей случаю скорбной миной, что его побудила к этому память о бывших товарищах по оружию, больших храбрецах, чем он, но менее удачливых и потому лежащих сейчас в безымянных могилах, аристократическая челюсть миссис Элсинг отвисла. Долли Мерриуэзер говорила ей со слов Скарлетт, что капитан Батлер якобы служил в армии, но она, конечно, этому не поверила. Никто не верил.
— Вы служили в армии? А в какой роте.., в каком полку? Ретт назвал.
— Ах, в артиллерии! Все мои знакомые были либо в кавалерии, либо в пехоте. А, ну тогда понятно… — Она в замешательстве умолкла, ожидая увидеть ехидную усмешку в его глазах. Но он смотрел вниз и играл цепочкой от часов. — Я бы с превеликой радостью пошел в пехоту, — сказал он, делая вид, будто не понял ее намека. — Но когда узнали, что я учился в Вест-Пойнте — хотя, миссис Элсинг, из-за одной мальчишеской выходки я и не окончил академии, — меня поставили в артиллерию, в настоящую артиллерию, а не к ополченцам. Во время последней кампании нужны были люди, знающие дело. Вам ведь известно, какие огромные мы понесли потери, сколько артиллеристов было убито. Я в артиллерии чувствовал себя одиноко. Ни единого знакомого человека. По-моему, за всю службу я не встретил никого из Атланты.
— М-да! — смущенно протянула миссис Элсинг. Если он служил в армии, значит, она вела себя недостойно. Она ведь не раз резко высказывалась о его трусости и теперь, вспомнив об этих своих высказываниях, почувствовала себя виноватой. — М-да! А почему же вы никогда никому не рассказывали о своей службе в армии? Можно подумать, что вы стесняетесь этого.
Ретт посмотрел ей прямо в глаза — лицо его оставалось бесстрастным.
— Миссис Элсинг, — внушительно заявил он, — прошу вас поверить мне: я горжусь своей службой Конфедерации, как ничем, что когда-либо совершал или еще совершу. У меня такое чувство.., такое чувство…
— Тогда почему же вы все это скрывали?
— Как-то стыдно мне было говорить об этом в свете.., в свете некоторых моих тогдашних поступков. Миссис Элсинг сообщила миссис Мерриуэзер о полученном даре и о разговоре во всех его подробностях.
— И даю слово, Долли, он сказал, что ему стыдно, со слезами на глазах! Да, да, со слезами! Я сама чуть не расплакалась.
— Сущий вздор! — не поверив ни единому ее слову, воскликнула миссис Мерриуэзер. — Не верю я, чтобы слезы появились у него на глазах, как не верю и тому, что он был в армии. И все это я очень быстро выясню. Если он был в том артиллерийском полку, я доберусь до правды, потому что полковник Карлтон, который им командовал, женат на дочери одной из сестер моего деда, и я ему напишу.
Она написала полковнику Карлтону и была совершенно сражена, получив ответ, где весьма недвусмысленно и высоко оценивалась служба Ретта: прирожденный артиллерист, храбрый воин, настоящий джентльмен, который все выносит без жалоб, и к тому же человек скромный, даже отказавшийся от офицерского звания, когда ему его предложили.
— Ну и ну! — произнесла миссис Мерриуэзер, показывая письмо миссис Элсинг. — В себя не могу прийти от удивления! Возможно, мы и в самом деле не правы были, считая, что он не служил в армии. Возможно, нам следовало поверить Скарлетт и Мелани, которые говорили ведь, что он записался в армию в день падения Атланты. Но все равно он подлипала и мерзавец, и я его не люблю!
— А мне вот думается, — сказала неуверенно миссис Элсинг, — мне думается, что не такой уж он и плохой. Не может человек, сражавшийся за Конфедерацию, быть совсем, плохим. Это Скарлетт плохая. Знаете, Долли, мне в самом деле кажется, что он.., ну, словом, что он стыдится Скарлетт, но, будучи джентльменом, не показывает этого.
— Стыдится?! Ерунда! Оба они из одного куска материи выкроены. Откуда вы взяли такие глупости?
— Это не глупости, — возразила возмущенная миссис Элсинг. — Вчера, под проливным дождем, он ездил в карете со всеми тремя детьми — заметьте, там была и малютка — вверх и вниз по Персиковой улице и даже меня до дому подвез. И когда я сказала:
«Капитан Батлер, вы что, с ума сошли, зачем вы держите, детей в сырости! Почему не везете их домой?», он ни слова не ответил, но вид у него был смущенный. Тогда Мамушка вдруг говорит:
«В доме-то у нас полным-полно всяких белых подонков, так что деткам лучше быть под дождем, чем дома!» — А он что сказал?
— А что он мог сказать? Только посмотрел, сдвинув брови, на Мамушку, и промолчал. Вы же знаете, вчера днем Скарлетт устраивала большую партию в вист, и все эти вульгарные простолюдинки были там. И ему, я полагаю, не хотелось, чтобы они целовали его малышку.
— Ну и ну! — произнесла миссис Мерриуэзер, заколебавшись, но все еще держась прежних позиций. Однако на следующей неделе капитулировала и она.
Теперь у Ретта появился в банке свой стол. Что он делал за этим столом, никто из растерявшихся чиновников не знал, но ему принадлежал слишком большой пакет акций, чтобы они могли возражать против его присутствия. Через некоторое время они забыли о своих возражениях, ибо он держался спокойно, воспитанно и к тому же кое-что понимал в банковском деле и капиталовложениях. Так или иначе, он целый день проводил за своим столом, и все видели, как он корпит, а он решил показать, что, подобно своим респектабельным согражданам, трудится — и трудится вовсю.
Миссис Мерриуэзер, стремясь расширить свою и так уже процветающую торговлю пирогами, надумала занять две тысячи долларов в банке под залог дома. В займе ей отказали, так как под дом было выдано уже две закладных. Дородная дама вне себя от возмущения выкатывалась из банка, когда Ретт остановил ее, выяснил в чем дело и озабоченно сказал:
— Тут произошла какая-то ошибка, миссис Мерриуэзер. Ужасная ошибка. Кому-кому, а вам нечего волноваться по поводу обеспечения! Да я одолжил бы вам деньги под одно ваше слово! Даме, которая сумела развернуть такое предприятие, можно без риска поверить. Кому же еще давать банку деньги, как не вам. Так что посидите, пожалуйста, в моем кресле, а я займусь вашим делом.
Через некоторое время он вернулся и со спокойной улыбкой сказал, что, как он и думал, произошла ошибка. Две тысячи долларов ждут ее, и она может взять их когда захочет. А насчет ее дома — не будет ли она так любезна поставить свою подпись вот тут?
Миссис Мерриуэзер, все еще не придя в себя от нанесенного ей оскорбления, злясь на то, что приходится принимать услугу от человека, который ей неприятен и которому она не доверяет, не слишком любезно поблагодарила его.
Но он сделал вид, будто ничего не заметил. Провожая ее к двери, он сказал:
— Миссис Мерриуэзер, я всегда высоко ставил ваш жизненный опыт — можно мне с вами посоветоваться?
Она слегка кивнула, так что перья на ее шляпке чуть колыхнулись.
— Что вы делали, когда ваша Мейбелл была маленькая и сосала палец?
— Что, что?
— У меня Бонни сосет пальчик. Я никак не могу ее отучить.
— Необходимо отучить, — решительно заявила миссис Мерриуэзер. — Это испортит форму ее рта.
— Знаю! Знаю! А у нее такой хорошенький ротик. Но я никак не соображу, что делать.
— Ну, Скарлетт наверняка знает, — отрезала миссис Мерриуэзер. — Она ведь уже вырастила двоих детей.
Ретт опустил глаза на свои сапоги и вздохнул.
— Я пытался смазывать Бонни ноготки мылом, — сказал он, пропустив мимо ушей замечание насчет Скарлетт.
— Мылом?! Ба-а! Мыло тут не поможет. Я посыпала Мейбелл пальчик хинином, и должна сказать вам, капитан Батлер, она очень скоро перестала его сосать.
— Хинином! Вот уж никогда бы не подумал! Я просто не в состоянии выразить вам свою признательность, миссис Мерриуэзер. А то это очень меня тревожило.
Он улыбнулся ей такой приятной, такой благородной улыбкой, что миссис Мерриуэзер секунду стояла опешив. Однако прощаясь с ним, она уже улыбалась сама. Ей не хотелось признаваться миссис Элсинг в том, что она неверно судила об этом человеке, но, будучи женщиной честной, сказала, что в мужчине, который так любит своего ребенка, несомненно, есть что-то хорошее. Какая жалость, что Скарлетт совсем не интересуется своей прелестной дочуркой! Когда мужчина растит маленькую девочку — в этом есть что-то патетическое. Ретт прекрасно понимал, сколь душещипательна создаваемая им картина, а то, что это бросало тень на репутацию Скарлетт, ничуть не волновало его.
Как только малышка начала ходить, он то и дело брал ее с собой, и она сидела с ним либо в карете, либо впереди него в седле. Вернувшись домой из банка, он отправлялся с ней на прогулку по Персиковой улице и, держа ее за руку, старался приноровиться к ее шажкам и терпеливо отвечал на тысячи ее вопросов. В это время дня, на закате, люди обычно сидели у себя в палисадниках или на крыльце, а поскольку Бонн и была очень общительная и хорошенькая девочка с копной черных кудрей и ярко-голубыми глазками, почти все заговаривали с ней. Ретт никогда не встревал в эти разговоры, а стоял поодаль, исполненный отцовской гордости и благодарности за то, что его дочь окружают вниманием.
У Атланты была хорошая память, она отличалась подозрительностью и не скоро меняла однажды сложившееся мнение. Времена были тяжелые, и на всех, кто имел хоть что-то общее с Баллоком и его окружением, смотрели косо. Но с помощью Бонни, унаследовавшей обаяние отца и матери, Ретт сумел вбить клинышек в стену отчуждения, которой окружила его Атланта.
Бонни быстро росла, и с каждым днем становилось все яснее, что она — внучка Джералда О’Хара. У нее были короткие крепкие ножки, большие голубые, какие бывают только у ирландцев, глаза и маленький квадратный подбородочек, говоривший о решимости стоять на своем. У нее был горячий нрав Джералда, проявлявшийся в истериках, которые она закатывала, причем она тотчас успокаивалась, как только ее желания были удовлетворены. А когда отец находился поблизости, желания ее всегда поспешно удовлетворялись. Он баловал ее, невзирая на противодействие Мамушки и Скарлетт, ибо ему нравилось в ней все, кроме одного. Кроме боязни темноты.
До двух лет она быстро засыпала в спаленке, которую делила с Уэйдом и Эллой. А потом без всякой видимой причины начала всхлипывать, как только Мамушка выходила из комнаты и уносила с собой лампу. Затем она начала просыпаться ночью с криками ужаса, пугая двух других детей и приводя в смятение весь дом. Однажды пришлось даже вызвать доктора Мида, и Ретт был весьма резок с ним, когда тот объявил, что это всего лишь от дурных снов. От нее же никто ничего не мог добиться, кроме одного слова: «Темно».
Скарлетт считала это капризом и говорила, что девочку надо отшлепать. Она не желала идти на уступки и оставлять лампу в детской, ибо тогда Уэйд и Элла не смогут уснуть. Обеспокоенный Ретт пытался мягко выудить у дочки, в чем дело; услышав заявление жены, он холодно сказал, что если кого и следует отшлепать, так это Скарлетт, причем сам он готов этим заняться.
В итоге Ретт перевел Бонни из детской в комнату, где он жил теперь. Ее кроватку поставили рядом с его большой кроватью, и на столике всю ночь горела затененная лампа. Когда об этом стало известно, весь город загудел. Было что-то неделикатное в том, что девочка спит в комнате отца, даже если этой девочке всего два года. От этих пересудов Скарлетт пострадала двояко. Во-первых, все узнали, что она и ее муж спят в разных комнатах, а это уже само по себе не могло не произвести шокирующего впечатления. А во-вторых, все считали, что если уж девочка боится спать одна, ее место — рядом с матерью. Скарлетт же не могла объяснить всем и каждому, что она не в состоянии спать при свете, да к тому же и Ретт не допустил бы, чтобы девочка спала с ней.
— Вы в жизни не проснетесь, пока она не закричит, а если и проснетесь, то скорее всего отшлепаете ее, — отрезал он.
Скарлетт раздражало то, что Ретт придает такое значение ночным страхам Бонни, но она подумала, что со временем исправит дело и переведет девочку назад в детскую. Все дети боятся темноты, и единственное тут лекарство — твердость. Ретт упорствует только затем, чтобы выставить ее в глазах всех плохой матерью и таким путем отплатить за то, что она изгнала его из своей спальни.
Он ни разу не переступал порога ее комнаты и даже не брался за ручку двери с того вечера, когда она сказала ему, что не хочет больше иметь детей. Ужинал он почти всегда вне дома — до тех пор, пока страхи Бонни не побудили его прекратить все отлучки. А то, бывало, он проводил вне дома всю ночь, и Скарлетт, лежа без сна за плотно закрытой дверью, прислушивалась к бою часов, возвещавшему наступление утра, и раздумывала, где-то Ретт. Ей вспоминались его слова: «На свете есть немало других постелей, прелесть моя!» И хотя при этой мысли ее всю передергивало, ничего поделать она не могла. Что бы она ни сказала, мгновенно возникла бы ссора, и тогда он непременно намекнул бы на ее запертую дверь и на то, что это связано с Эшли. Да, эта его дурацкая затея, чтобы Бонни спала при свете — причем в его комнате, — объясняется, конечно же, всего лишь подлым желанием отплатить ей.
Она не понимала, почему Ретт придавал такое значение дурацким страхам Бонни, как не понимала и его привязанности к девочке — до одной страшной ночи. Никто в семье потом не мог забыть эту ночь.
В тот день Ретт встретил на улице человека, с которым они прорывали блокаду, и им, конечно, было о чем друг с другом поговорить. Куда они отправились беседовать и пить, Скарлетт не знала, но подозревала, что скорее всего — к Красотке Уотлинг. Ретт не приехал домой днем, чтобы погулять с Бонни, не приехал и к ужину. Бонни, просидевшая весь день у окна, с нетерпением дожидаясь своего папу, чтобы показать ему коллекцию жуков и тараканов, наконец, невзирая на слезы и протестующие крики, была уложена в постель Лу.
Возможно, Лу забыла зажечь лампу, а возможно, лампа сама погасла. Никто толком не знал, что произошло, но когда Ретт явился, наконец, домой сильно навеселе, в доме все было вверх дном, а отчаянные крики Бонни донеслись до него, когда он еще был в конюшне. Девочка проснулась в темноте и позвала папу, а его не было. И все неведомые ужасы, населявшие ее воображение, пробудились. Ни уговоры, ни несколько ламп, принесенных Скарлетт, не помогали: она не успокаивалась, и у Ретта, когда он в три прыжка взбежал по лестнице, был вид человека, встретившегося со смертью.
Взяв дочку на руки, он, наконец, уловил среди ее всхлипываний слово «темно» и в ярости повернулся к Скарлетт и негритянкам.
— Кто, потушил лампу? Кто оставил Бонни в темноте одну? Присси, я с тебя шкуру за это сдеру…
— Бог мне свидетель, мистер Ретт, это не я! Это Лу!
— Ради всего святого, мистер Ретт, ведь я…
— Заткнись! Ты знаешь мой приказ. Клянусь богом, я… убирайся отсюда. И чтоб духу твоего не было. Скарлетт, дайте ей денег, и чтобы я не видел ее, когда я спущусь. А сейчас все уходите, все!
Негритянки выскочили — незадачливая Лу с рыданиями, закрывшись передником. Но Скарлетт осталась. Ей было тяжело видеть, как любимое дитя тотчас успокоилось на руках у Ретта, тогда как у нее на руках девочка кричала не умолкая. Тяжело было видеть и то, как маленькие ручонки обвились вокруг его шеи, слышать, как девочка, задыхаясь, прерывисто рассказывала ему, что ее так напугало, а она, Скарлетт, ничего членораздельного не могла из дочки вытянуть.
— Значит, он сидел у тебя на грудке, — тихо проговорил Ретт. — И он был большой?
— Да, да! Такой страшенный, большущий. И когти…
— Ах, значит, и когти у него были. Ну, ладно. Я всю ночь просижу тут и пристрелю его, если он снова явится. — Ретт сказал это убежденно, мягко, и Бонни постепенно перестала всхлипывать. Уже почти успокоившись, на языке, понятном одному только Ретт, она принялась описывать чудовище, которое явилось к ней. Скарлетт, слушая, как Ретт беседует с девочкой, точно речь идет о чем-то реальном, почувствовала, что в ней закипает раздражение.
— Ради бога, Ретт…
Но он жестом велел ей молчать. Когда Бонни наконец уснула, он уложил ее в кроватку и накрыл одеяльцем.
— Я с этой негритянки живьем шкуру спущу, — тихим голосом сказал он. — Да и вы виноваты. Почему вы не зашли посмотреть, горит ли свет?
— Не валяйте дурака, Ретт, — шепотом ответила Скарлетт. — Бонни так себя ведет потому, что вы ей потакаете. Многие дети боятся темноты, но у них это проходит. Уэйд тоже боялся, но я не потворствовала ему. Пусть покричит ночь-другую…
— Пусть покричит?! — На секунду Скарлетт показалось, что он сейчас ударит ее. — Либо вы дура, либо самая бессердечная женщина на свете.
— Я не хочу, чтоб она выросла нервной и трусливой.
— Трусливой! Черта с два! Да в ней трусости ни на грош нет! Просто вы лишены воображения и, конечно, не можете понять, какие муки испытывает человек, который им наделен, — особенно ребенок. Если что-то с когтями и рогами явится и сядет к вам на грудь, вы скажете, чтобы оно убиралось к дьяволу, да? Именно так, черт подери. Припомните, мадам, что я присутствовал при том, как вы просыпались, пища, точно выпоротая кошка, только потому, что вам приснилось, будто вы бежали в тумане. И это было не так уж давно!
Скарлетт растерялась — она не любила вспоминать этот сон. К тому же Ретт успокаивал ее тогда почти так же, как сейчас Бонни, и это вносило смятение в ее мысли. Не желая над этим раздумывать, Скарлетт повела на него наступление с другой стороны.
— Просто вы во всем ей потакаете…
— И намерен потакать. Тогда она рано или поздно преодолеет свой страх и забудет о Нем. — В таком случае, — язвительно заметила Скарлетт, — раз уж вы решили стать нянькой, не мешало бы вам приходить вечером домой и для разнообразия — в трезвом виде.
— Я и буду приходить домой рано, но напиваться, если захочу, буду, как сапожник.
С тех пор Ретт стал рано приходить домой — он приезжал задолго до того, как надо было укладывать Бонни в постель. Садился подле нее и держал ее за руку, пока она, заснув, не расслабляла пальцы. Лишь тогда он на цыпочках спускался вниз, оставив гореть лампу и приоткрыв дверь, чтобы услышать, если девочка проснется и испугается. Он твердо решил: больше он не допустит, чтоб ее мучил страх. Весь дом следил за тем, чтобы в комнате, где спала Бонни, не потух свет. Скарлетт, Мамушка, Присси и Порк то и дело на цыпочках поднимались наверх и проверяли, горит ли лампа.
И приходил Ретт домой трезвым, но не Скарлетт добилась этого. На протяжении последних месяцев он много пил, хотя никогда не был по-настоящему пьян, и вот однажды вечером от него особенно сильно пахло виски. Придя домой, он подхватил с пола Бонни, посадил ее к себе на плечо и спросил:
— Ты, что же, не желаешь поцеловать своего любимого папку? Бонни сморщила курносый носишко и заерзала, высвобождаясь из объятий.
— Нет, — чистосердечно призналась она. — Фу, — Это я — фу?
— Фу как плохо пахнет. От дяди Эшли никогда плохо не пахнет.
— Черт бы меня подрал! — буркнул Ретт, опуская ее на пол. — Вот уж никогда не думал, что обнаружу в собственном доме поборницу воздержания!
Но с того дня он выпивал лишь по бокалу вина после ужина. Бонни, которой разрешалось допить последние капли, вовсе не находила запах вина таким уж плохим. А у Ретта воздержание привело к тому, что одутловатость, отяжелившая черты его лица, постепенно исчезла, круги под черными глазами стали не такими темными и обозначались не так резко. Бонни любила кататься на лошади, сидя впереди него в седле, поэтому Ретт проводил много времени на воздухе, и смуглое лицо его, покрывшись загаром, потемнело еще больше. Вид у него был цветущий, он то и дело смеялся и снова стал похож на того удалого молодого человека, который на удивление всей Атланты столь смело прорывал блокаду в начале войны.
Люди, никогда не любившие Ретта, теперь улыбались при виде крошечной фигурки, торчавшей перед ним в седле. Матроны, до сих пор считавшие, что ни одна женщина не может чувствовать себя спокойной в его обществе, начали останавливаться и беседовать с ним на улице, любуясь Бонни. Даже самые строгие пожилые дамы держались мнения, что мужчина, способный рассуждать о детских болезнях и воспитании ребенка, не может быть совсем уж скверным.

