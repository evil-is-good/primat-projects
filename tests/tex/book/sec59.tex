\chapter{\ }

Всем было известно, что Бонни Батлер ни в чем не знает удержу и что ей нужна твердая рука, но все так любили девочку, что ни у кого не хватало духу проявить необходимую твердость. Впервые она вышла из повиновения во время поездки с отцом. Когда она была с Реттом в Новом Орлеане и Чарльстоне, ей позволяли допоздна сидеть со взрослыми и она часто засыпала у отца на руках в театре, ресторанах и за карточным столом. С тех пор только силой можно было заставить ее лечь в постель одновременно с послушной Эллой. Пока Бонни была с Реттом, он позволял ей носить любые платья, и с той поры она поднимала скандал всякий раз, как Мамушка пыталась надеть на нее бумажное платье и передничек вместо голубого тафтового платья с кружевным воротничком.
Вернуть то, что было упущено, пока девочка жила вне дома, и позже, когда Скарлетт заболела и находилась в Таре, не представлялось уже возможным. Бонни росла, и Скарлетт пыталась приструнивать ее, пыталась смягчить ее нрав и не слишком баловать, но все эти усилия почти ничего не давали. Ретт всегда брал сторону ребенка, какими бы нелепыми ни были желания Бонни и как бы возмутительно она себя ни вела. Он поощрял ее, когда она говорила, подражая взрослым, и относился к ней, как к взрослой, выслушивая с Серьезным видом ее суждения и прикидываясь, будто следует им. В результате Бонни могла оборвать кого угодно из старших, перечила отцу и осаживала его. Он же только смеялся и не разрешал Скарлетт даже хлопнуть девочку по руке в наказание.
«Если бы она не была такой милой, ласковой девчушкой, она была бы просто невыносима, — мрачно размышляла Скарлетт, неожиданно осознав, что дочка может помериться с ней силой воли. — Она обожает Ретта, и он мог бы заставить ее лучше себя вести, если бы хотел».
Однако Ретт не выказывал ни малейшего желания заставлять Бонни вести себя как следует. Что бы она ни делала, все признавалось правильным, и попроси она луну, она бы получила ее, сумей отец ее достать. Он бесконечно гордился хорошенькой мордочкой своей дочки, ее кудряшками, ямочками, изящными движениями. Ретту нравилось ее зубоскальство, ее задор, милая манера выказывать ему свою любовь и привязанность. Хотя избалованная и капризная, Бонни вызывала такую всеобщую любовь, что у Ретта не хватало духу даже пытаться ее обуздать. Он был для нее богом, средоточием ее маленького мирка и слишком ценил это, боясь наставлениями все разрушить.
Она льнула к нему как тень. Она будила его, не дав ему выспаться, сидела рядом с ним за столом и ела то с его тарелки, то со своей, ездила с ним в одном седле на лошади и никому, кроме Ретта, не позволяла себя раздевать и укладывать в кроватку, стоявшую рядом с его большой кроватью.
Скарлетт забавляло и трогало то, как ее маленькая дочка деспотично правит отцом. Кто бы мог подумать, что именно Ретт так серьезно воспримет отцовство? Но порою жало ревности пронзало Скарлетт, так как Бонни в четыре года лучше понимала Ретта, чем когда-либо понимала его сама Скарлетт, и лучше, чем Скарлетт, справлялась с ним.
Когда Бонни исполнилось четыре года, Мамушка принялась ворчать по поводу того, что нехорошо, мол, дитяти, к тому же — девочке, ездить «верхом в седле впереди своего папочки, да еще задравши платье». Ретт с вниманием отнесся к этому замечанию, как относился вообще ко всем замечаниям Мамушки по поводу воспитания девочек. В итоге появился маленький бело-бурый шотландский пони с длинной шелковистой гривой и длинным хвостом, а в придачу — крошечное седло, инкрустированное серебром. Пони, разумеется, предназначался для всех троих детей, и Ретт купил также седло для Уэйда. Но Уэйд предпочитал общество сенбернара, а Элла вообще боялась животных. Итак, пони стал собственностью всеобщей любимицы, и нарекли его Мистер Батлер. Радость Бонни омрачалась лишь тем, что она не могла ездить верхом, как отец, но когда он объяснил ей, насколько труднее сидеть в дамском седле, она успокоилась и быстро научилась кататься. Ретт несказанно гордился ее хорошей посадкой и крепкой рукой.
— Подождите, что еще будет, когда она вырастет и станет охотиться, — похвалялся он. — Никто не сможет сравниться с ней. Я увезу ее тогда в Виргинию. Вот где настоящая охота! И в Кентукки, где ценят хороших наездников. Когда дело дошло до костюма для верховой езды, девчушке было предоставлено право выбрать цвет материи, и, как всегда, она выбрала голубой.
— Но, деточка! Не этот же голубой бархат! Голубой бархат — это мне для вечернего платья, — рассмеялась Скарлетт. — А маленькие девочки носят добротное черное сукно. — И, увидев, как сошлись крохотные черные бровки, мать добавила: — Ради всего святого, Ретт, да скажите вы ей, что это не годится — ведь платье мигом станет грязным.
— Ну, пусть у нее будет костюм из голубого бархата. Если он испачкается, сошьем ей новый, — как ни в чем не бывало сказал Ретт.
Итак, Бонни получила голубой бархатный костюм для верховой езды с длинной юбкой, ниспадавшей на бок пони, и черную шапочку с красным пером, потому что рассказы тети Мелли про перо, которое носил Джеф Стюарт, воспламенили воображение девочки. В ясные погожие дни отец с дочерью катались по Персиковой улице, и Ретт придерживал своего большого вороного коня, чтобы тот шел в ногу с толстоногим пони. Иной раз Ретт и Бонни скакали по тихим дорогам вокруг города, спугивая кур, собак и детей, — Бонни хлестала Мистера Батлера кнутом, спутанные локоны ее развевались по ветру, а Ретт твердой рукой придерживал своего коня, чтобы девочка считала, что Мистер Батлер выигрывает состязание.
Удостоверившись в хорошей посадке, крепкой руке и полном бесстрашии Бонни, Ретт решил, что настало время обучить ее брать на лошади препятствия — совсем невысокие, в пределах возможностей коротконогого Мистера Батлера. Для этого Ретт построил на заднем дворе низкий барьер и платил Уошу, одному из малолетних племянников дядюшки Питера, двадцать пять центов в день, чтобы тот учил Мистера Батлера прыгать. Начали они с прыжков через перекладину, закрепленную на высоте двух дюймов от земли, и постепенно добрались до одного фута.
Эта затея была встречена всеми тремя заинтересованными сторонами в штыки, — а именно: Уошем, Мистером Батлером и Бонни. Уош боялся лошадей, и только поистине королевское вознаграждение могло побудить его заставлять упрямого пони по двадцать раз в день скакать через барьер; Мистер Батлер, хладнокровно сносивший выходки маленькой хозяйки, когда она дергала его за хвост, или позволявший без конца осматривать себе копыта, считал, что творец создал его вовсе не затем, чтобы переносить свою толстую тушу через перекладину; Бонни, не терпевшая, чтобы кто-то еще ездил на ее пони, приплясывала от нетерпения, пока Мистера Батлера учили уму-разуму.
Когда Ретт наконец решил, что пони достаточно хорошо обучен новому делу и ему можно доверить дочку, волнению малышки не было конца. Она победоносно совершила первый прыжок, и теперь уже езда рядом с отцом потеряла для нее все свое очарование. Скарлетт лишь посмеивалась над восторгами и бахвальством отца и дочки. Она, однако, считала, что, как только новизна этой затеи притупится, Бонни увлечется чем-то другим и все обретут мир и покой. Но этот вид спорта Бонни не приедался. От беседки в дальнем конце заднего двора до барьера тянулась утоптанная дорожка, и все утро во дворе раздавались возбужденные вопли. Дедушка Мерриуэзер, объехавший в 1849 году всю страну, говорил, что именно так кричат апачи, удачно сняв с кого-нибудь скальп.
Прошла неделя, и Бонни попросила, чтобы ей подняли перекладину на полтора фута над землей — Когда тебе исполнится шесть лет, — сказал отец, — тогда ты сможешь прыгать выше и я куплю тебе лошадку побольше. У Мистера Батлера коротковаты ноги.
— Ничего подобного! Я перепрыгивала через розовые кусты тети Мелли, а они во-о какие большие!
— Нет, надо подождать, — сказал отец, впервые настаивая на своем. Но твердость его постепенно таяла под воздействием бесконечных настойчивых требований и истерик дочки.
— Ну, ладно, — рассмеявшись, сказал он как-то утром и передвинул повыше узкую белую планку. — Если упадешь, не плачь и не вини меня.
— Мама! — закричала Бонни, запрокинув голову кверху, к окну спальни Скарлетт. — Мама! Смотри! Папа сказал, что я могу прыгать выше!
Скарлетт, расчесывавшая в это время волосы, подошла к окну и улыбнулась возбужденному маленькому существу, такому нелепому в своей перепачканной голубой амазонке.
«Право же, надо будет сшить ей другой костюм, — подумала она. — Только одному богу известно, как я заставлю ее снять всю эту грязь».
— Мама, смотри!
— Смотрю, милочка, — сказала улыбаясь Скарлетт. Ретт поднял малышку и посадил на пони, и Скарлетт при виде ее прямой спинки и горделивой посадки головы вдруг ощутила прилив гордости за дочь и крикнула:
— Какая же ты у меня красавица, моя бесценная!
— Ты тоже, — великодушно откликнулась Бонни и, ударив Мистера Батлера ногой под ребра, галопом понеслась в дальний конец двора к беседке.
— Мама, смотри, как я сейчас прыгну! — крикнула она и ударила хлыстом пони.
«Смотри, как я сейчас прыгну!» Словно прозвонил колокол, будя воспоминания. Что-то зловещее было в этих словах. Что именно? Почему она не может сразу вспомнить? Скарлетт смотрела вниз на свою маленькую дочурку, изящно сидевшую на скачущем пони, и лоб ее прорезала глубокая морщина, а в груди вдруг разлился холод. Бонни мчалась во весь опор, черные кудри ее подскакивали, голубые глаза блестели.
«Глаза у нее совсем как у моего папы, — подумала Скарлетт, — голубые глаза ирландки, и вообще она во всем похожа на него».
И, подумав о Джералде, она сразу вспомнила то, что так старалась выискать в памяти, — вспомнила отчетливо, словно при свете молнии, вдруг озарившем все вокруг, и сердце у нее остановилось. Она вдруг услышала ирландскую песню, услышала стремительный топот копыт вверх по выгону в Таре, услышала беспечный голос, такой похожий на голос ее девочки: «Эллин! Смотри, как я сейчас прыгну!» — Нет, — закричала она. — Нет! Стой, Бонни, стой!
Она еще не успела высунуться из окна, как услышала страшный треск дерева, хриплый крик Ретта, увидела взлетевший голубой бархат и копыта пони, взрыхлившие землю. Затем Мистер Батлер поднялся на ноги и затрусил прочь с пустым седлом.




На третий вечер после смерти Бонни Мамушка медленно проковыляла вверх по ступенькам, ведущим на кухню в доме Мелани. Она была вся в черном — от больших мужских ботинок, разрезанных, чтобы свободнее было пальцам, до черной косынки на голове Ее мутные старые глаза были воспалены, веки покраснели, и от всей монументальной фигуры веяло горем. На лице ее застыло выражение грустного удивления, словно у старой обезьяны, но губы были решительно сжаты.
Она тихо сказала несколько слов Дилси, и та покорно кивнула, словно в их старой вражде наступило перемирие. Дилси поставила тарелки, которые она держала в руках, и прошла через чуланчик в столовую. Минуту спустя Мелани уже появилась на кухне с салфеткой в руке и взволнованно спросила:
— С мисс Скарлетт что-нибудь…
— Мисс Скарлетт держится как всегда, — медленно произнесла Мамушка. — Я не хочу мешать вам ужинать, мисс Мелли. Я могу и подождать — еще успею ведь сказать вам, что у меня на уме.
— Ужин может подождать, — сказала Мелани. — Дилси, подавай, что осталось. Пойдем со мной, Мамушка.
Мамушка вперевалку последовала за ней — через кухню, мимо столовой, где во главе стола сидел Эшли, рядом с ним — Бо, а напротив — двое детей Скарлетт, отчаянно стучавших суповыми ложками. Счастливые голоса Уэйда и Эллы наполняли комнату: им ведь столь долгое пребывание у тети Мелли казалось чем-то вроде пикника. Тетя Мелли была всегда такая добрая, а сейчас особенно. Смерть младшей сестренки почти не произвела на них впечатления. Бонни упала со своей лошадки, и мама долго плакала. А тогда тетя Мелли взяла их к себе поиграть с Бо, и им давали кекс и сладкий пирог, стоило только попросить.
Мелани провела Мамушку в малую гостиную, установленную книжными шкафами, прикрыла дверь и предложила ей присесть на софу.
— Я как раз собиралась прийти к вам после ужина, — сказала она. — Теперь, раз матушка капитана Батлера приехала, похороны, очевидно, будут завтра утром.
— Похороны. Вот с этим-то я к вам и пришла, — сказала Мамушка. — Мисс Мелли, мы все в большой беде, я пришла к вам за помощью. Одни страдания, моя ласточка, одни страдания.
— Мисс Скарлетт что, слегла? — озабоченно спросила Мелани. — Я ведь почти не видела ее с тех пор, как Бонни… Она заперлась у себя в комнате, а капитана Батлера не было дома и…
Вдруг слезы потекли по черному лицу Мамушки. Мелани села рядом с ней, принялась гладить ее по плечу, и через какое-то время Мамушка приподняла свою черную юбку и вытерла глаза.
— Вы должны пойти помочь нам, мисс Мелли. Я-то делаю все что могу, да толку чуть.
— Мисс Скарлетт… Мамушка выпрямилась:
— Мисс Мелли, вы-то не хуже меня знаете мисс Скарлетт. Сколько страдало это бедное дитя, дай ей только бог силы все вынести… А уж это последнее горе и вовсе разбило ей сердце. Но она выстоит. А пришла-то я к вам из-за мистера Ретта.
— Мне так хотелось повидать его, но всякий раз, как я к вам заходила, он был либо в городе, либо сидел, запершись у себя в комнате с… А Скарлетт ходила словно привидение и молчала… Да говори же скорее, Мамушка. Ты ведь знаешь, что я помогу, если только это в моих силах.
Мамушка вытерла нос тыльной стороной ладони.
— Я говорю вам, мисс Скарлетт все может выдержать, что господь пошлет, потому как ей уже много испытаний было послано, а вот мистер Ретт… Мисс Мелли, он ведь никогда ничего не терпел, ежели ему не по нраву, никогда, ничегошеньки. Вот из-за него-то я к вам и пришла.
— Но…
— Мисс Мелли, вам надо пойти сегодня к нам — сейчас, вечером. — В голосе Мамушки звучала настоятельная мольба. — Может, мистер Ретт послушает вас. Он Ведь всегда так высоко вас ставил.
— Ах, Мамушка, да что же это? О чем ты говоришь? Мамушка распрямила плечи.
— Мисс Мелли, мистер Ретт, он.., совсем ума решился. Не хочет, чтобы мы увозили маленькую мисс.
— Ума решился? Ну, что ты, Мамушка, нет!
— Я не вру. Чистая правда. Он не даст нам похоронить дитятко. Он так мне сам и сказал — только час назад.
— Но не может же он… Ведь он же…
— Вот потому я и говорю, что он ума решился.
— Но почему…
— Мисс Мелли, я вам не все сказала. Не должна я говорить, да только вы ведь все равно нам как родная. И только вам я и могу сказать. Я вам все говорю. Вы же знаете, как он любил свое дитятко. Никогда еще я не видела, чтоб мужчина, черный ли, белый, так любил свое дитятко. Он сразу как с ума сошел, когда доктор Мид сказал: «У нее шейка сломалась». Он тогда схватил ружье, побежал и пристрелил этого бедного пони, а я, клянусь богом, думала, он пристрелит и себя. Я совсем было растерялась: мисс Скарлетт лежит в обмороке, все соседи по дому бегают — туда-сюда, а мистер Ретт держит свою дочку и не дает мне даже вымыть ей личико, а оно все в земле было измазано. А когда мисс Скарлетт пришла в себя, я подумала: слава тебе господи! Теперь они хоть утешат друг дружку. — Слезы снова полились, но на этот раз Мамушка даже не пыталась их утирать. — Да только как она пришла в себя, кинулась в комнату, где он сидел с мисс Бонни на руках, и говорит: «Отдайте мне моего ребенка, вы убили ее».
— Ах, нет! Не могла она так сказать!
— Да, мэм, так и сказала. Слово в слово: «Вы убили ее». А мне так стало жалко мистера Ретта, и я как заплачу, потому вид у него был точно у побитой собаки. Я и сказала: «Отдайте дитятко няне. Не позволю я, чтобы такое говорили над бедной моей маленькой мисс», — и забрала я у него доченьку, и отнесла к ним в комнату, и вымыла ей личико. Но я все слышала, что они говорили, и у меня прямо кровь стыла — чего они только не наговорили друг другу. Мисс Скарлетт обозвала его убивцем» — зачем он позволил деточке прыгать так высоко. А он сказал, что мисс Скарлетт плевать было на Бонни и на всех своих детей ей наплевать…
— Замолчи, Мамушка! Не рассказывай мне больше. Нехорошо, чтобы ты мне такое рассказывала! — воскликнула Мелани, стараясь отогнать от себя страшную картину, возникшую перед се глазами, пока она слушала Мамушку.
— Я знаю, не мое это дело вам говорить, но уж больно у меня на душе накипело, так что я сама не знаю, что и говорю. Значит, потом взял он ее и сам отнес к гробовщику, а когда принес назад, положил на кроватку у себя в комнате. Тут мисс Скарлетт говорит — надо ей лежать в гостиной в гробике, так я думала, мистер Ретт ударит ее. Но он только холодно ей сказал: «Ей место в моей комнате». А потом повернулся ко мне и говорит: «Мамушка, последи, чтоб она лежала тут, пока я вернусь». А сам вскочил на лошадь и ускакал. И до самого заката не возвращался. А как вернулся домой, вижу — он выпивши, крепко выпивши, но только, как всегда, держится. Влетел он в дом, и ни с мисс Скарлетт, ни с мисс Питти, ни с кем из леди, которые пришли навестить нас, — ни слова. Взлетел по лестнице, распахнул дверь в свою комнату, да как закричит мне… А когда я прибежала — быстро бежала, как только могла, — стоит он у кроватки, а в комнате темнотища — я его едва видела, потому как ставни-то ведь были закрыты.
Он и говорит мне этак грубо: «Открой ставни — темно». Я их открываю, а он смотрит на меня и, ей-богу, мисс Мелли, у меня прямо колени подогнулись — такой он страшный был. А потом и говорит: «Принеси свечей. Да побольше. И пусть все горят. И не смей закрывать ставни и опускать шторы. Разве ты не знаешь, что мисс Бонни боится темноты?» Расширенные от ужаса глаза Мелани встретились с глазами Мамушки, и та многозначительно кивнула.
— Так и сказал: «Мисс Бонни боится темноты». — Мамушку передернуло. — Принесла я ему с десяток свечей, а он говорит:
«Уходи!» А потом запер дверь и сидел там с маленькой мисс, и не открыл дверь даже мисс Скарлетт — даже когда она принялась стучать и кричать. И так оно уже два дня. И про похороны он ничего не говорит. Утром открывает дверь, потом снова запирает, садится на лошадь и — в город. А потом на закате приезжает пьяный, снова запирается, и ничего не ест, и ни чуточки не спит. А тут приехала его матушка, старая мисс Батлер, — приехала из Чарльстона на похороны, и мисс Сьюлин и мистер Уилл из Тары приехали, да только мистер Ретт ни с кем не желает говорить. Ох, мисс Мелли, это ужас что! А будет и хуже, и люди начнут говорить — скандал, да и только.
— Ну, а сегодня вечером, — помолчав немного, продолжала Мамушка и снова вытерла нос тыльной стороной ладони, — сегодня вечером мисс Скарлетт подстерегла его наверху, когда он воротился, и вошла с ним в комнату, и говорит: «Похороны будут завтра утром». А он говорит: «Только устройте похороны, и я вас завтра же убью».
— Ах, он, должно быть, помешался!
— Да, мэм. А потом она заговорила так тихо, и я ничего не слышала — слышала только, как он снова сказал, что мисс Бонни боится темноты, а в могиле очень темно. А немного погодя мисс Скарлетт сказала: «Чего вы так убиваетесь? Сами же убили ее, чтобы потешить свою гордость». А он ей говорит: «Неужели у вас нет жалости?» А она говорит: «Нет. И ребенка у меня тоже нет. Я устала от того, как вы себя ведете с тех пор, как погибла Бонни. Это же скандал — о вас весь город говорит. Вы все время пьяный, и если думаете, я такая глупая, что не знаю, где вы время проводите, то вы просто дурак. Я же знаю, что вы все дни у этой твари — Красотки Уотлинг».
— Ах, Мамушка, нет!
— Да, мэм, так она сказала. И, мисс Мелли, это правда. Негры — они узнают очень многое куда быстрее белых людей, и я знала, где он был, да только никому слова не сказала. А он и не отпирался. Говорит: «Да, мэм, именно там я и был. И нечего вам прикидываться — вам ведь все равно. Веселый дом — это рай для несчастного по сравнению с этим адом, в котором я живу. А у Красотки — самое доброе сердце на свете. Она не говорила мне, что я убил моего ребенка».
— О-о! — воскликнула Мелани, пораженная в самое сердце. Ее собственная жизнь текла так мирно, она была так защищена, так ограждена всеми любившими ее людьми, так полна добра, что картина, возникавшая из рассказа Мамушки, была выше ее понимания, она не могла этому поверить. И однако, в мозгу ее шевельнулось воспоминание, которое она поспешила отодвинуть в глубь памяти, как попыталась бы забыть чью-то случайно увиденную наготу. В тот день, когда Ретт плакал, уткнувшись головой ей в колени, он упоминал имя Красотки Уотлинг. Но любил-то он Скарлетт, она не могла ошибиться, и конечно же, Скарлетт любит его. Что же встало между ними? Как могут муж и жена так больно ранить друг друга, вооружившись каждый самым острым ножом?
А Мамушка тем временем с трудом продолжала свой рассказ:
— Потом мисс Скарлетт вышла из комнаты, белая как простыня, но губы сжаты эдак крепко, упрямо, увидала меня, что я стою там, и говорит: «Похороны завтра, Мамушка». И прошла мимо, как привидение. Тут сердце у меня перевернулось, потому как ежели мисс Скарлетт что скажет, так она и сделает. И мистер Ретт тоже — как скажет, так и сделает. А ведь он сказал, что убьет ее, ежели она это сделает. Тут уж я совсем растерялась, мисс Мелли, потому как есть у меня на совести одно дело и очень оно меня давит. Мисс Мелли, это я ведь напугала маленькую мисс темнотой-то.
— Ох, Мамушка, ну, какое это имеет значение — сейчас.
— Да нет, мэм, имеет. Из-за этого-то вся и беда. И вот надумала я: скажу-ка я все мистеру Ретту — пусть он убьет меня, да только не могу я молчать, потому как это лежит у меня на совести. Я в дверь — шасть, покуда он не запер ее, и говорю: «Мистер Ретт. Я пришла вам исповедаться». А он повернулся ко мне, точно сумасшедший, и говорит: «Уходи!» — и клянусь богом, никогда еще я так не пугалась! А все равно говорю ему: «Пожалуйста, мистер Ретт, уж вы дайте мне сказать, не то я просто помру. Ведь это я напугала маленькую мисс темнотой-то». И тут, мисс Мелли, опустила я голову и стала ждать, когда он меня ударит. А он молчит. И тогда я говорю: «Я ведь ничего дурного не хотела. Да только, мистер Ретт, дитятко-то ведь было уж больно неосторожное, ничего она не боялась. Как все лягут спать, она вылезет из кроватки и ну по всему дому бегать босая. А я беспокоилась, потому как боялась — вдруг она себя зашибет. Вот я и сказала ей, что там в темноте живут привидения и оборотни».
И знаете, мисс Мелли, что он сделал? Лицо у него стало такое доброе, подошел он ко мне и положил мне руку на плечо. А это в первый раз так. И говорит: «Она была очень храбрая, верно? Только темноты боялась, а больше не боялась ничего». Я тут как расплачусь, а он и говорит: «Ну-ну, Мамушка». И погладил меня. «Ну, Мамушка, не надо так убиваться. Я рад, что ты мне сказала. Я знаю, ты любишь мисс Бонни, и раз ты ее любишь, значит, все это пустяки. Важно то, что у человека на сердце». Ну, я тут немножечко прибодрилась, собралась с духом и говорю: «Мистер Ретт, а как же насчет похорон-то будет?» Тут он повернулся ко мне точно бешеный — глаза сверкают — и говорит: «Господи боже мой, я-то думал, хоть ты понимаешь, раз никто больше не понимает! Да неужто, ты думаешь, я позволю, чтоб мое дитя положили в темноту, когда она так боится темноты? Так и слышу, как она кричала, когда просыпалась в темноте. А я не хочу ее пугать». Тут, мисс Мелли, я и поняла, что он ума решился. Пьяный он и не спит и не ест, но это — что. Главное, он совсем рехнулся. Вытолкал меня за зверь и говорит: «Убирайся к черту отсюда!» А я иду вниз по лестнице и думаю себе: он говорит, что никаких похорон не будет, а мисс Скарлетт говорит — завтра утром, и он говорит, что пристрелит ее. А в доме полно родственников, и все соседи уже чешут языки и кудахтают, точно куры на насесте, вот я и подумала про вас, мисс Мелли. Вы должны пойти помочь нам.
— Ох, Мамушка, не могу я в такое вмешиваться!
— Если вы не можете, кто же может-то?
— Ну, что я могу сделать, Мамушка?
— Не знаю я, мисс Мелли, но что-то вы можете сделать. Поговорите с мистером Реттом, глядишь — он вас и послушает. Он ведь очень высоко вас ставит, мисс Мелли, может, вы этого и не знаете, а только это так. Сколько раз я слыхала, как он говорил, что только вы и есть настоящая леди.
Мелани поднялась в смятении, сердце у нее мучительно заколотилось при одной мысли о встрече с Реттом. Она вся похолодела, понимая, что придется спорить с человеком, судя по описаниям Мамушки, потерявшим разум от горя У нее заныло сердце, стоило ей подумать, что придется войти в ярко освещенную комнату, где лежит маленькая девочка, которую она так любила. Что делать? Что сказать Ретту, чтобы утешить его горе и вернуть ему разум? С минуту Мелани стояла, не зная на что решиться, и вдруг сквозь закрытую дверь до нее донесся звонкий смех ее мальчика. И словно ножом по сердцу резанула мысль, что было бы, если бы он умер. Что, если бы ее Бо лежал сейчас наверху, застывший и холодный, а его веселый смех навсегда угас.
— О! — испуганно воскликнула она и мысленно прижала сына к сердцу: она поняла, каково сейчас Ретту. Если бы Бо умер, разве она позволила бы его увезти, чтоб его положили где-то под дождем и ветром, в темноте? — Ох, бедный, бедный капитан Батлер! — воскликнула она. — Я сейчас же пойду к нему, немедленно.
Она поспешила в столовую, тихо сказала несколько слов Эшли и, к удивлению своего мальчика, крепко прижала его к себе и пылко поцеловала светлые кудряшки.
Она выбежала из дома без шляпы, все еще сжимая в руке обеденную салфетку, и шла так быстро, что старые ноги Мамушки еле поспевали за ней. Войдя в парадный холл дома Скарлетт, она наспех кивнула сидевшим в библиотеке — испуганной мисс Питтипэт, величественной старухе миссис Батлер, Уиллу и Сьюлин, затем быстро поднялась по лестнице; за ней, задыхаясь, следовала Мамушка. На секунду Мелани задержалась, проходя мимо закрытой двери Скарлетт, но Мамушка прошептала:
— Нет, мэм, не надо этого делать.
Площадку Мелани пересекла уже гораздо медленнее и, подойдя к комнате Ретта, остановилась. С минуту она постояла в нерешительности, словно хотела повернуться и бежать. Затем, призвав на помощь все свое мужество, как маленький солдатик перед тем, как идти в атаку, постучала в дверь и тихо сказала:
— Впустите меня, пожалуйста, капитан Батлер. Это миссис Уилкс. Я хочу увидеть Бонни.
Дверь быстро отворилась, и Мамушка, отступившая в глубину лестничной площадки, где было темно, увидела огромный черный силуэт Ретта на фоне ярко горящих свечей. Он пошатывался, и до Мамушки долетел запах виски. Секунду он смотрел на Мелли, затем взял ее за плечо, втолкнул в комнату и закрыл дверь.
Мамушка тихонько прокралась к двери и устало опустилась на стоявший возле нее стул, слишком маленький для ее могучего тела. Она сидела совсем тихо, беззвучно плакала и молилась. Время от времени она поднимала подол платья и вытирала глаза. Сколько ни напрягала она слух, но не могла разобрать ни слова — из-за двери доносился лишь тихий прерывистый гул голосов. После бесконечно долгого ожидания дверь приотворилась и выглянула Мелани — лицо у нее было белое, напряженное.
— Принеси мне кофейник с кофе, да побыстрей, и несколько сандвичей.
Когда дьявол вселялся в Мамушку, она начинала двигаться со стремительностью стройной шестнадцатилетней негритяночки, а сейчас ей так хотелось попасть в комнату Ретта и ее разбирало такое любопытство, что она двигалась еще быстрее. Но надежды ее не оправдались, ибо Мелани лишь приоткрыла дверь и взяла у нее поднос. Долгое время Мамушка прислушивалась, напрягая свой острый слух, но не могла различить ничего, кроме позвякиванья серебра о фарфор да приглушенного, мягкого голоса Мелани. Потом она услышала, как заскрипела кровать под тяжелым телом, и вскоре — стук сапог, сбрасываемых на пол. Через некоторое время в двери появилась Мелани. Но как Мамушка ни старалась, ничего разглядеть в комнате она не смогла. Вид у Мелани был усталый и на ресницах блестели слезы, но лицо было снова спокойное.
— Пойди скажи Скарлетт, что капитан Батлер согласен, чтобы похороны состоялись завтра утром, — прошептала она.
— Слава тебе господи! — воскликнула Мамушка. — Да как же это…
— Не кричи так. Он сейчас заснет. И еще. Мамушка, скажи мисс Скарлетт, что я буду здесь всю ночь, и принеси мне кофе. Принеси сюда…
— В эту самую комнату?
— Да, я обещала капитану Батлеру, что, если он заснет, я посижу с Бонни всю ночь. А теперь иди, скажи все мисс Скарлетт, чтобы она больше не волновалась.
Мамушка прошла через площадку — под ее тяжестью ходуном заходили половицы, — и сердце ее, успокоившись, пело: «Аллилуйя, аллилуйя». У двери в комнату Скарлетт она приостановилась и задумалась — благодарность к Мелани боролась в ней с любопытством.
«И как это мисс Мелани удалось, ума не приложу. Видно, ей помогают ангелы. Про то, что завтра похороны, я мисс Скарлетт скажу, а вот про то, что мисс Мелли будет всю-ночь сидеть с маленькой мисс, лучше утаю. Не понравится это мисс Скарлетт».

