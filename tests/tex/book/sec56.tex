\chapter{\ }

Ретт отсутствовал целых три месяца, и за все это время Скарлетт не получила от него ни слова. Она не знала ни где он, ни сколько продлится его отсутствие. Она даже понятия не имела, вернется ли он вообще. Все это время она занималась своими делами, высоко держа голову и глубоко страдая в душе. Она не очень хорошо себя чувствовала, но, побуждаемая Мелани, каждый день бывала в лавке и старалась хотя бы внешне поддерживать интерес к лесопилкам. Впервые лавка тяготила ее, и хотя она принесла тройной доход по сравнению с предыдущим годом — деньги так и текли рекой, — Скарлетт не могла заставить себя интересоваться этим делом, была резка и груба с приказчиками. Лесопилка Джонни Гэллегера тоже процветала, и на лесном складе без труда продавали все подчистую, но что бы Джонни ни говорил и ни делал, все раздражали ее. Джонни, будучи, как и она, ирландцем, наконец вспылил от ее придирок и, пригрозив, что уйдет, произнес на этот счет длинную тираду, которая кончилась так: «На этом я умываю, мэм, руки, и проклятье Кромвеля да падет на ваш дом». Чтобы утихомирить его, Скарлетт пришлось долго и униженно перед ним извиняться.
На лесопилке Эшли она не была ни разу. Не заходила она и в контору на лесном складе, если могла предположить, что он там. Она знала, что он избегает ее, знала, что ее частое присутствие в доме по настоятельной просьбе Мелани было для него мукой. Они никогда не оставались вдвоем, ни разу не говорили, а ей не терпелось задать ему один вопрос. Ей так хотелось знать, не возненавидел ли он ее, а также что он сказал Мелани, но Эшли держался от нее на расстоянии и беззвучно молил не заговаривать с ним. Ей было невыносимо видеть его лицо, постаревшее, измученное раскаянием, а то, что его лесопилка каждую неделю приносила лишь убытки, тем более раздражало ее, но она молчала.
Его беспомощность выводила ее из себя. Она не знала, что он мог бы сделать, чтобы улучшить положение, но считала, что он должен что-то сделать. Вот Ретт — тот непременно что-то предпринял бы. Ретт всегда что-то предпринимал — пусть даже не то, что надо, — и она невольно уважала его за это.
Теперь, когда злость на Ретта и его оскорбления прошла, ей стало недоставать его, и она все больше и больше скучала по нему по мере того, как шли дни, а вестей от него не было. Из сложного клубка чувств, который он оставил в ней, — восторга и гнева, душевного надрыва и уязвленной гордости, — родилась меланхолия и, точно ворон, уселась на ее плече. Она тосковала по Ретту, ей недоставало легкой дерзости его анекдотов, вызывавших у нее взрывы хохота, его иронической усмешки, которая сразу ставила все на свои места, не давая преувеличивать беды, — недоставало даже его издевок, больно коловших ее, вызывавших злобные реплики в ответ. Больше же всего ей недоставало его присутствия, недоставало человека, которому можно все рассказать. А лучшего слушателя, чем Ретт, и пожелать было трудно. Она могла без зазрения совести, даже с гордостью, рассказывать ему, как ободрала кого-нибудь точно липку, и он лишь аплодировал ей. А другим она не могла даже намекнуть на нечто подобное, ибо это лишь шокировало бы их.
Ей одиноко было без Ретта и без Бонни. Она скучала по малышке больше, чем могла предположить. Вспоминая последние жестокие слова Ретта об ее отношении к Уэйду и к Элле, она старалась заполнить ими пустые часы. Но все было ни к чему. Слова Ретта и поведение детей открыли Скарлетт глаза на страшную, больно саднившую правду. Пока дети были маленькие, она была слишком занята, слишком поглощена заботами о том, где достать денег, слишком была с ними резка и нетерпима и не сумела завоевать ни их доверие, ни любовь. А теперь было слишком поздно или, быть может, у нее не хватало терпения или ума проникнуть в тайну их сердечек.
Элла! Скарлетт крайне огорчилась, поняв, что Элла — неумная девочка, но это было именно так. Ее умишко ни на чем не задерживался — мысли порхали, как птички с ветки на ветку, и даже когда Скарлетт принималась ей что-то рассказывать, Элла с детской непосредственностью прерывала ее, задавая вопросы, не имевшие никакого отношения к рассказу, и прежде чем Скарлетт успевала дать пояснения, забывала, о чем спрашивала. Что же до Уэйда.., возможно, Ретт прав. Возможно, мальчик боится ее.
Это казалось Скарлетт странным и обидным. Ну, почему сын, единственный сын, должен бояться ее? Когда она пыталась втянуть Уэйда в разговор, на нее смотрели бархатные, карие глаза Чарльза, мальчик ежился и смущенно переминался с ноги на ногу. А вот с Мелани он болтал без умолку и показывал ей все содержимое своих карманов, начиная с червей для рыбной ловли и кончая обрывками веревок.
Мелани умела обращаться с детишками. Тут уж ничего не скажешь. Ее Бо был самым воспитанным и самым прелестным ребенком в Атланте. Скарлетт куда лучше ладила с ним, чем с собственным сыном, потому что маленький Бо не стеснялся со взрослыми и, увидев ее, тут же, без приглашения, залезал к ней на колени. Какой это была прелестный блондинчик — весь в Эшли! Вот если бы Уэйд был как Бо… Конечно, Мелани могла так много дать сыну потому, что это было ее единственное дитя, да к тому же не было у нее таких забот и не работала она, как Скарлетт. Во всяком случае, Скарлетт пыталась таким образом оправдаться перед собой, однако элементарная честность вынуждала ее признать, что Мелани любит детей и была бы рада, если бы у нее был их десяток. Недаром она с таким теплом относилась к Уэйду и ко всем соседским малышам.
Скарлетт никогда не забудет, как однажды, приехав к Мелани, чтобы забрать Уэйда, она шла по дорожке и вдруг услышала клич повстанцев, очень точно воспроизведенный ее сыном — тем самым Уэйдом, который дома был всегда тише мышки. А вслед за Криком Уэйда раздался пронзительный тоненький взвизг Бо. Войдя в гостиную, она обнаружила, что эти двое, вооружившись деревянными мечами, атакуют диван. Оба мгновенно умолкли, а из-за дивана поднялась Мелани, смеясь и подбирая рассыпанные шпильки, которыми она пыталась заколоть свои непослушные кудри.
— Это Геттисберг, — пояснила она. — Я изображаю янки, и мне, конечно, сильно досталось. А это генерал Ли, — указала она на Бо, — а это генерал Пиккет, — И она обняла за плечи Уэйда.
Да, Мелани умела обращаться с детьми, и тайны этого Скарлетт никогда не понять.
«По крайней мере, — подумала. Скарлетт, — хоть Бонни любит меня, и ей нравится со мной играть». Но честность вынуждала ее признать, что Бонни куда больше предпочитает Ретта. Да к тому же она может вообще больше не увидеть Бонни. Ведь Ретт, возможно, находится сейчас в Персии или в Египте и — как знать? — возможно, намерен остаться там навсегда.
Когда доктор Мид сказал Скарлетт, что она беременна, она была потрясена, ибо ожидала услышать совсем другой диагноз — что у нее разлитие желчи и нервное перенапряжение. Но тут она вспомнила ту дикую ночь и покраснела. Значит, в те минуты высокого наслаждения был зачат ребенок, хотя память о самом наслаждении и отодвинула на задний план то, что произошло потом. Впервые в жизни Скарлетт обрадовалась, что у нее будет ребенок. Хоть бы мальчик! Хороший мальчик, а не такая мямля, как маленький Уэйд. Как она будет заботиться о нем! Теперь, когда у нее есть для ребенка свободное время и деньги, которые облегчат его путь по жизни, как она будет счастлива заняться им! Она хотела было тотчас написать Ретту на адрес матери в Чарльстон. Силы небесные, теперь-то он уж должен вернуться домой! А что, если он задержится и ребенок родится без него?! Она же ничего не сможет объяснить ему потом! Но если написать, он еще подумает, что она хочет, чтобы он вернулся, и только станет потешаться над ней. А он не должен знать, что она хочет, чтобы он был рядом или что он нужен ей.
Она порадовалась, что подавила в себе желание написать Ретту, когда получила письмо от тети Полин из Чарльстона, где, судя по всему, гостил у своей матери Ретт. С каким облегчением узнала она, что он все еще в Соединенных Штатах, хотя письмо тети Полин само по себе вызвало у нее вспышку злости. Ретт зашел с Бонни навестить ее и тетю Евлалию, и уж как Полин расхваливала девочку.
«До чего же она хорошенькая! А когда вырастет, станет просто красавицей. Но ты, разумеется, понимаешь, что любому мужчине, который вздумает за ней ухаживать, придется иметь дело с капитаном Батлером, ибо никогда еще я не видела такого преданного отца. А теперь, дорогая моя, хочу тебе кое в чем признаться. До встречи с капитаном Батлером я считала, что твой брак с ним — страшный мезальянс, ибо у нас в Чарльстоне никто не слышал о нем ничего хорошего и все очень жалеют его семью. Мы с Евлалией даже не были уверены, следует ли нам его принимать, но ведь в конце концов милая крошка, с которой он собрался к нам прийти, — наша внучка. Когда же он появился у нас, мы были приятно удивлены — очень приятно — и поняли, что христиане не должны верить досужим сплетням. Он совершенно очарователен. И к тому же, как нам кажется, хорош собой — такой серьезный и вежливый. И так предан тебе и малышке.
А теперь, моя дорогая, я должна написать тебе о том, что дошло до наших ушей, — мы с Евлалией сначала и верить этому не хотели. Мы, конечно, слышали, что ты порою трудишься в лавке, которую оставил тебе мистер Кеннеди. Эти слухи доходили до нас, но мы их отрицали. Мы понимали, что в те первые страшные дни после конца войны это было, возможно, необходимо — такие уж были тогда условия жизни. Но сейчас ведь в этом нет никакой необходимости, поскольку, как мне известно, капитан Батлер — более чем обеспечен и, кроме того, вполне способен управлять вместо тебя любым делом и любой собственностью. Нам, просто необходимо было знать, насколько справедливы эти слухи, и мы вынуждены были задать капитану Батлеру некоторые вопросы, хотя это и было для нас крайне неприятно.
Он нехотя сообщил нам, что ты каждое утро проводишь в лавке и никому не позволяешь вести за тебя бухгалтерию. Он признался также, что у тебя есть лесопилка или лесопилки (мы не стали уточнять, будучи крайне расстроены этими сведениями, совсем для нас новыми), что побуждает тебя разъезжать одной или в обществе какого-то бродяги, который, по словам капитана Батлера, — просто убийца. Мы видели, как это переворачивает ему душу, и решили, что он самый снисходительный — даже слишком снисходительный — муж. Скарлетт, это надо прекратить. Твоей матушки уже нет на свете, чтобы сказать тебе это, и вместо нее обязана тебе сказать это я. Подумай только, каково будет твоим детям, когда они вырастут и узнают, что ты занималась торговлей! Как им будет горько, когда они узнают, что тебя могли оскорблять грубые люди и что своими разъездами по лесопилкам ты давала повод для неуважительных разговоров и сплетен. Такое неженское поведение…» Скарлетт ругнулась и, не дочитав письма, отшвырнула его. Она так и видела тетю Полин и тетю Евлалию, которые сидят в своем ветхом домишке на Бэттери и осуждают ее, а ведь сами еле сводят концы с концами и умерли бы с голоду, если бы она, Скарлетт, не помогала им каждый месяц. Неженское поведение? Да если бы она, черт побери, не занималась неженскими делами, у тети Полин и тети Евлалии не было бы сейчас, наверное, крыши над головой. Черт бы побрал этого Ретта — зачем он рассказал им про лавку, про бухгалтерию и про лесопилки! Рассказал, значит, нехотя, да? Она-то знала с каким удовольствием он изображал из себя перед старухами этакого серьезного, заботливого, очаровательного человека, преданного мужа и отца. А самому доставляло несказанное наслаждение расписывать им ее занятия в лавке, на лесопилках и в салуне. Не человек, а дьявол. И почему только он так любит делать людям гадости?
Однако злость Скарлетт очень скоро сменилась апатией. За последнее время из ее жизни исчезло многое, придававшее ей остроту. Вот если бы вновь познать былое волнение и радость от присутствия Эшли.., вот если бы Ретт вернулся домой и смешил бы ее, как прежде.




Они вернулись домой без предупреждения. Об их возвращении оповестил лишь стук сундуков, сбрасываемых на пол в холле, да голосок Бонни, крикнувшей: «Мама!» Скарлетт выскочила из своей комнаты наверху и увидела дочурку, пытавшуюся забраться по ступенькам, высоко поднимая свои коротенькие, толстенькие ножки. К груди она прижимала полосатого котенка.
— Бабушка мне дала, — возбужденно объявила Бонни, поднимая котенка за шкирку.
Скарлетт подхватила ее на руки и принялась целовать, благодаря бога за то, что присутствие девочки избавляет ее от встречи с Реттом наедине. Глядя поверх головки Бонни, она увидела, как он внизу, в холле, расплачивался с извозчиком. Он посмотрел вверх, увидел ее и, широким взмахом руки сняв панаму, склонился в поклоне. Встретившись с ним взглядом, она почувствовала, как у нее подпрыгнуло сердце. Каков бы он ни был, что бы он ни сделал, но он дома, и она была рада.
— А где Мамушка? — спросила Бонни, завертевшись на руках у Скарлетт, и та нехотя опустила ребенка на пол.
Да, ей будет куда труднее, чем она предполагала, достаточно небрежно поздороваться с Реттом. А уж как сказать ему о ребенке, которого она ждет!.. Она смотрела на его лицо, пока он поднимался по ступенькам, — смуглое беспечное лицо, такое непроницаемое, такое замкнутое. Нет, сейчас она ничего ему не скажет. Не может сказать, И однако же о такого рода событии должен прежде всего знать муж — муж, который обрадуется, услышав. Но она не была уверена, что он обрадуется.
Она стояла на площадке лестницы, прислонившись к перилам, и думала, поцелует он ее или нет. Он не поцеловал. Он сказал лишь:
— Что-то вы побледнели, миссис Батлер. Что, румян в продаже нет?
И ни слова о том, что он скучал по ней — пусть даже этого на самом деле не было. По крайней мере мог бы поцеловать ее при Мамушке, которая уже уводила Бонни в детскую. Ретт стоял рядом со Скарлетт на площадке и небрежно оглядывал ее.
— Уж не потому ли вы так плохо выглядите, что тосковали по мне? — спросил он, и хотя губы его улыбались, в глазах не было улыбки.
Так вот, значит, как он намерен себя с ней держать. Столь же отвратительно, как всегда. Внезапно ребенок, которого она носила под сердцем, из счастливого дара судьбы превратился в тошнотворное бремя, а этот человек, стоявший так небрежно, держа широкополую панаму у бедра, — в ее злейшего врага, причину всех зол. И потому в глазах ее появилось ожесточение — ожесточение, которого он не мог не заметить, и улыбка сбежала с его лица.
— Если я бледная, то по вашей вине, а вовсе не потому, что скучала по вас, хоть вы и воображаете, что это так. На самом же деле… — О, она собиралась сообщить ему об этом совсем иначе, но слова сами сорвались с языка, и она бросила ему, не задумываясь над тем, что их могут услышать слуги: — Дело в том, что у меня будет ребенок!
Он судорожно глотнул, и глаза его быстро скользнули по ее фигуре. Он шагнул было к ней, словно хотел дотронуться до ее плеча, но она увернулась, и в глазах ее было столько ненависти, что лицо его стало жестким.
— Вот как! — холодно произнес он. — Кто же счастливый отец? Эшли?
Она вцепилась в балясину перил так крепко, что уши вырезанного на них льва до боли врезались ей в ладонь. Даже она, которая так хорошо знала его, не ожидала такого оскорбления. Конечно, это шутка, но шутка слишком чудовищная, чтобы с нею мириться. Ей хотелось выцарапать ему ногтями глаза, чтобы не видеть в них этого непонятного сияния.
— Да будьте вы прокляты! — сказала она голосом, дрожавшим от ярости. — Вы.., вы же знаете, что это ваш ребенок. И мне он не нужен, как и вам. Ни одна.., ни одна женщина не захочет иметь ребенка от такой скотины. Хоть бы.., о господи, хоть бы это был чей угодно ребенок, только не ваш!
Она увидела, как вдруг изменилось его смуглое лицо — задергалось от гнева или от чего-то еще, словно его ужалили.
«Вот! — подумала она со жгучим злорадством. — Вот! Наконец-то я причинила ему боль!» Но лицо Ретта уже снова приняло обычное непроницаемое выражение; он пригладил усики с одной стороны.
— Не огорчайтесь, — сказал он и, повернувшись, пошел дальше, — может, у вас еще будет выкидыш.
Все закружилось вокруг нее: она подумала о том, сколько еще предстоит вынести до родов — изнурительная тошнота, уныло тянущееся время, разбухающий живот, долгие часы боли. Ни один мужчина обо всем этом понятия не имеет. И он еще смеет шутить! Да она сейчас расцарапает его. Только вид крови на его смуглом лице способен утишить боль в ее сердце. Она стремительно подскочила к нему точно кошка, но он, вздрогнув от неожиданности, отступил и поднял руку, чтобы удержать ее. Остановившись на краю верхней, недавно натертой ступеньки, она размахнулась, чтобы ударить его, но, наткнувшись на его вытянутую руку, потеряла равновесие. В отчаянном порыве она попыталась было уцепиться за балясину, но не смогла. И полетела по лестнице вниз головой, чувствуя, как у нее внутри все разрывается от боли. Ослепленная болью, она уже и не пыталась за что-то схватиться и прокатилась так на спине до конца лестницы.




Впервые в жизни Скарлетт лежала в постели больная — если не считать тех случаев, когда она рожала, но это было не в счет. Тогда она не чувствовала себя одинокой и ей не было страшно — сейчас же она лежала слабая, растерянная, измученная болью. Она знала, что серьезно больна — куда серьезнее, чем ей говорили, — и смутно сознавала, что может умереть. Сломанное ребро отзывалось болью при каждом вздохе, ушибленные лицо и голова болели, и все тело находилось во власти демонов, которые терзали ее горячими щипцами и резали тупыми ножами и лишь ненадолго оставляли в покое, настолько обессиленную, что она не успевала прийти в себя, как они возвращались. Нет, роды были совсем не похожи на это. Через два часа после рождения Уэйда, и Эллы, и Бонни она уже уплетала за обе щеки, а сейчас даже мысль о чем-либо, кроме холодной воды, вызывала у нее тошноту.
Как, оказывается, легко родить ребенка и как мучительно — не родить! Удивительно, что, даже несмотря на боль, у нее сжалось сердце, когда она узнала, что у нее не будет ребенка. И еще удивительнее то, что это был первый ребенок, которого она действительно хотела иметь. Она попыталась понять, почему ей так хотелось этого ребенка, но мозг ее слишком устал. Она не в состоянии была думать ни о чем — разве что о том, как страшно умирать. А смерть присутствовала в комнате, и у Скарлетт не было сил противостоять ей, бороться с нею. И ей было страшно. Ей так хотелось, чтобы кто-то сильный сидел рядом, держал ее за руку, помогал ей сражаться со смертью, пока силы не вернутся к ней, чтобы она сама могла продолжать борьбу.
Боль прогнала злобу, и Скарлетт хотелось сейчас, чтобы рядом был Ретт. Но его не было, а заставить себя попросить, чтобы он пришел, она не могла.
В последний раз она видела его, когда он подхватил ее на руки в темном холле у подножия лестницы; лицо у него было белое, искаженное от страха, и он хриплым голосом звал Мамушку. Потом еще она смутно помнила, как ее несли наверх, а дальше все терялось во тьме. А потом — боль, снова боль, и комната наполнилась жужжанием голосов, звучали всхлипыванья тети Питтипэт, и резкие приказания доктора Мида, и топот ног, бегущих по лестнице, и тихие шаги на цыпочках в верхнем холле. А потом — слепящий свет, сознание надвигающейся смерти и страх, наполнивший ее желанием крикнуть имя, но вместо крика получился лишь шепот.
Однако жалобный этот шепот вызвал мгновенный отклик, и откуда-то из темноты, окружавшей постель, раздался нежный напевный голос той, кого она звала:
— Я здесь, дорогая. Я все время здесь.
Смерть и страх начали постепенно отступать, когда Мелани взяла ее руку и осторожно приложила к своей прохладной щеке. Скарлетт попыталась повернуть голову, чтобы увидеть ее лицо, но не смогла. Мелли ждет ребенка, а к тому подступают янки. Город в огне, и надо спешить, спешить Но ведь Мелли ждет ребенка и спешить нельзя. Надо остаться с ней, пока не родится ребенок, и не падать духом, потому что Мелли нужна ее сила. Мелли мучила ее — снова горячие щипцы впились в ее тело, снова ее стали резать тупые ножи, и боль накатывалась волнами. Надо крепко держаться за руку Мелли.
Но доктор Мид все-таки пришел, хоть он и очень нужен солдатам в лазарете; она услышала, как он сказал:
— Бредит. Где же капитан Батлер?
Вокруг была темная ночь, а потом становилось светло, и то у нее должен был родиться ребенок, то у Мелани, которая кричала в муках, но, в общем, Мелли все время была тут, и Скарлетт чувствовала ее прохладные пальцы, и Мелани в волнении не всплескивала зря руками и не всхлипывала, как тетя Питти. Стоило Скарлетт открыть глаза и сказать: «Мелли?» — и голос Мелани отвечал ей. И как правило, ей хотелось еще шепнуть: «Ретт… Я хочу Ретта», — но она, словно во сне, вспоминала, что Ретт не хочет ее, и перед ней вставало лицо Ретта — темное, как у индейца, и его белые зубы, обнаженные в усмешке. Ей хотелось, чтобы он был с ней, но он не хочет.
Однажды она сказала: «Мелли?», — и голос Мамушки ответил:
«Тихо, детка», — и она почувствовала прикосновение холодной тряпки к своему лбу, и в испуге закричала: «Мелли! Мелани!» — и кричала снова и снова, но Мелани долго не приходила. В это время Мелани сидела на краю кровати Ретта, а Ретт, пьяный, рыдал, лежа на полу, — всхлипывал и всхлипывал, уткнувшись ей в колени.
Выходя из комнаты Скарлетт, Мелани всякий раз видела, как он сидит на своей кровати — дверь в комнату он держал открытой — и смотрит на дверь через площадку. В комнате у него было не убрано, валялись окурки сигар, стояли тарелки с нетронутой едой. Постель была смята, не заправлена, он сидел на ней небритый, осунувшийся и без конца курил. Он ни о чем не спрашивал Мелани, когда видел ее. Она сама обычно на минуту задерживалась у двери и сообщала: «Мне очень жаль, но ей хуже», или:
«Нет, она вас еще не звала. Она ведь в бреду», или: «Не надо отчаиваться, капитан Батлер. Давайте я приготовлю вам горячего кофе или чего-нибудь поесть. Вы так заболеете».
Ей всегда было бесконечно жаль его, хотя от усталости и недосыпания она едва ли способна была что-то чувствовать. Как могут люди так плохо говорить о нем — называют его бессердечным, порочным, неверным мужем, когда она видит, как он худеет, видит, как мучается?! Несмотря на усталость, она всегда стремилась, сообщая о том, что происходит в комнате больной, сказать это подробнее. А он глядел на нее, как грешник, ожидающий Страшного суда, — словно ребенок, внезапно оставшийся один во враждебном мире. Правда, Мелани ко всем относилась, как к детям.
Когда же, наконец, она подошла к его двери, чтобы сообщить радостную весть, что Скарлетт стало лучше, зрелище, представшее ее взору, было для нее полной неожиданностью. На столике у кровати стояла полупустая бутылки виски, и в комнате сильно пахло спиртным. Ретт посмотрел на нее горящими остекленелыми глазами, и челюсть у него затряслась, хоть он и старался крепко стиснуть зубы.
— Она умерла?
— Нет, что вы! Ей гораздо лучше.
Он произнес: «О господи», — и уткнулся головой в ладони. Мелани увидела, как задрожали, словно от озноба, его широкие плечи, — она с жалостью глядела на него и вдруг с ужасом поняла, что он плачет. Мелани ни разу еще не видела плачущего мужчину и, уж конечно же, не представляла себе плачущим Ретта — такого бесстрастного, такого насмешливого, такого вечно уверенного в себе.
Эти отчаянные, сдавленные рыдания испугали ее. Мелани в страхе подумала, что он совсем пьян, а она больше всего на свете боялась пьяных. Но он поднял голову, и, увидев его глаза, она тотчас вошла в комнату, тихо закрыла за собой дверь и подошла к нему. Она ни разу еще не видела плачущего мужчину, но ей пришлось успокаивать стольких плачущих детей. Она мягко положила руку ему на плечо, и он тотчас обхватил ее ноги руками. И не успела она опомниться, как уже сидела у него на кровати, а он уткнулся головой ей в колени и так сильно сжал ее ноги, что ей стало больно.
Она нежно поглаживала его черную голову, приговаривая, успокаивая:
— Да будет вам! Будет! Она скоро поправится.
От этих слов Мелани он лишь крепче сжал ее ноги и заговорил — быстро, хрипло, выплескивая все, словно поверяя свои секреты могиле, которая никогда их не выдаст, — впервые в жизни выплескивая правду, безжалостно обнажая себя перед Мелани, которая сначала ничего не понимала и держалась с ним по-матерински. А он все говорил — прерывисто, уткнувшись головой ей в колени, дергая за фалды юбки. Иной раз слова его звучали глухо, словно сквозь вату, иной раз она слышала их отчетливо, — безжалостные, горькие слова признания и унижения; он говорил такое, чего она ни разу не слышала даже от женщины, посвящал ее в тайную тайн, так что кровь приливала к щекам Мелани, и она благодарила бога за то, что Ретт не смотрит на нее.
Она погладила его по голове, точно перед ней был маленький Бо, и сказала:
— Замолчите, капитан Батлер! Вы не должны говорить мне такое! Вы не в себе! Замолчите!
Но неудержимый поток слов хлестал из него, он хватался за ее платье, точно за последнюю надежду.
Он обвинял себя в каких-то непонятных ей вещах, бормотал имя Красотки Уотлинг, а потом вдруг, с яростью встряхнув ее, воскликнул:
— Я убил Скарлетт! Я убил ее. Вы не понимаете. Она же не хотела этого ребенка и…
— Да замолчите! Вы просто не в себе! Она не хотела ребенка?! Да какая женщина не хочет…
— Нет! Нет! Вы хотите детей. А она не хочет. Не хочет иметь от меня…
— Перестаньте!
— Вы не понимаете. Она не хотела иметь ребенка, а я ее принудил. Этот.., этот ребенок.., ведь все по моей вине. Мы же не спали вместе…
— Замолчите, капитан Батлер! Нехорошо это…
— А я был пьян, я был вне себя, мне хотелось сделать ей больно.., потому что она причинила мне боль. Мне хотелось.., и я принудил ее, но она-то ведь не хотела меня. Она никогда меня не хотела. Никогда, я так старался.., так старался и…
— О, прошу вас!
— И я ведь ничего не знал о том, что она ждет ребенка, до того дня.., когда она упала. А она не знала, где я был, и не могла написать мне и сообщить.., да она бы и не написала мне, даже если б знала. Говорю вам.., говорю вам: я бы сразу приехал домой.., если бы только узнал.., не важно, хотела бы она, чтобы я приехал, или нет…
— О да, я уверена, что вы бы приехали!
— Бог ты мой, как я дурил эти недели, дурил и пил! А когда она мне сказала — там, на лестнице.., как я себя повел? Что я сказал? Я рассмеялся и сказал: «Не волнуйтесь. Может, у вас еще будет выкидыш». И тогда она…
Мелани побледнела и расширенными от ужаса глазами посмотрела на черную голову, метавшуюся, тычась в ее колени. Послеполуденное солнце струилось в раскрытое окно, и она вдруг увидела — словно впервые, — какие у него большие смуглые сильные руки, какие густые черные волосы покрывают их. Она невольно вся сжалась. Эти руки казались ей такими хищными, такими безжалостными, и, однако же, они беспомощно цеплялись сейчас за ее юбки…
Неужели до него дошла эта нелепая ложь насчет Скарлетт и Эшли, он поверил и приревновал? Действительно, он уехал из города сразу же после того, как разразился скандал, но… Нет, этого быть не может. Капитан Батлер и раньше всегда уезжал неожиданно. Не мог он поверить сплетне. Слишком он разумный человек. Если бы дело было в Эшли, он наверняка постарался бы его пристрелить! Или по крайней мере потребовал бы от него объяснения!
Нет, этого быть не может. Просто он пьян, и слишком измотан, и в голове у него немного помутилось, как бывает, когда у человека бред и он несет всякую дичь. Мужчины не обладают такой выносливостью, как женщины. Что-то расстроило его, быть может, он поссорился со Скарлетт и сейчас в своем воображении раздувает эту ссору. Возможно, что-то из того, о чем он тут говорил, и правда. Но все правдой быть не может. И уж во всяком случае, это последнее признание! Ни один мужчина не сказал бы такого женщине, которую он любит столь страстно, как этот человек любит Скарлетт. Мелани никогда еще не сталкивалась со злом, никогда не сталкивалась с жестокостью, и сейчас, когда они впервые предстали перед ней, она не могла этому поверить. Ретт пьян и болен. А больным детям не надо перечить.
— Да будет вам! Будет! — приговаривала она. — Помолчите. Я все понимаю.
Он резко вскинул голову и, посмотрев на нее налитыми кровью глазами, сбросил с себя ее руки.
— Нет, клянусь богом, вы ничего не поняли! Вы не можете понять! Вы.., слишком вы добрая, чтобы понять. Вы мне не верите, а все, что я сказал, — правда, и я — пес. Вы знаете, почему я так поступил? Я с ума сходил, я обезумел от ревности. Я всегда был ей безразличен, и вот я подумал, что сумею сделать так, что не буду ей безразличен. Но ничего не вышло. Она не любит меня. Никогда не любила. Она любит…
Горящие пьяные глаза его встретились с ее взглядом, и он умолк с раскрытым ртом, словно впервые осознав, с кем говорит. Лицо у Мелани было белое, напряженное, но глаза, в упор смотревшие на него, были ласковые, полные сочувствия, неверия. Эти мягкие карие глаза светились безмятежностью, из глубины их смотрела такая наивность, что у Ретта возникло ощущение, будто ему дали пощечину, и его затуманенное алкоголем сознание немного прояснилось, а стремительный поток безумных слов прервался. Он что-то пробормотал, отведя от Мелани взгляд, и быстро заморгал, словно пытаясь вернуться в нормальное состояние.
— Я — скотина, — пробормотал он, снова устало тыкаясь головой ей в колени. — Но не такая уж большая скотина. И хотя я все рассказал вам, вы ведь мне не поверили, да? Вы слишком хорошая, чтобы поверить. До вас я ни разу не встречал по-настоящему хорошего человека. Вы не поверите мне, правда?
— Нет, не поверю, — примирительно сказала Мелани и снова погладила его по голове. — Она поправится. Будет вам, капитан Батлер! Не надо плакать! Она поправится.

