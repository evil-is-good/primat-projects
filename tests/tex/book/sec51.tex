\chapter{\ }

Когда наконец Скарлетт почувствовала, что она снова в состоянии выходить, она велела Лу зашнуровать корсет как можно туже — только бы выдержали тесемки. Затем обмерила себе талию. Двадцать дюймов! Она громко охнула. Вот что получается, когда рожаешь детей! Да у нее теперь талия, как у тети Питти, даже шире, чем у Мамушки.
— Ну-ка, затяни потуже, Лу. Постарайся, чтобы было хоть восемнадцать с половиной дюймов, иначе я не влезу ни в одно платье.
— Тесемки лопнут, — сказала Лу. — Просто в талии вы располнели, мисс Скарлетт, и ничего уж тут не поделаешь.
«Что-нибудь да сделаем, — подумала Скарлетт, отчаянно дернув платье, которое требовалось распороть по швам и выпустить на несколько дюймов. — Просто никогда не буду больше рожать».
Да, конечно, Бонни была прелестна и только украшала свою мать, да и Ретт обожал ребенка, но больше Скарлетт не желала иметь детей. Что тут придумать, она не знала, ибо с Реттом вести себя, как с Фрэнком, она не могла. Ретт нисколько не боялся ее. Да и удержать его будет трудно, когда он так по-идиотски ведет себя с Бонни, — наверняка захочет иметь сына в будущем году, хоть и говорит, что утопил бы мальчишку, если бы она его родила. Ну так вот, не будет у него больше от нее ни мальчишки, ни девчонок. Для любой женщины хватит троих детей.
Когда Лу заново сшила распоротые швы, разгладила их и застегнула на Скарлетт платье, Скарлетт велела заложить карету и отправилась на лесной склад. По пути настроение у нее поднялось, и она забыла о своей талии: ведь на складе она увидит Эшли и сядет с ним просматривать бухгалтерию. И если ей повезет, то они какое-то время будут вдвоем. А видела она его в последний раз задолго до рождения Бонни. Ей не хотелось встречаться с ним, когда ее беременность стала бросаться в глаза. А она привыкла видеть его ежедневно — пусть даже рядом всегда кто-то был. Привыкла к кипучей деятельности, связанной с торговлей лесом, — ей так всего этого не хватало, пока она сидела взаперти. Конечно, теперь ей вовсе не нужно было работать. Она вполне могла продать лесопилки и положить деньги на имя Уэйда и Эллы. По это означало бы, что она почти не будет видеть Эшли — разве что в обществе, когда вокруг тьма народу. А работать рядом с Эшли доставляло ей огромное удовольствие.
Подъехав к складу, она с удовлетворением увидела, какие высокие стоят штабеля досок и сколько покупателей толпится возле Хью Элсинга. Увидела она и шесть фургонов, запряженных мулами, и негров-возчиков, грузивших лес. «Шесть фургонов! — подумала она с гордостью. — И всего этого я достигла сама!» Эшли вышел на порог маленькой конторы, чтобы приветствовать ее, — глаза его светились радостью; подав Скарлетт руку, он помог ей выйти из кареты и провел в контору — так, словно она была королевой.
Но когда она просмотрела его бухгалтерию и сравнила с книгами Джонни Гэллегера, радость ее померкла. Лесопилка Эшли еле покрывала расходы, тогда как Джонни заработал для нее немалую сумму. Она решила промолчать, но Эшли, видя, как она глядит на лежавшие перед ней два листа бумаги, угадал ее мысли.
— Скарлетт, мне очень жаль. Могу лишь сказать в свое оправдание, что лучше бы вы разрешили мне нанять вольных негров вместо каторжников. Мне кажется, я бы добился больших успехов. — Негров?! Да мы бы прогорели из-за одного жалованья, которое пришлось бы им платить. А каторжники — это же дешевле дешевого. Если Джонни может с их помощью получать такой доход…
Эшли смотрел поверх се плеча на что-то, чего она не могла видеть, и радостный свет в его глазах потух.
— Я не могу заставлять каторжников работать так, как Джонни Гэллегер. Я не могу вгонять людей в гроб.
— Бог ты мой! Джонни просто удивительно с ними справляется. А у вас, Эшли, слишком мягкое сердце. Надо заставлять их больше работать. Джонни говорил мне: если какому нерадивому каторжнику вздумается отлынивать от работы, он заявляет, что заболел, и вы отпускаете его на целый день. Боже правый, Эшли! Так не делают деньги. Стеганите его разок-другой, и любая хворь мигом пройдет, кроме, может, сломанной ноги…
— Скарлетт! Скарлетт! Перестаньте! Я не могу слышать от вас такое, — воскликнул Эшли и так сурово посмотрел на нее, что она умолкла. — Да неужели вы не понимаете, что это же люди… И среди них есть больные, голодные, несчастные и… О господи, просто видеть не могу, какой жестокой вы из-за него стали — это вы-то, всегда такая мягкая…
— Я стала такой — из-за кого?
— Я должен был вам это сказать, хоть и не имею права. И все же должен. Этот ваш Ретт Батлер. Он отравляет все, к чему бы ни прикоснулся. Вот он женился на вас, такой мягкой, такой щедрой, такой нежной, хоть и вспыльчивой порою, и вы стали.., жесткая, грубая — от одного общения с ним.
— О! — выдохнула Скарлетт: чувство вины боролось в ней с чувством радости оттого, что Эшли неравнодушен к ней, что она ему по-прежнему мила. Слава богу, он думает, что Ретт виноват в том, что она считает каждый пенни. А на самом-то деле Ретт не имел к этому никакого отношения и вся вина — только ее, но, в конце концов, на совести Ретта достаточно много черных пятен, не страшно, если прибавится еще одно.
— Будь вашим мужем любой другой человек, я, может быть, так бы не огорчался, но… Ретт Батлер! Я вижу, что он с вами сделал. Вы сами не заметили, как вслед за ним стали мыслить сухо и жестко. О да, я знаю, что не должен этого говорить… Он спас мне жизнь, и я благодарен ему, но как бы я хотел, чтобы это был кто угодно, только не он! И я не имею права говорить с вами так, точно…
— Ах, нет, Эшли, имеете.., как никто другой!
— Говорю вам, это просто невыносимо: видеть, как вы, такая тонкая, огрубели под его влиянием, знать, что ваша красота, ваше обаяние принадлежат человеку, который… Когда я думаю, что он дотрагивается до вас, я…
«Сейчас он поцелует меня! — в экстазе подумала Скарлетт. — И не я буду в этом виновата!» Она качнулась к нему. Но он отступил на шаг, словно вдруг осознав, что сказал слишком много, — сказал то, чего не собирался говорить.
— Простите великодушно, Скарлетт. Я.., я намекал, что ваш муж — не джентльмен, а сам сейчас наговорил такое, что меня тоже джентльменом не назовешь. Никто не имеет права принижать мужа перед женой. Мне нет оправдания, кроме.., кроме… — Он запнулся, лицо его исказилось. Она ждала, затаив дыхание. — У меня вообще нет оправданий.
Бешеная работа шла в мозгу Скарлетт, пока она ехала в карете домой. Вообще нет оправданий, кроме — кроме того, что он любит ее! И мысль, что она спит в объятиях Ретта, вызывает в нем такую ярость, на какую она не считала его способным. Что ж, она может это понять. Если бы не сознание, что его отношения с Мелани волею судеб сводятся к отношениям брата и сестры, она сама жила бы как в аду. Значит, объятия Ретта огрубили ее, сделали более жесткой! Что ж, раз Эшли так считает, она вполне может обойтись без этих объятий. И она подумала, как это будет сладостно и романтично, если оба они с Эшли решат блюсти физическую верность друг другу, хотя каждый и связан супружескими узами с другим человеком. Эта идея завладела ее воображением и пришлась ей по душе. Да к тому же была тут и практическая сторона. Ведь это значит, что ей не придется больше рожать.
Когда Скарлетт приехала домой и отпустила карету, восторженное состояние, овладевшее ею после слов Эшли, начало испаряться, стоило ей представить себе, как она скажет Ретту, что хочет иметь отдельную спальню — со всеми вытекающими отсюда последствиями. Это будет нелегко. Да и как она скажет Эшли, что не допускает больше до себя Ретта, потому что он, Эшли, так пожелал? Кому нужна такая жертва, если никто не знает о ней? До чего же это тяжкое бремя — скромность и деликатность! Если бы только она могла говорить с Эшли так же откровенно, как с Реттом! А, да ладно. Найдет она способ намекнуть Эшли на истинное положение вещей.
Она поднялась но лестнице и, открыв дверь в детскую, обнаружила Ретта — он сидел у колыбельки Понни, держа Эллу на коленях, а Уэйд стоял рядом и показывал ему свои сокровища, спрятанные в карманах. Какое счастье, что Ретт любит детей и так ими занимается! Есть ведь отчимы, которые терпеть не могут детей от предыдущих браков. — Я хочу поговорить с тобой, — сказала Скарлетт и прошла в их спальню. Лучше покончить с этим сразу, пока она полна решимости не иметь больше детей и любовь Эшли придает ей силы для разговора.
— Ретт, — без всяких предисловий начала она, как только он закрыл за собой дверь спальни, — я не хочу иметь больше детей, это решено.
Если его и поразило ее неожиданное заявление, то он и виду не подал. Он не спеша подошел к креслу и, опустившись в него, откинулся на спинка.
— Я ведь уже говорил тебе, моя кошечка, еще прежде чем родилась Ионии, что мне безразлично, будет у нас один ребенок или двадцать.
Ловко он уходит от главного, как будто появление детей не имеет никакого отношения к тому, что тому предшествует.
По-моему, троих вполне достаточно. Я вовсе не намерена каждый год носить по ребенку.
Три мне кажется вполне подходящим числом.
— Ты прекрасно понимаешь, — начала было она и покраснела от смущения. — Ты понимаешь, что я хочу сказать?
— Понимаю. А ты сознаешь, что я могу подать на развод, если ты откажешься выполнять свои супружеские обязанности?
— У тебя хватит низости придумать такое! — воскликнула она, раздосадованная тем, что все идет не так, как она наметила. — Да если бы в тебе было хоть немного благородства, ты бы.., ты бы вел себя, как… Ну, словом, посмотри на Эшли Уилкса. Мелани ведь не может иметь детей, и он…
— Образцовый джентльмен чик, этот Эшли, — докончил за нее Ретт, и в глазах его появился странный блеск. — Прошу, продолжай свою речь.
Скарлетт поперхнулась, ибо речь ее была окончена и ей нечего было больше сказать. Теперь она поняла, сколь глупо было надеяться, что она сумеет по-доброму договориться о столь сложном деле, особенно с таким эгоистичным мерзавцем, как Ретт.
— Ты сегодня ездила на свой лесной склад, да?
— А какое это имеет отношение к нашему разговору?
— Ты ведь любишь собак, Скарлетт? Что ты предпочитаешь — чтобы собака лежала на сене или находилась при тебе?
Намек не дошел до ее сознания, захлестнутого яростью и разочарованием.
Ретт легко вскочил на ноги, подошел к ней и, взяв за подбородок, рывком поднял ее голову.
— Какое же ты дитя! У тебя было трое мужей, но ты до сих пор не знаешь мужской природы. Ты, видимо, считаешь мужчину кем-то вроде старухи, перешагнувшей возрастной рубеж.
Он игриво ущипнул ее за подбородок и убрал руку. Подняв черную бровь, он долго холодно смотрел на жену.
— Пойми вот что, Скарлетт. Если бы ты и твоя постель все еще представляла для меня интерес, никакие замки и никакие уговоры не удержали бы меня. И мне не было бы стыдно, так как мы с тобой заключили сделку — сделку, которой я верен и которую ты сейчас нарушаешь. Что ж, спи одна в своей девственной постельке, прелесть моя.
— Ты что же, хочешь сказать… — возмутилась Скарлетт, — что тебе безразлично.
— Я ведь надоел тебе, да? Ну, так мужчинам женщины надоедают куда быстрее. Будь святошей, Скарлетт. Огорчаться по этому поводу я не стану. Мне все равно. — Он с усмешкой пожал плечами. — По счастью, в мире полно постелей, а в постелях хватает женщин.
— Ты хочешь сказать, что готов дойти до…
— Дорогая моя невинность! Конечно. Удивительно уже то, что я воздерживался от этого так долго. Я никогда не считал супружескую верность добродетелью.
— Я на ночь буду запирать свою дверь!
— К чему утруждать себя? Если я захочу тебя, никакой замок меня не удержит.
Он повернулся с таким видом, точно считал разговор оконченным, и вышел ил комнаты. Скарлетт слышала, как он вошел в детскую, где дети бурно приветствовали его. Она села. Вот она и добилась своего. Этого хотела она, и этого хотел Эшли. Но она не чувствовала себя счастливой. Гордость ее была задета: ее оскорбляла самая мысль, что Ретт отнесся так легко к ее словам, что он вовсе не жаждет обладать ею, что он ее равняет с другими женщинами в других постелях.
Ей очень хотелось придумать какой-то способ деликатно намекнуть Эшли, что она и Ретт физически больше не муж и жена. Но Скарлетт понимала, что это невозможно. Она заварила какую-то чудовищную кашу и уже отчасти жалела о своих словах. Ей будет недоставать долгих забавных разговоров с Реттом в постели, когда кончик его сигары светился в темноте. Ей будет недоставать его объятий, когда она пробуждалась в ужасе от кошмарных снов, а ей ведь не раз снилось, что она бежит сквозь холодный густой туман. Внезапно почувствовав себя глубоко несчастной, она уткнулась головой в подлокотник кресла и расплакалась.

