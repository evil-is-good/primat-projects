\chapter{\ }

Скарлетт была в Мариетте, когда пришла срочная телеграмма от Ретта. Поезд в Атланту отходил через десять минут, и она села в него, взяв с собой лишь ридикюль и оставив Уэйда и Эллу в отеле на попечении Присси.
Атланта находилась всего в двадцати милях, но поезд полз бесконечно долго по мокрой, осенней, освещенной тусклым дневным светом равнине и останавливался у каждого переезда. До крайности взволнованная вестью от Ретта, Скарлетт сгорала от нетерпения и при каждой остановке буквально готова была кричать. А поезд шел не спеша сквозь леса, чуть тронутые усталым золотом, мимо красных холмов, все еще изрезанных серпантинами окопов, мимо бывших артиллерийских редутов и заросших сорняком воронок, вдоль дороги, по которой с боями отступали солдаты Джонстона. Каждая остановка, каждый перекресток, объявляемые кондуктором, носили имя сражения, были местом битвы. В свое время это вызывало бы у Скарлетт страшные воспоминания, но сейчас ей было не до них.
Телеграмма Ретта гласила: «Миссис Уилкс заболела. Немедленно возвращайтесь».
Сумерки уже спустились, когда поезд подошел к Атланте, — пелена мелкого дождя затянула город. На улицах тускло горели газовые фонари — желтые круги в тумане. Ретт ждал ее на вокзале с каретой. Лицо его испугало Скарлетт еще больше, чем телеграмма. Она никогда прежде не видела у него такого бесстрастного лица.
— Она не… — вырвалось у Скарлетт.
— Нет. Она еще жива. — Ретт помог Скарлетт сесть в карету. — К дому миссис Уилкс, и гони быстрее, — приказал он кучеру.
— А что с ней? Я и не знала, что она больна. Ведь еще на прошлой неделе она выглядела как всегда. С ней произошел несчастный случай? Ах, Ретт, и, конечно же, это не так серьезно, как вы…
— Она умирает, — сказал Ретт, и голос его был так же бесстрастен, как и лицо. — Она хочет вас видеть.
— Нет, только не Мелли! Ах, нет. Что же с ней такое?
— У нее был выкидыш.
— Вы… Вы… Но, Ретт, она же… — И Скарлетт умолкла. Эта весть совсем лишила ее дара речи.
— А вы что, разве не знали, что она ждет ребенка? У Скарлетт не было сил даже покачать головой.
— Ах, ну конечно, наверное, не знали. Я думаю, она никому не говорила. Ей хотелось устроить всем сюрприз. Но я знал.
— Вы знали? Но не она же вам сказала?
— Ей не надо было мне говорить. Я знал. Она была такая.., такая счастливая последние два месяца, что я понял: ничего другого быть не могло.
— Но, Ретт, доктор ведь говорил, что она умрет, если вздумает еще раз рожать!
— Вот она и умирает, — сказал Ретт. И обращаясь к кучеру: — Ради всего святого, ты что, не можешь ехать быстрее?
— Но, Ретт, она, конечно же, не умирает! Я ведь не.., я не.., не…
— Она не такая сильная, как вы. Она вообще никогда не отличалась особой силой. Одно Только и было у нее — это сердце.
Карета, покачиваясь, остановилась перед приземистым домом, и Ретт помог Скарлетт выйти. Дрожащая, испуганная, вдруг бесконечно одинокая, она крепко ухватилась за его локоть.
— А вы разве не зайдете со мной, Ретт?
— Нет, — сказал он и снова сел в карету.
Скарлетт взлетела вверх по ступенькам, пересекла крыльцо и распахнула дверь. Внутри при желтом свете лампы она увидела Эшли, тетю Питти и Индию. Скарлетт подумала: «А что здесь делает Индия? Мелани ведь сказала, чтобы она не смела переступать порог этого дома». Все трое поднялись при виде нее — тетя Питти кусала дрожащие губы; сраженная горем Индия без всякой ненависти смотрела на нее. У Эшли был вид сомнамбулы, и когда он подошел к Скарлетт и положил руку ей на плечо, он и заговорил, как сомнамбула.
— Она звала вас, — сказал он. — Она вас звала.
— Могу я ее видеть? — Скарлетт повернулась к закрытой двери комнаты Мелани.
— Нет. Там сейчас доктор Мид. Я рад, что вы приехали, Скарлетт.
— Я тут же выехала. — Скарлетт сбросила шляпку и накидку, — Поезд… Но она же не… Скажите мне, что ей лучше, правда, Эшли? Да говорите же! И не смотрите так! Она же не…
— Она все звала вас, — сказал Эшли и посмотрел Скарлетт в глаза.
И она увидела в его глазах ответ на все свои вопросы. На секунду сердце у нее остановилось, а потом какой-то странный страх, более сильный, чем тревога, более сильный, чем горе, забился в ее груди. «Этого не может быть, — думала она, стараясь отогнать тревогу. — Врачи ошибаются. Я не могу поверить, что это правда. Не могу позволить себе так думать. Если я так подумаю, то закричу. Буду думать о чем-то другом».
— Я этому не верю! — воскликнула она, с вызовом глядя на этих троих с осунувшимися лицами и словно призывая их опровергнуть ее слова. — Ну, почему Мелани ничего не сказала мне? Я бы никогда не поехала в Мариетту, если б знала!
Глаза у Эшли ожили, в них появилась мука.
— Она никому ничего не говорила, Скарлетт, тем более — вам. Она боялась, что вы станете ругать ее, если узнаете. Ей хотелось дождаться трех.., словом, пока она не будет уверена, что все в порядке, а тогда удивить вас всех, и посмеяться, и сказать, как не правы были врачи. И она была так счастлива. Вы же знаете.., как она относится к детям.., как ей хотелось иметь девочку. И все шло хорошо, пока.., а потом без всяких причин.
Дверь из комнаты Мелани тихо отворилась, и в холле появился доктор Мид. Он прикрыл за собой створки и постоял с минуту, уткнувшись седой бородой в грудь, глядя на внезапно застывших четверых людей. Последней он увидел Скарлетт. Когда он подошел к ней, она прочла в его глазах скорбь, а также неприязнь и презрение, и сердце у нее испуганно заныло от чувства вины.
— Значит, прибыли, наконец, — сказал он. Она еще не успела ничего сказать, а Эшли уже шагнул к закрытой двери.
— Нет, вам пока нельзя, — сказал доктор. — Она хочет говорить со Скарлетт.
— Доктор, — сказала Индия, положив руку ему на рукав. Голос у нее звучал безжизненно, но так умоляюще, что брал за душу сильнее слов. — Разрешите мне увидеть ее хоть на минуту. Я ведь здесь с самого утра и все жду, но она… Разрешите мне увидеть ее хоть на минуту. Я хочу сказать ей.., я должна ей сказать, что была не права.., не права кое в чем.
Произнося эти слова, она не смотрела ни на Эшли, ни на Скарлетт, но доктор Мид остановил строгий ледяной взгляд на Скарлетт.
— Я постараюсь, мисс Индия, — отрывисто сказал он. — Но только при условии, если вы пообещаете мне не отнимать у нее силы, заставляя слушать ваше признание в том, что вы были не правы. Она знает, что вы были не правы, и вы только взволнуете ее, если начнете извиняться.
— Прошу вас, доктор Мид… — робко начала было Питти.
— Мисс Питти, вы же знаете, что только закричите и тут же упадете в обморок.
Питти возмущенно выпрямилась во весь свой невысокий рост и посмотрела на доктора в упор. Глаза ее были сухи и вся фигурка исполнена достоинства.
— Ну, хорошо, душа моя, немного позже, — сказал доктор, чуть подобрев. — Пойдемте, Скарлетт.
Они подошли на цыпочках к закрытой двери, и доктор вдруг резко схватил Скарлетт за плечо.
— Вот что, мисс, — быстро зашептал он, — никаких истерик и никаких признаний у ложа умирающей, иначе, клянусь богом, я сверну вам шею! И не смотрите на меня с этим вашим невинным видом. Вы знаете, что я имею в виду. Мисс Мелли умрет легко, а вы не облегчите своей совести, если скажете ей про Эшли. Я никогда еще не обидел ни одной женщины, но если вы сейчас хоть в чем-то признаетесь, вы мне за это ответите.
И прежде чем она успела что-либо сказать, он распахнул дверь, втолкнул ее в комнату и закрыл за ней дверь. Маленькая комнатка, обставленная дешевой мебелью из черного ореха, была погружена в полумрак, лампа прикрыта газетой. Все здесь было маленькое и аккуратное, как в комнате школьницы: узкая кровать с низкой спинкой, простые тюлевые занавески, подхваченные с боков, чистые, выцветшие лоскутные коврики, — все было непохоже на роскошную спальню Скарлетт с тяжелой резной мебелью, драпировками из розовой парчи и ковром, вытканным розами.
Мелани лежала в постели, тело ее под одеялом казалось совсем худеньким и плоским, как у девочки. Две черные косы обрамляли лицо, закрытые глаза были обведены багровыми кругами. При виде ее Скарлетт замерла и остановилась, прислонясь к двери. Несмотря на полумрак в комнате, она увидела, что лицо Мелани — желто-воскового цвета. Жизнь ушла из него, и нос заострился. До этой минуты Скарлетт надеялась, что доктор Мид ошибается, но сейчас она все поняла. В госпиталях во время войны она видела слишком много таких лиц с заострившимися чертами и знала, о чем это говорит.
Мелани умирает, но мозг Скарлетт какое-то время отказывался это воспринять. Мелани не может умереть. Это невозможно. Бог не допустит, чтобы она умерла, — ведь она так нужна ей, Скарлетт. Никогда прежде Скарлетт и в голову не приходило, что Мелани так ей нужна. Но сейчас эта мысль пронзила ее до глубины души. Она полагалась на Мелани, как полагалась на саму себя, но до сих пор этого не сознавала. А теперь Мелани умирает, и Скарлетт почувствовала, что не сможет жить без нее. Идя на цыпочках через комнату к тихо лежавшей фигуре, чувствуя, как от панического страха сжимается сердце, Скарлетт поняла, что Мелани была ее мечом и щитом, ее силой и ее утешением.
«Я должна удержать ее! Я не могу допустить, чтобы она вот так ушла!» — думала она и, шурша юбками, села у постели. Она поспешно схватила лежавшую на покрывале влажную руку и снова испугалась — такой холодной была эта рука.
— Это я, Мелли, — прошептала она.
Мелани чуть приоткрыла глаза и, словно удостоверившись, что это в самом деле Скарлетт, опять закрыла их. Потом она перевела дух и шепотом произнесла:
— Обещай!
— О, все что угодно!
— Бо.., присмотри за ним.
Скарлетт могла лишь кивнуть, так у нее сдавило горло, и она тихонько пожала руку, которую держала в своей руке, давая понять, что все выполнит.
— Я поручаю его тебе. — Легкое подобие улыбки мелькнуло на лице Мелани. — Я ведь уже поручала его твоим заботам.., помнишь.., до того, как он родился.
Помнит ли она? Да разве может она забыть то время? Так же отчетливо, как если бы тот страшный день снова наступил, она почувствовала удушающую жару сентябрьского полдня, вспомнила свой ужас перед янки, услышала топот ног отступающих солдат, и в ушах ее снова зазвенел голос Мелани, которая просила, чтобы Скарлетт взяла к себе ребенка, если она умрет.., вспомнила и то, как она ненавидела Мелани в тот день, как надеялась, что та умрет.
«Я убила ее, — подумала она в суеверном ужасе. — Я так часто желала ей смерти, и вот господь услышал меня и теперь наказывает».
— Ах, Мелли, не надо так говорить. Ты же знаешь, что выздоровеешь…
— Нет. Обещай.
Скарлетт глотнула воздуха.
— Ты же знаешь, что я буду относиться к нему как к родному сыну.
— Колледж? — спросил слабый, еле слышный голос Мелани.
— О да! И Гарвард, и Европа, и все, чего он захочет.., и.., и.., пони.., и уроки музыки… Ах, Мелли, ну пожалуйста, попытайся! Сделай над собой усилие!
Снова наступила тишина, и по лицу Мелани видно было, что она старается собраться с силами, чтобы еще что-то сказать.
— Эшли, — сказала она. — Ты и Эшли… — Голос ее задрожал и умолк.
При упоминании имени Эшли сердце у Скарлетт остановилось и захолодело, стало Как камень. Значит, все это время Мелани знала. Скарлетт уткнулась головой в одеяло, и в горле ее застряло невырвавшееся рыдание. Мелани знает. Сейчас Скарлетт уже не было стыдно, она ничего не чувствовала, кроме диких угрызений совести за то, что столько лет причиняла боль этой мягкой, ласковой женщине. Мелани все знала — и, однако же, оставалась ее верным другом. Ах, если бы можно было прожить заново эти годы! Она никогда бы не позволила себе даже встретиться взглядом с Эшли.
«Господи, господи, — молилась она про себя, — пожалуйста, сохрани ей жизнь! Я воздам ей сторицей. Я к ней буду хорошо относиться. Пока буду жива, я никогда больше словом не перемолвлюсь с Эшли, если только ты дашь ей поправиться!» — Эшли, — слабым голосом произнесла Мелани и дотронулась пальцами до склоненной головы Скарлетт. Ее большой и указательный пальцы потянули волосы Скарлетт не сильнее, чем это сделал бы грудной младенец. Скарлетт сразу поняла, что это значило, — поняла, что Мелани хочет, чтобы она взглянула на нее. Но она не могла, не могла встретиться взглядом с Мелани и прочесть в ее глазах, что та все знает.
— Эшли, — снова прошептала Мелани, и Скарлетт постаралась взять себя в руки. Ведь когда она в день Страшного суда посмотрит в лицо господа бога и прочтет в его глазах свой приговор, ей будет не страшнее, чем сейчас. И хотя все внутри у нее сжалось, она подняла голову.
Она увидела все те же темные любящие глаза, сейчас запавшие и уже затуманенные смертью, все тот же нежный рот, устало боровшийся за каждый вздох. В лице Мелани не было ни упрека, ни обвинения, ни страха — лишь тревога, что она не сумеет найти в себе силы, чтобы произнести нужные слова.
Скарлетт была настолько потрясена увиденным, что даже не почувствовала облегчения. Потом, взяв в ладони руку Мелани, она ощутила, как теплая волна благодарности затопляет ее, и впервые со времени детства смиренно, не думая о себе, произнесла молитву:
«Благодарю тебя господи. Я знаю, что недостойна, но все же благодарю тебя за то, что ты позволил ей не знать».
— Так что же насчет Эшли, Мелли?
— Ты.., присмотришь за ним?
— О да.
— Он простужается.., так легко.
Пауза.
— Присмотри.., за его делами. Понимаешь?
— О да, понимаю. Я присмотрю.
Сделав над собой огромное усилие, Мелани добавила:
— Эшли — он такой.., непрактичный. Только смерть могла подвигнуть Мелани на такую нелояльность.
— Присмотри за ним, Скарлетт.., но только чтобы он не знал.
— Я присмотрю за ним и за его делами, и так, что он никогда об этом не узнает. Просто буду иногда давать ему советы.
На лице Мелани появилась слабая, но удовлетворенная улыбка, когда она встретилась глазами со Скарлетт. Этот обмен взглядами как бы скреплял сделку между ними, и забота о том, чтобы оградить Эшли Уилкса от слишком жестокого мира, причем оградить так, чтобы его мужская гордость при этом не пострадала, перешла от одной женщины к другой.
Теперь усталое лицо уже не было таким напряженным, точно обещание, которое дала Скарлетт, принесло покой Мелани.
— Ты такая ловкая.., такая мужественная, ты всегда была так добра ко мне…
При этих словах рыдания подступили к горлу Скарлетт, и она зажала рукой рот. Она сейчас заревет как ребенок, закричит: «Я была сущим дьяволом! Я столько причинила тебе зла! Ничего я для тебя никогда не делала! Я все делала только для Эшли».
Она резко поднялась, прикусив палец, чтобы сдержаться. И на память ей снова пришли слова Ретта: «Она любит вас. Так что придется вам нести и этот крест». И этот крест стал сейчас еще тяжелее. Худо было уже то, что она с помощью всевозможных хитростей пыталась отобрать Эшли у Мелани. А теперь все становилось еще хуже — оттого что Мелани, слепо доверявшая ей всю жизнь, уносила с собой в могилу ту же любовь и то же доверие. Нет, сейчас она не в состоянии ничего больше сказать. Она не может даже повторить: «Попытайся сделать над собой усилие». Она должна дать Мелани спокойно уйти, без борьбы, без слез, без горя.
Дверь слегка приотворилась, и на пороге появился доктор Мид, повелительным жестом требуя, чтобы Скарлетт ушла. Она склонилась над постелью, давясь слезами, и, взяв руку Мелани, приложила ее к своей щеке.
— Спи спокойно, — сказала она голосом более твердым, чем даже сама ожидала.
— Обещай мне… — послышался шепот, теперь уже совсем, совсем тихий.
— Все что угодно, дорогая.
— Капитан Батлер — будь добра к нему. Он.., он так тебя любит.
«Ретт?» — недоумевая, подумала Скарлетт; слова Мелани показались ей пустым звуком.
— Да, конечно, — машинально сказала она и, легонько поцеловав руку Мелани, опустила ее на кровать.
— Скажите дамам, чтобы они шли сейчас же, — шепнул доктор, когда она проходила мимо него.
Затуманенными от слез глазами Скарлетт увидела, как Индия и Питти проследовали за доктором в комнату, придерживая юбки, чтобы не шуршать. Дверь за ними закрылась, и в доме наступила тишина. Эшли нигде не было видно. Скарлетт приткнулась головой к стене, словно капризный ребенок, которого поставили в угол, и потерла сдавленное рыданиями горло. За дверью уходила из жизни Мелани. И вместе с ней уходила сила, на которую, сама того не понимая, столько лет опиралась Скарлетт. Почему, ну, почему она до сих пор не сознавала, как она любит Мелани и как та ей нужна? Ну, кому могло бы прийти в голову, что в этой маленькой некрасивой Мелани заключена такая сила? В Мелани, которая до слез стеснялась чужих людей, боялась вслух выразить свое мнение, страшилась неодобрения пожилых дам, в Мелани, у которой не хватило бы храбрости прогнать гуся! И однако же…
Мысленно Скарлетт вернулась на многие годы назад, к тому жаркому дню в Таре, когда серый дым стлался над распростертым телом в синем мундире, а на верху лестницы с саблей Чарльза в руке стояла Мелани. Скарлетт вспомнила, как подумала тогда: «Какая глупость! Ведь Мелани этой саблей даже взмахнуть не могла бы!» Но сейчас она знала, что, случись такая необходимость, Мелани ринулась бы вниз по лестнице и убила бы того янки — или была бы убита сама.
Да, Мелани в тот день стояла, сжимая клинок в маленькой руке, готовая сразиться за нее, Скарлетт. И сейчас, с грустью оглядываясь назад, Скарлетт поняла, что Мелани всегда стояла рядом с клинком в руке, — стояла неназойливо, словно тень, любя ее, сражаясь за нее со слепой страстной преданностью, сражаясь с янки, с пожаром, с голодом, с нищетой, с общественным мнением и даже со своими любимыми родственниками.
Скарлетт почувствовала, как мужество и уверенность в своих силах покидают ее, ибо она поняла, что этот клинок, сверкавший между нею и миром, сейчас навеки вкладывается в ножны.
«Мелани — единственная подруга, которая по-настоящему меня любила. Да она для меня и была как мама, и все, кто знал ее, всегда цеплялись за ее юбки».
Внезапно у нее возникло такое чувство, будто это Эллин лежит за закрытой дверью и вторично покидает мир. И Скарлетт вдруг снова очутилась в Таре, а вокруг нее шумел враждебный мир, и она была в отчаянии от сознания, что не сможет смотреть жизни в лицо, не чувствуя за своим плечом необычайную силу этой слабой, мягкой, нежной женщины.




Скарлетт стояла в холле, испуганная, не зная, на что решиться; яркий огонь в камине гостиной отбрасывал на стены высокие призрачные тени. В доме царила полнейшая тишина. И эта тишина проникала в нее, словно мелкий, все пропитывающий дождь. Эшли! Где же Эшли?
Она зашла в гостиную в поисках его — так продрогшая собака ищет огня, — но Эшли там не было. Надо его найти. Она открыла силу Мелани, и свою зависимость от этой силы в ту минуту, когда теряла Мелани навсегда, но ведь остался же Эшли. Остался Эшли — сильный, мудрый, способный оказать поддержку. В Эшли и его любви она будет черпать силу — чтобы побороть свою слабость, черпать мужество — чтобы прогнать свои страхи, черпать успокоение, — чтобы облегчить горе.
«Должно быть, он у себя в комнате», — подумала она и, пройдя на цыпочках через холл, тихонько постучала к нему в дверь. Ответа не последовало, и она открыла дверь. Эшли стоял у комода, глядя на заштопанные перчатки Мелани. Сначала он взял одну перчатку и посмотрел на нее, точно никогда прежде не видел, потом осторожно положил, словно, она была стеклянная, и взял другую.
Скарлетт дрожащим голосом произнесла: «Эшли!» Он медленно повернулся и посмотрел на нее. Его серые глаза уже не были затуманенными, отчужденными, а — широко раскрытые — смотрели на нее. И она увидела в них страх, схожий с ее страхом, беспомощность еще большую, чем та, которую испытывала она, растерянность более глубокую, чем она когда-либо знала. Чувство страха, обуявшее ее в холле, стало еще острее, когда она увидела лицо Эшли. Она подошла к нему.
— Я боюсь, — сказала она. — Ах, Эшли, обними меня. Я так боюсь!
Он не шевельнулся, а лишь смотрел на нее, обеими руками сжимая перчатку. Скарлетт дотронулась до его плеча и прошептала:
— Что с тобой?
Его глаза пристально смотрели на нее, ища, отчаянно ища чего-то и не находя. Наконец он заговорил, и голос его был какой-то чужой.
— Мне недоставало тебя, — сказал он. — Я хотел побежать, чтобы найти тебя — как ребенок, который ищет утешения, — а нашел я сейчас такого же ребенка, только еще более испуганного, который прибежал ко мне.
— Но ты же.., ты же не боишься! — воскликнула она. — Ты никогда ничего не боялся… А вот я… Ты всегда был такой сильный…
— Если я был когда-либо сильный, то лишь потому, что она стояла за моей спиной, — сказал он, голос его сломался, он посмотрел на перчатку и разгладил на ней пальцы. — И.., и.., вся сила, какая у меня была, уходит вместе с ней.
В его тихом голосе было такое безысходное отчаяние, что Скарлетт убрала руку с его плеча и отступила. В тяжелом молчании, воцарившемся между ними, она почувствовала, что сейчас действительно впервые в жизни поняла его.
— Как же так… — медленно произнесла она, — как же так, Эшли? Ты, значит, любил ее?
— Она была моей единственной сбывшейся мечтой, — с трудом произнес он, — она жила и дышала и не развеивалась от соприкосновения с реальностью.
«Мечта! — подумала Скарлетт, чувствуя, как в ней зашевелилось былое раздражение. — Вечно у него эти мечты! И никакого здравого смысла!» И с тяжелым сердцем она не без горечи сказала:
— Какой же ты был глупый, Эшли. Неужели ты не видел, что она стоила миллиона таких, как я?
— Прошу тебя, Скарлетт! Если бы ты только знала, сколько я выстрадал с тех пор, как…
— Сколько ты выстрадал! Ты думаешь, что я… Ах, Эшли, тебе следовало бы знать уже много лет назад, что ты любил ее, а не меня! Почему же ты этого не понял? Все могло бы быть иначе, а теперь.., ах, тебе бы следовало понять и не держать меня на привязи, рассуждая о чести и жертвах! Если бы ты сказал мне это много лет назад, я бы… Это нанесло бы мне смертельный удар, но я бы как-нибудь выстояла. А ты выяснил это только сегодня, когда Мелли умирает, и сейчас уже слишком поздно что-либо изменить. Ах, Эшли, мужчинам положено знать такое — не женщинам! Тебе бы следовало понять, что все это время ты любил ее, а меня лишь хотел, как.., как Ретт хочет эту Красотку Уотлинг!
Он съежился от этих слов, но продолжал смотреть на нее, взглядом моля замолчать, утешить. Каждая черточка в его лице подтверждала правоту ее слов. И даже то, как он стоял, опустив плечи, говорило, что он казнит себя куда сильнее, чем могла бы казнить она. Он стоял перед ней молча, сжимая перчатку, словно это была рука все понимающего друга, и в наступившей тишине Скарлетт почувствовала, как ее возмущение тает, сменяясь жалостью, смешанной с презрением… Совесть заговорила в ней. Она же пинает ногами поверженного, беззащитного человека, а она обещала Мелани заботиться о Нем. «Не успела я дать ей обещание, как наговорила ему кучу обидных, колючих слов, хотя не было никакой необходимости говорить их ему — ни мне, ни кому-либо другому. Он знает правду, и она убивает его, — в отчаянии думала Скарлетт. — Он ведь так и не стал взрослым. Как и я, он — ребенок и в ужасе оттого, что теряет Мелани. И Мелани знала, что так будет, — Мелани знала его куда лучше, чем я. Вот почему она просила — в одних и тех же выражениях, — чтобы я присмотрела за ним и за Бо. Сумеет ли Эшли все это вынести? Я сумею. Я что угодно вынесу. Мне пришлось уже столько вынести. А он не сможет — ничего не сможет вынести без нее».
— Извини меня, дорогой, — мягко сказала она, раскрывая ему объятия. — Я знаю, каково тебе. И помни: она ничего не знает.., она никогда даже не подозревала.., бог был милостив к нам.
Он стремительно шагнул к ней и, зажмурясь, обнял. Она приподнялась на цыпочки, прижалась к нему своей теплой щекой и, стремясь успокоить его, погладила по затылку.
— Не плачь, хороший. Ей хочется, чтобы ты был мужествен. Она захочет тебя увидеть, и ты должен держаться мужественно. Она не должна заметить, что ты плакал. Это расстроит ее.
Он сжал ее так сильно, что ей стало больно дышать, и прерывающимся голосом зашептал на ухо:
— Что я буду делать? Я не смогу.., не смогу жить без нее!
«Я тоже», — подумала она и внутренне содрогнулась, представив себе долгие годы жизни без Мелани. Но она тут же крепко взяла себя в руки. Эшли полагается на нее, Мелани полагается на нее. И она подумала — как когда-то думала в Таре, лежа пьяная, измученная, в лунном свете: «Ноша создана для плеч, достаточно сильных, чтобы ее нести». Что ж, у нее сильные плечи, а у Эшли — нет. Она распрямила свои плечи, готовясь принять на них эту ношу, и спокойно, — хотя на душе у нее было далеко не спокойно, — поцеловала его мокрую щеку, без страсти, без томления, без лихорадки, легко и нежно.
— Ничего.., как-нибудь справимся, — сказала она. Дверь, ведущая в холл, со стуком распахнулась, и доктор Мид резко, повелительно крикнул:
— Эшли, скорей!
«Боже мой! Ее не стало! — подумала Скарлетт. — И Эшли даже не успел с ней попрощаться! Но может быть, еще…» — Скорей! — воскликнула она, подталкивая его к двери, ибо он стоял словно громом пораженный. — Скорей!
Она открыла дверь и вытолкнула его в холл. Подгоняемый ее словами, он побежал, продолжая сжимать в руке перчатку. Скарлетт услышала его быстрые шаги, когда он пересекал холл, и звук захлопываемой двери.
Она сказала: «Боже мой!» — и, добредя до кровати, села, уронив голову на руки. Она вдруг почувствовала такую усталость, какой еще не испытывала в жизни. Когда раздался звук захлопнувшейся двери, у Скарлетт словно что-то надорвалось, словно лопнула державшая ее пружина. Она почувствовала, что измучена, опустошена. Горе, угрызения совести, страх, удивление — все исчезло. Она устала, и мозг ее работал тупо, механически — как часы на камине.
Из этого унылого тумана, обволакивавшего ее, выплыла одна мысль. Эшли не любит ее и никогда по-настоящему не любил, но и это не причинило ей боли. А ведь должно было бы. Она должна была бы впасть в отчаяние, горевать, проклинать судьбу. Она ведь так долго жила его любовью. Эта любовь поддерживала ее во многие мрачные минуты жизни, и тем не менее — такова была истина. Он не любит ее, а ей все равно. Ей все равно, потому что и она не любит его. Она не любит его, и потому, что бы он ни сделал, что бы ни сказал, это не способно больше причинить ей боль.
Она легла на кровать и устало опустила голову на подушку. Ни к чему пытаться прогнать эту мысль, ни к чему говорить себе:
«Но я-то люблю его. Я любила его многие годы. А любовь не может в одну минуту превратиться в безразличие».
Но оказывается, может превратиться и превратилась.
«А ведь на самом деле он таким никогда и не был — только в моем воображении, — отрешенно думала она. — Я любила образ, который сама себе создала, и этот образ умер, как умерла Мелли. Я смастерила красивый костюм и влюбилась в него. А когда появился Эшли, такой красивый, такой ни на кого не похожий, я надела на него этот костюм и заставила носить, не заботясь о том, годится он ему или нет. Я не желала видеть, что он такое на самом деле. Я продолжала любить красивый костюм, а вовсе не его самого».
Теперь, оглядываясь на много лет назад, она увидела себя в платье из канифаса зелеными цветочками, — она стояла на солнце в Таре и смотрела как зачарованная на молодого всадника, чьи светлые волосы блестели на солнце как серебряный шлем. Сейчас она отчетливо понимала, что все это была лишь детская причуда, столь же ничего не значащая, как и ее капризное желание иметь аквамариновые сережки, которые она и выклянчила у Джералда. Как только она получила эти сережки, они утратили для нее всякую ценность, как утрачивало ценность все, что она получала, — кроме денег. Вот так же и Эшли — он тоже не был бы ей дорог, если бы в те далекие дни первого знакомства она могла удовлетворить свое тщеславие и отказаться выйти за него замуж. Окажись он в ее власти, стань он пылким, страстным, ревнивым, надутым, молящим, как другие юноши, эта безумная влюбленность, которая владела ею, давным-давно прошла бы, рассеялась, как легкий туман под лучами солнца, лишь только она встретила бы другого мужчину.
«Какая же я была идиотка, — с горечью думала она. — А теперь вот за все это расплачиваюсь. То, чего я так часто желала, — случилось. Я желала, чтобы Мелли умерла и чтобы Эшли стал моим. И вот теперь она умерла и он мой — и он мне не нужен. Его проклятое понятие о чести заставит его спросить меня, не соглашусь ли я развестись с Реттом и выйти за него замуж. Выйти за него замуж? Да он не нужен мне, даже преподнесите мне его на серебряном блюде! Все равно он будет висеть на моей шее до конца моих дней. И пока я буду жива, мне придется заботиться о нем, следить, чтобы он не умер с голоду и чтобы никто не задел его чувств. Просто у меня появится еще одно дитя, которое будет цепляться за мои юбки. Я потеряла возлюбленного и приобрела еще одного младенца. И не пообещай я Мелли, я.., мне было бы все равно, даже если бы я никогда больше его не увидела».

