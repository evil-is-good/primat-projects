\documentclass[a4paper,12pt]{article}
\usepackage{graphicx}
\usepackage[utf8]{inputenc}
\usepackage[russian]{babel}
\usepackage{amsmath,amssymb}
\usepackage{tikz}
\usepackage{pgfplots}
\pgfplotsset{compat=1.7}

\begin{document}
\part*{Подбор сечений консольных и двухопорных балок, работающих на поперечный изгиб}
Для Необходимо построить три балки и определить сечения для этих балок так, чтобы возникающие, под действием заданных систем сил, напряжения не превышали допустимые для заданного материала. Балки, формы сечений и характеристики приведены на рис. 1, рис. 2 и табл. 3 (в том числе значения допустимых напряжений).
Процесс построение балки практически не отличается от описанного в предыдущих заданиях (напр. третьей части). Иным образом только задаётся жескость. Теперь она залается не численно, а в виде сечения. В каждом подзадании свой способ определения сечения.

\includegraphics[width=0.9\linewidth]{2i}

\includegraphics[width=0.9\linewidth]{3i}

\includegraphics[width=0.9\linewidth]{4i}
% \end{figure}
\\
Необходимые для расчета данные: 

\includegraphics[width=0.9\linewidth]{1i}
\\
Для выполнения задания требуется сконструировать ферму из своего варианта. \\

Для этого зайдите в закладку «Узлы и Элементы», нажмите кнопку «Узлы» 
(первая в строке) и затем кнопку «Ввод узлов». Откроется окно «Ввод узлов», 
в нем, сначала оставляете координаты нулевыми и нажимаете кнопку «Добавить». 
\begin{figure}[h]
 \center{\includegraphics[width=0.5\linewidth]{a1}}
 \label{ris:a1}
\end{figure}
Таким образом вы создадите первую точку в начале координат. После этого меняйте 
значения координат X и Z, после каждого нажатия кнопки «Добавить» будет 
добавляться новая точка (кто бы мог подумать).  Создайте все необходимые для 
вашего варианта точки. Если получились лишние или неправильные точки, то их 
можно удалить нажав кнопу «Удаление узлов» на инструментальной панели, а затем 
выделив неугодные точки курсором мыши, после нажатия кнопки «Ок», на панели, они исчезнут. \\

После задания всех точек нужно соединить их стержнями. Для этого во вкладке 
«Узлы и Элементы» нажмите кнопку «Элементы» (вторая на панели) и затем на 
кнопку «Добавление стержней».  Жмите сначала на одну точку, а затем на 
соседнею, между ними образуется стержень. Таким же нехитрым образом задайте все стержни.
Далее задайте жесткость стержней, опоры и нагрузки, как это было показано в 
части 2 обучающего материала. \\

Прежде чем выполнить линейный расчет в дереве проекта, 
в разделе «Исходные данные» откройте подраздел 
«Специальные исходные данные» и  нажмите на 
«Нагрузки от фрагмента схемы».
\begin{figure}[h]
 \center{\includegraphics[width=0.4\linewidth]{a2}}
\end{figure}
В открывшемся 
окне в поле «Список элементов» перечислите номера 
всех элементов фермы. Например как в примере с первого 
элемента по двенадцатый. Затем нажмите кнопку «Новый участок». 
Далее, в поле «Список узлов» причислите 
узлы в которых заданны опоры. После этого можете выполнить линейный расчет. \\

В результатах, в графическом анализе можно посмотреть получившиеся 
эпюры как во второй части.  Для получения реакций опор, в результатов 
нажмите на «Печать таблиц».
\begin{figure}[h]
 \center{\includegraphics[width=0.5\linewidth]{a3}}
\end{figure}
Затем выберите «Нагрузки от фрагмента схемы», 
нажмите «Формирование документа» и затем «Просмотр результатов». Сгенерируется 
текстовый файл с таблицей в которой представлены реакции выбранных узлов. \\

\bf
Не забывайте контролировать входные и выходные единицы измерения. Например, 
при формировании документа в «Параметрах вывода» или в разделе «Опции» $\to$ «Единицы измерений».
По умолчанию, в SCAD-е, единицы измерения силы – тонны, а во всех наших расчетах кН. 



\end{document}
