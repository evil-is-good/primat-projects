\documentclass[a4paper,12pt]{article}
\usepackage{graphicx}
\usepackage[utf8]{inputenc}
\usepackage[russian]{babel}
\usepackage{amsmath,amssymb}
\usepackage{tikz}
\usepackage{pgfplots}
\pgfplotsset{compat=1.7}

\begin{document}
\part*{Часто возникающие проблемы}

% \begin{enumerate}
\section{Единицы измерения} 
        Чтобы результаты совпадали с ответами, и с ними было бы проще работать, 
        задайте единицы измерения. Для этого войдите в меню «Опции $\to$ Единицы измерений». 
        Задайте единицы входные и выходные как показано на рисунках. 
        Не забудьте поставить галочки «Использовать по умолчанию» 
        чтобы каждый раз заново не проделывать эту операцию. \\
        \begin{center}
            \includegraphics[width=0.7\linewidth]{em1} \\
            \includegraphics[width=0.7\linewidth]{em2}
        \end{center}
        \section{Рабочие директории}
        Чтобы scad не ругался на то что не может найти файл 
        или получить доступ задайте рабочие директории. 
    Откройте меню «Опции $\to$ Назначение рабочих директорий». 
        Укажите расположение двух директорий sdata и swork на рабочем столе. 
        Важно: проделать это надо до того как вы создадите проект 
        иначе директории заменить не получится и придётся создавать проект заново.\\
        \begin{center}
        \includegraphics[width=0.5\linewidth]{wd}
        \end{center}
% \end{enumerate}



\end{document}
