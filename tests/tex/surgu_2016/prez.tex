
\documentclass{beamer} 
\usepackage{beamerthemesplit} 

\usetheme{Madrid}%Frankfurt} 
%Boadilla}
\usecolortheme{default}

\setbeamertemplate{footline}{%
    \hspace{0.94\paperwidth}%
    \usebeamerfont{title in head/foot}%
    \large{\insertframenumber}%
}

\setbeamertemplate{navigation symbols}{}

% Выпишем часть возможных стилей, некоторые из них могут содержать

% дополнительные опции

% default, Bergen, Madrid, AnnArbor,Pittsburg, Rochester, 

% Antiles, Montpellier, Berkley, Berlin, CambridgeUS

% 

% Далее пакеты, необходимые вам для создания презентации
\usepackage[T2A]{fontenc}
\usepackage[utf8]{inputenc}
\usepackage[english,russian]{babel}
\usepackage{amssymb,amsfonts,amsmath,mathtext}
\usepackage{cite,enumerate,float,indentfirst}
\usepackage{graphicx}
% \usepackage[usenames]{color}
\usepackage{colortbl}
\usepackage{graphicx,xcolor}

%\DeclareSymbolFont{letters}{OT1}{cmr}{m}{n}

\usefonttheme[stillsansserifsmall]{serif}
%\usefonttheme[onlymath]{serif}

\title{Моделирование и расчет композитного обжимного кольца}
\author{А.Ф. Власко, аспирант}
\institute{Сургутский государственный университет (Сургут) \\
\vspace{0.5cm}
Научный руководитель --- д-р.ф.-м.н. {\bf Г.Л. Горынин}}
\date{Сургут, 2016}


\begin{document} 

\maketitle

\frame{ \frametitle{Периодическая среда} 

\begin{figure}
    \begin{center}
        \includegraphics[scale=0.2]{text7303}
    \end{center}
\end{figure}

}

\frame{ \frametitle{Обжимное кольцо изготовленное из волокнистого материала} 

% \begin{figure}
%     \begin{center}
%         \includegraphics[scale=0.45]{ring}
%     \end{center}
% \end{figure}
%
% \begin{figure}
%     \begin{center}
%         \includegraphics[scale=0.45]{slice}
%     \end{center}
% \end{figure}

\begin{figure}[h]
    \begin{center}
        \begin{minipage}[h]{0.4\linewidth}
            \includegraphics[width=1\linewidth]{ring} \\ \center{Кольцо}
            % \caption{Кольцо.} %% подпись к рисунку
        \end{minipage}
        \hfill 
        \begin{minipage}[h]{0.4\linewidth}
        \includegraphics[width=1\linewidth]{slice} \\ \center{Поперечное сечение кольца}
            % \caption{Сечение кольца.}
        \end{minipage}
    \end{center}
\end{figure}

}

\frame{ \frametitle{Поставновка задачи} 

\text{Уравнения равновесия}
\begin{equation}
        \mathrm{\frac{\partial \sigma_{\alpha x}}{\partial x} +
        \frac{\partial \sigma_{\alpha y}}{\partial y} + 
        \frac{\partial \sigma_{\alpha z}}{\partial z} + F_{\alpha} = 0,}
\end{equation}

\text{Закон Гука}
\begin{equation}
    \begin{array}{c}
        \mathrm{\sigma_{\alpha \beta} = \sum\limits_{\varphi,\psi \in \{x,y,z\}} 
        E_{\alpha \beta \varphi \psi} 
        \frac{\partial u_{\varphi}}{\partial \psi}}; 
        \mathrm{\alpha, \beta \in \{x,y,z\}}
    \end{array} 
\end{equation}

\text{Условия сопряжения материалов}
\begin{equation}
    \mathrm{[\sigma_{\alpha n}] = 0, [u_{\alpha}] = 0}
\end{equation}

\text{Формула контактных напряжений}
\begin{equation}
        \mathrm{[\sigma_{\alpha n}] = 
        [\sigma_{\alpha x}]n_x +
        [\sigma_{\alpha y}]n_y +
        [\sigma_{\alpha z}]n_z}
\end{equation}

\text{Граничное условие на внутренней поверхности кольца}
\begin{equation}
\mathrm{
    \sigma_{\alpha n} = P
}
\end{equation}
} 



\frame{ \frametitle{Поставновка задачи в безразмерных величинах} 

\begin{equation}
    \begin{array}{c}
        \mathrm{\frac{\partial \sigma_{\alpha x}}{\partial x}\varepsilon +
        \frac{\partial \sigma_{\alpha y}}{\partial y}\varepsilon + 
        \frac{\partial \sigma_{\alpha z}}{\partial z}\varepsilon + F_{\alpha} = 0,} \\ \\
        \mathrm{\sigma_{\alpha \beta} = \sum\limits_{\varphi,\psi \in \{x,y,z\}} 
        E_{\alpha \beta \varphi \psi} 
        \frac{\partial u_{\varphi}}{\partial \psi} \varepsilon} \\ \\ 
        \mathrm{\alpha, \beta \in \{x,y,z\}}
    \end{array} 
\end{equation}

\text{Правила обезразмеривания}
\begin{equation}
    \begin{array}{c}
    \mathrm{x \leftrightarrow \frac{x}{L},
    y \leftrightarrow \frac{y}{L},
    z \leftrightarrow \frac{z}{L},
    u_{\alpha} \leftrightarrow \frac{u_{\alpha}}{\widetilde{u}},} \\ \\
    \mathrm{E_{\alpha \beta \varphi \psi} \leftrightarrow  
    \frac{E_{\alpha \beta \varphi \psi}}{\widetilde{E}},
    \sigma_{\alpha \beta} \leftrightarrow \frac{\sigma_{\alpha \beta}}{\widetilde{E}},
    F_{\alpha} \leftrightarrow \frac{F_{\alpha}h}{\widetilde{E}},} \\ \\
    \end{array} 
\end{equation}

\begin{equation*}
    \mathrm{\varepsilon = \frac{h}{L} \ll 1}
\end{equation*}

}

\frame{ \frametitle{Асимптотическое расщепление} 

\begin{equation}
\mathrm{u_{\alpha}^{(n)} = v_{\alpha}^{(n)} + \sum\limits_{\varphi \in \{x,y,z\}}
\sum\limits_{k=1}^{n}\sum\limits_{k_x+k_y+k_z=k}\left( 
\left(U_{\beta}^{v_{\varphi}}\right)^{\bar{k}}
\frac{\partial^k v_{\varphi}^{(n)}}{\partial \bar{r}^{\bar{k}}}\varepsilon^k \right)}
\end{equation}

\begin{equation}
\mathrm{\sigma_{\alpha\beta}^{(n)} = \sum\limits_{\varphi \in \{x,y,z\}}
\sum\limits_{k=1}^{n}\sum\limits_{k_x+k_y+k_z=k}\left( 
\left(\tau_{\alpha\beta}^{v_{\varphi}}\right)^{\bar{k}}
\frac{\partial^k v_{\varphi}^{(n)}}{\partial \bar{r}^{\bar{k}}}\varepsilon^k \right)}
\end{equation}

\begin{equation*}
    \mathrm{\varepsilon = \frac{h}{L} \ll 1}
\end{equation*}

\text{Обозначения}
\begin{equation}
    \begin{array}{c}
        \mathrm{\bar{k}=\{k_x,k_y,k_z\} =
        k_x\bar{\text{э}}_x+k_y\bar{\text{э}}_y+k_z\bar{\text{э}}_z,} \\
        \mathrm{|\bar{k}|=k=k_x+k_y+k_z,} \\ 
        \mathrm{\partial \bar{r}^{\bar{k}}= \partial x^{k_x} \partial y^{k_y} \partial z^{k_z}}, \\
        \mathrm{k_{\alpha} \ge 0, k_{\alpha} \in \mathbb{Z}}, \\
\alpha \in \{x,y,z\}
    \end{array} 
\end{equation}

}

\frame{ \frametitle{Алгоритм решения задачи} 

\begin{figure}
    \begin{center}
        \includegraphics[scale=0.45]{alg}
    \end{center}
\end{figure}
% \begin{figure}\centering
%         \def\svgwidth{10cm} % если надо изменить размер
%     \input{alg.pdf_tex}
%     % \caption{...}\label{...}
% \end{figure}

}

\frame{ \frametitle{Ячейковая система координат} 

\begin{figure}
    \begin{center}
        \includegraphics[scale=0.2]{cell_xi}
    \end{center}
\end{figure}

\begin{equation}
    \mathrm{\xi_x,\xi_y \in [0,1]}
\end{equation}

\begin{equation}
    \mathrm{x = x_i + \xi_x\varepsilon, \
    y = y_i + \xi_y\varepsilon, \
z = z.}
\end{equation}

\begin{equation*}
    \mathrm{x_i, y_i} - \text{координаты левого нижнего угла i-той ячейки}
\end{equation*}

}

\frame[t]{ \frametitle{Краевые задачи на ячейке} 

% \text{9 краевых задач $\eta, \lambda \in \{x,y,z\}$:}
\begin{equation}\label{eq:assimptotic}
    \begin{array}{c}
        \mathrm{\frac
        {\partial 
        \left(\tau_{\alpha x}^{v_{\eta}}\right)^{\overline{k}}}
        {\partial x} +
        \frac
        {\partial
        \left(\tau_{\alpha y}^{v_{\eta}}\right)^{\overline{k}}}
        {\partial y} 
    +\left(\tau_{\alpha x}^{v_{\eta}}\right)^{\overline{k}-\overline{\text{э}}_{x}}
    +\left(\tau_{\alpha y}^{v_{\eta}}\right)^{\overline{k}-\overline{\text{э}}_{y}}
    +\left(\tau_{\alpha z}^{v_{\eta}}\right)^{\overline{k}-\overline{\text{э}}_{z}}
= 0} \\ \\
    \end{array} 
\end{equation}
\begin{equation*}
    \begin{array}{c}
        \mathrm{\left(\tau_{\alpha \beta}^{v_{\eta}}\right)^{\overline{k}}=
            \sum\limits_{\varphi,\psi \in \{x,y,z\}} E_{\alpha \beta \varphi \psi} 
        (\frac{\partial \left(U_{\varphi}^{v_{\eta}}\right)^{\overline{k}}}
    {\partial \xi_{\psi}} + 
\left(U_{\varphi}^{v_{\eta}}\right)^{\overline{k}-\overline{\text{э}}_{\psi}})} \\
        \mathrm{\alpha, \eta \in \{x,y,z\}, |\bar{k}| \in \{1,2\}}
    \end{array} 
\end{equation*}

\text{Условия сопряжения материалов}
\begin{equation}
    \mathrm{\left[\left(\tau_{\alpha n}^{v_{\eta}}\right)^{\overline{k}}\right] = 0, 
            \left[\left(U_{\varphi}^{v_{\eta}}\right)^{\overline{k}}\right] = 0}
\end{equation}

\text{Условия периодичности ячейковых функций}
\begin{equation}
    \begin{array}{c}
        \mathrm{\left. \left(U_{\varphi}^{v_{\eta}}\right)^{\overline{k}} 
        \right|_{\xi_{\gamma} = 0} =
        \left. \left(U_{\varphi}^{v_{\eta}}\right)^{\overline{k}} 
        \right|_{\xi_{\gamma} = 1}} \\ \\
        \mathrm{\left.\left(\tau_{\alpha \gamma}^{v_{\eta}}\right)^{\overline{k}}
        \right|_{\xi_{\gamma} = 0} =
        \left.\left(\tau_{\alpha \gamma}^{v_{\eta}}\right)^{\overline{k}}
        \right|_{\xi_{\gamma} = 1}}, 
        \mathrm{\gamma \in \{x, y\}}
    \end{array} 
\end{equation}

}

\frame[t]{ \frametitle{Краевые задачи на ячейке} 

\begin{equation}\label{eq:assimptotic}
    \begin{array}{c}
        \mathrm{\frac
        {\partial 
        \left(\tau_{\alpha x}^{v_{\eta}}\right)^{\overline{k}}}
        {\partial x} +
        \frac
        {\partial
        \left(\tau_{\alpha y}^{v_{\eta}}\right)^{\overline{k}}}
        {\partial y} 
    +\left(\tau_{\alpha x}^{v_{\eta}}\right)^{\overline{k}-\overline{\text{э}}_{x}}
    +\left(\tau_{\alpha y}^{v_{\eta}}\right)^{\overline{k}-\overline{\text{э}}_{y}}
    +\left(\tau_{\alpha z}^{v_{\eta}}\right)^{\overline{k}-\overline{\text{э}}_{z}}
= 0} \\ \\
    \end{array} 
\end{equation}
\begin{equation*}
    \begin{array}{c}
        \mathrm{\left(\tau_{\alpha \beta}^{v_{\eta}}\right)^{\overline{k}}=
            \sum\limits_{\varphi,\psi \in \{x,y,z\}} E_{\alpha \beta \varphi \psi} 
        (\frac{\partial \left(U_{\varphi}^{v_{\eta}}\right)^{\overline{k}}}
    {\partial \xi_{\psi}} + 
\left(U_{\varphi}^{v_{\eta}}\right)^{\overline{k}-\overline{\text{э}}_{\psi}})} \\
\mathrm{\alpha, \eta \in \{x,y,z\}, |\bar{k}| \in \{1,2\}}
    \end{array} 
\end{equation*}

\text{Дополнительное условие, накладываемое на решение}
\begin{equation}
    \mathrm{\int_0^1\int_0^1
        \left(U_{\varphi}^{v_{\eta}}\right)^{\overline{k}}d\xi_xd\xi_y = 0}
\end{equation}

}

\frame{ \frametitle{Вычисление упругих макроконстант}

\text{Формула вычисление упругих макроконстант}
\begin{equation}
    \begin{array}{c}
        \mathrm{\widetilde{E}_{\alpha \beta \eta \lambda} =
        \left \langle 
        \left(\tau_{\alpha y}^{v_{\eta}}\right)^{\overline{k}}
    \right \rangle =} \\ \\
        \mathrm{= \langle 
        E_{\alpha \beta \eta \lambda} 
        \rangle +
        \left\langle 
        \sum\limits_{\varphi,\psi \in \{x,y\}} E_{\alpha \beta \varphi \psi} 
        \frac{\partial \left(U_{\varphi}^{v_{\eta}}\right)^{\overline{k}}}
        {\partial \xi_{\psi}} \right\rangle +}  \\ \\
        \mathrm{+ \left\langle \sum\limits_{\psi \in \{x,y\}} E_{\alpha \beta z \psi} 
        \frac{\partial \left(U_{z}^{v_{\eta}}\right)^{\overline{k}}}
        {\partial \xi_{\psi}}  
    \right\rangle} \\ \\
        \mathrm{\alpha, \beta, \eta \in \{x,y,z\}, |\bar{k}| = 1}
    \end{array} 
\end{equation}

\text{Оператор усреднения}
\begin{equation}
        \langle \_ \rangle = \int\limits_0^1\int\limits_0^1\_d\xi_xd\xi_y
\end{equation}

}

\frame{ \frametitle{Кольцо изготовленное из однородного макроматериала} 

\begin{figure}
    \begin{center}
        \includegraphics[scale=0.4]{ring}
    \end{center}
\end{figure}

}

\frame{ \frametitle{Поставновка задачи в однородном макроматериала} 

\text{Уравнения равновесия}
\begin{equation}
        \mathrm{\frac{\partial \widetilde{\sigma}_{\alpha x}}{\partial x}\varepsilon +
            \frac{\partial \widetilde{\sigma}_{\alpha y}}{\partial y}\varepsilon + 
        \frac{\partial \widetilde{\sigma}_{\alpha z}}{\partial z}\varepsilon + F_{\alpha} = 0,}
\end{equation}

\text{Закон Гука}
\begin{equation}
        \mathrm{\widetilde{\sigma}_{\alpha \beta} = \sum\limits_{\varphi,\psi \in \{x,y,z\}} 
                \widetilde{E}_{\alpha \beta \varphi \psi} 
        \frac{\partial v_{\varphi}}{\partial \psi} \varepsilon};
        \mathrm{\alpha, \beta \in \{x,y,z\}}
\end{equation}

\text{Граничное условие на внутренней поверхности кольца}
\begin{equation}
\mathrm{
    \sigma_{\alpha n} = P
}
\end{equation}
}

\frame{ \frametitle{Сборка итогового решения} 

\begin{equation}
\mathrm{u_{\alpha}^{(n)} = v_{\alpha}^{(n)} + \sum\limits_{\varphi \in \{x,y,z\}}
\sum\limits_{k=1}^{n}\sum\limits_{k_x+k_y+k_z=k}\left( 
\left(U_{\beta}^{v_{\varphi}}\right)^{\bar{k}}
\frac{\partial^k v_{\varphi}^{(n)}}{\partial \bar{r}^{\bar{k}}}\varepsilon^k \right)}
\end{equation}

\begin{equation}
\mathrm{\sigma_{\alpha\beta}^{(n)} = \sum\limits_{\varphi \in \{x,y,z\}}
\sum\limits_{k=1}^{n}\sum\limits_{k_x+k_y+k_z=k}\left( 
\left(\tau_{\alpha\beta}^{v_{\varphi}}\right)^{\bar{k}}
\frac{\partial^k v_{\varphi}^{(n)}}{\partial \bar{r}^{\bar{k}}}\varepsilon^k \right)}
\end{equation}
}

\frame{ \frametitle{Параметры колец} 

\begin{table}[H]
    \begin{center}
        \begin{tabular}{|c|c|c|c|}
            \hline
            \multicolumn{4}{|c|}{Широкое кольцо} \\
            \hline
            Внутренний радиус & Ширина & Высота & Приложенная сила\\
            \hline
            20 & 5 & 1 & 1\\
            \hline
            \multicolumn{4}{|c|}{Узкое кольцо} \\
            \hline
            Внутренний радиус & Ширина & Высота & Приложенная сила\\
            \hline
            20 & 1 & 5 & 1\\
            \hline
        \end{tabular}
    \end{center}
\end{table} 
}

\frame{ \frametitle{Свойства материалов} 

\begin{table}[H]
    \begin{center}
        \begin{tabular}{|c|c|}
            \hline
            \multicolumn{2}{|c|}{Связующее. Эпоксидная смола.} \\
            \hline
            Модуль Юнга & Коэффициент Пуассона\\
            \hline
            5 & 0.22\\
            \hline
            \multicolumn{2}{|c|}{Включение. Карбоновое волокно.} \\
            \hline
            Модуль Юнга & Коэффициент Пуассона\\
            \hline
            100 & 0.38\\
            \hline
        \end{tabular}
    \end{center}
\end{table} 
}

\frame{ \frametitle{Напряжения $\sigma_\theta$ для широкого кольца} 

\begin{figure}[h]
    \begin{center}
        \begin{minipage}[h]{0.5\linewidth}
            \includegraphics[width=1\linewidth]{/home/primat/projects/tests/new_struct/sources/ring_fig/stress_yy_W}
            % \caption{Кольцо.} %% подпись к рисунку
        \end{minipage}
        \hfill 
        \begin{minipage}[h]{0.4\linewidth}
        \includegraphics[width=1\linewidth]{/home/primat/projects/tests/new_struct/sources/ring_fig/stress_line_yy_W} 
        %\\ \center{Поперечное сечение кольца}
            % \caption{Сечение кольца.}
        \end{minipage}
    \end{center}
\end{figure}
}

\frame{ \frametitle{Напряжения $\sigma_z$ для широкого кольца} 

\begin{figure}[h]
    \begin{center}
        \begin{minipage}[h]{0.5\linewidth}
            \includegraphics[width=1\linewidth]{/home/primat/projects/tests/new_struct/sources/ring_fig/stress_zz_W}
            % \caption{Кольцо.} %% подпись к рисунку
        \end{minipage}
        \hfill 
        \begin{minipage}[h]{0.4\linewidth}
        \includegraphics[width=1\linewidth]{/home/primat/projects/tests/new_struct/sources/ring_fig/stress_line_zz_W} 
        %\\ \center{Поперечное сечение кольца}
            % \caption{Сечение кольца.}
        \end{minipage}
    \end{center}
\end{figure}
}

\frame{ \frametitle{Напряжения $\sigma_\theta$ для узкого кольца} 

\begin{figure}[h]
    \begin{center}
        \begin{minipage}[h]{0.5\linewidth}
            \includegraphics[width=1\linewidth]{/home/primat/projects/tests/new_struct/sources/ring_fig/stress_yy_H}
            % \caption{Кольцо.} %% подпись к рисунку
        \end{minipage}
        \hfill 
        \begin{minipage}[h]{0.4\linewidth}
        \includegraphics[width=1\linewidth]{/home/primat/projects/tests/new_struct/sources/ring_fig/stress_line_yy_H} 
        %\\ \center{Поперечное сечение кольца}
            % \caption{Сечение кольца.}
        \end{minipage}
    \end{center}
\end{figure}
}

\frame{ \frametitle{Напряжения $\sigma_z$ для узкого кольца} 

\begin{figure}[h]
    \begin{center}
        \begin{minipage}[h]{0.5\linewidth}
            \includegraphics[width=1\linewidth]{/home/primat/projects/tests/new_struct/sources/ring_fig/stress_zz_H}
            % \caption{Кольцо.} %% подпись к рисунку
        \end{minipage}
        \hfill 
        \begin{minipage}[h]{0.4\linewidth}
        \includegraphics[width=1\linewidth]{/home/primat/projects/tests/new_struct/sources/ring_fig/stress_line_zz_H} 
        %\\ \center{Поперечное сечение кольца}
            % \caption{Сечение кольца.}
        \end{minipage}
    \end{center}
\end{figure}
}

\frame{ \frametitle{Сравнение напряжений $\sigma_\theta$ для широкого и узкого колец} 

\begin{figure}[h]
    \begin{center}
        \begin{minipage}[h]{0.4\linewidth}
            \includegraphics[width=1\linewidth]{/home/primat/projects/tests/new_struct/sources/ring_fig/stress_line_yy_W} \\ \center{Широкое кольцо}
            % \caption{Кольцо.} %% подпись к рисунку
        \end{minipage}
        \hfill 
        \begin{minipage}[h]{0.4\linewidth}
        \includegraphics[width=1\linewidth]{/home/primat/projects/tests/new_struct/sources/ring_fig/stress_line_yy_H} \\ \center{Узкое кольцо}
        %\\ \center{Поперечное сечение кольца}
            % \caption{Сечение кольца.}
        \end{minipage}
    \end{center}
\end{figure}
}

\frame{ \frametitle{Сравнение напряжений $\sigma_z$ для широкого и узкого колец} 

\begin{figure}[h]
    \begin{center}
        \begin{minipage}[h]{0.4\linewidth}
            \includegraphics[width=1\linewidth]{/home/primat/projects/tests/new_struct/sources/ring_fig/stress_line_zz_W} \\ \center{Широкое кольцо}
            % \caption{Кольцо.} %% подпись к рисунку
        \end{minipage}
        \hfill 
        \begin{minipage}[h]{0.4\linewidth}
        \includegraphics[width=1\linewidth]{/home/primat/projects/tests/new_struct/sources/ring_fig/stress_line_zz_H} \\ \center{Узкое кольцо}
        %\\ \center{Поперечное сечение кольца}
            % \caption{Сечение кольца.}
        \end{minipage}
    \end{center}
\end{figure}
}


\end{document}
















\frame{ \frametitle{Сечение вдоль которого демонстрируются напряжения} 

\begin{figure}
    \begin{center}
        \includegraphics[scale=0.4]{hole_line}
    \end{center}
\end{figure}
}

\frame{ \frametitle{Деформации в макроматериале $\mathrm{\omega_{yy} = \frac{\partial v_y}{\partial y}}$} 

\begin{figure}
    \begin{center}
        \includegraphics[scale=0.2]{makro_stress_y_y}
    \end{center}
\end{figure}
}

\frame{ \frametitle{Волокна} 

\begin{figure}
    \begin{center}
        \includegraphics[scale=0.4]{layer}
    \end{center}
\end{figure}
}

\frame{ \frametitle{Ячейковые напряжения} 

\begin{figure}
    \begin{center}
        \includegraphics[scale=0.2]{cell_stress_y_y}
    \end{center}
\end{figure}
}

\frame{ \frametitle{Итоговое напряжение в неоднородном материале при $\varepsilon=0.1$} 

\begin{figure}
    \begin{center}
        \includegraphics[scale=0.2]{stress_y_y_10}
    \end{center}
\end{figure}
}

\frame{ \frametitle{Итоговое напряжение в неоднородном материале при $\varepsilon=0.01$} 

\begin{figure}
    \begin{center}
        \includegraphics[scale=0.2]{stress_y_y}
    \end{center}
\end{figure}
}

\frame{ \frametitle{Итоговое напряжение в неоднородном материале при $\varepsilon=0.01$} 

\begin{figure}
    \begin{center}
        \includegraphics[scale=0.2]{stress_x_x}
    \end{center}
\end{figure}
}

\frame{ \frametitle{Асимптотическое расщепление напряжения, ограниченное n=2} 

\begin{equation}
    \begin{array}{c}
\mathrm{\sigma_{\alpha\beta}^{(n)} = \sum\limits_{\varphi \in \{x,y,z\}}
\sum\limits_{k_x+k_y+k_z=1}\left( 
\left(\tau_{\alpha\beta}^{v_{\varphi}}\right)^{\bar{k}}
\frac{\partial v_{\varphi}^{(n)}}{\partial \bar{r}^{\bar{k}}}\varepsilon \right) +}\\
\mathrm{\sum\limits_{\varphi \in \{x,y,z\}}
\sum\limits_{k_x+k_y+k_z=2}\left( 
\left(\tau_{\alpha\beta}^{v_{\varphi}}\right)^{\bar{k}}
\frac{\partial^2 v_{\varphi}^{(n)}}{\partial \bar{r}^{\bar{k}}}\varepsilon^2 \right)}
    \end{array} 
\end{equation}
}

\frame{ \frametitle{Вторые производные от перемещений в макроматериале $\mathrm{\omega_{yyx} = \frac{\partial v_y}{\partial y \partial x}}$} 

\begin{figure}
    \begin{center}
        \includegraphics[scale=0.2]{makro_stress_y_y_x}
    \end{center}
\end{figure}
}

\frame{ \frametitle{Ячейковые напряжения, второе приближение} 

\begin{figure}
    \begin{center}
        \includegraphics[scale=0.2]{cell_stress_y_y_2}
    \end{center}
\end{figure}
}

\end{document}
% \begin{figure}[h]
%     \begin{center}
%         \begin{minipage}[h]{0.4\linewidth}
%             \includegraphics[width=1\linewidth]{path6562-8-2-6}
%             \caption{Исходное изображение.} %% подпись к рисунку
%             \label{ris:experimoriginal} %% метка рисунка для ссылки на него
%         \end{minipage}
%         \hfill 
%         \begin{minipage}[h]{0.4\linewidth}
%             \includegraphics[width=1\linewidth]{stress}
%             \caption{Закодированное изображение.}
%             \label{ris:experimcoded}
%         \end{minipage}
%     \end{center}
% \end{figure}

}

\frame{ \frametitle{Напряжение $\sigma_{yy}/\sigma_{p}$} 

\begin{figure}
    \begin{center}
        \includegraphics[scale=0.5]{sigma_yy}
    \end{center}
\end{figure}
}

%\frame{ \frametitle{Одна ячейка} 

%\begin{figure}
%    \begin{center}
%        \includegraphics[scale=0.3]{cell}
%    \end{center}
%\end{figure}

%}

% \frame{ \frametitle{Верхняя и нижняя оценки Рейсса-Фойгта} 
%
% \begin{figure}
%     \begin{center}
%         \includegraphics[scale=0.2]{cell}
%     \end{center}
% \end{figure}
%
% \begin{equation}
%     \frac{1}{\frac{\Theta^I}{E_{\alpha\beta\varphi\psi}^I} +
%     \frac{\Theta^B}{E_{\alpha\beta\varphi\psi}^B}} \le
%     \widetilde{E}_{\alpha\beta\varphi\psi} \le
%     \Theta^IE_{\alpha\beta\varphi\psi}^I +
%     \Theta^BE_{\alpha\beta\varphi\psi}^B
% \end{equation}
%
% \begin{equation}
%     \begin{array}{c}
%         \Theta^I - \text{доля включения (коэффициент армирования)} \\ \\
%         \Theta^B - \text{доля связующего}
%     \end{array} 
% \end{equation}
%
% }
%
% \frame{ \frametitle{Верхняя и нижняя оценки Хашина-Штрикмана} 
%
% \begin{equation}
%     \begin{array}{c}
%     K^B + \frac{\Theta^I}{1/(K^I-K^B) + \Theta^B/(K^B+G^B)} \le 
%     \widetilde{K}_{xy} \le \\
%     K^I + \frac{\Theta^B}{1/(K^B-K^I) + \Theta^I/(K^I+G^I)}
%     \end{array} 
% \end{equation}
%
% \begin{equation}
%     \begin{array}{c}
%     G^B + \frac{\Theta^I}{1/(G^I-G^B) + \Theta_B(K^B+2G^B)/(2G^B(K^B+G^B))} \le 
%     \widetilde{G}_{xy} \le \\
%     G^I + \frac{\Theta^B}{1/(G^B-G^I) + \Theta_I(K^I+2G^I)/(2G^I(K^I+G^I))} \le 
%     \end{array} 
% \end{equation}
%
% \begin{equation}
%     \begin{array}{c}
%     G^B + \frac{\Theta^I}{1/(G^I-G^B) + \Theta^B/(2G^B))} \le 
%     \widetilde{G}_{xz} \le \\ 
%     G^I + \frac{\Theta^B}{1/(G^B-G^I) + \Theta^I/(2G^I)} 
%     \end{array} 
% \end{equation}
%
% \begin{equation}
%     \begin{array}{c}
%     \frac{\Theta^I\Theta^B}{\Theta^I/K^B+\Theta^B/K^I+1/G^B} \le 
%     \frac{\widetilde{E}_{zz} - \Theta^IE^I - \Theta^BE^B}{4(\nu^I-\nu^B)^2} \le \\ 
%     \frac{\Theta^I\Theta^B}{\Theta^I/K^B+\Theta^B/K^I+1/G^I} 
%     \end{array} 
% \end{equation}
%
% \begin{equation}
%     \begin{array}{c}
%     \frac{\Theta^I\Theta^B}{\Theta^I/K^B+\Theta^B/K^I+1/G^B} \le 
%     \frac{\widetilde{\nu}_{xz} - \Theta^I\nu^I - \Theta^B\nu^B}
%     {(\nu^I-\nu^B)(1/K^B-1/K^I)} \le \\ 
%     \frac{\Theta^I\Theta^B}{\Theta^I/K^B+\Theta^B/K^I+1/G^I} 
%     \end{array} 
% \end{equation}
%
% }
%
% \frame{ \frametitle{Выражения Хашина-Хилла для полидисперсной модели} 
%
% \begin{equation}
%     \begin{array}{c}
%     \widetilde{K}_{xy} = k^B + \frac{G^B}{3} + \\
%     +\frac{\Theta^I}{1/[k^I-k^B+1/3(G^I-G^B)]+(1-\Theta^I)/(k^B+4/3G^B)}
%     \end{array} 
% \end{equation}
%
% \begin{equation}
%     \begin{array}{c}
%     \widetilde{G}_{xz} = 
%     \frac{G^I(1+\Theta^I)+G^B(1-\Theta^I)}
%     {G^I(1-\Theta^I)+G^B(1+\Theta^I)} G^B
%     \end{array} 
% \end{equation}
%
% \begin{equation}
%     \begin{array}{c}
%     \widetilde{E}_{zz} = \Theta^IE^I+\Theta^BE^B+ \\
%     +\frac{4\Theta^I\Theta^B(\mu^I-\nu^B)^2G^B}
%     {\Theta^BG^B/(k^I-G^I/3)+\Theta^IG^B/(k^B-G^B/3)+1}
%     \end{array} 
% \end{equation}
%
% \begin{equation}
%     \begin{array}{c}
%         \widetilde{\nu}_{xz} = \Theta^I\nu^I + \Theta^B\nu^B + \\
%         +\frac
%         {\Theta^I\Theta^B(\nu^I-nu^B)(G^B/(k^B-G^B/3)-G^B/(k^I-G^I/3))}
%         {\Theta^BG^B/(k^I-G^I/3)+\Theta^IG^B/(k^B-G^B/3)+1}
%     \end{array} 
% \end{equation}
%
% }

% \frame[t]{ \frametitle{Краевые задачи на ячейке} 
%
% \text{9 краевых задач $\eta, \lambda \in \{x,y,z\}$:}
% \begin{equation}
%     \begin{array}{c}
%         \mathrm{\frac
%         {\partial 
%         \left(\tau_{\alpha x}^{v_{\eta}}\right)^{\overline{\text{э}}_{\lambda}}}
%         {\partial x} +
%         \frac
%         {\partial
%         \left(\tau_{\alpha y}^{v_{\eta}}\right)^{\overline{\text{э}}_{\lambda}}}
%         {\partial y} = 0} \tag{\ref{eq:assimptotic}}
%     \end{array} 
% \end{equation}
% \begin{equation*}
%     \begin{array}{c}
%         \mathrm{\left(\tau_{\alpha \beta}^{v_{\eta}}\right)^{\overline{\text{э}}_{\lambda}}=
%         E_{\alpha \beta \eta \lambda} + 
%         \sum\limits_{\varphi,\psi \in \{x,y\}} E_{\alpha \beta \varphi \psi} 
%         \frac{\partial \left(U_{\varphi}^{v_{\eta}}\right)^{\overline{\text{э}}_{\lambda}}}
%         {\partial \xi_{\psi}}} 
%         \mathrm{+ \sum\limits_{\psi \in \{x,y\}} E_{\alpha \beta z \psi} 
%         \frac{\partial \left(U_{z}^{v_{\eta}}\right)^{\overline{\text{э}}_{\lambda}}}
%         {\partial \xi_{\psi}}} \\ 
%         \mathrm{\alpha \in \{x,y\}}
%     \end{array} 
% \end{equation*}
%
% \text{Равенства, сокращающие число необходимых задач до шести}
% \begin{equation}
%     \mathrm{
%         \left(\tau_{\alpha \beta}^{v_{\eta}}\right)^{\overline{\text{э}}_{\lambda}} = 
%         \left(\tau_{\alpha \beta}^{v_{\lambda}}\right)^{\overline{\text{э}}_{\eta}},
%         \left(U_{\varphi}^{v_{\eta}}\right)^{\overline{\text{э}}_{\lambda}} =
%         \left(U_{\varphi}^{v_{\lambda}}\right)^{\overline{\text{э}}_{\eta}}
%     }
% \end{equation}
% }

\frame{ \frametitle{}
}


\end{document}

\frame{ \frametitle{Периодические ячейки с различными формами поперечных сечений арматурных волокон} 

\begin{figure}
    \begin{center}
        \includegraphics[scale=0.3]{cells}
    \end{center}
\end{figure}

\begin{equation}
    \begin{array}{c}
        \mathrm{w = b - \sqrt{b^2 - S^I}, b \approx 0.75} \\ \\
        \mathrm{S^I - \text{Площадь включения}}
    \end{array} 
\end{equation}

}

\frame{ \frametitle{Модуль Юнга $E_x$} 

\begin{figure}
    \begin{center}
        \includegraphics[scale=0.3]
        {../../elastic_test_on_cell/sources/i_10/article_nowosib/10_Ex}
    \end{center}
\end{figure}

}


\frame{ \frametitle{Модуль сдвига $G_{xy}$} 

\begin{figure}
    \begin{center}
        \includegraphics[scale=0.3]
        {../../elastic_test_on_cell/sources/i_10/article_nowosib/10_Mxy}
    \end{center}
\end{figure}

}

\frame{ \frametitle{Модуль сдвига $G_{xz}$} 

\begin{figure}
    \begin{center}
        \includegraphics[scale=0.3]
        {../../elastic_test_on_cell/sources/i_10/article_nowosib/10_Mxz}
    \end{center}
\end{figure}

}

\frame{ \frametitle{Коэффициент Пуасона $\nu_{xy}$} 

\begin{figure}
    \begin{center}
        \includegraphics[scale=0.3]
        {../../elastic_test_on_cell/sources/i_10/article_nowosib/10_Nxy}
    \end{center}
\end{figure}

}

\frame{ \frametitle{Коэффициент Пуасона $\nu_{xz}$} 

\begin{figure}
    \begin{center}
        \includegraphics[scale=0.3]
        {../../elastic_test_on_cell/sources/i_10/article_nowosib/10_Nxz}
    \end{center}
\end{figure}

}

\frame{ \frametitle{Коэффициент Пуасона $\nu_{zx}$} 

\begin{figure}
    \begin{center}
        \includegraphics[scale=0.3]
        {../../elastic_test_on_cell/sources/i_10/article_nowosib/10_Nzx}
    \end{center}
\end{figure}

}

\frame{ \frametitle{Максимальное относительное отклонение значений, 
вычисленных по формуле Хашина-Хилла от численных результатов} 

% \begin{center}
%     \begin{tabular}{cccccccc}
%          & $E_x$ & $E_z$ & $G_{xy}$ & $G_{xz}$ & $\nu_{xy}$ & $\nu_{xz}$ & $\nu_{zx}$  \\
%         Квадрат & 0.110334 & 6.04E-05 & 0.20481 & 0.022313 & 0.524838 & 0.112269 & 0.004455 \\
%         Круг & 0.215916 & 0.00361 & 0.129223 & 0.176678 & 0.373822 & 0.22775 & 0.015489 \\
%         Крестовина & 0.1385 & 0.000987 & 0.160598 & 0.125731 & 0.480372 & 0.163576 & 0.033257 \\
%         Трубка & 0.224928 & 0.001328 & 0.160598 & 0.210531 & 0.481037 & 0.259006 & 0.050899 
%     \end{tabular}
% \end{center}

% \begin{table}[H]
\text{Отношение модулей Юнга $\frac{E^I}{E^B}=10$}
\begin{center}
    \begin{tabular}{|c|c|c|c|c|c|c|c|}
        \hline
         & $E_x$ & $E_z$ & $G_{xy}$ & $G_{xz}$ & $\nu_{xy}$ & $\nu_{xz}$ & $\nu_{zx}$  \\
        \hline
    Квадрат & 0.110 & \textcolor{blue}{0.0} & \textcolor{red}{0.204} & \textcolor{blue}{0.022} & \textcolor{red}{0.525} & 0.112 & \textcolor{blue}{0.004} \\
        \hline
        Круг & \textcolor{red}{0.216} & \textcolor{blue}{0.0} &   0.129 & 0.176 & \textcolor{red}{0.374} & \textcolor{red}{0.228} & \textcolor{blue}{0.015} \\
        \hline
        Крестовина & 0.13 &\textcolor{blue}{0.001} & 0.160 & 0.126 & \textcolor{red}{0.480} & 0.163 & \textcolor{blue}{0.033} \\
        \hline
        Трубка &  \textcolor{red}{0.225} & \textcolor{blue}{0.001} & 0.160 & \textcolor{red}{0.210} & \textcolor{red}{0.481} & \textcolor{red}{0.259} & \textcolor{blue}{0.051} \\
        \hline
    \end{tabular}
\end{center}
\text{Отношение модулей Юнга $\frac{E^I}{E^B}=100$}
\begin{center}
    \begin{tabular}{|c|c|c|c|c|c|c|c|}
        \hline
         & $E_x$ & $E_z$ & $G_{xy}$ & $G_{xz}$ & $\nu_{xy}$ & $\nu_{xz}$ & $\nu_{zx}$  \\
        \hline
        Квадрат & \textcolor{red}{0.193} & \textcolor{blue}{0.0} & \textcolor{red}{0.49} & \textcolor{blue}{0.044} & \textcolor{red}{2.768} & \textcolor{red}{0.193} & \textcolor{blue}{0.004} \\
        \hline
        Круг & \textcolor{red}{0.435} & \textcolor{blue}{0.037} & \textcolor{red}{0.412} & \textcolor{red}{0.35} & \textcolor{red}{1.658} & \textcolor{red}{0.479} & \textcolor{blue}{0.013} \\
        \hline
        Крестовина & \textcolor{red}{0.261} & \textcolor{blue}{0.001} & \textcolor{red}{0.256} & \textcolor{red}{0.311} & \textcolor{red}{1.305} & \textcolor{red}{0.307} & \textcolor{blue}{0.069} \\
        \hline
        Трубка & \textcolor{red}{0.47} & \textcolor{blue}{0.001} & \textcolor{red}{0.256} & \textcolor{red}{0.55} & \textcolor{red}{1.326} & \textcolor{red}{0.517} & 0.112 \\
        \hline
    \end{tabular}
\end{center}
% \end{table} 
}
\end{document} 
