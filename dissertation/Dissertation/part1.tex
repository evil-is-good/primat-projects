\chapter{Для затравки ай} \label{chapt1}

Композитный материал -- искуственно созданный материал, обладающий неоднородными физическими свойствами.
В данной работе ему ставится в соответствие математическая модель, описываемая материальными функциями определяющих соотношений. Материальные функции
разрывные по координатам.
Определяющее соотношение имеет вид:
% искусственно созданный неоднородный сплошной материал, состоящий из двух или более компонентов с чёткой границей раздела между ними. 
\begin{equation}
    \label{eq:BasicDefRelations}
\mathfrak{b} = \mathfrak{F}(\mathfrak{a}, \vec{x}),
\end{equation}
где, $\mathfrak{a}$ может быть, например, градиентом температуры, в случае задачи темплопроводности или тензором деформации, в случае
задачи упругости; $\mathfrak{b}$ будет, соответственно, вектором теплопроводности, либо тензором напряжения; $\vec{x}$ -- вектор координат.

В вырожении \ref{eq:BasicDefRelations} явно задана зависимость материальных функций от координат.

Периодический композитный материал -- композитный материал, материальные функции, которого, периодические по координатам.
\begin{equation}
    \label{eq:BasicPeriodicFuction}
    \mathfrak{F}(\mathfrak{a}, \vec{x} + n_i\vec{a_i}) = \mathfrak{F}(\mathfrak{a}, \vec{x}),
\end{equation}

где, $\vec{a_i}$ -- постоянные векторы, $n_i$ -- произвольные числа.

Периодические композитные материалы, по типу периодичности, можно разделить на 3 вида:

1-периодические композитные материалы (слоистые) (рис. \ref{img:one_period}).
Физические свойства изменяются периодически вдоль одного направления (на рис. \ref{img:one_period} направление $Ox$). В доль двух других направлений
физические свойства не изменяются.

\begin{equation}
    \begin{array}{ccc}
        \vec{a}_x = \left(\begin{array}{c}h_x\\0\\0\end{array}\right) & 
        \vec{a}_y = \left(\begin{array}{c}0\\0\\0\end{array}\right) & 
        \vec{a}_z = \left(\begin{array}{c}0\\0\\0\end{array}\right)
    \end{array}
\end{equation}

% \begin{figure} [h] 
%     \center
%     \includegraphics [scale=0.5] {one_period}
%     \caption{1-периодчекая среда.} 
%     \label{img:one_period}  
% \end{figure}
\begin{figure}[h]
  \def\svgwidth{1.5\textwidth}%
  \resizebox{\textwidth}{!}{%
    % GNUPLOT: LaTeX picture with Postscript
\begingroup
  \makeatletter
  \providecommand\color[2][]{%
    \GenericError{(gnuplot) \space\space\space\@spaces}{%
      Package color not loaded in conjunction with
      terminal option `colourtext'%
    }{See the gnuplot documentation for explanation.%
    }{Either use 'blacktext' in gnuplot or load the package
      color.sty in LaTeX.}%
    \renewcommand\color[2][]{}%
  }%
  \providecommand\includegraphics[2][]{%
    \GenericError{(gnuplot) \space\space\space\@spaces}{%
      Package graphicx or graphics not loaded%
    }{See the gnuplot documentation for explanation.%
    }{The gnuplot epslatex terminal needs graphicx.sty or graphics.sty.}%
    \renewcommand\includegraphics[2][]{}%
  }%
  \providecommand\rotatebox[2]{#2}%
  \@ifundefined{ifGPcolor}{%
    \newif\ifGPcolor
    \GPcolorfalse
  }{}%
  \@ifundefined{ifGPblacktext}{%
    \newif\ifGPblacktext
    \GPblacktexttrue
  }{}%
  % define a \g@addto@macro without @ in the name:
  \let\gplgaddtomacro\g@addto@macro
  % define empty templates for all commands taking text:
  \gdef\gplbacktext{}%
  \gdef\gplfronttext{}%
  \makeatother
  \ifGPblacktext
    % no textcolor at all
    \def\colorrgb#1{}%
    \def\colorgray#1{}%
  \else
    % gray or color?
    \ifGPcolor
      \def\colorrgb#1{\color[rgb]{#1}}%
      \def\colorgray#1{\color[gray]{#1}}%
      \expandafter\def\csname LTw\endcsname{\color{white}}%
      \expandafter\def\csname LTb\endcsname{\color{black}}%
      \expandafter\def\csname LTa\endcsname{\color{black}}%
      \expandafter\def\csname LT0\endcsname{\color[rgb]{1,0,0}}%
      \expandafter\def\csname LT1\endcsname{\color[rgb]{0,1,0}}%
      \expandafter\def\csname LT2\endcsname{\color[rgb]{0,0,1}}%
      \expandafter\def\csname LT3\endcsname{\color[rgb]{1,0,1}}%
      \expandafter\def\csname LT4\endcsname{\color[rgb]{0,1,1}}%
      \expandafter\def\csname LT5\endcsname{\color[rgb]{1,1,0}}%
      \expandafter\def\csname LT6\endcsname{\color[rgb]{0,0,0}}%
      \expandafter\def\csname LT7\endcsname{\color[rgb]{1,0.3,0}}%
      \expandafter\def\csname LT8\endcsname{\color[rgb]{0.5,0.5,0.5}}%
    \else
      % gray
      \def\colorrgb#1{\color{black}}%
      \def\colorgray#1{\color[gray]{#1}}%
      \expandafter\def\csname LTw\endcsname{\color{white}}%
      \expandafter\def\csname LTb\endcsname{\color{black}}%
      \expandafter\def\csname LTa\endcsname{\color{black}}%
      \expandafter\def\csname LT0\endcsname{\color{black}}%
      \expandafter\def\csname LT1\endcsname{\color{black}}%
      \expandafter\def\csname LT2\endcsname{\color{black}}%
      \expandafter\def\csname LT3\endcsname{\color{black}}%
      \expandafter\def\csname LT4\endcsname{\color{black}}%
      \expandafter\def\csname LT5\endcsname{\color{black}}%
      \expandafter\def\csname LT6\endcsname{\color{black}}%
      \expandafter\def\csname LT7\endcsname{\color{black}}%
      \expandafter\def\csname LT8\endcsname{\color{black}}%
    \fi
  \fi
  \setlength{\unitlength}{0.0500bp}%
  \begin{picture}(7200.00,5040.00)%
    \gplgaddtomacro\gplbacktext{%
      \csname LTb\endcsname%
      \put(942,1289){\makebox(0,0){\strut{} 0}}%
      \put(1266,1230){\makebox(0,0){\strut{} 0.1}}%
      \put(1590,1170){\makebox(0,0){\strut{} 0.2}}%
      \put(1914,1111){\makebox(0,0){\strut{} 0.3}}%
      \put(2238,1051){\makebox(0,0){\strut{} 0.4}}%
      \put(2562,992){\makebox(0,0){\strut{} 0.5}}%
      \put(2886,933){\makebox(0,0){\strut{} 0.6}}%
      \put(3209,873){\makebox(0,0){\strut{} 0.7}}%
      \put(3533,814){\makebox(0,0){\strut{} 0.8}}%
      \put(3856,754){\makebox(0,0){\strut{} 0.9}}%
      \put(4180,695){\makebox(0,0){\strut{} 1}}%
      \put(4398,755){\makebox(0,0){\strut{} 0}}%
      \put(4585,858){\makebox(0,0){\strut{} 0.1}}%
      \put(4772,961){\makebox(0,0){\strut{} 0.2}}%
      \put(4959,1064){\makebox(0,0){\strut{} 0.3}}%
      \put(5146,1167){\makebox(0,0){\strut{} 0.4}}%
      \put(5333,1270){\makebox(0,0){\strut{} 0.5}}%
      \put(5520,1373){\makebox(0,0){\strut{} 0.6}}%
      \put(5707,1475){\makebox(0,0){\strut{} 0.7}}%
      \put(5894,1578){\makebox(0,0){\strut{} 0.8}}%
      \put(6081,1681){\makebox(0,0){\strut{} 0.9}}%
      \put(6268,1784){\makebox(0,0){\strut{} 1}}%
      \put(920,2070){\makebox(0,0)[r]{\strut{}-1}}%
      \put(920,2413){\makebox(0,0)[r]{\strut{}-0.5}}%
      \put(920,2755){\makebox(0,0)[r]{\strut{} 0}}%
      \put(920,3098){\makebox(0,0)[r]{\strut{} 0.5}}%
      \put(920,3441){\makebox(0,0)[r]{\strut{} 1}}%
    }%
    \gplgaddtomacro\gplfronttext{%
    }%
    \gplbacktext
    \put(0,0){\includegraphics{num1}}%
    \gplfronttext
  \end{picture}%
\endgroup
%
  }
  \caption{Test with ``figure'' environment.}
\end{figure}
2-периодические коомпозитные материалы (армированные) (рис. \ref{img:two_period}).
Физические свойства изменяются периодически вдоль двух направлений (на рис. \ref{img:two_period} $Ox$ и $Oy$). 
В доль третьего направления ($Oz$) физические свойства не изменяются. 

\begin{equation}
    \begin{array}{ccc}
        \vec{a}_x = \left(\begin{array}{c}h_x\\0\\0\end{array}\right) & 
        \vec{a}_y = \left(\begin{array}{c}0\\h_y\\0\end{array}\right) & 
        \vec{a}_z = \left(\begin{array}{c}0\\0\\0\end{array}\right)
    \end{array}
\end{equation}

% \begin{figure} [h] 
%     \center
%     \includegraphics [scale=0.5] {two_period}
%     \caption{2-периодчекая среда.} 
%     \label{img:two_period}  
% \end{figure}

3-периодические коомпозитные материалы (рис. \ref{img:three_period}).
Физические свойства изменяются периодически вдоль всех трёх направлений.

\begin{equation}
    \begin{array}{ccc}
        \vec{a}_x = \left(\begin{array}{c}h_x\\0\\0\end{array}\right) & 
        \vec{a}_y = \left(\begin{array}{c}0\\h_y\\0\end{array}\right) & 
        \vec{a}_z = \left(\begin{array}{c}0\\0\\h_z\end{array}\right)
    \end{array}
\end{equation}

% \begin{figure} [h] 
%     \center
%     \includegraphics [scale=0.5] {three_period}
%     \caption{3-периодчекая среда.} 
%     \label{img:three_period}  
% \end{figure}

Область в которой определяющие соотношения \ref{eq:BasicDefRelations} непрерывны по координате будем называть компонентом композита.
Композит ожет иметь два и более компонент. Если композит двухкомпонентный, то один из компонентом можно называть связующим (матрицей), а другой
включениями (в случае армированного композита арматурой, волокнами).

%\newpage
%============================================================================================================================

\clearpage
